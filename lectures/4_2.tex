% !TEX root = ../main.tex

\section{Багатовимірний нормальний розподіл}
\subsection{Виведення характеристичної функції та щільності}

Розглянемо випадковий вектор $\vec{\xi} = (\xi_1, \xi_2, ..., \xi_n)$, $\xi_k \sim {N}(a_k^o, \sigma_k)$ для $k=1,...,n$, координати незалежні у сукупності.
За властивостями характеристичної функції та щільності $\chi_{\vec{\xi}}(\vec{t}) = \prod\limits_{k=1}^n \chi_{\xi_k}(t_k)$,
$f_{\vec{\xi}}(\vec{x}) = \prod\limits_{k=1}^n f_{\xi_k}(x_k)$.

\begin{gather*}
    \chi_{\vec{\xi}}(\vec{t}) = \prod\limits_{k=1}^n e^{i a_k^o t_k - \frac{\sigma_k^2 t_k^2}{2}} = \exp\left\{i(\vec{a^o}, \vec{t}) - \frac{1}{2}\sum\limits_{k=1}^n \sigma_k^2 t_k^2\right\}
    \\
    f_{\vec{\xi}}(\vec{x}) = \prod\limits_{k=1}^n \frac{1}{\sqrt{2\pi}\sigma_k} e^{-\frac{(x-a_k^o)^2}{2\sigma_k^2}} = \frac{1}{(2\pi)^{n/2}\prod_{k=1}^n \sigma_k} \exp \left\{ -\frac{1}{2} \sum_{k=1}^n \frac{(x-a_k^o)^2}{\sigma_k^2}\right\}
\end{gather*}

Введемо матрицю $D = \begin{pmatrix}
    \sigma_1^2 & 0 & \cdots & 0 \\
    0 & \sigma_2^2 & \cdots & 0 \\
    \vdots & \vdots & \ddots & \vdots \\
    0 & 0 & \cdots & \sigma_n^2
\end{pmatrix}$, $D^{-1} = \begin{pmatrix}
    1/\sigma_1^2 & 0 & \cdots & 0 \\
    0 & 1/\sigma_2^2 & \cdots & 0 \\
    \vdots & \vdots & \ddots & \vdots \\
    0 & 0 & \cdots & 1/\sigma_n^2
\end{pmatrix}$.

\noindent Тоді отримаємо
\begin{gather}
    \chi_{\vec{\xi}}(\vec{t}) = \exp\left\{i(\vec{a^o}, \vec{t}) - \frac{1}{2}(D\vec{t}, \vec{t})\right\}
    \\
    f_{\vec{\xi}}(\vec{x}) = \frac{1}{(2\pi)^{n/2} \sqrt{{\det{D}}}} \exp \left\{ -\frac{1}{2} \left( D^{-1}(\vec{x} - \vec{a^o}), (\vec{x} - \vec{a^o})\right)\right\}
\end{gather}
Причому в такому випадку $D$ --- кореляційна матриця.

Тепер розглянемо вектор $\vec{\eta} = A\vec{\xi}$. За властивістю 
$\chi_{\vec{\eta}}(\vec{t}) = \chi_{\vec{\xi}}(A^{*}\vec{t})$. Маємо
\begin{gather*}
    \chi_{\vec{\eta}}(\vec{t}) = \exp\left\{i(\vec{a^o}, A^{*}\vec{t}) - \frac{1}{2}(DA^{*}\vec{t}, A^{*}\vec{t})\right\} =
    \exp\left\{i(A\vec{a^o}, \vec{t}) - \frac{1}{2}(ADA^{*}\vec{t}, \vec{t})\right\}
\end{gather*}
Доведемо, що $K = ADA^{*}$ --- симетрична й невід'ємно визначена, тому її можна вважати кореляційною матрицею.
\begin{enumerate}
    \item $K^{*} = \left( ADA^{*}\right)^{*} = \left( A^{*}\right)^{*}DA^{*} = ADA^{*} = K$.
    \item $\forall \vec{u} \in \mathbb{R}^n$ $\left( K \vec{u}, \vec{u}\right) = \left( ADA^{*} \vec{u}, \vec{u}\right) =
    \left( DA^{*} \vec{u}, A^{*}\vec{u}\right) = \left[ A^{*}\vec{u} = \vec{v}\right] = \left( D\vec{v}, \vec{v}\right) > 0$ для $\vec{v}\neq \vec{0}$.
\end{enumerate}

Нехай в $n$-вимірному евклідовому просторі задано вектор $\vec{a}$ та симетричну додатно визначену матрицю $K$.
Існує ортогональне перетворення $U$, яке дає можливість записати $K=UDU^{*}$, причому на діагоналі $D$ стоять строго додатні
власні числа $K$. Тоді функцію вигляду $\exp\left\{i(\vec{a}, \vec{t}) - \frac{1}{2}(K\vec{t}, \vec{t})\right\}$ 
можна розглядати як характеристичну функцію випадкового вектора $\vec{\eta} = U \vec{\xi}$, де $\vec{\xi}$ ---
вектор, координати якого незалежні у сукупності та мають нормальний розподіл.

\begin{definition}
    $n$\emph{-вимірним гауссівським вектором} $\vec{\xi}$ називається випадковий вектор,
    характеристична функція якого має вигляд $\chi_{\vec{\xi}}(\vec{t}) = \exp\left\{i(\vec{a}, \vec{t}) - \frac{1}{2}(K\vec{t}, \vec{t})\right\}$,
    де $K$ --- кореляційна матриця, а $\vec{a} = \left( E\xi_1, E\xi_2, ..., E\xi_n\right)$ --- центр розсіювання.
\end{definition}
\noindent \textbf{Позначення:} $\vec{\xi} \sim {N}( \vec{a}, K)$, ${N}( \vec{0}, \mathbb{I})$ --- стандартний нормальний розподіл.
\begin{exercise}
    Перевірити, що $\vec{a}$ дійсно є центром розсіювання.
\end{exercise}
Знайдемо щільність сумісну розподілу такого вектору:
\begin{gather*}
    P\left\{\vec{\eta} \in C \subset \mathbb{R}^n\right\} = P\left\{U\vec{\xi}\in C\right\} = 
    P\left\{\vec{\xi}\in U^{-1}C\right\} = 
    \int\limits_{U^{-1}C} f_{\vec{\xi}}(\vec{x}) d\vec{x} = \\
    \int\limits_{U^{-1}C} \frac{1}{(2\pi)^{n/2} \sqrt{{\det{D}}}} \exp \left\{ -\frac{1}{2} \left( D^{-1}(\vec{x} - \vec{a^o}), (\vec{x} - \vec{a^o})\right)\right\} d\vec{x} =
    \left[ \vec{x} - \vec{a^o} = U^{-1}(\vec{y} - \vec{a}) = \right. \\ \left. = U^{*}(\vec{y} - \vec{a}), d\vec{x} = d\vec{y}\right] = 
    \frac{1}{(2\pi)^{n/2} \sqrt{{\det{D}}}} \int\limits_{C} \exp \left\{ -\frac{1}{2} \left( D^{-1}(U^{*}\vec{y} - U^{*}\vec{a}), (U^{*}\vec{y} - U^{*}\vec{a})\right)\right\} d\vec{y} = \\
    = \left[ UD^{-1}U^{*} = K^{-1}\right] =
    \frac{1}{(2\pi)^{n/2} \sqrt{{\det{K}}}} \int\limits_{C} \exp \left\{ -\frac{1}{2} \left( K^{-1}(\vec{y} - \vec{a}), (\vec{y} - \vec{a})\right)\right\} d\vec{y}
\end{gather*}
Оскільки $P\left\{\vec{\eta} \in C\right\} = \int\limits_{C} f_{\vec{\eta}}(\vec{y}) d\vec{y}$,
отримуємо \textbf{щільність розподілу} гауссівського вектора $\vec{\xi} \sim {N}(\vec{a}, K)$:
\begin{equation}
    f_{\vec{\xi}}(\vec{x}) = \frac{1}{(2\pi)^{n/2} \sqrt{{\det{K}}}} \exp \left\{ -\frac{1}{2} \left( K^{-1}(\vec{x} - \vec{a}), (\vec{x} - \vec{a})\right)\right\}
\end{equation}

\subsection{Властивості гауссівських векторів}
\begin{enumerate}
    \item Якщо $\vec{\xi}$ --- гауссівський вектор, то всі його координати гауссівські,
    а будь-яка підсистема теж є гауссівським вектором.
    \begin{proof}
        $\chi_{\vec{\xi}}(0,0,...,t_j, ..., 0) = \chi_{\xi_j}(t) = 
        e^{i a t_j - \frac{1}{2}\sigma_j^2 t_j^2} \Rightarrow \xi_j \sim {N}(a_j, \sigma_j^2)$.

        Аналогічно для будь-якої підсистеми $\vec{\eta} = (\xi_{k_1}, \xi_{k_2}, ..., \xi_{k_n})$.
    \end{proof}
    \begin{remark}
        Обернене твердження, взагалі кажучи, не є вірним.

        Розглянемо випадковий вектор із щільністю
        $$ f_{\vec{\xi}}(x, y) = \frac{1}{2\pi} \left( 
            \left( \sqrt{2} e^{-x^2/2} - e^{-x^2}\right)e^{-y^2} + 
            \left( \sqrt{2} e^{-y^2/2} - e^{-y^2}\right)e^{-x^2}\right)$$

        Очевидно, це не щільність нормального розподілу. Знайдемо щільності розподілу координат $\xi_1$ та $\xi_2$.
        \begin{gather*}
            f_{\xi_1}(x) = \int\limits_{-\infty}^{+\infty}f_{\vec{\xi}}(x, y) dy =
            \frac{\sqrt{2}}{2\pi} e^{-x^2/2}\int\limits_{-\infty}^{+\infty}e^{-y^2}dy -
            \frac{1}{2\pi}e^{-x^2}\int\limits_{-\infty}^{+\infty}e^{-y^2}dy \; +\\
            + \; \frac{\sqrt{2}}{2\pi}e^{-x^2}\int\limits_{-\infty}^{+\infty} e^{-y^2/2}dy - 
            \frac{1}{2\pi} e^{-x^2}\int\limits_{-\infty}^{+\infty}e^{-y^2}dy =
            \frac{\sqrt{2}}{2\pi} e^{-x^2/2}\cdot \sqrt{\pi} - \frac{1}{2\pi}e^{-x^2} \cdot \sqrt{\pi} \; + \\
            + \; \frac{\sqrt{2}}{2\pi}e^{-x^2} \cdot \sqrt{2\pi} - \frac{1}{2\pi}e^{-x^2}\cdot \sqrt{\pi} = 
            \frac{1}{\sqrt{2\pi}}e^{-x^2/2} - \frac{1}{2\sqrt{\pi}}e^{-x^2} + \frac{1}{\sqrt{\pi}}e^{-x^2} - \frac{1}{2\sqrt{\pi}}e^{-x^2} \; = \\ 
            \\ = \; \frac{1}{\sqrt{2\pi}}e^{-x^2/2} \Rightarrow \xi_1 \sim {N}(0, 1).
        \end{gather*}
        Аналогічно $\xi_2 \sim {N}(0, 1)$. Отже, координати вектора мають нормальний розподіл, а сам вектор --- ні.
    \end{remark}
    \item Для гауссівського випадкового вектора поняття незалежності та некорельованості координат є еквівалентними.
    \begin{proof}
        Було доведено, що з незалежності координат випливає ії некорельованість.

        Якщо координати некорельовані, то кореляційна матриця $K = \mathrm{diag}(\sigma_1^2, \sigma_2^2, ..., \sigma_n^2)$.
        Тоді $\left( K \vec{t}, \vec{t}\right) = \sum_{k=1}^n \sigma_k^2 t_k^2$.
        Тоді $\chi_{\vec{\xi}}(\vec{t}) = \exp\left\{i(\vec{a}, \vec{t}) - \frac{1}{2}\left( K \vec{t}, \vec{t}\right)\right\} = \prod\limits_{k=1}^n e^{i a_k t_k - \frac{1}{2}\sigma_k^2 t_k^2} = \prod\limits_{k=1}^n \chi_{\xi_k}(t_k)$
        і координати незалежні.
    \end{proof}
    \item В $n$-вимірному евклідовому просторі $\mathbb{R}^n$ завжди можна перейти до
    ортонормованого базису з власних векторів матриці $K$, в якому $K$ приймає діагональний вигляд.
    Отже, в базисі з власних векторів матриці $K$ координати відповідного гауссівського вектора стають незалежними.
    \item \emph{Афінне перетворення гауссівських векторів}.

    $\vec{\xi} \sim {N}(\vec{a}, K)$. $A : \mathbb{R}^n \rightarrow \mathbb{R}^m$ --- матриця, $\vec{b} \in \mathbb{R}^m$, $\vec{\eta} = A\vec{\xi} + \vec{b}$.
    За властивістю характеристичної функції $\chi_{\vec{\eta}}(\vec{t}) = e^{i(\vec{b}, \vec{t})}\cdot\chi_{\vec{\xi}}(\vec{A^{*}t}) =
    \exp\left\{i(\vec{b}, \vec{t})\right\}\cdot \exp\left\{i(\vec{a}, A^{*}\vec{t}) - \frac{1}{2}\left( K A^{*}\vec{t}, A^{*}\vec{t}\right)\right\} =
    \exp\left\{i(A\vec{a} + \vec{b}, \vec{t}) - \frac{1}{2}\left( A K A^{*}\vec{t}, \vec{t}\right)\right\}$. Отже, $\vec{\eta} \sim {N}(A\vec{a} + \vec{b}, A K A^{*})$.
\end{enumerate}
\begin{example}
    \begin{enumerate}
        \item Задано вектор $\vec{\xi} \sim {N}(\vec{a}, K)$. Знайти розподіл $\eta = \xi_1 + \xi_2 + ... + \xi_n$.

        $\eta = \begin{pmatrix}
            1 & 1 & \cdots & 1
        \end{pmatrix}\cdot
        \begin{pmatrix}
            \xi_1 & \xi_2 & \cdots & \xi_n
        \end{pmatrix}^T = A\vec{\xi}$.
        $E\eta = A\vec{a} = E\xi_1 + E\xi_2 + ... + E\xi_n = \sum\limits_{k=1}^n E\xi_k$.
        $D\eta = A K A^{*} = \begin{pmatrix}
            1 & 1 & \cdots & 1
        \end{pmatrix}\cdot
        \cdot K \cdot \begin{pmatrix}
            1 & 1 & \cdots & 1
        \end{pmatrix}^T$. Якщо $K = \mathrm{diag}(\sigma_1^2, \sigma_2^2, ..., \sigma_n^2)$, то $D\eta = \sum\limits_{k=1}^n D\xi_k$,
        а в загальному випадку у цій сумі ще будуть доданки вигляду $2K\xi_i\xi_j$.
    \end{enumerate}
\end{example}