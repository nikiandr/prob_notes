% !TEX root = ../../main.tex
\section{Інтервальне оцінювання невідомих параметрів ГС}
Нехай $\xi$ --- ГС, $F_{\xi}(x, \theta_1, \theta_2, ..., \theta_k)$ --- її функція 
розподілу, $\theta_1, \theta_2, ..., \theta_k$ --- набір параметрів. 

При інтервальному оцінюванні деякого параметра ГС ми шукаємо статистики 
$\theta^*_1(\vec{\xi})$, $\theta^*_2(\vec{\xi})$ такі, що:
\begin{gather*}
    \P\{\theta \in (\theta^*_1, \theta^*_2)\} = \gamma \text{ (або }
    \geq \gamma \text{ )}
\end{gather*}

Фактично шукається інтервал, в який з наперед заданою ймовірністю потрапляє 
оцінюваний параметр $\theta$.

\begin{definition}
    Інтервал $(\theta^*_1, \theta^*_2)$ називається довірчим інтервалом для параметра 
    $\theta$, $\gamma$ називається надійністю довірчого інтервалу.
\end{definition}

Довірчі інтервали можуть бути симетричними відносно $\theta$, тобто:
\begin{gather*}
    \P\{|\theta - \theta^*| < \varepsilon\} = \gamma
\end{gather*}

Тоді сам довірчий інтервал матиме вигляд 
$(\theta^* - \varepsilon, \theta^* + \varepsilon)$, $\varepsilon$ --- точність 
довірчого інтервалу.

Таким же чином довірчі інтервали можуть бути односторонніми:
\begin{gather*}
    \P\{\theta > \theta_1^*\} = \gamma \\
    \text{або} \\
    \P\{\theta < \theta_2^*\} = \gamma
\end{gather*}

\section{Побудова довірчого інтервалу для гаусcівської ГС при відомій дисперсії}
Маємо ГС $\xi \sim N(a, \sigma^2)$. Будемо оцінювати параметр $a$ ГС. Довірчий 
інтервал будемо будувати у симетричній формі:
\begin{gather*}
    \P\{|a - a^*| < \varepsilon\} = \gamma
\end{gather*}
В якості оцінки параметра $a = \E\xi$ беремо вибіркове середнє: $a^* = 
\overline{\xi} = \frac{1}{n} \sum\limits_{k=1}^n \xi_k$. Вибіркове середнє гаусcівської 
випадкової вибірки матиме гаусcівський розподіл (т.я. це є афінне перетворення 
гауссівського випадкового вектору) з матсподіванням $a$ (в силу незміщенності 
оцінки) та дисперсією $\frac{\sigma^2}{n}$:
\begin{gather*}
    \D \overline{\xi} = \D \left( 
        \frac{1}{n} \sum\limits_{k=1}^n \xi_k
    \right) = \frac{1}{n^2} \D \left( 
        \sum\limits_{k=1}^n \xi_k
    \right) = \left[\xi_k\text{ --- незалежні}\right] = 
    \frac{1}{n^2}\sum\limits_{k=1}^n \D\xi_k = \frac{1}{n^2}n\sigma^2 = 
    \frac{\sigma^2}{n}
\end{gather*}
Таким чином матимемо, що:
\begin{gather*}
    \P\{|a - a^*| < \varepsilon\} = \P \{a - \varepsilon < a^* < a + \varepsilon\} = 
    2\Phi\left(\frac{\varepsilon\sqrt{n}}{\sigma}\right) = \gamma
\end{gather*}

Знайдемо значення $\varepsilon$ з таблиці значень функції $\Phi(x)$. Тоді матимемо, 
що з ймовірністю $\gamma$ параметр $a$ належить інтервалу $(a^*_{\text{знач.}} - 
\varepsilon, a^*_{\text{знач.}} + \varepsilon)$.