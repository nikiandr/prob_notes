% !TEX root = ../../main.tex
\section{Точкові оцінки}
Нехай $\xi$ --- ГС, а $F_{\xi}(x, \theta_1, \theta_2, ..., \theta_k)$ --- її функція розподілу, де
$\theta_1, \theta_2, ..., \theta_k$ --- набір параметрів. Тип самої функцію розподілу у випадку точкового оцінювання
вважається відомим, невідомими є параметри. Наприклад, $\lambda$ в експоненційному законі чи $a$ та $\sigma$ в нормальному.
\begin{definition}
    \emph{Точковою оцінкою} $\theta^*$ невідомого параметру $\theta$ називається деяка статистика
    $\theta^*(\xi_1, ..., \xi_n)$, значення якої на конкретній
    реалізації вибірки приймається за наближене значення $\theta$.
\end{definition}
Зрозуміло, що точкова оцінка $\theta^*$, на відміну від параметру $\theta$, є випадковою величиною, 
яка залежить від закону розподілу $\xi$ та обсягу вибірки. Безумовно, можна ввести багато функцій від результатів спостережень, 
які можна брати в якості $\theta^*$. Наприклад, якщо параметр $\theta$ є математичним сподіванням $\xi$, 
то за оцінку математичного сподівання за результатами спостережень можна взяти середнє арифметичне, моду, медіану, 
півсуму найбільшого та найменшого значень вибірки тощо. 
Отже, яку статистику краще обрати? Назвати <<найкращою>> оцінку ту, яка найбільш близька до істинного значення оцінюваного параметру, 
неможливо, оскільки точкова оцінка --- випадкова величина. Таким чином, робити висновки про якість оцінки варто не по її конкретним значенням, 
а по її розподілу. В зв'язку з цим розглянемо вимоги, що висувають до точкових оцінок.

\subsection{Незміщені точкові оцінки}
 \begin{definition}
    Точкова оцінка $\theta^*$ параметру $\theta$ називається \emph{незміщеною}, якщо
    \begin{gather}\label{estim_unbiased}
        E\theta^*(\xi_1, ..., \xi_n) = \theta \text{ для всіх } n\in\mathbb{N}
    \end{gather} 
    і \emph{асимптотично незміщеною}, якщо $\underset{n\to\infty}{\lim} E\theta^*(\xi_1, ..., \xi_n) = \theta$.
 \end{definition}
 \begin{example} 
    Розглянемо деякі важливі незміщені оцінки.
    \begin{enumerate}
        \item Вибіркове середнє --- незміщена оцінка математичного сподівання:
        $\overline{\xi} = \frac{1}{n}\sum\limits_{k=1}^n \xi_k$, $E\overline{\xi} = \frac{1}{n}\sum\limits_{k=1}^n E\xi_k = \frac{1}{n} \cdot{n} \cdot{E\xi} = E\xi$,
        оскільки всі $\xi_k$ однаково розподілені.
        \item Вибіркова дисперсія --- незміщена оцінка дисперсії у випадку відомого математичного сподівання $E\xi$:
        $D^*\xi = \frac{1}{n}\sum\limits_{k=1}^n \left(\xi_k - E\xi \right)^2$, 
        $E\left( D^* \xi\right) = \frac{1}{n}\sum\limits_{k=1}^n E\left(\xi_k - E\xi \right)^2 = D\xi$ знову через однаковий розподіл $\xi_k$.
        \item Дослідимо вибіркову дисперсію, але у випадку невідомого математичного сподівання. 
        \begin{gather*}
            D^*\xi = \frac{1}{n}\sum\limits_{k=1}^n \left(\xi_k - \overline{\xi} \right)^2 = 
            \frac{1}{n}\sum\limits_{k=1}^n \left((\xi_k - E\xi) + (\overline{\xi} - E\xi) \right)^2 =  \\
            = \frac{1}{n}\sum\limits_{k=1}^n \left(\xi_k - E\xi \right)^2 - 2\left(\overline{\xi} - E\xi\right)\cdot 
            \underbrace{\frac{1}{n}\sum\limits_{k=1}^n \left(\xi_k - E\xi\right)}_{\overline{\xi} - E\xi} + \left(\overline{\xi} - E\xi\right)^2 = \\
            = \frac{1}{n}\sum\limits_{k=1}^n \left(\xi_k - E\xi \right)^2 - \left(\overline{\xi} - E\xi\right)^2
        \end{gather*}
        Перший доданок --- це формула для вибіркової дисперсії при відомому математичному сподіванні, а 
        $E\xi = E\overline{\xi}$, тому
        \begin{gather*}
            E\left( D^* \xi\right) = D\xi - E\left(\overline{\xi} - E\xi\right)^2= D\xi - D\overline{\xi} \\
            D\overline{\xi} = D\left(\frac{1}{n}\sum\limits_{k=1}^n \xi_k \right) = \frac{1}{n^2} \sum\limits_{k=1}^n D\xi_k = \frac{1}{n} D\xi
        \end{gather*}
        Дві останні рівності одержані через незалежність та однаковий розподіл $\xi_k$. Отже, маємо $E\left( D^* \xi\right) = \left( 1- \frac{1}{n}\right) D\xi$,
        тому ця оцінка є лише асимптотично незміщеною. Проте, оцінка $D^{**}\xi = \frac{n}{n-1} D^*\xi$ буде незміщеною.
        Таким чином, якщо $E\xi$ невідоме, то \emph{виправлена вибіркова дисперсія} 
        $D^{**}\xi = \frac{1}{n-1} \sum\limits_{k=1}^n \left(\xi_k - \overline{\xi} \right)^2$ є незміщеною оцінкою дисперсії.
        \item Нехай $\xi \sim \mathrm{U}\left< a; b\right>$, перевіримо незміщеність $a^* = \underset{1\leq k \leq n}{\min}\xi_k$.
        Знайдемо $Ea^*$. Як відомо, 
        $f_{\min}(x) = n \left(1-F_{\xi}(x)\right)^{n-1} f_{\xi}(x) = n\left( 1- \frac{x-a}{b-a}\right)^{n-1}\frac{1}{b-a} = n\cdot\frac{(b-x)^{n-1}}{(b-a)^n}$, якщо
        $x \in \left< a; b\right>$, та $0$ інакше.
        \begin{gather*}
            Ea^* = \frac{n}{(b-a)^n} \int_a^b x(b-x)^{n-1} dx = \left[ b-x = t \right] = 
            \frac{n}{(b-a)^n} \int_0^{b-a} (b-t)t^{n-1} dt = \\
            = \frac{n}{(b-a)^n} \left.\left( \frac{bt^n}{n} - \frac{t^{n+1}}{n+1}\right)\right|_0^{b-a} = 
            \frac{n}{(b-a)^n} \left(\frac{b(b-a)^n}{n} - \frac{(b-a)^{n+1}}{n+1}\right) = \\
            = b - \frac{n(b-a)}{n+1} = \frac{bn + b - nb + na}{n+1} = a \cdot \frac{n}{n+1} + \frac{b}{n+1} \neq a
         \end{gather*}
         Але $\underset{n\to\infty}{\lim} Ea^* = a$, тому $a^*$ є асимптотично незміщеною оцінкою.
    \end{enumerate}
 \end{example}
 \begin{exercise}
     Перевірити, що $b^* = \underset{1\leq k \leq n}{\max}\xi_k$ є асимптотично незміщеною оцінкою для параметра $b$ у випадку $\xi \sim \mathrm{U}\left< a; b\right>$.
     Користуючись вже дослідженими оцінками $a^*$ та $b^*$, знайти незміщені оцінки для параметрів $a$ і $b$ (підказка: це будуть деякі лінійні комбінації $a^*$ та $b^*$).
 \end{exercise}