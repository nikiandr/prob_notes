% !TEX root = ../../main.tex
\section{Точкові оцінки}
Нехай $\xi$ --- ГС, а $F_{\xi}(x, \theta_1, \theta_2, ..., \theta_k)$ --- її функція розподілу, де
$\theta_1, \theta_2, ..., \theta_k$ --- набір параметрів. Тип самої функцію розподілу у випадку точкового оцінювання
вважається відомим, невідомими є параметри. Наприклад, $\lambda$ в експоненційному законі чи $a$ та $\sigma$ в нормальному.
\begin{definition}
    \emph{Точковою оцінкою} $\theta^*$ невідомого параметру $\theta$ називається деяка статистика
    $\theta^*(\xi_1, ..., \xi_n)$, значення якої на конкретній
    реалізації вибірки приймається за наближене значення $\theta$. Іноді вводиться позначення $\theta^*_n$,
    щоб вказати залежність оцінки від обсягу вибірки.
\end{definition}
Зрозуміло, що точкова оцінка $\theta^*$, на відміну від параметру $\theta$, є випадковою величиною, 
яка залежить від закону розподілу $\xi$ та обсягу вибірки. Безумовно, можна ввести багато функцій від результатів спостережень, 
які можна брати в якості $\theta^*$. Наприклад, якщо параметр $\theta$ є математичним сподіванням $\xi$, 
то за оцінку математичного сподівання за результатами спостережень можна взяти середнє арифметичне, моду, медіану, 
півсуму найбільшого та найменшого значень вибірки тощо. 
Отже, яку статистику краще обрати? Назвати <<найкращою>> оцінку ту, яка найбільш близька до істинного значення оцінюваного параметру, 
неможливо, оскільки точкова оцінка --- випадкова величина. Таким чином, робити висновки про якість оцінки варто не по її конкретним значенням, 
а по її розподілу. В зв'язку з цим розглянемо вимоги, що висувають до точкових оцінок.

\subsection{Незміщені оцінки}
 \begin{definition}
    Точкова оцінка $\theta^*$ параметру $\theta$ називається \emph{незміщеною}, якщо
    \begin{gather}\label{estim_unbiased}
        E\theta^*_n = \theta \text{ для всіх } n\in\mathbb{N}
    \end{gather} 
    і \emph{асимптотично незміщеною}, якщо $\underset{n\to\infty}{\lim} E\theta^*_n = \theta$. 
 \end{definition}
 \begin{example} 
    Розглянемо деякі важливі незміщені оцінки.
    \begin{enumerate}
        \item \emph{Вибіркове середнє} --- незміщена оцінка математичного сподівання:
        $\overline{\xi} = \frac{1}{n}\sum\limits_{k=1}^n \xi_k$, $E\overline{\xi} = \frac{1}{n}\sum\limits_{k=1}^n E\xi_k = \frac{1}{n} \cdot{n} \cdot{E\xi} = E\xi$,
        оскільки всі $\xi_k$ однаково розподілені.
        \item \emph{Вибіркова дисперсія} --- незміщена оцінка дисперсії у випадку відомого математичного сподівання $E\xi$:
        $D^*\xi = \frac{1}{n}\sum\limits_{k=1}^n \left(\xi_k - E\xi \right)^2$, 
        $E\left( D^* \xi\right) = \frac{1}{n}\sum\limits_{k=1}^n E\left(\xi_k - E\xi \right)^2 = D\xi$ знову через однаковий розподіл $\xi_k$.
        \item Дослідимо вибіркову дисперсію, але у випадку невідомого математичного сподівання. 
        \begin{gather*}
            D^*\xi = \frac{1}{n}\sum\limits_{k=1}^n \left(\xi_k - \overline{\xi} \right)^2 = 
            \frac{1}{n}\sum\limits_{k=1}^n \left((\xi_k - E\xi) + (\overline{\xi} - E\xi) \right)^2 =  \\
            = \frac{1}{n}\sum\limits_{k=1}^n \left(\xi_k - E\xi \right)^2 - 2\left(\overline{\xi} - E\xi\right)\cdot 
            \underbrace{\frac{1}{n}\sum\limits_{k=1}^n \left(\xi_k - E\xi\right)}_{\overline{\xi} - E\xi} + \left(\overline{\xi} - E\xi\right)^2 = \\
            = \frac{1}{n}\sum\limits_{k=1}^n \left(\xi_k - E\xi \right)^2 - \left(\overline{\xi} - E\xi\right)^2
        \end{gather*}
        Перший доданок --- це формула для вибіркової дисперсії при відомому математичному сподіванні, а 
        $E\xi = E\overline{\xi}$, тому
        \begin{gather*}
            E\left( D^* \xi\right) = D\xi - E\left(\overline{\xi} - E\xi\right)^2= D\xi - D\overline{\xi} \\
            D\overline{\xi} = D\left(\frac{1}{n}\sum\limits_{k=1}^n \xi_k \right) = \frac{1}{n^2} \sum\limits_{k=1}^n D\xi_k = \frac{1}{n} D\xi
        \end{gather*}
        Дві останні рівності одержані через незалежність та однаковий розподіл $\xi_k$. Отже, маємо $E\left( D^* \xi\right) = \left( 1- \frac{1}{n}\right) D\xi$,
        тому ця оцінка є лише асимптотично незміщеною. Проте, оцінка $D^{**}\xi = \frac{n}{n-1} D^*\xi$ буде незміщеною.
        Таким чином, якщо $E\xi$ невідоме, то \emph{виправлена вибіркова дисперсія} 
        $D^{**}\xi = \frac{1}{n-1} \sum\limits_{k=1}^n \left(\xi_k - \overline{\xi} \right)^2$ є незміщеною оцінкою дисперсії.
        \item Нехай $\xi \sim \mathrm{U}\left< a; b\right>$, перевіримо незміщеність $a^*_n = \underset{1\leq k \leq n}{\min}\xi_k$.
        Знайдемо $Ea^*_n$. Як відомо, 
        $f_{\min}(x) = n \left(1-F_{\xi}(x)\right)^{n-1} f_{\xi}(x) = n\left( 1- \frac{x-a}{b-a}\right)^{n-1}\frac{1}{b-a} = n\cdot\frac{(b-x)^{n-1}}{(b-a)^n}$, якщо
        $x \in \left< a; b\right>$, та $0$ інакше.
        \begin{gather*}
            Ea^*_n = \frac{n}{(b-a)^n} \int_a^b x(b-x)^{n-1} dx = \left[ b-x = t \right] = 
            \frac{n}{(b-a)^n} \int_0^{b-a} (b-t)t^{n-1} dt = \\
            = \frac{n}{(b-a)^n} \left.\left( \frac{bt^n}{n} - \frac{t^{n+1}}{n+1}\right)\right|_0^{b-a} = 
            \frac{n}{(b-a)^n} \left(\frac{b(b-a)^n}{n} - \frac{(b-a)^{n+1}}{n+1}\right) = \\
            = b - \frac{n(b-a)}{n+1} = \frac{bn + b - nb + na}{n+1} = a \cdot \frac{n}{n+1} + \frac{b}{n+1} \neq a
         \end{gather*}
         Але $\underset{n\to\infty}{\lim} Ea^*_n = a$, тому $a^*_n$ є асимптотично незміщеною оцінкою.
    \end{enumerate}
 \end{example}
 \begin{exercise}
     Перевірити, що $b^*_n = \underset{1\leq k \leq n}{\max}\xi_k$ є асимптотично незміщеною оцінкою для параметра $b$ у випадку $\xi \sim \mathrm{U}\left< a; b\right>$.
     Користуючись вже дослідженими оцінками $a^*_n$ та $b^*_n$, знайти незміщені оцінки для параметрів $a$ і $b$ (підказка: це будуть деякі лінійні комбінації $a^*_n$ та $b^*_n$).
 \end{exercise}

\subsection{Конзистентні оцінки}
\begin{definition}
    Точкова оцінка $\theta^*$ параметру $\theta$ називається \emph{конзистентною}, якщо
    \begin{gather}\label{estim_consis}
        \forall \; \varepsilon > 0: \underset{n \to \infty}{\lim} P\left\{|\theta^*_n - \theta| \geq \varepsilon\right\}= 0
        \Leftrightarrow \theta^*_n \overset{\mathrm{P}}{\longrightarrow} \theta, n \to \infty
    \end{gather}
\end{definition}
Конзистентність оцінки можна перевіряти за означенням. Але для незміщених та асимптотично незміщених оцінок є
\emph{достатня умова} конзистентності.
\begin{proposition*}
    Якщо $\theta^*_n$ --- незміщена чи асимптотично незміщена оцінка та $D\theta^*_n \to 0$, $n \to \infty$,
    то ця оцінка є конзистентною.
\end{proposition*}
\begin{proof}
    Якщо $\theta^*_n$ незміщена, то 
    $P\left\{|\theta^*_n - \theta| \geq \varepsilon\right\}=P\left\{|\theta^*_n - E\theta^*_n| \geq \varepsilon\right\} 
    \leq \frac{D\theta^*_n}{\varepsilon^2} \to 0$.
    Якщо $\theta^*_n$ асимптотично незміщена, то $\underset{n\to\infty}{\lim} E\theta^*_n = \theta$,
    звідки за критерієм збіжності в середньому квадратичному до константи маємо 
    $\theta^*_n \overset{\text{СК}}{\longrightarrow} \theta \Rightarrow \theta^*_n \overset{\mathrm{P}}{\longrightarrow} \theta$.
\end{proof}
\begin{remark}
    Також є поняття \emph{сильно конзистентної оцінки}, де збіжність за ймовірностю замінюється збіжністю майже напевно.
    Такі оцінки надалі розглядати не будемо через складність дослідження такої збіжності в загальному випадку. 
\end{remark}
\begin{example}
    Перевіримо конзистентність оцінок, незміщеність чи асимптотичну незміщеність яких вже дослідили.
    \begin{enumerate}
        \item \emph{Вибіркове середнє} $\overline{\xi} = \frac{1}{n}\sum\limits_{k=1}^n \xi_k$ є конзистентною оцінкою
        за законом великих чисел (і навіть сильно конзистентною).
        \item \emph{Вибіркова дисперсія} $D^*\xi = \frac{1}{n}\sum\limits_{k=1}^n \left(\xi_k - E\xi \right)^2$
        є незміщеною оцінкою дисперсії в разі відомого $E\xi$. 
        $D\left(D^*\xi\right) = \frac{1}{n^2}\sum\limits_{k=1}^n D\left(\xi_k - E\xi \right)^2 =
        \frac{1}{n} D\left(\xi - E\xi \right)^2 \to 0, n\to\infty$, тому ця оцінка конзистентна.
        \item \emph{Виправлена вибіркова дисперсія} $D^{**}\xi = \frac{n}{n-1} D^*\xi$ є незміщеною оцінкою дисперсії в разі невідомого $E\xi$. 
        $D\left(D^{**}\xi\right) = \left(\frac{n}{n-1} \right)^2 \cdot D\left(D^*\xi\right) \to 0, n\to\infty$, тому ця оцінка конзистентна.
        \item Нехай $\xi \sim \mathrm{U}\left< a; b\right>$, $a^*_n = \underset{1\leq k \leq n}{\min}\xi_k$ --- асимптотично незміщена оцінка $a$.
        Перевірятимемо конзистентність за означенням:
        \begin{gather*}
            \varepsilon > 0, \; P\left\{|a^*_n - a| > \varepsilon\right\} = 
            P\left\{a - \varepsilon < a^*_n < a + \varepsilon\right\} = \int_{a}^{a+\varepsilon} f_{\min}(x) dx = \\
            = \frac{n}{(b-a)^n}\int_{a}^{a+\varepsilon} (b-x)^{n-1} dx =
            -\frac{n}{(b-a)^n} \cdot \left. \frac{(b-x)^n}{n} \right|_{a}^{a+\varepsilon} = \\
            = \frac{1}{(b-a)^n} \cdot \left( (b-a)^n  - (b-a-\varepsilon)^n\right) = 
            1 - \left(1 - \frac{\varepsilon}{b-a}\right)^n \to 1, n\to\infty
        \end{gather*} 
        Отже, оцінка є конзистентною.
    \end{enumerate}
\end{example}
\begin{exercise}
    Перевірити конзистентність оцінки $b^*_n = \underset{1\leq k \leq n}{\max}\xi_k$ для $\xi \sim \mathrm{U}\left< a; b\right>$.
\end{exercise}
\begin{remark}
    Конзистентність та незміщеність вибіркового середнього, виправленої та звичайної вибіркових дисперсій було встановлено для будь-якого розподілу $\xi$.
\end{remark}

\subsection{Ефективні оцінки}
Основне питання задачі оцінювання параметрів розподілу --- наскільки великою є похибка. 
Введені означення незміщеної та конзистентної оцінки показують, відповідно, чи правильні в середньому значення цієї оцінки,
та чи покращується точність оцінювання зі збільшенням обсягу вибірки.
Зрозуміло, що для оцінювання параметру $\theta$ можна запропонувати декілька незміщених оцінок, значення яких за визначенням
зосереджені навколо справжнього значення $\theta$. Природно вимагати від <<найкращої>> такої оцінки найменшої можливої дисперсії.
\begin{definition}
    Нехай $\Theta_n$ --- множина усіх незміщених оцінок параметру $\theta$ за вибірками фіксованого обсягу $n$. 
    Оцінка $\theta^*_{\text{еф}}$ називається
    \emph{ефективною}, якщо 
    \begin{gather}\label{estim_eff}
        D\theta^*_{\text{еф}} = \underset{\theta^* \in \Theta_n}{\inf} D\theta^*
    \end{gather}
\end{definition}
Означення ефективності не надто сприяє дослідженню оцінки. По-перше, не завжди
легко обчислити дисперсію оцінки. По-друге, навіть якщо її вдасться обчислити, то немає гарантії, що ця дисперсія буде найменшою серед дисперсій усіх оцінок
з $\Theta_n$. Перевіряти ефективність оцінок допомагає нерівність Рао-Крамера.

\begin{theorem*}[нерівність Рао-Крамера]
    Якщо $\theta^*$ --- незміщена оцінка, то $D\theta^* \geq \frac{1}{I(\theta)}$, 
    де $I(\theta) = $
\end{theorem*}