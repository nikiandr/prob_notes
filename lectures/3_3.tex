% !TEX root = ../main.tex
\section{Числові характеристики випадкових векторів}

\subsection{Математичне сподівання випадкових векторів}
\begin{definition}
    \emph{Математичним сподіванням $E{\vec{\xi}}$} 
    випадкового вектора $\vec{\xi}$ називається вектор 
    $\left(E{\xi_1}, ..., E{\xi_n}\right)^T$.
\end{definition}
\begin{remark}
    Математичне сподівання двовимірного випадкового вектора 
    називається \emph{центром розсіювання}.
\end{remark}

\noindent \textbf{Способи знаходження:}
\begin{enumerate}
    \item Знайти закони розподілу окремих координат, а далі --- математичні сподівання цих координат.
    \item Знайти математичні сподівання координат одразу.
    Для $\xi = (\xi_1, \xi_2)^T$:
    
    $E\xi_1 = \begin{cases}
        \sum\limits_i \sum\limits_j x_i p_{ij}, & \vec{\xi} \text{ --- ДВВ} \\
        \int\limits_{-\infty}^{+\infty} \int\limits_{-\infty}^{+\infty} x f_{\vec{\xi}}(x,y) dx dy, & \vec{\xi} \text{ --- НВВ}
    \end{cases}$,
    $E\xi_2 = \begin{cases}
        \sum\limits_j \sum\limits_i y_j p_{ij}, & \vec{\xi} \text{ --- ДВВ} \\
        \int\limits_{-\infty}^{+\infty} \int\limits_{-\infty}^{+\infty} y f_{\vec{\xi}}(x,y) dx dy, & \vec{\xi} \text{ --- НВВ}
    \end{cases}$.
    
    В загальному випадку для НВВ $E\xi_k = \int\limits_{-\infty}^{+\infty}...\int\limits_{-\infty}^{+\infty} x_k f_{\vec{\xi}}(\vec{x}) dx_1 ... dx_n$.
\end{enumerate}

\subsection{Мішані початкові та центральні 
            моменти випадкових векторів}
\begin{definition}
    \emph{Мішаним початковим моментом} порядку 
    $k_1+k_2+...+k_n$ 
    $\left(k_i \in \mathbb{N}\right)$
    випадкового вектора 
    $\vec{\xi} = \left(\xi_1, ..., \xi_n\right)^T$
    називається число
    \begin{equation*}
        \alpha_{k_1+k_2+...+k_n} = 
        E{\xi_1^{k_1}...\xi_n^{k_n}}
    \end{equation*}
\end{definition}
\begin{remark}
    $E\xi_1 = \alpha_{1+0+...+0}, E\xi_2 = \alpha_{0+1+...+0},
    ..., E\xi_n = \alpha_{0+0+...+1}$.
\end{remark}
\begin{definition}
    \emph{Мішаним центральним моментом} порядку 
    $k_1+k_2+...+k_n$
    $\left(k_i \in \mathbb{N}\right)$ 
    випадкового вектора
    $\vec{\xi} = \left(\xi_1, ..., \xi_n\right)^T$
    називається число
    \begin{equation*}
        \beta_{k_1+k_2+...+k_n} = 
        E{\mathring{\xi}_1^{k_1}
        ...
        \mathring{\xi}_n^{k_n}},
        \mathring{\xi_k} = \xi_k - E\xi_k
    \end{equation*}
\end{definition}
\begin{remark}
    Всі центральні моменти 1-го порядку --- 
    нульові.
\end{remark}
Центральні моменти порядку $0+...+0+2+0+...+0$ --- дисперсії відповідних координат:
    $\forall k = \overline{1,n}$: 
    $\beta_{0+...+0+\underset{k}{2}
    +0+...+0} = E\mathring{\xi}_k^2 = 
    D\xi_k$

Дисперсії координат задають розсіювання 
вздовж відповідних координатних осей.

\begin{definition}
    \emph{Кореляційним моментом} або 
    \emph{коваріацією} випадкових величин 
    $\xi_i$ та $\xi_j$ називається змішаний 
    центральний момент 
    порядку
    $0+...+0+\underset{i}{1}+0+...+0+\underset{j}{1}+0+....+0$:
    \begin{equation*}
        {cov}(\xi_i,\xi_j) = K\xi_i\xi_j = \beta_{0+...+0+\underset{i}{1}+0+...
        +0+\underset{j}{1}+0+...+0}
        =
        E(\xi_i-E\xi_i)(\xi_j-E\xi_j)
    \end{equation*}
\end{definition}
\begin{remark}
    ${cov}(\xi_k, \xi_k) = E(\xi_k-E\xi_k)^2 = D\xi_k$.
\end{remark}
\begin{definition}
   \emph{Кореляційною матрицею} випадкового 
   вектора $\vec{\xi}$ називається матриця, у яку зібрано усі 
   кореляційні моменти випадкового вектора
   \begin{equation*}
       K = 
       \begin{pmatrix}
           D\xi_1 & K\xi_1\xi_2 & \cdots & K\xi_1\xi_n \\
           K\xi_1\xi_2 & D\xi_2 & \cdots & K\xi_2\xi_n \\
           \vdots & \vdots & \ddots & \vdots \\
           K\xi_1\xi_n & K\xi_2\xi_n & \cdots & D\xi_n
       \end{pmatrix}
   \end{equation*} 
\end{definition}
\begin{remark}
    $\mathring{\vec{\xi}} = \left(\mathring{\xi_1}, ..., \mathring{\xi_n}
    \right)^T$.
    Тоді $K = E\,\mathring{\vec{\xi}}\,\mathring{\vec{\xi}}^T$.
\end{remark}

\noindent \textbf{Властивості кореляційного моменту випадкових 
величин}
\begin{enumerate}
    \item $K\xi_i\xi_j = K\xi_j\xi_i$.
    \begin{proof}
        Випливає з означення кореляційного моменту.
    \end{proof}
    \item $\eta_1 = a_1\xi_1 + b_1$, $\eta_2 = a_2\xi_2 + b_2$.
    Тоді $K\eta_1\eta_2 = a_1a_2K\xi_1\xi_2$.
    \begin{proof}
        $K\eta_1\eta_2 = E(\eta_1 - E\eta_1)(\eta_2 - E\eta_2) = 
        E(a_1\xi_1 + b_1 - E(a_1\xi_1 + b_1))
        (a_2\xi_2 + b_2 - E(a_2\xi_2 + b_2)) = 
        Ea_1(\xi_1 - E\xi_1)a_2(\xi_2 - E\xi_2) = 
        a_1a_2E(\xi_1 - E\xi_1)(\xi_2 - E\xi_2) = 
        a_1a_2K\xi_1\xi_2$.
    \end{proof}

    \item Зручна формула для обчислення кореляційного моменту.
    \begin{equation}
        K\xi_1\xi_2 = E\xi_1\xi_2 - E\xi_1 E\xi_2
    \end{equation}
    \begin{proof}
        $K\xi_1\xi_2 = E(\xi_1 - E\xi_1)(\xi_2 - E\xi_2) = 
        E(\xi_1\xi_2 - \xi_1E\xi_2 - \xi_2E\xi_1 + E\xi_1E\xi_2) = 
        E\xi_1\xi_2 - E\xi_1E\xi_2 - E\xi_2E\xi_1 + E\xi_1\xi_2 = 
        E\xi_1\xi_2 - E\xi_1E\xi_2$
    \end{proof}
    
    \item $D(a\xi_1 \pm b\xi_2) = a^2D\xi_1 \pm 2abK\xi_1\xi_2 + 
    b^2D\xi_2$

    \begin{proof}
        $D(a\xi_1 \pm b\xi_2) = 
        E(a\xi_1 \pm b\xi_2 - E(a\xi_1 \pm b\xi_2))^2 =
        E(a(\xi_1-E\xi_1)\pm b(\xi_2 - E\xi_2))^2 = 
        a^2 E\mathring{\xi}_1^2 \pm 2abE\mathring{\xi}_1 E\mathring{\xi} _2 
        + b^2 E\mathring{\xi}_2^2 = 
        a^2D\xi_1 \pm 2abK\xi_1\xi_2 + b^2D\xi_2$
    \end{proof}
\end{enumerate}

\subsection{Коваріація як скалярний добуток випадкових величин}
Зафіксуємо ймовірнісний простір $\left\{\Omega, \mathcal{F}, P\right\}$.
Позначимо $\mathcal{L}^2(\Omega) = \left\{\xi : \Omega \rightarrow \mathbb{R} \;|\; E\xi^2 < +\infty\right\}$.
Оскільки $D\xi = E\xi^2 - (E\xi)^2 \geq 0$, то для $\xi \in \mathcal{L}^2(\Omega)$ $|E\xi| < +\infty$.
Таким чином, для будь-яких $\xi, \eta \in \mathcal{L}^2(\Omega)$ 
існує ${cov}(\xi,\eta)$, бо $|{cov}(\xi,\eta)| \leq \sqrt{D\xi} \cdot \sqrt{D\eta}$.

Для коваріації маємо 
${cov}(a\xi_1 + b\xi_2,\eta) = E(a\xi_1+b\xi_2)\eta - E(a\xi_1 + b\xi_2)E\eta = 
a E\xi_1\eta + b E\xi_2\eta - a E\xi_1 E\eta - b E\xi_2 E\eta = a\cdot{cov}(\xi_1,\eta) + b\cdot{cov}(\xi_2,\eta)$.
Також було доведено ${cov}(\xi, \eta) = {cov}(\eta, \xi)$ та ${cov}(\xi, \xi) \geq 0$.
Виконуються всі умови скалярного добутку, окрім ${cov}(\xi, \xi) = 0 \Rightarrow \xi = 0$.
Ця умова буде виконуватися, якщо розглядати інший простір випадкових величин:
$\mathcal{L}_0^2(\Omega) = \left\{\xi : \Omega \rightarrow \mathbb{R} \;|\; E\xi = 0, E\xi^2 < +\infty\right\}$.
\begin{remark}
    Якщо $\xi \in \mathcal{L}^2(\Omega)$, то $\mathring{\xi} \in \mathcal{L}_0^2(\Omega)$.
\end{remark}
\begin{exercise}
    Перевірити, що $\mathcal{L}_0^2(\Omega)$ є лінійним простором.
\end{exercise}

\noindent \textbf{Висновок:} ${cov}(\xi, \eta)$ задає \emph{скалярний добуток} на $\mathcal{L}_0^2(\Omega)$.

Таким чином, властивість $|{cov}(\xi,\eta)| \leq \sqrt{D\xi} \cdot \sqrt{D\eta}$ ---
це нерівність Коші-Буняковського, а кореляційна матриця $K$ --- це матриця Грама системи
випадкових величин $\left\{\xi_1, \xi_2, ..., \xi_n\right\}$.
З курсу лінійної алгебри відомо, що якщо ці випадкові величини лінійно незалежні, то матриця $K$ є невиродженою.