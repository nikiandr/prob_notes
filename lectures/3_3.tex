% !TEX root = ../main.tex
\section{Числові характеристики випадкових векторів}

\subsection{Математичне сподівання випадкових векторів}
$\vec{\xi} = \left(\xi_1, ..., \xi_n\right)^T$
\begin{definition}
    \emph{Математичним сподіванням $E_{\vec{\xi}}$} 
    випадкового вектора $\vec{\xi}$ називається вектор 
    $\left(E{\xi_1}, ..., E{\xi_n}\right)^T$.
\end{definition}
\begin{remark}
    Математичне сподівання двовимірного випадкового вектора 
    називається \emph{центром розсіювання}.
\end{remark}

\noindent \textbf{Способи знаходження}

Знаходимо закони розподілу окремих координат, з цього - 
математичні сподівання координат і об'єднуємо їх в один вектор.

\begin{equation}
    E{\xi_i} = 
    \underbrace{
        \int\limits_{-\infty}^{+\infty} 
        ... 
        \int\limits_{-\infty}^{+\infty}
    }_{n} x_i f_{\vec{\xi}}(\vec{x})dx_i...dx_n 
    \;\;\;\;
    \forall i = \overline{1,n}
\end{equation}

\subsection{Мішані початкові та центральні 
            моменти випадкових векторів}
\begin{definition}
    \emph{Мішаним початковим моментом} порядку 
    $k_1+k_2+...+k_n$ 
    $\left(k_i \in \mathbb{N} \; \forall i = \overline{1,n}
    \right)$
    випадкового вектора 
    $\vec{\xi} = \left(\xi_1, ..., \xi_n\right)^T$
    називається число
    \begin{equation*}
        \alpha_{k_1+k_2+...+k_n} = 
        E{\xi_1^{k_1}...\xi_n^{k_n}}
    \end{equation*}
\end{definition}
\begin{remark}
    Зауважимо, що
    \begin{center}
        \begin{tabular}{c}
            $E\xi_1 = \alpha_{1+0+...+0}$ \\
            $E\xi_2 = \alpha_{0+1+...+0}$ \\
            $...$ \\
            $E\xi_n = \alpha_{0+0+...+1}$
        \end{tabular}
    \end{center}
\end{remark}
\begin{definition}
    \emph{Мішаним центральним моментом} порядку 
    $k_1+k_2+...+k_n$
    $\left(k_i \in \mathbb{N} \; \forall i = \overline{1,n}
    \right)$ 
    випадкового вектора
    $\vec{\xi} = \left(\xi_1, ..., \xi_n\right)^T$
    називається число
    \begin{equation*}
        \beta_{k_1+k_2+...+k_n} = 
        E{\mathring{\xi_1}^{k_1}
        ...
        \mathring{\xi_n}^{k_n}}
    \end{equation*}
    де $\mathring{\xi_i} = \xi_i - 
    E\xi_i$.
\end{definition}
\begin{remark}
    Всі центральні моменти 1-го порядку - 
    нульові.

    $\forall i = \overline{1,n}$: 
    $\beta_{0+...+0+\underset{i}{2}
    +0+...+0} = E\mathring{\xi_i}^2 = 
    D\xi_i$
\end{remark}

Дисперсії координат задають розсіювання 
вздовж відповідної координатної осі.

\begin{definition}
    \emph{Кореляційним моментом} або 
    \emph{коваріацією} випадкових величин 
    $\xi_i$ та $\xi_j$ називається змішаний 
    центральний момент 
    порядку
    $0+...+0+1+0+...+0+1+0+....+0$ 
    \begin{equation*}
        \beta_{0+...+\underset{i}{1}+...
        +\underset{j}{1}+...+0}
        =
        E(\xi_i-E\xi_i)(\xi_j-E\xi_j)
    \end{equation*}
\end{definition}

\noindent \textbf{Позначення:} 
$cov(\xi_i,\xi_j)$ або 
$K\xi_i\xi_j$.

\begin{definition}
   \emph{Кореляційною матрицею} випадкового 
   вектора $\vec{\xi}$ називається матриця, у яку зібрано усі 
   кореляційні моменти випадкового вектора
   \begin{equation*}
       K = 
       \begin{pmatrix}
           D\xi_1 & K\xi_1\xi_2 & ... & K\xi_1\xi_n \\
           K\xi_1\xi_2 & D\xi_2 & ... & K\xi_2\xi_n \\
           ... & ... & ... & ...\\
           K\xi_1\xi_n & K\xi_2\xi_n & ... & D\xi_n
       \end{pmatrix}
   \end{equation*} 
\end{definition}
\begin{remark}[1]
    Діагональні елементи кореляційної матриці - дисперсії, бо 
    $\forall i = \overline{1,n} $: $K\xi_i\xi_i = D_{\xi_i}$
\end{remark}
\begin{remark}[2]
    $\mathring{\vec{\xi}} = \left(\mathring{\xi_1}, ..., \mathring{\xi_n}
    \right)^T$.
    Тоді $K = E\,\mathring{\vec{\xi}}\,\mathring{\vec{\xi}}^T$.
\end{remark}

\noindent \textbf{Властивості кореляційного моменту випадкових 
величин}
\begin{enumerate}
    \item $K\xi_i\xi_j = K\xi_j\xi_i$, $K\xi_i\xi_i = D\xi_i$
    \begin{proof}
        Напряму випливає з визначення кореляційного моменту та дисперсії.
    \end{proof}
    \item $\eta_1 = a_1\xi_1 + b_1$, $\eta_2 = a_2\xi_2 + b_2$
    
    Тоді $K\eta_1\eta_2 = a_1a_2K\xi_1\xi_2$
    \begin{proof}
        $K\eta_1\eta_2 = E(\eta_1 - E\eta_1)(\eta_2 - E\eta_2) = 
        E(a_1\xi_1 + b_1 - E(a_1\xi_1 + b_1))
        (a_2\xi_2 + b_2 - E(a_2\xi_2 + b_2)) = 
        Ea_1(\xi_1 - E\xi_1)a_2(\xi_2 - E\xi_2) = 
        a_1a_2E(\xi_1 - E\xi_1)(\xi_2 - E\xi_2) = 
        a_1a_2K\xi_1\xi_2$
    \end{proof}

    \item Зручна формула для знаходження кореляційного моменту.
    \begin{equation*}
        K\xi_1\xi_2 = E\xi_1\xi_2 - E\xi_1 E\xi_2
    \end{equation*}
    \begin{proof}
        $K\xi_1\xi_2 = E(\xi_1 - E\xi_1)(\xi_2 - E\xi_2) = 
        E(\xi_1\xi_2 - \xi_1E\xi_2 - \xi_2E\xi_1 + E\xi_1E\xi_2) = 
        E\xi_1\xi_2 - E\xi_1E\xi_2 - E\xi_2E\xi_1 + E\xi_1\xi_2 = 
        E\xi_1\xi_2 - E\xi_1E\xi_2$
    \end{proof}
    
    \item $D(a\xi_1 \pm b\xi_2) = a^2D\xi_1 \pm 2abK\xi_1\xi_2 + 
    b^2D\xi_2$

    \begin{proof}
        $D(a\xi_1 \pm b\xi_2) = 
        E(a\xi_1 \pm b\xi_2 - E(a\xi_1 \pm b\xi_2))^2 =
        E(a(\xi_1-E\xi_1)\pm b(\xi_2 - E\xi_2))^2 = 
        a^2 E\mathring{\xi_1}^2 \pm 2abE\mathring{\xi_1}E\mathring{\xi_2} 
        + b^2 E\mathring{\xi_2}^2 = 
        a^2D\xi_1 \pm 2abK\xi_1\xi_2 + b^2D\xi_2$
    \end{proof}
\end{enumerate}