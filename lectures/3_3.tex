% !TEX root = ../main.tex
\section{Числові характеристики випадкових векторів}

\subsection{Математичне сподівання випадкових векторів}
$\vec{\xi} = \left(\xi_1, ..., \xi_n\right)^T$
\begin{definition}
    \emph{Математичним сподіванням $E_{\vec{\xi}}$} 
    випадкового вектора $\vec{\xi}$ називається вектор 
    $\left(E{\xi_1}, ..., E{\xi_n}\right)^T$.
\end{definition}
\begin{remark}
    Математичне сподівання двовимірного випадкового вектора 
    називається \emph{центром розсіювання}.
\end{remark}

\noindent \textbf{Способи знаходження}

Знаходимо закони розподілу окремих координат, з цього - 
математичні сподівання координат і об'єднуємо їх в один вектор.

\begin{equation}
    E{\xi_i} = 
    \underbrace{
        \int\limits_{-\infty}^{+\infty} 
        ... 
        \int\limits_{-\infty}^{+\infty}
    }_{n} x_i f_{\vec{\xi}}(\vec{x})dx_i...dx_n 
    \;\;\;\;
    \forall i = \overline{1,n}
\end{equation}

\subsection{Мішані початкові та центральні 
            моменти випадкових векторів}
\begin{definition}
    \emph{Мішаним початковим моментом} порядку 
    $k_1+k_2+...+k_n$ 
    $\left(k_i \in \mathbb{N} \; \forall i = \overline{1,n}
    \right)$
    випадкового вектора 
    $\vec{\xi} = \left(\xi_1, ..., \xi_n\right)^T$
    називається число
    \begin{equation*}
        \alpha_{k_1+k_2+...+k_n} = 
        E{\xi_1^{k_1}...\xi_n^{k_n}}
    \end{equation*}
\end{definition}
\begin{remark}
    Зауважимо, що
    \begin{center}
        \begin{tabular}{c}
            $E\xi_1 = \alpha_{1+0+...+0}$ \\
            $E\xi_2 = \alpha_{0+1+...+0}$ \\
            $...$ \\
            $E\xi_n = \alpha_{0+0+...+n}$
        \end{tabular}
    \end{center}
\end{remark}