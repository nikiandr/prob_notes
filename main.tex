\documentclass{report}

%%% Работа с языком
%\usepackage{cmap}					    % поиск в PDF				    % русские буквы в формулах
\usepackage[T2A]{fontenc}			% кодировка
\usepackage[utf8]{inputenc}             % кодировка исходного текста
\usepackage[english,ukrainian]{babel}	% локализация и переносы	    
\usepackage{indentfirst}

\usepackage[a4paper, top=25mm, bottom=25mm, left=30mm, right=30mm]{geometry}

\usepackage{amsmath,amsfonts,amssymb,amsthm,mathtools} % AMS

\usepackage{multicol} % Несколько колонок
\usepackage{wrapfig} % Картинки посреди текста
\usepackage{chngcntr} % для нумерации формул
\usepackage{enumerate} % чтоб можно было делать римскую нумерацию
\usepackage{mdwlist}

%%%% Теоремы, определения и т.д. и т.п.
\theoremstyle{plain} % Это стиль по умолчанию, его можно не переопределять.
\newtheorem{theorem}{Теорема}
\newtheorem{proposition}{Твердження}
\renewcommand{\qedsymbol}{$\blacktriangle$}
 
\theoremstyle{definition} % "Определение"
\newtheorem{definition}{Означення}[section]
\newtheorem*{example}{Приклад}

\theoremstyle{remark}
\newtheorem*{remark}{Зауваження}
\newtheorem*{exercise}{Вправа}

\counterwithout{equation}{chapter} % нумерация формул
\counterwithin*{equation}{section}

%%%% Картинки
\usepackage{graphicx}
\usepackage{tikz}
\usetikzlibrary{patterns}
\usetikzlibrary{shapes.geometric}
\usetikzlibrary{arrows.meta}
\usetikzlibrary{shapes.geometric}
\graphicspath{{pictures/}}
\DeclareGraphicsExtensions{.pdf,.png,.jpg}


\author{\Huge Каніовська І.Ю.}
\title{
    \textbf{\fontsize{40}{48}\selectfont Конспект лекцій \\з теорії ймовірностей \\та математичної \\\vspace{0.64em}статистики}
    }
\date{}

\begin{document} 
    \maketitle
    \tableofcontents
    \part{Теорія ймовірностей}
    \chapter{Випадкові події}
        % !TEX root = ../main.tex

\section{Основні поняття теорії ймовірності}
\subsection*{Стохастичний експеримент (СЕ)}
\begin{definition}
    Стохастичним експериментом (випробуванням) називається експеримент, 
    який можна повторювати неодноразово зберігаючи певні умови і результат цього 
    експерименту заздалегідь передбачити неможливо.
\end{definition}
\begin{definition}
    Будь-який результат СЕ називається подією.
\end{definition}
\begin{example}
    СЕ - кидання кубика, подія - випало 6 очок.
\end{example}

\textbf{Події бувають:}
\begin{enumerate}
    \item Випадкові (можуть відбутися чи не відбутися при проведенні СЕ)
    \item Неможливі (ніколи не відбуваються при проведенні СЕ)
    \item Вірогідні (завжди відбуваються при проведенні СЕ)
\end{enumerate}
\subsection*{Теоретико-множинний підхід до основних понять ТЙ}

%qw
        % !TEX root = ../main.tex

\newlength\Radius
\setlength\Radius{2cm}

\section{Геометричні ймовірності. Аксіоми теорії ймовірностей.}
\subsection{Геометрична модель ймовірності}
\begin{example}
    Нехай точка кидається навмання на відрізок $\left[a; b\right]$. 
    Яка ймовірність її 
    потрапляння в $\left<\alpha; \beta\right> \subset  \left[a; b\right]$?
    \newline
    Розглянемо подію $A = \left\{ 
        \text{точка потрапила в} \left<\alpha; \beta\right>
    \right\}$.
    \newline
    \hbox to \hsize{\hfil{
        \begin{tikzpicture}
            \draw [fill] (0, 0) circle [radius = 0.05];
            \node [below] at (0, 0) {$a$};
            \node [above] at (0, 0) { };
            \node [below] at (8, 0) {$b$};
            \draw [-{Straight Barb}] (6,0) to (2,0) {};
            \draw [-{Straight Barb}] (2,0) to (6,0) {};
            \node [below] at (2, 0) {$\alpha$};
            \node [below] at (6, 0) {$\beta$};
            \draw [fill] (8, 0) circle [radius = 0.05];
            \draw [thick] (0, 0) -- (8, 0);
        \end{tikzpicture}
    }\hfil}
    $P(A) = k\cdot l_{\left<\alpha; \beta\right>}\; \text{для деякого}\; k > 0$.
    З іншого боку, $P(\Omega) = 1 = k\cdot l_{\left[a; b\right]}$. Таким чином 
    $k = \frac{1}{l_{\left[a; b\right]}} = \frac{1}{b-a}$.
    Тому $P(A) = \frac{l_{\left<\alpha; \beta\right>}}{l_{\left[a; b\right]}}$.
\end{example}
Нехай простір елементарних подій інтерпретується як замкнена область в 
$ \mathbb{R} ^n$, а випадкові події --- її підмножини. В якості $\sigma$-алгебри 
подій $\mathcal{F}$ беремо підмножини, що мають міру $\mu$. Тоді в якості ймовірності 
деякої події $A$ беремо $P(A) = \frac{\mu(A)}{\mu(\Omega)}$. 
Наприклад, в $\mathbb{R}^1$ беремо міру <<довжина>>, в $\mathbb{R}^2$ --- <<площа>>, а в $\mathbb{R}^3$ --- <<об'єм>>.

Ймовірності, що знаходяться таким чином називаються \emph{геометричними}, а сама модель 
називається \emph{геометричною моделлю ймовірності}.
Геометрична модель може використовуватись, 
коли $\Omega$ має геометричну інтерпретацію як замкнена область в $\mathbb{R}^n$,
а елементарні події --- рівноможливі.

\begin{example}[задача Бюффона]\label{buffon}
    На площині накреслені паралельні прямі на відстані $2a$, на них кидається 
    голка довжиною $2l,\; l \leq a $. Яка ймовірність того, що голка перетне 
    яку-небудь пряму?
    \newline
    \hbox to \hsize{\hfil{
        \begin{tikzpicture}[scale = 0.5]
            \draw [thick] (0, 0) -- (8, 0);
            \draw [thick] (0, 2) -- (8, 2);
            \draw [thick] (0, 4) -- (8, 4);
            \draw [thick] (0, 6) -- (8, 6);
            \draw [{Straight Barb}-{Straight Barb}] (7.5,4) to (7.5,6);
            \node [right] at (7.5, 5) {$2a$}; 
        \end{tikzpicture}
        \begin{tikzpicture}[scale = 1]
            \draw (0, 0) -- (4, 0);
            \draw (0, 2) -- (4, 2);
            \draw [thick] (0.5, -0.5)-- (2.5, 1.5);
            \draw [{Straight Barb}-{Straight Barb}] (3.5,0) to (3.5,2);
            \draw [{Straight Barb}-{Straight Barb}] (1.5,0) to (1.5,0.5);
            \node [right] at (3.5, 1) {$2a$};
            \node [right] at (1.5, 0.25) {$ x $};
            \draw [-{Straight Barb}](2.5, 0) to [out = 90, in = 0](1.75, 0.75);
            \node [right] at (2.28, 0.53) {$ \varphi $};
        \end{tikzpicture}
    }\hfil}
    Достатньо розглядати лише дві прямі. $\Omega = \left\{(\varphi, x)\in 
    \mathbb{R}^2: \varphi \in \left[0; \pi\right], x \in \left[0; a\right] 
    \right\}$
    \newline
    При такій побудові простору елементарних подій подія $A = \left\{\text{голка 
    перетне пряму}\right\} =$
    \newline
    $\left\{(\varphi, x)\in 
    \Omega: x\leq l\sin\varphi\right\}$
    \newline
    \hbox to \hsize{\hfil{
        \begin{tikzpicture}
            \draw [<->] (4, 0) -- (0, 0) -- (0, 2.5);
            \node [left] at (0, 2.5) {$x$};
            \node [below] at (4, 0) {$\varphi$};
            \draw [domain=0:pi,fill=gray] plot (\x, {1*sin(\x r)});
            \node [below] at (0, 0) {$0$};
            \node [below] at (pi, 0) {$\pi$};
            \draw [dashed] (0,1) to (pi/2,1);
            \node [left] at (0, 1) {$l$};
            \draw (0,2) -- (pi, 2) -- (pi, 0);
            \node [left] at (0, 2) {$a$};
            \node [below left] at (pi, 2) {$\Omega$};
            \node [above] at (pi/2, 0.25) {$A$};
        \end{tikzpicture}
    }\hfil}
    $P(A) = \frac{S(A)}{S(\Omega)},\; S(\Omega) = \pi a,\;S(A) = \int_0^\pi 
    l\sin\varphi\,\mathrm{d}x = l\left.(-\cos\varphi)\right|^\pi_0 = 2l$.
    Отже, $P(A) = \frac{2l}{\pi a}$.
    \newline
    Якщо провести $n$ кидань голки, у $m$ з яких голка потрапить на пряму, то за допомогою приблизної рівності $\frac{m}{n} 
    \approx \frac{2l}{\pi a}$ можна приблизно обчислити число $\pi$. 
    Наприклад, якщо взяти $n=5000, m=2532, l/a = 0.8$, то отримаємо $\pi \approx 3.1596$.
\end{example}
\begin{example}[парадокс Бертрана]
    Нехай в колі радіусом $R$ навмання обирається хорда. Яка ймовірність того, 
    що її довжина буде більшою за довжину сторони правильного трикутника, 
    вписаного в це коло?
    \newline
    Існують три способи розв'язання цієї задачі, причому кожен з них дає різний результат.
    
    \textbf{Спосіб 1.}
    З міркувань симетрії обирається якийсь фіксований діаметр кола і розглядається
    всі перпендикулярні до нього хорди. Серед них і обирається навмання довільна хорда.
    Очевидно, що кожна хорда у цьому випадку може бути однозначно визначена своєю серединою,
    тобто кожній хорді можна взаємно однозначно поставити у відповідність координату її середини,
    якщо за початок відліку взяти якийсь з кінців фіксованого діаметру.
    \newline
    \hbox to \hsize{\hfil{
        \begin{tikzpicture}[scale = 1.5]
            \draw (0, 0) circle [radius = 1]; 
            \draw (0, 1) -- (-0.86602540378, -0.5); 
            \draw (0, 1) -- (0.86602540378, -0.5);
            \draw (0.86602540378, -0.5) -- (-0.86602540378, -0.5);
            \draw [->] (-1,0) to (1, 0);
            \node [left] at (-1, 0) {$0$};
            \node [right] at (1, 0) {$2R$};
            \draw [dashed] (0, -1) to (0, 1);
            \draw [dashed] (-0.3, -0.954) to (-0.3, 0.954);
            \draw [dashed] (-0.7, -0.714) to (-0.7, 0.714);
            \draw [dashed] (0.3, -0.954) to (0.3, 0.954);
            \draw [dashed] (0.7, -0.714) to (0.7, 0.714);
            \draw [ultra thick] (-0.5, 0) -- (0.5, 0);
            \draw [ultra thick] (-0.5, 0.05) -- (-0.5, -0.05);
            \draw [ultra thick] (0.5, 0.05) -- (0.5, -0.05);
            \node [below] at (-0.5, 0) {$\frac{R}{2}$};
            \node [below] at (0.5, 0) {$\frac{3R}{2}$};
        \end{tikzpicture}
    }\hfil}
    Таким чином, множина всіх значень координати середини хорди $\Omega = \left[0; 2R\right]$. 
    Множина, що відповідає події --- це відрізок $A = \left[\frac{R}{2}; \frac{3R}{2}\right]$.
    В якості міри візьмемо довжину. Тоді $P(A) = \frac{l(A)}{l(\Omega)} = \frac{R}{2R}= \frac{1}{2}$.
    
    \textbf{Спосіб 2.}
    В цьому способі пропонується розглядати тільки хорди з одним закріпленим кінцем.
    Кожній хорді поставимо у відповідність ту частину дуги кола, яка потрапляє у кут $\varphi$,
    що утворює хорда з дотичною, проведеною через точку закріплення кінця всіх хорд.
    \newline
    \hbox to \hsize{\hfil{
        \begin{tikzpicture}[scale = 1.5]
            \draw (0, 0) circle [radius = 1];
            \draw (0, 1) -- (-0.86602540378, -0.5); 
            \draw (0, 1) -- (0.86602540378, -0.5);
            \draw (0.86602540378, -0.5) -- (-0.86602540378, -0.5);
            \draw (-1.2, 1) -- (1.2, 1);
            \draw [fill] (0, 1) circle [radius = 0.05];
            \draw [fill] (0.17, -0.985) circle [radius = 0.05];
            \draw [fill] (1, 0) circle [radius = 0.05];
            \draw [fill] (-0.967, -0.256) circle [radius = 0.05];
            \draw (0.16, 1) arc (0:-46:0.16);
            \draw (0.19, 1) arc (0:-46:0.19);
            %\node [above] at (0, 1);
            \draw [dashed] (0, 1) to (0.17, -0.985);
            \draw [dashed] (0, 1) to (1, 0);
            \draw [dashed] (0, 1) to (-0.967, -0.256);
            \node [below right] at (0.14, 1) {$_\varphi$};
            \draw [ultra thick](0.86602540378, -0.5) to [out = 242, in = 0](0, -1);
            \draw [ultra thick](0, -1) to [out = 180, in = 300](-0.86602540378, -0.5);
        \end{tikzpicture}
    }\hfil}
    Тоді множина $\Omega = \left[ 0; \pi\right]$, а $A = \left[ \frac{\pi}{3}; \frac{2\pi}{3}\right]$.
    Знову візьмемо за міру довжину і отримаємо $P(A) = \frac{\pi /3}{\pi} = \frac{1}{3}$.
    
    \textbf{Спосіб 3.}
    Розглядаються всі хорди кола. Кожній з них взаємно однозначно ставиться у відповідність точка її середини $(x,y)$,
    якщо за початок координат прийняти центр кола.
    \newline
    \hbox to \hsize{\hfil{
        \begin{tikzpicture}[scale = 1.8]
            \draw (0, 0) circle [radius = 1];
            \draw (0, 0) circle [radius = 0.5];
            \draw (0, 0) [fill] circle [radius = 0.05];
            \draw (0, 1) -- (-0.86602540378, -0.5); 
            \draw (0, 1) -- (0.86602540378, -0.5);
            \draw (0.86602540378, -0.5) -- (-0.86602540378, -0.5);
            \draw (-0.737, 0.675) [fill] circle [radius = 0.05];
            \draw (0.878, 0.479) [fill] circle [radius = 0.05];
            \draw [dashed] (-0.737, 0.675) to (0.878, 0.479);
            \draw [dashed] (0, 0) to (0.0705, 0.577);
            \draw (0.0705, 0.577) [fill] circle [radius = 0.05];
        \end{tikzpicture}
    }\hfil}
    В такому випадку $\Omega = \left\{ (x,y)\in \mathbb{R}^2: x^2 + y^2 \leq R^2\right\}$,
    а $A = \left\{ (x,y)\in \Omega: x^2 + y^2 \leq \frac{R^2}{4}\right\}$.
    Тоді $P(A) = \frac{\pi R^2/4}{\pi R^2} = \frac{1}{4}$.

    Жозеф Бертран був противником геометричного означення ймовірності.
    Він казав, що воно не дає можливості однозначно обчислити ймовірність
    однієї і тієї ж випадкової події та використовував вищенаведений приклад
    для доведення своєї правоти. Дійсно, було отримано три різних відповіді при
    розв'язанні однієї задачі. Так в чому ж справа? Насправді, помилка полягає
    у різних трактуваннях поняття <<навмання обрана хорда>>.
    В кожному з трьох способів це трактування було різним, що й стало причиною різних відповідей.
\end{example}

\subsection{Аксіоми теорії ймовірностей}
\begin{enumerate}[I.]
    \item Побудова вимірного простору $\left\{ \Omega, \mathcal{F}\right\}$:
    \begin{enumerate}[\bfseries {A}1:]
        \item Будь-якому стохастичному експерименту можна поставити у відповідність
        простір елементарних подій $\Omega$.
        \item $\left(\forall n \in \mathbb{N}: A_n \in \mathcal{F} \right) \Rightarrow \left( \bigcup_{n=1}^{n (\infty)} A_n \in \mathcal{F}\right)$.
        \item $\left( A \in \mathcal{F}\right) \Rightarrow \left( \overline{A} \in \mathcal{F}\right)$.
        
        Дві останні аксіоми стосуються побудови алгебри чи $\sigma$-алгебри подій.
    \end{enumerate}
    \item Властивості ймовірності:
    \begin{enumerate}[\bfseries {P}1:]
        \item $\forall A \in \mathcal{F}: P(A)\geq 0$.
        \item $P(\Omega) = 1$ --- аксіома нормування.
        \item $\forall A_1, A_2, ..., A_n, ... \in \mathcal{F},  A_i \cap A_j \text{ при } i \neq j: P(\bigcup_{n=1}^{\infty} A_n) = \sum_{n=1}^{\infty} P(A_n)$ ---
        аксіома зліченної адитивності.
        Іноді у випадках <<простої>> алгебри (а не $\sigma$-алгебри) достатньо
        аксіоми скінченної адитивності \textbf{P3$'$}: 
        $\forall A_1, A_2, ..., A_n \in \mathcal{F},  A_i \cap A_j \text{ при } i \neq j: P(\bigcup_{k=1}^{n} A_k) = \sum_{k=1}^{n} P(A_k)$.
    \end{enumerate}
\end{enumerate}
\begin{definition}
    Нормована міра $P$, що введена на вимірному просторі $\left\{ \Omega, \mathcal{F}\right\}$,
    називається \emph{ймовірністю}. Всі події беруться з $\sigma$-алгебри подій.
    Трійка $\left\{ \Omega, \mathcal{F}, P\right\}$ називається 
    \emph{ймовірнісний простором} стохастичного експерименту.
\end{definition}

Ця система аксіом несуперечна, бо існують стохастичні експерименти,
які задовольняють цим аксіомам, але неповна, бо в різних задачах
теорії ймовірностей розглядаються різні ймовірнісні простори.

\subsection{Властивості ймовірності, що випливають з аксіом}
\begin{enumerate}
    \item Якщо $A_1, A_2, ..., A_n, ... \in \mathcal{F}$ утворюють повну групу 
    подій, то $P(\cup_{k=1}^\infty A_k) = 1$.
    \begin{proof}
        Випливає з аксіоми \textbf{P2}.
    \end{proof}
    \item $P(\overline{A}) = 1 - P(A)$.
    \begin{proof}
        $A \cup \overline{A} = \Omega \Rightarrow 1 \overset{P2}{=} P(\Omega) 
        = P(A \cup \overline{A}) \overset{P3'}{=} P(A) + P(\overline{A})$.
    \end{proof}
    \item $P(\varnothing) = 0$.
    \begin{proof}
        Наслідок з властивості 2 $(\varnothing = \overline{\Omega})$.
    \end{proof}
    \begin{remark}
        Якщо ймовірність події дорівнює нулю, то це не означає, що подія неможлива.
    \end{remark}
    \begin{example}
        СЕ --- кидання точки на відрізок $[a; b]$. A = \{\text{точка потрапила в 
        певну } $x \in [a; b]$\}. $A$ не є неможливою, проте $P(A) = 0$.
    \end{example}
    \item $A \subset B \Rightarrow P(A) \leq P(B)$.
    \begin{proof}
        $A \subset B \Rightarrow B = A \cup (B \backslash A) 
        \overset{P3'}{\Rightarrow} P(B) = P(A) + P(B \backslash A) 
        \overset{P1}{\geq} P(A)$.
    \end{proof}
    \item $\forall A \in \mathcal{F}: P(A) \leq 1$.
    \begin{proof}
        Наслідок з властивості 4, $A \subset \Omega$ та \textbf{P2}.
    \end{proof}
    \item $A \subset B: P(B \backslash A) = P(B) - P(A)$.
    \begin{proof}
        Наслідок з доведення властивості 4.
    \end{proof}
    \item $\forall A, B \in \mathcal{F}: P(A \cup B) = P(A) + P(B) - P(A \cap B)$.
    \begin{proof}
        $A \cup B = (A\backslash(A \cap B)) 
        \cup (A \cap B) 
        \cup (B\backslash(A \cap B))$ --- попарно несумісні події. 
        \newline
        З аксіоми \textbf{P3$'$}: $P(A \cup B) = P(A\backslash(A \cap B)) 
        + P(A \cap B) + P(B\backslash(A \cap B)) \overset{6}{=} P(A) - P(A \cap B) + P(A \cap B)
        + P(B) - P(A \cap B) = P(A) + P(B) - P(A \cap B)$.
    \end{proof}
    \item Узагальнення властивості 7 --- формула включення-виключення для ймовірностей: 
    
    \begin{math}
        \forall A_1, A_2, \dots, A_n \in \mathcal{F} 
    : P(\bigcup_{i=1}^n A_i) = \sum_{i=1}^n P(A_i) - \sum_{i < j}^n P(A_i \cap A_j)
    + \sum_{i < j < k}^n P(A_i \cap A_j \cap A_k) - ... + (-1)^{n-1}P(\bigcap_{i=1}^n A_i)
    \end{math}.
    \begin{exercise}
        Довести.
    \end{exercise}
\suspend{enumerate}
\begin{example}[задача про неуважну секретарку]
    Секретарка поклала $n$ листів в $n$ чистих конвертів, заклеїла ці конверти і тільки 
    після цього написала адреси. Яка ймовірність того, що хоча б один з листів дійде 
    за призначенням?
    \newline
    $A_i = \left\{i\text{-тий лист дійшов за призначенням}\right\}, i = 
    1,...,n. \;P(A_i) = \frac{1}{n} = \frac{(n-1)!}{n!}$.
    \newline
    $A = \left\{\text{хоча б один із листів дійшов за призначенням}\right\}, 
    A = \bigcup_{i=1}^n A_i$.
    \newline
    $P(A_i \cap A_j) = \frac{(n-2)!}{n!} = \frac{1}{n(n-1)}, i \neq j$.
    \newline
    $P(A_i \cap A_j \cap A_k) = \frac{1}{n(n-1)(n-2)}, i \neq j \neq k$.
    \newline
    ...
    \newline
    $P(A_1 \cap ... \cap A_n) = \frac{1}{n!}$.
    \newline
    За формулою включення-виключення $P(A) = n\cdot \frac{1}{n} - C_n^2 \cdot \frac{1}{n(n-1)}- \dots + (-1)^{n-1}\frac{1}{n!}
    = 1 - \frac{1}{2!} + \frac{1}{3!} - \dots +(-1)^{n-1} \cdot \frac{1}{n!} \approx 
    1 - \frac{1}{e} \approx 0.63$.
\end{example}
\resume{enumerate}
    \item $\forall A_1, A_2, ..., A_n, ... \in \mathcal{F}: 
    P(\bigcup_{k=1}^\infty A_k) \leq \sum_{k=1}^\infty P(A_k)$
    \begin{proof}
        Введемо події $B_1 = A_1, B_2 = \overline{A_1} \cap A_2, 
        B_3 = \overline{A_1} \cap \overline{A_2} \cap A_3, ..., B_n = 
        \overline{A_1} \cap \overline{A_2} \cap ...$ 
        \newline
        $... \cap \overline{A_{n-1}} \cap A_n$. Ці події попарно несумісні, 
        $\bigcup_{k=1}^\infty B_k = \bigcup_{k=1}^\infty A_k,\;B_k \subset A_k $ для всіх $k \in \mathbb{N}$.
        \newline
        $P(\bigcup_{k=1}^\infty A_k) = P(\bigcup_{k=1}^\infty B_k) 
        \overset{P3}{=} \sum_{k=1}^\infty P(B_k) \overset{4}{\leq} 
        \sum_{k=1}^\infty P(A_k)$.
    \end{proof}
    \item $\forall A_1, A_2, ..., A_n, ... \in \mathcal{F}: P(\bigcap_{k=1}^\infty
    A_k) = 1 - P(\overline{\bigcap_{k=1}^\infty A_k}) = 1 - P(\bigcup_{k=1}^\infty \overline{A_k}) \geq  
    1 - \sum_{k=1}^\infty P(\overline{A_k})$.
\end{enumerate}

\subsection{Теореми неперервності ймовірності}
\begin{theorem}\label{th:1}
    Нехай є монотонно неспадна послідовність подій $A_1 \subset A_2 \subset ... \subset A_n ... \in \mathcal{F}$.
    Тоді $P(\bigcup_{n=1}^{\infty} A_n) = \lim_{n\rightarrow \infty} P(A_n)$.
\end{theorem}
\begin{proof}
    $\bigcup_{n=1}^{\infty} A_n = A_1 \cup (A_2 \backslash A_1) \cup (A_3 \backslash A_2) \cup ... \cup (A_{n} \backslash A_{n-1}) \cup ...$ 

    \noindent З аксіоми \textbf{P3} $P(\bigcup_{n=1}^{\infty} A_n) = P(A_1) + \sum_{n=2}^{\infty} P(A_{n} \backslash A_{n-1})$.

    \noindent $S_n = P(A_1) + \sum_{k=2}^{n} P(A_{k} \backslash A_{k-1}) = P(A_1) + P(A_2) - P(A_1) + P(A_3) - P(A_2) + ... + P(A_n) - P(A_{n-1}) = P(A_n)$.
    Отже, $P(\bigcup_{n=1}^{\infty} A_n) = \lim_{n\rightarrow \infty} S_n = \lim_{n\rightarrow \infty} P(A_n)$.
\end{proof}
\begin{theorem}\label{th:2}
    Нехай є монотонно спадна послідовність подій $A_1 \supset A_2 \supset ... \supset A_n ... \in \mathcal{F}$.
    Тоді $P(\bigcap_{n=1}^{\infty} A_n) = \lim_{n\rightarrow \infty} P(A_n)$.
\end{theorem}
\begin{proof}
    $P(\bigcap_{n=1}^{\infty} A_n) = 1 - P(\bigcup_{n=1}^{\infty} \overline{A_n}) \overset{\text{теор. \refeq{th:1}}}{=} 1 -
    \lim_{n\rightarrow \infty} P(\overline{A_n}) = \lim_{n\rightarrow \infty} (1-P(\overline{A_n})) = \lim_{n\rightarrow \infty} P(A_n)$.
\end{proof}

        % !TEX root = ../main.tex
\section{Умовна ймовірність та її застосування}

\subsection{Поняття умовної ймовірності}
Припустимо, що спостерігається деякий експеримент, що описується класичною моделлю, а для події $B$ $\P(B)>0$.
\emph{Умовна ймовірність} $\P(A/B)$ --- це ймовірність події $A$ за умови, що відбулась подія $B$.
Наприклад, експеримент --- витягання карт з колоди, $A=\left\{\text{витягнуто даму пік}\right\}$, $B=\left\{\text{витягнуто карту чорної масті}\right\}$.

Оскільки експеримент описується класичною моделлю, то можемо позначити $m_B$ та $m_{A\cap B}$ кількості елементарних подій, що сприяють появами подій $B$ та $A \cap B$ відповідно.
Тоді $\P(A/B) = \frac{m_{A\cap B}}{m_B} = \frac{m_{A\cap B}/n}{m_B/n} = \frac{\P(A\cap B)}{\P(B)}$, де $n = \mathrm{card}(\Omega)$.

Для довільних ймовірнісних просторів ця формула вводиться як означення умовної ймовірності.
\begin{definition}
    Якщо подія $B$ має додатну ймовірність $\P(B)>0$, то \emph{умовна ймовірність} події $A$ за умови, що відбулась подія $B$,
    обчислюється за формулою 
    \begin{equation}\label{eq:cond_prob}
        \P(A/B) = \frac{\P(A\cap B)}{\P(B)}
    \end{equation}
\end{definition}
\noindent \textbf{Властивості умовної ймовірності:}
\begin{enumerate}
    \item $\P(A/B) \geq 0$.
    \item $\P(\Omega /B) = \P(B/B) = 1$.
    \item $ \forall A_1, A_2, ... , A_n, ... \in \mathcal{F}: A_i \cap A_j = \varnothing \text{ при } i \neq j : \\
    \P\left(\left(\bigcup\limits_{n=1}^{\infty} A_n\right)/B\right) = \frac{\P\left(\left(\bigcup\limits_{n=1}^{\infty} A_n\right)\cap B\right)}{\P(B)} = \frac{\P\left(\bigcup\limits_{n=1}^{\infty} (A_n\cap B)\right)}{\P(B)} \overset{\text{\textbf{P3}}}{=} \frac{\sum\limits_{n=1}^{\infty} \P(A_n \cap B)}{\P(B)} = \sum\limits_{n=1}^{\infty} \P(A_n/B)$.
\end{enumerate}
\vspace{1em}
Таким чином, для умовної ймовірності виконуються аксіоми \textbf{P1}, \textbf{P2} та \textbf{P3}. 
\begin{exercise}
    Нехай $\left\{ \Omega, \mathcal{F}, \P\right\}$ --- деякий ймовірнісний простір, а $B\in\mathcal{F}$ --- деяка
    подія з $\P(B) > 0$. Перевірити, що для простору елементарних подій $\Omega_B$, що складається
    з усіх елементарних подій, які містить $B$, $\mathcal{F}_B = \left\{ A\cap B : A \in \mathcal{F} \right\}$ є $\sigma$-алгеброю.
    Разом з вже доведеними властивостями умовної ймовірності $\P(\cdot / B)$ це означає, що $\left\{ \Omega_B, \mathcal{F}_B, \P(\cdot / B)\right\}$
    також є ймовірнісним простором, в якому $B$ є вірогідною подією.
\end{exercise}

\subsection{Незалежність подій}
\begin{definition}
    Дві події $A$ та $B$ називаються \emph{незалежними}, якщо
    \begin{equation}\label{eq:indep_events}
        \P(A/B) = \P(A) \text{ або } \P(B/A) = \P(B)
    \end{equation}
    Це означає, що на появу однієї події не впливає поява іншої.
\end{definition}
\noindent \textbf{Властивості незалежних подій:}
\begin{enumerate}
    \item Якщо події $A$ та $B$ незалежні, то $\P(A\cap B) = \P(A)\cdot \P(B)$. Це випливає з \eqref{eq:cond_prob} та \eqref{eq:indep_events}.
    \begin{remark}
        Несумісні події залежні, якщо їх ймовірності не нульові.
    \end{remark}

    \item Якщо події $A$ та $B$ незалежні, то пари $A$ та $\overline{B}$, 
    $\overline{A}$ та $B$, $\overline{A}$ та $\overline{B}$ --- теж незалежні.
    \begin{proof}
        Доведемо спочатку для пари $A$ та $\overline{B}$. $\P(A\cap \overline{B}) = \P(A\setminus (A\cap B)) = \P(A) - \P(A\cap B) = \P(A) - \P(A)\cdot \P(B) = \P(A)\cdot (1 - \P(B)) = \P(A)\cdot \P(\overline{B})$.
        Отже, $A$ та $\overline{B}$ --- незалежні. Доведення для інших пар аналогічне.
    \end{proof}
    \item Якщо $A$ та $B$, $A$ та $C$ незалежні, а $B$ та $C$ несумісні, то $A$ та $B\cup C$ --- незалежні події.
    \begin{proof}
        $\P(A\cap (B \cup C)) = \P((A \cap B) \cup (A \cap C)) = \P(A\cap B) + \P(A\cap C) = \P(A)\cdot \P(B) + \P(A)\cdot \P(C) = \P(A) \cdot (\P(B) + \P(C)) = \P(A)\cdot \P(B\cup C).$
    \end{proof}
\end{enumerate}

\begin{definition}
    Події $A_1, A_2, ..., A_n \in \mathcal{F}$ називаються \emph{незалежними у сукупності}, якщо
    \begin{equation}\label{eq:indep}
        \forall i_1, i_2, ..., i_k \in \{1,...,n\}: \P\left(\bigcap\limits_{j=1}^k A_{i_j}\right) = \prod\limits_{j=1}^k \P\left(A_{i_j}\right)
    \end{equation}
    Зокрема:
    \nopagebreak
    \begin{enumerate}
        %\nopagebreak
        \item $\forall i, j \in {1,..,n}, i\neq j: \P(A_i \cap A_j) = \P(A_i)\cdot \P(A_j)$ --- попарна незалежність.
        \item $\P\left(\bigcap\limits_{k=1}^n A_k\right) = \prod\limits_{k=1}^n \P(A_k)$.
    \end{enumerate}
\end{definition}
\begin{remark}
    Із незалежності подій у сукупності випливає попарна незалежність, 
    але наслідку в інший бік, взагалі кажучи, немає.
\end{remark}
\begin{example}[приклад Бернштейна]
    Дитина кидає на підлогу тетраедр, три грані якого розфарбовані відповідно зеленим, синім та червоним кольором,
    а четверта --- усіма трьома кольорами. Введемо події $\text{З}, \text{С}, \text{Ч}$, які означають, що випала грань,
    на якій є зелений, синій або червоний колір відповідно.

    $\P(\text{З}) = \P(\text{С}) = \P(\text{Ч}) = \frac{1}{2}$, 
    $\P(\text{З} \cap \text{С}) = \P(\text{З} \cap \text{Ч}) = \P(\text{С} \cap \text{Ч}) = \frac{1}{4}$,
    тому ці події є попарно незалежними. Але $\P(\text{З} \cap \text{С} \cap \text{Ч}) = \frac{1}{4} \neq \frac{1}{8}$,
    тому незалежності у сукупності немає.
\end{example}

\subsection{Теореми додавання та множення}
\noindent\textbf{Теорема множення.} 
З формули \eqref{eq:cond_prob} випливає, що
\begin{equation}\label{eq:mult_for_2}
    \P(A\cap B) = \P(A)\cdot \P(B/A) = \P(B) \cdot \P(A/B)
\end{equation}
Для трьох подій: $\P(A \cap B \cap C) = \P(A\cap B) \cdot \P(C/(A\cap B)) = \P(A)\cdot \P(B/A) \cdot \P(C/(A\cap B))$.
За методом математичної індукції неважко довести, що
\begin{equation}\label{eq:mult_for_n}
    \P\left( \bigcap\limits_{k=1}^{n} A_k\right) = \P\left(A_1\right) \cdot \P\left(A_2/A_1\right) \cdot \P\left( A_3 / \left( A_1 \cap A_2\right)\right) \cdot ... \cdot \P\left( A_n / \left( \bigcap\limits_{k=1}^{n-1} A_k\right) \right)
\end{equation}

Якщо $A_1, A_2, ..., A_n \in \mathcal{F}$ --- незалежні у сукупності, то, як було зазначено, з \eqref{eq:indep} випливає, що
формула \eqref{eq:mult_for_n} спрощується до $\P\left(\bigcap\limits_{k=1}^n A_k\right) = \prod\limits_{k=1}^n \P(A_k) = \prod\limits_{k=1}^n \left(1-\P(\overline{A_k})\right)$.

\begin{example}
    На 10 картках написано по одній букві так, що можна скласти слово <<математика>>.
    Дитина навмання витягує без повернення по одній картці. Яка ймовірність того, що після витягання 4 карток
    утвориться слово <<мама>>?

    Треба обчислити ймовірність $\P\left(M^{(1)} \cap A^{(2)} \cap M^{(3)} \cap A^{(4)}\right)$,
    де верхній індекс означає крок, на якому витягнуто літеру.
    $\P\left(M^{(1)}\right) = \frac{2}{10}$, $\P\left(A^{(2)} / M^{(1)}\right) = \frac{3}{9}$,
    $\P\left(M^{(3)}/\left(M^{(1)} \cap A^{(2)}\right)\right) = \frac{1}{8}$, $\P\left(A^{(4)} / \left(M^{(1)} \cap A^{(2)} \cap M^{(3)}\right)\right) = \frac{2}{7}$.
    Тому за формулою \eqref{eq:mult_for_n} $\P\left(M^{(1)} \cap A^{(2)} \cap M^{(3)} \cap A^{(4)}\right) = \frac{2 \cdot 3 \cdot 1 \cdot 2}{10 \cdot 9 \cdot 8 \cdot 7} = \frac{1}{420}$.
\end{example}

\noindent\textbf{Теорема додавання.} Нехай $A_1, A_2, ..., A_n \in \mathcal{F}$ --- незалежні у сукупності.
Тоді з рівностей $\P\left(\bigcup\limits_{k=1}^n A_k\right) = 1 - \P\left(\overline{\bigcup\limits_{k=1}^n A_k}\right) = 1 - \P\left(\bigcap\limits_{k=1}^n \overline{A_k}\right)$
випливає формула
\begin{equation}
    \P\left(\bigcup\limits_{k=1}^n A_k\right) = 1 - \prod\limits_{k=1}^n \P(\overline{A_k})
\end{equation}
Зокрема, для двох незалежних подій $A$ і $B$: $\P(A\cup B) = 1 - \P(\overline{A}) \cdot \P(\overline{B})$.

\subsection{Формула повної ймовірності}
Розглядаємо деякий ймовірнісний простір $\left\{ \Omega, \mathcal{F}, \P\right\}$.

Нехай $H_1, H_2, ..., H_n \in \mathcal{F}$ (може бути й зліченна кількість) --- деяка повна група подій,
які називаємо \emph{гіпотезами (припущеннями)}, а подія $A$ відбувається разом з якоюсь гіпотезою. Тоді має місце
\emph{формула повної ймовірності}:
\begin{equation}\label{eq:total_prob}
    \P\left( A \right) = \sum\limits_{k=1}^n \P(H_k)\cdot \P(A/H_k)
\end{equation}
\begin{proof}
    $\P(A) = \P(A \cap \Omega) = \P\left(A \cap \left(\bigcup\limits_{k=1}^n H_k\right)\right) = \P\left(\bigcup\limits_{k=1}^n (A\cap H_k)\right) = \sum\limits_{k=1}^n \P(A\cap H_k) = \sum\limits_{k=1}^n \P(H_k)\cdot \P(A/H_k)$.
\end{proof}
У випадку зліченної кількості гіпотез формула \eqref{eq:total_prob} доводиться аналогічно та перетворюється
на $\P\left( A \right) = \sum\limits_{k=1}^{\infty} \P(H_k)\cdot \P(A/H_k)$.

\begin{example}
    \begin{enumerate}
        \item У магазин постачають $80\%$ телефонів з Китаю, $15\%$ з В'єтнаму та $5\%$ з Кореї,
        причому бракованих відповідно $1\%$, $0.1\%$ та $0.01\%$.
        Знайти ймовірність події $A = \left\{ \text{куплений телефон буде бракованим}\right\}$.

        За умовою введемо гіпотези $H_1 = \left\{ \text{телефон з Китаю}\right\}$,
        $H_2 = \left\{ \text{телефон з В'єтнаму}\right\}$ та $H_3 = \left\{ \text{телефон з Кореї}\right\}$.
        $\P(H_1) = 0.8$, $\P(H_2) = 0.15$, $\P(H_3) = 0.05$,
        $\P(A/H_1) = 0.01$, $\P(A/H_2) = 0.001$, $\P(A/H_3) = 0.0001$.
        Тоді за формулою \eqref{eq:total_prob} $\P(A) = 0.8\cdot 0.01 + 0.15\cdot 0.001 + 0.05\cdot 0.0001 = 0.008155$.
        \item Серед $N$ екзаменаційних білетів $n$ <<щасливих>>, $n<N$.
        У якого студента ймовірність витягнути <<щасливий>> білет більша --- у того, хто тягне першим, чи того, хто тягне другим?

        Позначимо через $A$ і $B$ події, що <<щасливий>> білет витягнув перший та другий студент відповідно. За умовою $\P(A) = \frac{n}{N}$.
        Щоб скористатися формулою \eqref{eq:total_prob} для обчислення $\P(B)$, введемо гіпотези $H_1$ = $A$ та $H_2$ = $\overline{A}$.
        $\P(H_1) = \frac{n}{N}$, $\P(H_2) = \frac{N-n}{N}$.
        \\ $\P(B) = \P(H_1)\cdot \P(B/H_1) + \P(H_2)\cdot \P(B/H_2) = \frac{n}{N}\cdot \frac{n-1}{N-1} + \frac{N-n}{N}\cdot \frac{n}{N-1} = 
        \frac{n^2 - n + N\cdot n - n^2}{N\cdot(N-1)} = \frac{n\cdot (N-1)}{N\cdot(N-1)} = \frac{n}{N}$.
    \end{enumerate}
\end{example}

\subsection{Формула Баєса}
Як і раніше, нехай $H_1, H_2, ..., H_n \in \mathcal{F}$ --- повна група подій деякого СЕ, які називаємо гіпотезами, причому
перед проведенням експерименту відомі їх \emph{апріорні} ймовірності $\P(H_1), \P(H_2), ..., \P(H_n)$.
В результаті проведення експерименту відбулась деяка подія $A$.
Постає питання: чому рівні \emph{апостеріорні} ймовірності $\P(H_1/A), \P(H_2/A), ..., \P(H_n/A)$?
Тобто, чому дорівнюють ймовірності, що мала місце кожна з гіпотез за умови, що подія $A$ відбулась? На це питання дає відповідь \emph{формула Баєса}:
\begin{equation}\label{eq:bayes}
    \P(H_i/A) = \frac{\P(H_i \cap A)}{\P(A)} = \frac{\P(H_i) \cdot \P(A/H_i)}{\sum\limits_{k=1}^n \P(H_k)\cdot \P(A/H_k)}, i = 1,...,n
\end{equation}
% \begin{remark}
%     В найпростішому випадку, коли гіпотези лише дві (деяка $H$ та протилежна їх $\overline{H}$), формулу \eqref{eq:bayes} можна записати так:
%     \begin{gather*}
%         \P(H / A) = \frac{
%             \P(H) \cdot \P(A / H)
%         }{
%             \P(H) \cdot \P(A / H) + \P(\overline{H}) \cdot \P(A / \overline{H})
%         }
%     \end{gather*}
% \end{remark}
\begin{example}
    Нехай на деяку хворобу хворіє 1\% людей. Тестування показує наявність цієї хвороби у 99\% дійсно хворих людей,
    і у 2\% тих, що насправді не хворіють. Яка ймовірність того, що людина дійсно хвора, якщо тест має позитивний результат?
    
    Для застосування формули Баєса \eqref{eq:bayes} введемо гіпотези $H = \left\{\text{людина хвора}\right\}$, $\overline{H} = \left\{\text{людина не хвора}\right\}$ та
    подію $A = \left\{\text{тест дав позитивний результат}\right\}$.
    З умови $\P(H) = 0.01$, $\P(\overline{H}) = 0.99$, $\P(A / H) = 0.99$ та $\P(A / \overline{H}) = 0.02$.
    За формулою \eqref{eq:bayes}:
    $\P(H / A) = \frac{
        \P(H) \cdot \P(A / H)
    }{
        \P(\overline{H}) \cdot \P(A / \overline{H})
    } = \frac{
        0.01 \cdot 0.99
    }{
        0.01 \cdot 0.99 + 0.99 \cdot 0.02
    } = \frac{1}{3}
    $. 
    Отже, лише в третині випадків позитивний результат тесту означає, що людина дійсно хвора. Цей приклад показує,
    що для розуміння того, як добре деяка перевірка визначає деяку рідкісну ознаку, важливо знати не лише 
    ймовірність, з якою ця перевірка підтвердить наявність ознаки, а й те, як часто ця перевірка дає позитивний результат в тих випадках,
    де його насправді немає.
\end{example}
\begin{example}
    В групі з 10 осіб троє вчаться на <<5>>, четверо на <<4>>, двоє на <<3>> та один на <<2>>.
    Викладач підготував на екзамен 20 питань. Студенти, які вчаться на <<5>>, знають відповіді на всі,
    на <<4>> --- на 16, на <<3>> --- на 10, на <<2>> --- на 5.
    Екзаменаційний білет містить 3 питання. Деякий студент відповів правильно на всі 3.
    Яка ймовірність того, що він вчиться на <<2>>?

    Введемо гіпотези $H_1 = \left\{ \text{студент вчиться на <<5>>}\right\}$, $H_2 = \left\{ \text{студент вчиться на <<4>>}\right\}$,
    $H_3 = \left\{ \text{студент вчиться на <<3>>}\right\}$, $H_4 = \left\{ \text{студент вчиться на <<2>>}\right\}$.
    За умовою $\P(H_1) = \frac{3}{10}$, $\P(H_2) = \frac{4}{10}$, $\P(H_3) = \frac{2}{10}$, $\P(H_4) = \frac{1}{10}$. 
    Позначимо $A = \left\{ \text{студент відповів на всі три питання}\right\}$. 
    Тоді $\P(A/H_1) = 1$, $\P(A/H_2) = \frac{16 \cdot 15 \cdot 14}{20 \cdot 19 \cdot 18}$, 
    $\P(A/H_3) = \frac{10 \cdot 9 \cdot 8}{20 \cdot 19 \cdot 18}$, 
    $\P(A/H_4) = \frac{5 \cdot 4 \cdot 3}{20 \cdot 19 \cdot 18}$.
    За формулою \eqref{eq:bayes} шукана ймовірність $\P(H_4/A) = \frac{\P(H_4) \cdot \P(A/H_4)}{\sum\limits_{k=1}^4 \P(H_k)\cdot \P(A/H_k)} =
    \frac{0.1 \cdot 5 \cdot 4 \cdot 3}{0.3 \cdot 20 \cdot 19 \cdot 18 + 0.4 \cdot 16 \cdot 15 \cdot 14 + 0.2 \cdot 10 \cdot 9 \cdot 8 + 0.1 \cdot 5 \cdot 4 \cdot 3}
     = \frac{60}{35460} = \frac{1}{591}$.

\end{example}
        % !TEX root = ../main.tex

\section{Поліноміальна модель ймовірності. Схема Бернуллі.}
Припустимо, що події $A_1, A_2, \dots, A_k$ утворюють повну групу подій деякого 
СЕ, причому відомі ймовірності $$P(A_i) = p_i,\;\;\;\; \sum_{i=1}^k p_i = 1$$
Експеримент проводиться n разів, в кожному з яких може відбутись одна з 
подій $A_i$. 
\begin{definition}
    Випробування, що проводяться, називаються \emph{незалежними}, якщо 
    $$P(A_{i_1}^{(1)} \cap A_{i_2}^{(2)} \cap \dots \cap A_{i_n}^{(n)}) = 
    P(A_{i_1}^{(1)}) P(A_{i_2}^{(2)}) \dots P(A_{i_n}^{(n)}),\;\;\;
    i=\overline{1,k}$$
\end{definition}

Основна задача - знайти ймовірність того, що подія $A_1$ відбудеться $m_1$ разів, 
подія $A_2$ відбудеться $m_2$ разів і т.д, подія $A_k$ відбудеться $m_k$ разів, при 
чому $\sum_{j=1}^k m_j = n$. Такі ймовірності будемо позначати як $P_n(m_1, m_2, \dots, m_n)$.
$$P(\underbrace{A_1 \cap \dots \cap A_1}_{m_1 \text{разів}} 
\cap \underbrace{ A_2 \cap \dots \cap A_2}_{m_1 \text{разів}}
\cap \dots \cap  \underbrace{A_k \cap \dots \cap A_k}_{m_1 \text{разів}})
= p_1^{m_1} p_2^{m_2} \dots p_k^{m_k}$$

Це є лише один із можливих варіантів появи подій в серії випробувань. Всього таких варіантів 
рівно $C_n^{m_1, m_2, \dots, m_k} = \frac{n!}{m_1!m_2! \dots m_k!}$. Таким чином отримуємо:
$$P_n(m_1, \dots, m_k) = \frac{n!}{m_1!m_2! \dots m_k!} p_1^{m_1} p_2^{m_2} \dots p_k^{m_k}$$
\begin{definition}
    Імовірності, що обчислюємо за даною формулою, називаються \emph{поліноміальними}, а 
    сама схема - \emph{поліноміальною схемою ймовірності}. 
\end{definition}
\begin{remark}
    $$1 = (p_1 + p_2 + \dots + p_k)^n = \sum_{m_1 + \dots + m_k = n} P_n(m_1, \dots, m_k)$$
\end{remark}
\begin{example}
    Студент ІПСА за рівнем підготовки з ймовірністю $0,3$ вважається слабким студентом, 
    з ймовірністю $0,5$ вважається середнім студентом та 
    з ймовірністю $0,2$ - середнім студентом. Яка ймовірність того, що з 6 навмання 
    обраних студентів кількість слабких та сильних буде однаковою?
\end{example}
    \chapter{Випадкові величини}
        % !TEX root = ../main.tex

\section{Поняття випадкової величини та її задання}
\begin{definition}
    \emph{Випадковою величиною} називається дійснозначна вимірна функція, що здійснює 
    відображення з $\Omega$ в $\mathbb{R}$, тобто $\xi = \xi(\omega): \Omega 
    \rightarrow \mathbb{R}$.
\end{definition}
\begin{remark}
    Вимірність функції $\xi(\omega)$ означає, що 
    \begin{equation}
        \forall x \in \mathbb{R}: 
        \left\{ \omega \in \Omega\; |\; \xi(\omega) < x\right\} \in \mathcal{F}
    \end{equation} В курсі 
    функціонального аналізу доводиться, що якщо виконується (1), то
    \begin{equation}
        \left.\begin{aligned}
            \{\omega\;|&\;\xi(\omega) > x\}\\
            \{\omega\;|&\;\xi(\omega) \leq x\}\\
            \{\omega\;|&\;\xi(\omega) \geq x\}\\
            \{\omega\;|&\;a < \xi(\omega) < b\}
        \end{aligned}\right\} \in \mathcal{F}
    \end{equation}
\end{remark}

Всього існують два види випадкових величин --- дискретні та неперервні.

\subsection{Дискретні випадкові величини та їх задання}
\begin{definition}
    Випадкова величина $\xi = \xi(\omega)$ називається 
    \emph{дискретною випадковою величиною} (ДВВ), якщо вона набуває скінченну або зліченну 
    кількість значень.
\end{definition}
Для задання ДВВ крім знання значень випадкової величини необхідно знати ймовірності, 
з якими ці значення приймаються.
$$p_i = P\left\{\omega: \xi(\omega) = x_i\right\} = P(\xi = x_i)$$
$$\sum_{i=1}^\infty p_i = 1$$
\begin{definition}
    \emph{Законом розподілу (імовірностей)} ДВВ називається співвідношення, яке вказує, 
    які ця випадкова величина приймає значення і з якими імовірностями.

    Закон розподілу ДВВ записується у вигляді \emph{ряду розподілу}:

    \begin{tabular}{c|c|c|c|c|c}
        $\xi$ & $x_1$ & $x_2$ & ... & $x_n$ & ... \\
        \hline
        $p$ & $p_1$ & $p_2$ & ... & $p_n$ & ...
    \end{tabular}
    \hspace{40pt}
    $x_1 < x_2 < ... < x_n < ...,\;\; \sum\limits_i p_i = 1$
\end{definition}
\begin{example}
    $\xi$ задає кількість влучень при чотирьох пострілах з імовірністю влучення 
    $p = \frac{1}{2}$. Скласти закон розподілу цієї випадкової величини.

    $P(\xi = k) = C_4^k \left(\frac{1}{2}\right)^k \left(\frac{1}{2}\right)^{4-k} = 
    C_4^k \left(\frac{1}{2}\right)^4 = \frac{C_4^k}{2^4}$

    \begin{tabular}{c|c|c|c|c|c}
        $\xi$ & 0 & 1 & 2 & 3 & 4 \\
        \hline
        $p$ & $\frac{1}{2^4}$ & $\frac{4}{2^4}$ & $\frac{6}{2^4}$ & $\frac{5}{2^4}$ & 
        $\frac{1}{2^4}$
    \end{tabular}
\end{example}
        % !TEX root = ../main.tex
\section{Числові характеристики випадкових величин}
Закон розподілу випадкової величини в будь-якому вигляді (ряд розподілу для ДВВ, щільність розподілу для НВВ, функція розподілу в загальному випадку)
повністю задає випадкову величину. Однак в прикладних задачах іноді достатньо мати лише деяке сумарне уявлення про деякі характерні невипадкові риси розподілу.

\subsection{Математичне сподівання}
\begin{definition}\index{математичне сподівання}
    \emph{Математичним сподіванням} $\E\xi$ випадкової величини називається значення
    інтеграла Стілтьєса
    \begin{equation}\label{eq:e_xi}
        \E\xi = \int\limits_{-\infty}^{+\infty} x dF_\xi(x) = \begin{cases}
            \sum\limits_{k=1}^{n(\infty)} x_k \P\left\{\xi = x_k\right\}, & \xi \text{ --- ДВВ} \\
            \int\limits_{-\infty}^{+\infty} x f_\xi(x)dx, & \xi \text{ --- НВВ}
        \end{cases}
    \end{equation}
\end{definition}
Інтеграл або ряд \eqref{eq:e_xi} має збігатися \emph{абсолютно}, інакше кажуть, що
випадкова величина не має математичного сподівання. Ця вимога пояснюється тим, що, наприклад, у випадку ДВВ зі зліченною множиною значень не має бути різниці,
в якому порядку додавати всі значення.
\begin{example}\index{розподіл!Коші}
    НВВ, розподілена за законом Коші з щільністю $f_\xi(x) = \frac{1}{\pi (1+x^2)}$
    не має математичного сподівання, бо інтеграл $\frac{1}{\pi}\int\limits_{-\infty}^{+\infty} \frac{x}{1+x^2}dx$ 
    розбіжний.
\end{example}
Математичне сподівання характеризує середнє значення випадкової величини. Наприклад,
якщо ДВВ набуває скінченну кількість значень з однаковими ймовірностями, то математичним сподіванням є просто середнім арифметичним усіх цих значень.
Фізична інтерпретація --- центр мас системи точок $\left\{x_1, x_2, ..., x_n,...\right\}$ з масами $\left\{p_1, p_2, ..., p_n, ...\right\}$ (ДВВ) або стрижня,
розподіл маси в якому задано функцією щільності (НВВ).

\vspace{0.5em}
\noindent \textbf{Властивості математичного сподівання:}
\begin{enumerate}
    \item Математичне сподівання константи --- сама константа, оскільки
    її можна інтерпретувати як ДВВ, що приймає єдине значення з ймовірністю $1$:
    $c = const, \E c = c$.
    \item $\E \left(c\cdot\xi\right) = c\cdot \E\xi$ --- це властивість рядів та інтегралів.
    \item $\E\left( \xi_1 + \xi_2\right) = \E\xi_1 + \E\xi_2$.
    \begin{proof}[Доведення для ДВВ]
        $\P\left\{\xi_1 = x_k\right\} = p_k$, $\P\left\{\xi_2 = y_j\right\} = p_j$, $\P\left\{\xi_1 + \xi_2 = x_k + y_j\right\} = p_{kj}$.
        В цих позначеннях
        $\E\left( \xi_1 + \xi_2\right) = \sum\limits_k \sum\limits_j (x_k+y_j) p_{kj} =
        \sum\limits_k x_k \sum\limits_j p_{kj} + \sum\limits_j y_j \sum\limits_k p_{kj} = \sum\limits_k x_k p_k + \sum\limits_j y_j p_j = \E\xi_1 + \E\xi_2$.
    \end{proof}
\suspend{enumerate}
\begin{definition}\index{випадкова величина!незалежні величини}
    Дві випадкові величини $\xi_1$ та $\xi_2$, задані на одному ймовірнісному просторі, називаються \emph{незалежними}, якщо
    події $A=\left\{\omega : \; \xi_1(\omega) < x\right\}$ та
    $B=\left\{\omega : \; \xi_2(\omega) < y\right\}$ є незалежними для всіх $x, y \in \mathbb{R}$.
\end{definition}
\resume{enumerate}
    \item Якщо $\xi_1$ та $\xi_2$ незалежні, то $\E\xi_1\xi_2 = \E\xi_1 \cdot \E\xi_2$.
    \begin{proof}[Доведення для ДВВ]
        Позначимо $p_{kj} = \P\left\{\xi_1 = x_k, \xi_2 = y_j\right\}$.
        
        $\E\xi_1\xi_2 = \sum\limits_k \sum\limits_j x_k y_j p_{kj} = \sum\limits_k \sum\limits_j x_k y_j p_k p_j = \left( \sum\limits_k x_k p_k\right)\cdot \left( \sum\limits_j y_j p_j\right) = \E\xi_1 \cdot \E\xi_2$.
    \end{proof}
\end{enumerate}
\begin{remark}
    Властивості 3 та 4 для НВВ буде доведено в темі <<Функції випадкових аргументів>> (ст. \pageref{proof:expectation}).
\end{remark}

\begin{example}
    \begin{enumerate}
        \item Обчислити математичне сподівання двох ДВВ:
        
        \begin{center}
            \begin{tabular}{c c}
                \begin{tabular}{|c|c|c|}
                    \hline
                    $\xi_1$ & $-1$ & $1$ \\ 
                    \hline
                    $p$ & $1/2$ & $1/2$ \\
                    \hline
                \end{tabular} &
                \begin{tabular}{|c|c|c|}
                    \hline
                    $\xi_2$ & $-100$ & $100$ \\ 
                    \hline
                    $p$ & $1/2$ & $1/2$ \\
                    \hline
                \end{tabular}
            \end{tabular}
        \end{center}
        $\E\xi_1 = -1\cdot \frac{1}{2} + 1\cdot \frac{1}{2} = 0, \E\xi_2 = -100\cdot \frac{1}{2} + 100\cdot \frac{1}{2} = 0$.
        
        Цей приклад показує, що попри однакове значення математичного сподівання, можливі значення цих ДВВ знаходяться на різній відстані від нього,
        тому знання математичного сподівання ще не дає повного уявлення про закон розподілу випадкової величини.        
        \item Обчислити математичне сподівання НВВ, розподіленої за законом Сімпсона.\index{розподіл!Сімпсона}

        \begin{tabular}{c c}
            \begin{tikzpicture}[baseline={(current bounding box.center)}, yscale=1]
                \pgfmathsetmacro{\a}{0.7}
                \draw [->] (-2, 0) -- (2, 0);
                \draw [->] (0, -0.1) -- (0, 1.7);
                \draw [ultra thick] (-2, 0) -- (-\a, 0);
                \draw [ultra thick] (\a, 0) -- (2, 0);
                \draw [ultra thick] (-\a, 0) -- (0, 1/\a);
                \draw [ultra thick] (\a, 0) -- (0, 1/\a);
                \node [below] at (2, 0) {$x$};
                \node [left] at (0, 1.6) {$f_\xi(x)$};
                \node [below] at (\a, 0) {$a$};
                \node [below] at (-\a, 0) {$-a$};
                \node [right] at (0, 1/\a) {$\frac{1}{a}$};
            \end{tikzpicture} &
            $f_\xi(x) = \begin{cases}
                \frac{1}{a} \left(1 - \frac{|x|}{a}\right), & |x| \leq a \\
                0, & |x| > a
            \end{cases} \;(a>0)$
        \end{tabular}
        
        $\E\xi = \int\limits_{-\infty}^{+\infty} x f_\xi(x)dx = \int\limits_{-a}^a \frac{x}{a}\left(1 - \frac{|x|}{a}\right)dx = 0$, 
        оскільки інтегрується непарна функція по симетричному проміжку.
    \end{enumerate}
\end{example}

\subsection{Дисперсія}
Як було показано, випадкові величини можуть мати однакові математичні сподівання, але відхилення значень цих випадкових величин
від нього може бути різним. Треба ввести ще одну числову характеристику,
що описуватиме це.
\begin{definition}\index{дисперсія}
    \emph{Дисперсією} випадкової величини називається 
    \begin{equation}\label{eq:d_xi}
        \D\xi = \E\left(\xi-\E\xi\right)^2 = \int\limits_{-\infty}^{+\infty} \left(x-\E\xi\right)^2 dF_\xi(x) = \begin{cases}
            \sum\limits_{k=1}^{n(\infty)} \left(x_k-\E\xi\right)^2 \P\left\{\xi = x_k\right\}, & \xi \text{ --- ДВВ} \\
            \int\limits_{-\infty}^{+\infty} \left(x-\E\xi\right)^2 f_\xi(x)dx, & \xi \text{ --- НВВ}
        \end{cases}
    \end{equation}
\end{definition}
Дисперсія --- характеристика розсіювання випадкової величини навколо свого математичного сподівання.
Фізична інтерпретація --- момент інерції маси системи точок або стрижня (як і у випадку інтерпретації математичного сподівання)
відносно свого центру мас.
\begin{remark}
    Дисперсія визначена тоді і тільки тоді, коли математичне сподівання існує та є скінченним.
    Оскільки в \eqref{eq:d_xi} і члени ряду, і підінтегральна функція є невід'ємними, то
    дисперсія може бути або скінченною, або рівною $+\infty$.
\end{remark}

\begin{example}
    \begin{enumerate}
        \item Обчислити дисперсії двох ДВВ:
        
        \begin{center}
            \begin{tabular}{c c}
                \begin{tabular}{|c|c|c|}
                    \hline
                    $\xi_1$ & $-1$ & $1$ \\ 
                    \hline
                    $p$ & $1/2$ & $1/2$ \\
                    \hline
                \end{tabular} &
                \begin{tabular}{|c|c|c|}
                    \hline
                    $\xi_2$ & $-100$ & $100$ \\ 
                    \hline
                    $p$ & $1/2$ & $1/2$ \\
                    \hline
                \end{tabular}
            \end{tabular}
        \end{center}
        
        $\D\xi_1 = (-1)^2\cdot \frac{1}{2} + 1^2\cdot \frac{1}{2} = 1, \D\xi_2 = (-100)^2\cdot \frac{1}{2} + 100^2\cdot \frac{1}{2} = 10000$.
        Тут вже видно різницю між двома випадковими величинами: значення $\xi_2$ розташовані далеко від математичного сподівання і тому дисперсія більша, ніж у $\xi_1$.
        \item Обчислити дисперсію НВВ, розподіленої за законом Сімпсона.
        
        Відповідна щільність розподілу має вигляд $f_\xi(x) = \begin{cases}
            \frac{1}{a} \left(1 - \frac{|x|}{a}\right), & |x| \leq a \\
            0, & |x| > a
        \end{cases}$.
        Математичне сподівання вже було обчислено, воно рівне 0. Тому $\D\xi = \int\limits_{-\infty}^{+\infty} \left(x-\E\xi\right)^2 f_\xi(x)dx =
        \int\limits_{-a}^a \frac{x^2}{a}\left(1 - \frac{|x|}{a}\right)dx = \frac{2}{a}\int\limits_0^a {x^2}\left(1 - \frac{x}{a}\right)dx =
        \frac{2}{a}\cdot \left( \frac{x^3}{3} - \frac{x^4}{4a}\right)\Bigr\vert_0^a = \frac{2}{a} \cdot \left( \frac{a^3}{3} - \frac{a^3}{4}\right) = \frac{a^2}{6}$.
    \end{enumerate}
\end{example}

\noindent \textbf{Властивості дисперсії:}
\begin{enumerate}\index{середньоквадратичне відхилення}
    \item $\D\xi \geq 0$, $\sqrt{\D\xi} = \sigma_\xi$ --- \emph{середньоквадратичне (стандартне) відхилення}.
    \item $\D\xi = 0 \Leftrightarrow \xi = const$.
    \begin{proof}
        В одну сторону: $\D c = \E(c - \E c)^2 = \E(c - c)^2 = 0$. В іншу: якщо $\xi$ --- ДВВ,
        то з $\D\xi = 0$ випливає, що $\left(x_k-\E\xi\right)^2 = 0$ для всіх $k$,
        тому для всіх $k$ $x_k = \E \xi$ --- $\xi$ приймає лише одну значення.
        Для НВВ аналогічно випливає, що $\left(x-\E\xi\right)^2 f_\xi(x) \equiv 0$. Оскільки $\left(x-\E\xi\right)^2$
        рівне $0$ лише в точці $x=\E\xi$, то $f_\xi(x)$ може бути відмінним від 0 лише в цій точці. Але тоді $f_\xi(x)$ не може бути щільністю,
        тому що площа під кривою розподілу має бути рівною 1. Це свідчить про дискретність $\xi$,
        і тому за вже доведеним $\xi$ приймає лише одне значення.
    \end{proof}
    \item Для обчислення дисперсії більш зручною є наступна формула:

    $\D\xi = \E(\xi^2 - 2\xi \E\xi +(\E\xi)^2) = \E\xi^2 - 2(\E\xi)^2 + (\E\xi)^2 = \E\xi^2 - (\E\xi)^2$. 
    \item Для сталої $c$: $\D(c\cdot \xi) = c^2\cdot \D\xi$, оскільки $\D(c\cdot \xi) = \E\left(c\cdot\xi-\E(c\cdot\xi\right))^2 = \E\left(c\cdot\xi - c\cdot \E\xi\right)^2 = c^2 \cdot \E\left(\xi-\E\xi\right)^2 = c^2 \cdot \D\xi$.
    \item Для сталої $c$: $\D(\xi + c) = \E(\xi + c - \E(\xi + c))^2 = \E(\xi + c - \E\xi - \E c) = \E(\xi - \E\xi)^2 = \D\xi$.
\suspend{enumerate}
\begin{definition}\index{випадкова величина!центрована}
    \emph{Центрованою} випадковою величиною, що відповідає випадковій величині $\xi$, називається 
    випадкова величина $\mathring{\xi} = \xi - \E\xi$. Для неї $\E\mathring{\xi} = 0$ та 
    $\E\mathring{\xi^2} = \D\xi$.
\end{definition}
\resume{enumerate}
    \item Якщо випадкові величини $\xi_1$ та $\xi_2$ --- незалежні, то
    $\D\left(\xi_1 + \xi_2\right) = \D\xi_1 + \D\xi_2$.
    \begin{proof}
        $\D\left(\xi_1 + \xi_2\right) = \E((\xi_1 + \xi_2) - \E(\xi_1 + \xi_2))^2 = \E(\mathring{\xi}_1 + \mathring{\xi}_2)^2 =
        \E\mathring{\xi}_1^2 + 2\E\mathring{\xi}_1\mathring{\xi}_2 + \E\mathring{\xi}_2^2 = 
        \E\mathring{\xi}_1^2 + 2\E\mathring{\xi}_1 \E\mathring{\xi}_2 + \E\mathring{\xi}_2^2 =
        \E\mathring{\xi}_1^2 + \E\mathring{\xi}_2^2 = \D\xi_1 + \D\xi_2$.
    \end{proof}
\end{enumerate}

\subsection{Моменти випадкової величини}
\begin{definition}\index{момент!початковий}
    \emph{Початковим моментом k-того порядку} ($k \in \mathbb{N}$) ВВ $\xi$ називається
    математичне сподівання $k$-того степеня $\xi$:
    \begin{equation}\label{eq:e_alpha_k}
        \alpha_k = \E\xi^k = \int\limits_{-\infty}^{+\infty} x^k dF_\xi(x) = \begin{cases}
            \sum\limits_{m=1}^{n(\infty)} x_m^k \P\left\{\xi = x_m\right\}, & \xi \text{ --- ДВВ} \\
            \int\limits_{-\infty}^{+\infty} x^k f_\xi(x)dx, & \xi \text{ --- НВВ}
        \end{cases}
    \end{equation}
\end{definition}
\begin{definition}\index{момент!центральний}
    \emph{Центральним моментом k-того порядку} ($k \in \mathbb{N}$) ВВ $\xi$ називається
    \begin{equation}\label{eq:d_beta_k}
        \beta_k = \E\mathring{\xi}^k = \E\left(\xi-\E\xi\right)^k = \int\limits_{-\infty}^{+\infty} \left(x-\E\xi\right)^k dF_\xi(x) = \begin{cases}
            \sum\limits_{m=1}^{n(\infty)} \left(x_m-\E\xi\right)^k \P\left\{\xi = x_m\right\}, & \xi \text{ --- ДВВ} \\
            \int\limits_{-\infty}^{+\infty} \left(x-\E\xi\right)^k f_\xi(x)dx, & \xi \text{ --- НВВ}
        \end{cases}
    \end{equation}
\end{definition}
Математичне сподівання є початковим моментом першого порядку, а дисперсія --- центральним моментом другого порядку.
Узагальненням формули для дисперсії
$\D\xi = \beta_2 = \E\xi^2 - (\E\xi)^2 = \alpha_2 - \alpha_1^2$ є наступна формула,
що дозволяє виразити будь-який центральний момент через початкові моменти:
\begin{equation}
    \beta_k = \E\left(\xi-\E\xi\right)^k = \E\left( \sum\limits_{j=0}^k C_k^j \xi^j (-1)^{k-j} (\E\xi)^{k-j}\right) =
\sum\limits_{j=0}^k (-1)^{k-j} C_k^j \E\xi^j (\E\xi)^{k-j}
\end{equation}
\begin{exercise}
    Виразити через початкові моменти $\beta_3$ та $\beta_4$.
\end{exercise}
Розглядаються також \emph{абсолютні початкові моменти}\index{момент!початковий!абсолютний} $\E\vert \xi \vert ^k$, 
\emph{абсолютні центральні моменти}\index{момент!центральний!абсолютний} $\E\vert \xi - \E\xi \vert ^k$
та \emph{факторіальні моменти}\index{момент!факторіальний} $\gamma_k = \E\left( \xi (\xi - 1) ... (\xi - k + 1)\right)$.
\begin{remark}
    Якщо за змістом задачі випадкова величина має якісь одиниці вимірювання (наприклад, метри чи кілограми),
    то моменти $k$-того порядку мають відповідний порядок одиниць вимірювання.
    Наприклад, якщо деяка випадкова величина задає похибку вимірювання в міліметрах,
    то математичне сподівання теж буде у міліметрах, а дисперсія --- в квадратних міліметрах
    (але середньоквадратичне відхилення --- просто у міліметрах).
\end{remark}


\subsection{Мода та медіана випадкової величини}
\begin{definition}\index{мода}
    \emph{Модою} $\Mo\xi$ називається абсциса точки максимуму щільності 
    розподілу у випадку НВВ та значення, ймовірність 
    появи якого є найбільшим, у випадку ДВВ.

    \emph{Унімодальний закон} --- такий, що має лише одну моду. 
    \emph{Полімодальний закон} --- такий, що має декілька мод.
    \emph{Антимодальний закон} --- такий, що не має моди.
\end{definition}
\begin{definition}\index{медіана}
    \emph{Медіаною} $\Me\xi$ НВВ $\xi$ називається така точка $x_0$, для якої 
    $\P\left\{\xi < x_0\right\} = \P\left\{\xi \geq x_0\right\} 
    = \frac{1}{2} \Leftrightarrow F_\xi(x_0) = \frac{1}{2}$.
    Для ДВВ $\xi$ медіаною є така точка $x_0$, для якої одночасно
    виконуються нерівності $\P\left\{\xi < x_0\right\} \geq \frac{1}{2}$ та
    $\P\left\{\xi \geq x_0\right\} \geq \frac{1}{2}$.
    Медіана є окремим випадком \emph{квантиля}.
\end{definition}
\begin{definition}\index{квантиль}
    Точка $x_0$ називається \emph{квантилем q-го порядку}, якщо $F_\xi(x_0) = q$ 
    для НВВ $\xi$ або $F_\xi(x_0) \geq q$ та $F_\xi(x_0) \geq 1-q$ для ДВВ.
\end{definition}
\begin{example}
    Знайти моду та медіану НВВ, розподіленої за законом Релея.\index{розподіл!Релея}
    
    \begin{tabular}{c c}
        \begin{tikzpicture}[baseline={(current bounding box.center)}, yscale=2]
            \pgfmathsetmacro{\s}{1}
            \draw [->] (-0.5, 0) -- (5, 0);
            \draw [->] (0, -0.1) -- (0, 1);
            \node [below] at (5, 0) {$x$};
            \node [below left] at (0, 1) {$f_\xi(x)$};
            \draw [domain=0:5, smooth, variable = \x, ultra thick] plot ({\x}, {((\x/(\s^2)) * exp(-(\x)^2/(2*\s^2))});
            \draw [ultra thick] (-0.5, 0) -- (0, 0);
        \end{tikzpicture} &
        $f_\xi(x) = \begin{cases}
            \frac{x}{\sigma^2} e^{-\frac{x^2}{2\sigma^2}}, & x \geq 0 \\
            0, & x < 0
        \end{cases} (\sigma > 0)$ 
    \end{tabular}

    Для визначення моди знайдемо максимум $f_\xi(x)$ при $x\geq 0$.
    $f'_\xi(x) = \frac{\sigma^2 - x^2}{\sigma^4} e^{-\frac{x^2}{2\sigma^2}} = 0$ при $x=\sigma$.
    $f''_\xi(x) = \left( -\frac{3x}{\sigma^4} + \frac{x^3}{\sigma^6}\right) e^{-\frac{x^2}{2\sigma^2}}$,
    $f''_\xi(\sigma) = -\frac{2}{\sigma^3} e^{-\frac{1}{2}} < 0$. Отже, в точці $x=\sigma$ дійсно максимум $f_\xi(x)$, тому $\Mo\xi = \sigma$.

    Медіану знайдемо з рівності $\P\left\{\xi < x_0\right\} = 0.5$. $\P\left\{\xi < x_0\right\} = \int\limits_{-\infty}^{x_0}f_\xi(x)dx = 
    \int\limits_0^{x_0} \frac{x}{\sigma^2} e^{-\frac{x^2}{2\sigma^2}} dx = 1 - e^{-\frac{x_0^2}{2\sigma^2}}$.
    З рівняння $1 - e^{-\frac{x_0^2}{2\sigma^2}} = 0.5$ знаходимо $x_0 = \Me\xi = \sigma \sqrt{2\ln{2}}$.
\end{example}

\subsection{Асиметрія та ексцес випадкової величини}
\begin{definition}\index{асиметрія}
    \emph{Асиметрією} випадкової величини $\As\xi$ називається безрозмірна 
    числова характеристика, що дорівнює $\frac{\beta_3}{\sigma_\xi^3} = 
    \frac{\E(\xi - \E\xi)^3}{(\D\xi)^{3/2}}$. 
    Ця характеристика показує порушення чи наявність симетрії кривої розподілу відносно математичного сподівання.
\end{definition}
\begin{center}
    \begin{tabular}{c c c}
        \begin{tikzpicture}[yscale = 1.5]
            \draw [->] (-0.5, 0) -- (3.8, 0);
            \draw [->] (0, -0.1) -- (0, 1);
            \draw [domain=-0.5:3.7, smooth, variable = \x, ultra thick] plot ({\x}, {0.797884560803 * exp(-2*(\x-1)^2)});
            \draw [dashed] (1, 0) -- (1, 0.797884560803);
            \node [below] at (1, 0) {$\E\xi$};
        \end{tikzpicture} &
        \begin{tikzpicture}[yscale = 1.5]
            \draw [->] (-0.5, 0) -- (3.2, 0);
            \draw [->] (0, -0.1) -- (0, 1);
            \draw [ultra thick] (-0.5, 0) -- (0, 0);
            \draw [ultra thick] (3, 0) -- (3.1, 0);
            \draw [domain=0:3, smooth, variable = \x, ultra thick] plot ({\x}, {10 * (\x/3)^4 * (1-(\x/3))});
            \draw [dashed] (2.1428, 0) -- (2.1428, 0.744);
            \node [below] at (2.1428, 0) {$\E\xi$};
        \end{tikzpicture} &
        \begin{tikzpicture}[yscale = 1.5]
            \draw [->] (-0.5, 0) -- (3.2, 0);
            \draw [->] (0, -0.1) -- (0, 1);
            \draw [ultra thick] (-0.5, 0) -- (0, 0);
            \draw [domain=0:3, smooth, variable = \x, ultra thick] plot ({\x}, {10 * (\x/3) * (1-(\x/3))^4});
            \draw [dashed] (0.8571, 0) -- (0.8571, 0.744);
            \node [below] at (0.8571, 0) {$\E\xi$};
        \end{tikzpicture} \\
        $\As\xi = 0$ & $\As\xi < 0$ & $\As\xi > 0$ 
    \end{tabular}
\end{center}

\begin{definition}\index{ексцес}
    \emph{Ексцесом} випадкової величини $\Ex\xi$ називається безрозмірна 
    числова характеристика, що дорівнює $\Ex\xi = \frac{\beta_4}{\sigma_\xi^4} - 3 = 
    \frac{\E(\xi - \E\xi)^4}{(\D\xi)^{2}} - 3$.
    Ця характеристика показує, наскільки швидко крива розподілу 
    прямує до точки максимуму.
\end{definition}
\begin{center}
    \begin{tikzpicture}[yscale = 2]
        \draw [->] (-1, 0) -- (7, 0);
        \draw [->] (0, -0.1) -- (0, 1);
        \draw [domain=-1:7, smooth, variable = \x, ultra thick] plot ({\x}, {0.398942280401 * exp(-(\x-3)^2/2)});
        \draw [domain=-1:7, smooth, variable = \x, ultra thick] plot ({\x}, {0.797884560803 * exp(-2*(\x-3)^2)});
        \draw [domain=-1:7, smooth, variable = \x, ultra thick] plot ({\x}, {0.199471140201 * exp(-(\x-3)^2/8)});
        \draw [->] (2, 0.7) -- (2.4, 0.6);
        \node [left] at (2, 0.7) {$\Ex\xi > 0$};
        \draw [->] (3.7, 0.7) -- (3, 0.45);
        \node [right] at (3.7, 0.7) {$\Ex\xi = 0$};
        \draw [->] (5, 0.4) -- (3, 0.25);
        \node [right] at (5, 0.4) {$\Ex\xi < 0$};
    \end{tikzpicture}
\end{center}

\subsection{Генератриса (твірна функція) ДВВ}
\begin{definition}\index{генератриса (твірна функція)}
    Нехай $\xi$ --- ДВВ, що приймає цілі невід'ємні значення. 
    \emph{Генератрисою (твірною функцією)} цієї ДВВ називається функція комплексного аргументу
    \begin{equation}\label{eq:gen_func}
        G_\xi(z) = \sum\limits_{k=0}^{\infty} \P\left\{\xi = k\right\} z^k
    \end{equation}
\end{definition}
\noindent \textbf{Властивості генератриси:}
\begin{enumerate}
    \item Відповідний ряд рівномірно збігається принаймні в колі $|z|\leq 1$.
    \begin{proof}
        Випливає з того, що $\left| \sum\limits_{k=0}^{\infty} \P\left\{\xi = k\right\} z^k \right| \leq \sum\limits_{k=0}^{\infty} \P\left\{\xi = k\right\} = 1$.
    \end{proof}
    \item Генератриса --- аналітична функція в колі $|z|\leq 1$ (як наслідок властивості 1).
    \item За генератрисою можна відновити розподіл $\xi$:
    $\P\left\{\xi = k\right\} = \frac{1}{k!}\cdot G_\xi^{(k)}(0)$.
    \item За допомогою генератриси можна знайти факторіальні моменти $\xi$.
    \begin{proof}
        За означенням факторіальний момент $\gamma_m = \E\left( \xi (\xi - 1) ... (\xi - m + 1)\right)$.
        Для ДВВ, що розглядаються, він рівний $\sum\limits_{k=0}^{\infty} k(k-1)(k-2)...(k-m+1) \P\left\{\xi = k\right\}$.
        З іншого боку, $G'_\xi(z) = \sum\limits_{k=0}^{\infty} k \P\left\{\xi = k\right\} z^{k-1}$,
        $G''_\xi(z) = \sum\limits_{k=0}^{\infty} k(k-1) \P\left\{\xi = k\right\} z^{k-2}$ і так далі,
        $G^{(m)}_\xi(z) = \sum\limits_{k=0}^{\infty} k(k-1)(k-2)...(k-m+1) \P\left\{\xi = k\right\} z^{k-m}$.
        Отже, $\gamma_m = G^{(m)}_\xi(1)$.
    \end{proof}
\end{enumerate}
Як наслідок властивості 4, можна також знайти математичне сподівання та дисперсію.
Факторіальний та початковий моменти першого порядку збігаються, тому $\E\xi = G'_\xi(1)$.
Дисперсію знайдемо за допомогою другого факторіального моменту: $\gamma_2 = \E\left( \xi (\xi - 1)\right) = \E\xi^2 - \E\xi = G''_\xi(1)$,
звідки $\E\xi^2 = G''_\xi(1) + G'_\xi(1)$. 
Отже, $\D\xi = \E\xi^2 - (\E\xi)^2 = G''_\xi(1) + G'_\xi(1) - \left( G'_\xi(1)\right)^2$.
\begin{exercise}
    Виразити через похідні генератриси центральні моменти 3-го та 4-го порядків.
\end{exercise}
        % !TEX root = ../main.tex
\section{Деякі закони розподілу випадкових величин}

\subsection{Біноміальний розподіл}
\noindent\textbf{Означення:}
    ДВВ $\xi$ розподілена за \emph{біноміальним законом}, 
    якщо набуває значень $0,1,...,n$ з ймовірностями \begin{equation}
        \P\left\{\xi = k\right\} = C_n^k p^k q^{n-k}, q = 1 - p
    \end{equation}
\textbf{Коротке позначення:} $\xi \sim \mathrm{Bin}(n, p)$.
    $n$ і $p$ --- параметри закону, $n\in \mathbb{N}$, $p\in (0;1)$.

Окремим випадком біноміального закону є \emph{розподіл Бернуллі}: $\xi \sim \mathrm{Bin}(1, p)$.
За законом Бернуллі розподілені випадкові величини-індикатори $\mathds{1}_A = \begin{cases}
    0, & \text{подія }A\text{ не відбулась}\\ 1, & \text{подія }A\text{ відбулась}
\end{cases}$ при $\P(A) = p$.

\noindent\textbf{Ряд розподілу:}
\begin{center}
    \begin{tabular}{|c|c|c|c|c|c|c|}
        \hline
        $\xi$ & $0$ & $1$ & $2$ & $...$ & $n-1$ & $n$ \\
        \hline
        $p$ & $q^n$ & $C_n^1 pq^{n-1}$ & $C_n^2 p^2 q^{n-2}$ & $...$ & $C_n^{n-1}p^{n-1}q$ & $p^n$ \\
        \hline
    \end{tabular}
\end{center}

\noindent\textbf{Полігон розподілу:} приклад для $n=5$, $p=0.6$.
\begin{center}
    \begin{tikzpicture}
        \pgfmathsetmacro{\s}{5};
        \pgfmathsetmacro{\p}{0.6};
        \pgfmathsetmacro{\q}{1-\p};
        \pgfmathsetmacro{\n}{5};
        \draw [->] (-1,0) -- (\n+1, 0);
        \draw [->] (0, -0.1*\s) -- (0, 0.4*\s);
        \draw [thick] (0,\q^\n*\s) -- (1, {\n*\p*\q^(\n-1)*\s}) -- (2, {(\n*(\n-1)/2)*\p^2*\q^(\n-2)*\s});
        \draw [thick] (2, {(\n*(\n-1)/2)*\p^2*\q^(\n-2)*\s}) -- (3, {(\n*(\n-1)*(\n-2)/6)*\p^3*\q^(\n-3)*\s});
        \draw [thick] (3, {(\n*(\n-1)*(\n-2)/6)*\p^3*\q^(\n-3)*\s}) -- (4, {\n*\p^4*\q^(\n-4)*\s});
        \draw [thick] (4, {\n*\p^4*\q^(\n-4)*\s}) -- (5, \p^\n*\s);
        \draw [fill] (0,\q^\n) circle [radius = 0.05];
        \draw [fill] (1, {\n*\p*\q^(\n-1)*\s}) circle [radius = 0.05];
        \draw [fill] (2, {(\n*(\n-1)/2)*\p^2*\q^(\n-2)*\s}) circle [radius = 0.05];
        \draw [fill] (3, {(\n*(\n-1)*(\n-2)/6)*\p^3*\q^(\n-3)*\s}) circle [radius = 0.05];
        \draw [fill] (4, {\n*\p^4*\q^(\n-4)*\s}) circle [radius = 0.05];
        \draw [fill] (5, \p^\n*\s) circle [radius = 0.05];
        \node [below left] at (0, 0) {0};
        \foreach \k in {1,...,\n}:
            \node [below] at (\k, 0) {\k};
        \node [below] at (\n+1, 0) {$x$};
        \node [left] at (0, 0.4*\s) {$p$};
    \end{tikzpicture}
\end{center}

\noindent\textbf{Функція розподілу:} приклад для $n=5$, $p=0.6$.
\begin{center}
    \begin{tabular}{c c}
        $
            F_\xi(x) = \begin{cases}
                0, & x \leq 0 \\
                q^n, & 0 < x \leq 1 \\
                q^n + npq^{n-1}, & 1 < x \leq 2 \\
                \dots \\
                1, & x > n
            \end{cases}
        $ &
        \begin{tikzpicture}[baseline={(current bounding box.center)}, yscale=2.5]
            \pgfmathsetmacro{\p}{0.6};
            \pgfmathsetmacro{\q}{1-\p};
            \pgfmathsetmacro{\n}{5};
            \draw [->] (-1,0) -- (\n+1, 0);
            \draw [->] (0, -0.1) -- (0, 1.2);
            \draw [ultra thick] (-1, 0) -- (0,0);
            \draw [ultra thick] [<-] (0,\q^\n) -- (1, \q^\n);
            \draw [ultra thick] [<-] (1, {\q^\n + \n*\p*\q^(\n-1)}) -- (2, {\q^\n + \n*\p*\q^(\n-1)});
            \draw [ultra thick] [<-] (2, {\q^\n + \n*\p*\q^(\n-1) + (\n*(\n-1)/2)*\p^2*\q^(\n-2)}) -- (3, {\q^\n + \n*\p*\q^(\n-1) + (\n*(\n-1)/2)*\p^2*\q^(\n-2)});
            \draw [ultra thick] [<-] (3, {1 - (\q^\n + \n*\p*\q^(\n-1) + (\n*(\n-1)/2)*\p^2*\q^(\n-2))}) -- (4, {1 - (\q^\n + \n*\p*\q^(\n-1) + (\n*(\n-1)/2)*\p^2*\q^(\n-2))});
            \draw [ultra thick] [<-] (4, {1 - (\q^\n + \n*\p*\q^(\n-1))}) -- (5, {1 - (\q^\n + \n*\p*\q^(\n-1))});
            \draw [ultra thick] [<-] (5, 1) -- (6, 1);
            \draw [dashed] (1, 0) -- (1, {\q^\n + \n*\p*\q^(\n-1)});
            \draw [dashed] (2, 0) -- (2, {\q^\n + \n*\p*\q^(\n-1) + (\n*(\n-1)/2)*\p^2*\q^(\n-2)});
            \draw [dashed] (3, 0) -- (3, {1 - (\q^\n + \n*\p*\q^(\n-1) + (\n*(\n-1)/2)*\p^2*\q^(\n-2))});
            \draw [dashed] (4, 0) -- (4, {1 - (\q^\n + \n*\p*\q^(\n-1))});
            \draw [dashed] (5, 0) -- (5, 1);
            \node [below left] at (0, 0) {0};
            \foreach \k in {1,...,\n}:
                \node [below] at (\k, 0) {\k};
            \draw [dashed] (0, 1) -- (\n, 1);
            \node [left] at (0, 1) {1};
            % \node [right] [align=center] at (3.5, 0.2) {Приклад для \\ $n = \n, p = \p$};
            \node [below] at (\n+1, 0) {$x$};
            \node [left] at (0, 1.2) {$F_\xi(x)$};
        \end{tikzpicture}
    \end{tabular}
\end{center}

Для дослідження числових характеристик скористаємося генератрисою розподілу:

$G_\xi(z) = \sum\limits_{k=0}^{n} \P\left\{\xi = k\right\} z^k = \sum\limits_{k=0}^{n} C_n^k p^k q^{n-k} z^k = (pz+q)^n$,
звідки 
$G'_\xi(z) = np(pz+q)^{n-1}$, $G''_\xi(z) = n(n-1)p^2(pz+q)^{n-2}$.
Тому 
$G'_\xi(1) = np(p+q)^{n-1} = np$, $G''_\xi(1) = n(n-1)p^2 = n^2p^2 - np^2$, $G''_\xi(1) + G'_\xi(1) - \left( G'_\xi(1)\right)^2 = n^2p^2 - np^2 + np - n^2p^2 = np(1-p) = npq$.
Отже, знайдено значення математичного сподівання та дисперсії.

\begin{samepage}
\noindent\textbf{Числові характеристики:}
\begin{enumerate}
    \item $\E\xi = np$.
    \item $\D\xi = npq$, $\sigma_\xi = \sqrt{npq}$.
    \item $\Mo\xi = \begin{cases}
        \left[np+p\right], & \text{якщо } np+p \text{ не ціле}\\
        np+p, np-q, & \text{якщо } np+p \text{ ціле}
    \end{cases}$ --- як найбільш ймовірна кількість успіхів у схемі Бернуллі.
    \item $\Me\xi$ --- одне зі значень $\left[np\right] - 1$, $\left[np\right]$, $\left[np\right] + 1$.
\end{enumerate}
\end{samepage}

\noindent\textbf{Застосування:} якщо проводиться $n$ незалежних випробувань з ймовірністю успіху $p$, 
то $\xi \sim \mathrm{Bin}(n, p)$ задає кількість успіхів.

\subsection{Геометричний розподіл}
\noindent\textbf{Означення:}
    ДВВ $\xi$ розподілена за \emph{геометричним законом}, 
    якщо набуває значень $1,2,3,..$ з ймовірностями \begin{equation}
        \P\left\{\xi = k\right\} = pq^{k-1}, q = 1 - p
    \end{equation}
    \textbf{Коротке позначення:} $\xi \sim \mathrm{Geom}(p)$.
    $p$ --- параметр закону, $p\in (0;1)$.    

\noindent\textbf{Ряд розподілу:}
\begin{center}
    \begin{tabular}{|c|c|c|c|c|c|c|}
        \hline
        $\xi$ & $1$ & $2$ & $3$ & $...$ & $k$ & $...$ \\
        \hline
        $p$ & $p$ & $pq$ & $pq^2$ & $...$ & $pq^{k-1}$ & $...$ \\
        \hline
    \end{tabular}
\end{center}

\noindent\textbf{Полігон розподілу:} приклад для $p=0.5$.
\begin{center}
    \begin{tikzpicture}[baseline={(current bounding box.center)}]
        \pgfmathsetmacro{\s}{4};
        \pgfmathsetmacro{\p}{0.5};
        \pgfmathsetmacro{\q}{1-\p};
        \pgfmathsetmacro{\n}{5};
        \draw [->] (-0.3,0) -- (\n+1, 0);
        \draw [->] (0, -0.1*\s) -- (0, 0.6*\s);
        \foreach \k in {1,...,\n}:
            \draw [thick] (\k, {\p*\q^(\k - 1)*\s}) -- (\k+1, {\p*\q^(\k)*\s});
        \foreach \k in {1,...,\n}:
            \draw [fill] (\k, {\p*\q^(\k - 1)*\s}) circle [radius = 0.05];
        \node [below left] at (0, 0) {0};
        \foreach \k in {1,...,\n}:
            \node [below] at (\k, 0) {\k};
        \node [below] at (\n+1, 0) {$x$};
        \node [left] at (0, 0.6*\s) {$p$};
    \end{tikzpicture}
\end{center}

\noindent\textbf{Функція розподілу:} приклад для $p=0.5$.

\begin{tabular}{c c}
    $
        F_\xi(x) = \begin{cases}
            0, & x \leq 1 \\
            p, & 1 < x \leq 2 \\
            p+pq, & 2 < x \leq 3 \\
            \dots \\
            \sum\limits_{m=0}^{k-1}pq^m = 1-q^k, & k < x \leq k+1 \\
            \dots
        \end{cases}
    $ &
    \begin{tikzpicture}[baseline={(current bounding box.center)}, yscale=2.5, xscale=0.88]
        \pgfmathsetmacro{\p}{0.5};
        \pgfmathsetmacro{\q}{1-\p};
        \pgfmathsetmacro{\n}{5};
        \draw [->] (-0.3,0) -- (\n+1, 0);
        \draw [->] (0, -0.1) -- (0, 1.2);
        \draw [ultra thick] (-0.3, 0) -- (1,0);
        \foreach \k in {1,...,\n}:
            \draw [ultra thick] [<-] (\k, 1-\q^\k) -- (\k+1, 1-\q^\k);
        \node [below left] at (0, 0) {0};
        \foreach \k in {1,...,\n}:
            \node [below] at (\k, 0) {\k};
        \draw [dashed] (0, 1) -- (\n+1, 1);
        \node [left] at (0, 1) {1};
        \node [below] at (\n+1, 0) {$x$};
        \node [left] at (0, 1.2) {$F_\xi(x)$};
    \end{tikzpicture}
\end{tabular}

Для дослідження числових характеристик скористаємося генератрисою розподілу:

$G_\xi(z) = \sum\limits_{k=1}^{\infty} \P\left\{\xi = k\right\} z^k = \sum\limits_{k=1}^{\infty} pq^{k-1} z^k = \frac{pz}{1-qz}$, якщо $\left| qz\right|<1$, звідки
$G'_\xi(z) = \frac{p}{(1-qz)^2}$, $G''_\xi(z) = \frac{2pq}{(1-qz)^3}$. Тому
$G'_\xi(1) = \frac{1}{p}$, $G''_\xi(1) = \frac{2q}{p^2}$, $G''_\xi(1) + G'_\xi(1) - \left( G'_\xi(1)\right)^2 = \frac{2q}{p^2} + \frac{1}{p} - \frac{1}{p^2} = \frac{2q+p-1}{p^2} = \frac{q}{p^2}$.
Отже, знайдено значення математичного сподівання та дисперсії.

\noindent\textbf{Числові характеристики:}
\begin{enumerate}
    \item $\E\xi = \frac{1}{p}$.
    \item $\D\xi = \frac{q}{p^2}$, $\sigma_\xi = \frac{\sqrt{q}}{p}$.
    \item $\Mo\xi = 1$.
    \item $\Me\xi = \left[ \frac{-1}{\log_2(1-p)}\right]$.
\end{enumerate}

\noindent\textbf{Застосування:} якщо проводяться незалежні випробування з ймовірністю успіху $p$ до першого успішного,
    то $\xi$, що задає кількість проведених випробувань, має розподіл $\mathrm{Geom}(p)$.

\begin{exercise}
    Записати ряд розподілу, функцію розподілу, генетратрису та обчислити
    числові характеристики для іншого означення геометричного розподілу, 
    де $\xi$ приймає значення $0,1,2,...$ з ймовірностями $\P\left\{\xi = k\right\} = pq^k$.
\end{exercise}
Геометричний розподіл \textbf{<<не має пам'яті>>}: $\P\left\{\xi = n+m / \xi \geq n\right\} = \P\left\{\xi = m\right\}$.
Це означає, що кількість минулих <<невдач>> не впливає на кількість майбутніх <<невдач>>.
\begin{exercise}
    Довести цю властивість і те, що геометричний розподіл --- 
    єдиний дискретний розподіл, що <<не має пам'яті>>.
\end{exercise}
До геометричного закону можна звести \emph{закон розподілу Паскаля} $\xi \sim \mathrm{Pas}(a)$,
що задається $\P\left\{\xi = k\right\} = \frac{a^k}{(1+a)^{k+1}}, k = 0,1,2,...$ для $a>0$,
заміною $p=\frac{1}{a+1}$. Для нього $\E\xi = a$, $\D\xi = a + a^2$.

\subsection{Розподіл Пуассона}
\noindent\textbf{Означення:}
    ДВВ $\xi$ розподілена за \emph{законом Пуассона}, 
    якщо набуває значень $0,1,2,..$ з ймовірностями \begin{equation}
        \P\left\{\xi = k\right\} = \frac{a^k}{k!}e^{-a}
    \end{equation}
    \textbf{Коротке позначення:} $\xi \sim \mathrm{Poiss}(a)$.
    $a$ --- параметр закону, $a > 0$.

\noindent\textbf{Ряд розподілу:}
\begin{center}
    \begin{tabular}{|c|c|c|c|c|c|c|}
        \hline
        $\xi$ & $0$ & $1$ & $2$ & $...$ & $k$ & $...$ \\
        \hline
        $p$ & $e^{-a}$ & $ae^{-a}$ & $\frac{a^2}{2}e^{-a}$ & $...$ & $\frac{a^k}{k!}e^{-a}$ & $...$\\
        \hline
    \end{tabular}
\end{center}

\noindent\textbf{Полігон розподілу:} приклад для $a=2$.
\begin{center}
    \begin{tikzpicture}[baseline={(current bounding box.center)}]
        \pgfmathsetmacro{\s}{5};
        \pgfmathsetmacro{\a}{2};
        \pgfmathsetmacro{\n}{5};
        \draw [->] (-0.3,0) -- (\n+1, 0);
        \draw [->] (0, -0.1*\s) -- (0, 0.3*\s);
        \foreach \k in {0,...,\n}:
            \draw [thick] (\k, {(\a)^(\k)/factorial(\k)*e^(-\a)*\s}) -- (\k+1, {(\a)^(\k+1)/factorial(\k+1)*e^(-\a)*\s});
        \foreach \k in {0,...,\n}:
            \draw [fill] (\k, {(\a)^(\k)/factorial(\k)*e^(-\a)*\s}) circle [radius = 0.05];
        \node [below left] at (0, 0) {0};
        \foreach \k in {1,...,\n}:
            \node [below] at (\k, 0) {\k};
        \node [below] at (\n+1, 0) {$x$};
        \node [left] at (0, 0.3*\s) {$p$};
    \end{tikzpicture}
\end{center}

\noindent\textbf{Функція розподілу:} приклад для $a=2$.

\begin{tabular}{c c}
    $
        F_\xi(x) = \begin{cases}
            0, & x \leq 0 \\
            e^{-a}, & 0 < x \leq 1 \\
            e^{-a}+ae^{-a}, & 1 < x \leq 2 \\
            \dots \\
            \sum\limits_{m=0}^{k-1}\frac{a^m}{m!}e^{-a}& k-1 < x \leq k \\
            \dots
        \end{cases}
    $ &
    \begin{tikzpicture}[baseline={(current bounding box.center)}, yscale=2.5, xscale=0.88]
        \pgfmathsetmacro{\a}{2};
        \pgfmathsetmacro{\n}{5};
        \draw [->] (-0.3,0) -- (\n+1, 0);
        \draw [->] (0, -0.1) -- (0, 1.2);
        \draw [ultra thick] (-0.3, 0) -- (0,0);
        \draw [ultra thick] [<-] (0, {e^(-\a)}) -- (1, {e^(-\a)});
        \draw [ultra thick] [<-] (1, {e^(-\a)*(1 + \a)}) -- (2, {e^(-\a)*(1 + \a)});
        \draw [ultra thick] [<-] (2, {e^(-\a)*(1 + \a + \a^2/2)}) -- (3, {e^(-\a)*(1 + \a + \a^2/2)});
        \draw [ultra thick] [<-] (3, {e^(-\a)*(1 + \a + \a^2/2 + \a^3/6)}) -- (4, {e^(-\a)*(1 + \a + \a^2/2 + \a^3/6)});
        \draw [ultra thick] [<-] (4, {e^(-\a)*(1 + \a + \a^2/2 + \a^3/6 + \a^4/24)}) -- (5, {e^(-\a)*(1 + \a + \a^2/2 + \a^3/6 + \a^4/24)});
        \draw [ultra thick] [<-] (5, {e^(-\a)*(1 + \a + \a^2/2 + \a^3/6 + \a^4/24 + \a^5/120)}) -- (6, {e^(-\a)*(1 + \a + \a^2/2 + \a^3/6 + \a^4/24 + \a^5/120)});
        \node [below left] at (0, 0) {0};
        \foreach \k in {1,...,\n}:
            \node [below] at (\k, 0) {\k};
        \draw [dashed] (0, 1) -- (\n+1, 1);
        \node [left] at (0, 1) {1};
        \node [below] at (\n+1, 0) {$x$};
        \node [left] at (0, 1.2) {$F_\xi(x)$};
    \end{tikzpicture}
\end{tabular}

Для дослідження числових характеристик скористаємося генератрисою розподілу:

$G_\xi(z) = \sum\limits_{k=0}^{\infty} \P\left\{\xi = k\right\} z^k = e^{-a} \sum\limits_{k=0}^{\infty} \frac{a^k}{k!}z^k = e^{-a(1-z)}$, звідки
$G'_\xi(z) = ae^{az-a}$, $G''_\xi(z) = a^2e^{az-a}$. Тому
$G'_\xi(1) = a$, $G''_\xi(1) = a^2$, $G''_\xi(1) + G'_\xi(1) - \left( G'_\xi(1)\right)^2 = a^2 + a - a^2 = a$.
Отже, знайдено значення математичного сподівання та дисперсії.

\noindent\textbf{Числові характеристики:}
\begin{enumerate}
    \item $\E\xi = a$.
    \item $\D\xi = a$, $\sigma_\xi = \sqrt{a}$.
    \item $\Mo\xi = \left[a\right]$.
    \item $\Me\xi \approx \left[a + 1/3 - 0.02/a\right]$.
\end{enumerate}

\noindent\textbf{Застосування:} як вже було показано в асимптотичній формулі Пуассона \eqref{eq:poiss_aprox},
цим розподілом можна наблизити біноміальний, якщо ймовірність появи події дуже мала. Також цей розподіл
має кількість подій в \hyperref[poiss_proc]{потоці Пуассона}.

\subsection{Потік Пуассона}\label{poiss_proc}
\emph{Потоком} називають послідовність подій, які наступають в певні моменти часу одна за одною.
Потік характеризується випадковою величиною $\xi(t)$ --- 
кількістю подій, що наступили протягом проміжку часу $\left[ 0; t\right)$.
Ймовірності $\P\left\{\xi(t) = m\right\}$ позначатимемо $p_m(t)$.

Накладемо на потік такі вимоги:
\begin{enumerate}
    \item \emph{Стаціонарність (однорідність)} --- кількість подій, що наступають
    за певний проміжок часу, залежить лише від довжини проміжку і не залежить
    від того, де цей проміжок розташований на часовій осі.
    \item \emph{Відсутність післядії} --- якщо проміжки часу не перетинаються, то кількості подій,
    які за ці проміжки відбулися, є незалежними подіями.
    \item \emph{Ординарність} --- події наступають поодинці, тобто
    $\P\left\{\xi(t+\Delta t) - \xi(t) = 1\right\} = \lambda\cdot\Delta t + o(\Delta t)$
    ($\lambda > 0$ --- параметр інтенсивності) і 
    $\P\left\{\xi(t+\Delta t) - \xi(t) \geq 2\right\} = o(\Delta t)$.
\end{enumerate}
\begin{definition}
    Потік, що має перелічені властивості, називається \emph{потоком Пуассона}.
\end{definition}
\begin{remark}
    Насправді, умова $\P\left\{\xi(t+\Delta t) - \xi(t) = 1\right\} = \lambda\cdot\Delta t + o(\Delta t)$
    є надлишковою і її можна вивести з інших умов, але для спрощення приймемо її як частину означення.
\end{remark}

Отримаємо явний вираз для ймовірностей $p_m(t)$ у потоці Пуассона.
Позначимо $\tilde{p}_k(t+\Delta t) = \P\left\{\xi(t+\Delta t) - \xi(t) = k\right\}$.
Тоді $p_m(t+\Delta t) = p_m(t)\cdot \tilde{p}_0(t+\Delta t) + p_{m-1}(t)\cdot \tilde{p}_1(t+\Delta t) +
p_{m-2}(t)\cdot \tilde{p}_2(t+\Delta t) + ... + p_0(t)\cdot \tilde{p}_m(t+\Delta t)$ для $m \geq 1$,
для $m=0$ $p_0(t+\Delta t) = p_0(t)\cdot\tilde{p}_0(t+\Delta t)$. Застосуємо ординарність:

$p_0(t+\Delta t) = p_0(t) \cdot (1 - \lambda \cdot \Delta t + o(\Delta t))$

$p_m(t+\Delta t) = p_m(t) \cdot (1 - \lambda \cdot \Delta t + o(\Delta t)) + p_{m-1}(t) \cdot (\lambda \cdot \Delta t + o(\Delta t)) + o(\Delta t)$

\noindent Розкриємо дужки:

$p_0(t+\Delta t) - p_0(t) = -\lambda \cdot \Delta t \cdot p_0(t) + o(\Delta t)$
\nopagebreak

$p_m(t+\Delta t) - p_m(t) = -\lambda \cdot \Delta t \cdot p_m(t) + \lambda\cdot\Delta t \cdot p_{m-1}(t) + o(\Delta t)$

\noindent Поділимо на $\Delta t$ та перейдемо до границі при $\Delta t \rightarrow 0$.
Отримаємо систему диференціальних рівнянь:

$\begin{cases}
    p'_0(t) = - \lambda p_0(t) \\
    p'_m(t) = - \lambda p_m(t) + \lambda p_{m-1}(t), \; m \geq 1
\end{cases}$

\noindent Щоб розв'язати цю систему, введемо генератрису ймовірностей $p_m(t)$:

$G_\xi(z, t) = \sum\limits_{m=0}^{\infty} \P\left\{\xi(t) = m\right\} z^m = \sum\limits_{m=0}^{\infty} p_m(t) z^m$ 

\noindent Помножимо кожне рівняння отриманої системи на $z$ у відповідному степені та складемо:

$\sum\limits_{m=0}^{\infty} p'_m(t) z^m = -\lambda \sum\limits_{m=0}^{\infty} p_m(t) z^m + \lambda \sum\limits_{m=1}^{\infty} p_{m-1}(t) z^m = 
-\lambda \sum\limits_{m=0}^{\infty} p_m(t) z^m + \lambda z \sum\limits_{m=0}^{\infty} p_{m}(t) z^m$

\noindent Отримали диференціальне рівняння для генератриси:

$\frac{\partial G_\xi}{\partial t} = - \lambda G_\xi + \lambda z G_\xi = -\lambda(1-z) G_\xi$

\noindent Його розв'язком буде $G_\xi(z, t) = C(z) \cdot e^{-\lambda(1-z)t}$.
Оскільки $G_\xi(z, 0) = 1$, то $C(z) = 1$.

Отже, генератриса рівна $G_\xi(z, t) = e^{-\lambda(1-z)t}$ --- це генератриса закону Пуассона з параметром $a=\lambda t$.

Таким чином, $p_m(t) = \frac{(\lambda t)^m}{m!}e^{-\lambda t}$. 
Тепер зрозумілою стає назва параметру $\lambda$ (<<інтенсивність>>): $\E\xi = \lambda t$, $\lambda = \frac{\E\xi}{t}$ --- середня кількість подій за одиницю часу.

\begin{example}
    На АТС за годину в середньому за годину надходить 180 дзвінків. Яка ймовірність того,
    що за 2 хвилини надійде більше 4 дзвінків, якщо вважати, що моменти надходження дзвінків
    утворюють потік Пуассона?

    За умовою можна виписати інтенсивність надходження дзвінків: $\lambda = \frac{180}{60} = 3$ дзвінків за хвилину,
    тому кількість дзвінків за 2 хвилини матиме розподіл $\mathrm{Poiss}(3\cdot 2)$.
    Спочатку можна знайти ймовірність того, що надійде 4 або менше дзвінків:
    $\left(1 + 6 + \frac{6^2}{2!} + \frac{6^3}{3!} + \frac{6^4}{4!}\right)e^{-6} \approx 0.285$.
    Отже, шукана ймовірність приблизно дорівнює $1 - 0.285 = 0.715$.

\end{example}

\subsection{Рівномірний розподіл}
\noindent\textbf{Означення:}
    НВВ $\xi$ розподілена за \emph{рівномірним законом}, 
    якщо її щільність розподілу має вигляд \begin{equation}
        f_\xi(x) = \begin{cases}
            \frac{1}{b-a}, & x \in \left<a; b\right> \\
            0, & x \notin \left<a; b\right>
        \end{cases}
    \end{equation}
\textbf{Коротке позначення:} $\xi \sim \mathrm{U}\left<a; b\right>$.
    $a$ і $b$ --- параметри закону, $a<b$, $a, b \in \mathbb{R}$.

Розподіл $\mathrm{U}\left<0; 1\right>$ називається \emph{стандартним рівномірним розподілом}.

\begin{remark}
    Насправді, не важливо, що саме розуміти під проміжком $\left<a; b\right>$
    ($[a; b]$, $(a, b)$, $[a, b)$ чи $(a, b]$), оскільки $\P\left\{\xi = a \right\} = \P\left\{\xi = b \right\} = 0$,
    бо $\xi$ є неперервною.
\end{remark}

\begin{samepage}
    \noindent \textbf{Крива розподілу:}
    \begin{center}
        \begin{tikzpicture}[yscale = 2]
            \pgfmathsetmacro{\a}{1};
            \pgfmathsetmacro{\b}{3};
            \draw [->] (-2, 0) -- (5, 0);
            \draw [->] (0, -0.5) -- (0, 1);
            \draw [ultra thick] (\a, {1/(\b -\a)}) -- (\b, {1/(\b -\a)});
            \draw [dashed] (\a, 0) -- (\a, {1/(\b -\a)});
            \draw [dashed] (\b, 0) -- (\b, {1/(\b -\a)});
            \draw [ultra thick] [->] (-2, 0) -- (\a, 0);
            \draw [ultra thick] [<-] (\b, 0) -- (4.9, 0);
            \node [below] at (\a, -0.03) {$a$};
            \node [below] at (\b, 0) {$b$};
            \draw [dashed] (0, {1/(\b -\a)}) -- (\a, {1/(\b -\a)});
            \node [left] at (0, {1/(\b -\a)}) {$\frac{1}{b-a}$};
            \node [below] at (5, 0) {$x$};
            \node [left] at (0, 0.9) {$f_\xi(x)$};
        \end{tikzpicture}
    \end{center}
\end{samepage}

\noindent \textbf{Функція розподілу:}
\begin{center}
    \begin{tabular}{c c}
        $
            F_\xi(x) = \begin{cases}
                0, & x \leq a \\
                \frac{x-a}{b-a}, & a< x \leq b \\
                1, & x > b
            \end{cases}
        $ &
        \begin{tikzpicture}[baseline={(current bounding box.center)}, yscale=1.5]
            \pgfmathsetmacro{\a}{1};
        \pgfmathsetmacro{\b}{3};
        \draw [->] (-2, 0) -- (5, 0);
        \draw [->] (0, -0.5) -- (0, 1.2);
        \draw [ultra thick] (-2, 0) -- (\a, 0);
        \draw [domain=\a:\b, smooth, variable = \x, ultra thick] plot ({\x}, {(\x-\a)/(\b-\a)});
        \draw [ultra thick] (\b, 1) -- (5, 1);
        \node [below] at (\a, -0.03) {$a$};
        \node [below] at (\b, 0) {$b$};
        \node [below] at (5, 0) {$x$};
        \node [left] at (0, 1.2) {$F_\xi(x)$};
        \draw [dashed] (0, 1) -- (\b, 1);
        \draw [dashed] (\b, 0) -- (\b, 1);
        \node [left] at (0, 1) {$1$};
        %\node [right] [align=center] at (2.5, 0.3) {Приклад для \\ $a = \a, b = \b$};
        \end{tikzpicture}
    \end{tabular}
\end{center}

\noindent\textbf{Числові характеристики:}
\begin{enumerate}
    \item $\E\xi = \int\limits_{-\infty}^{+\infty} x f_\xi(x)dx = \int\limits_{a}^{b} \frac{x}{b-a}dx = \frac{a+b}{2}$.
    \item $\D\xi = \E\xi^2 - (\E\xi)^2 = \int\limits_{a}^{b} \frac{x^2}{b-a}dx - \frac{(a+b)^2}{4} = \frac{(b-a)^2}{12}$, $\sigma_\xi = \frac{b-a}{2\sqrt{3}}$.
    \item $\Mo\xi$ --- будь-яка точка з $\left<a; b\right>$.
    \item $\Me\xi = \frac{a+b}{2}$.
    \item $\As\xi = 0$.
    \item $\Ex\xi = -\frac{6}{5}$.
\end{enumerate}
% \begin{exercise}
%     Обчислити $\Me\xi$, $\As\xi$ та $\Ex\xi$ та перевірити наведені значення.
% \end{exercise}

\noindent\textbf{Застосування:} отримання вибірок з інших законів розподілу;
похибка округлення до найближчої поділки приладу при ціні поділки $a$ має розподіл $\mathrm{U}\left<-\frac{a}{2}; \frac{a}{2}\right>$.

\subsection{Експоненційний (показниковий) розподіл}
\noindent\textbf{Означення:}
    НВВ $\xi$ розподілена за \emph{експоненційним законом}, 
    якщо її щільність розподілу має вигляд \begin{equation}
        f_\xi(x) = \begin{cases}
            \lambda e^{-\lambda x}, & x \geq 0 \\
            0, & x < 0
        \end{cases}
    \end{equation}
\textbf{Коротке позначення:} $\xi \sim \mathrm{Exp}(\lambda)$.
    $\lambda > 0$ --- параметр закону.

\noindent \textbf{Крива розподілу:}
\begin{center}
    \begin{tikzpicture}[yscale = 2]
        \pgfmathsetmacro{\l}{1};
        \draw [->] (-2, 0) -- (5, 0);
        \draw [->] (0, -0.5) -- (0, 1.2);
        \draw [ultra thick] [->] (-2, 0) -- (0, 0);
        \draw [domain=0:5, smooth, variable = \x, ultra thick] plot ({\x}, {\l*e^(-\l*\x)});
        \node [below] at (5, 0) {$x$};
        \node [left] at (0, 1.2) {$f_\xi(x)$};
        \node [below right] at (0, 0) {$0$};
        \node [left] at (0, \l) {$\lambda$};
    \end{tikzpicture}
\end{center}

\noindent \textbf{Функція розподілу:}
\begin{center}
    \begin{tabular}{c c}
        $
            F_\xi(x) = \begin{cases}
                \lambda \int\limits_0^x e^{-\lambda t} dt = 1 - e^{-\lambda x}, & x> 0 \\
                0, & x \leq 0
            \end{cases}
            $ &
        \begin{tikzpicture}[baseline={(current bounding box.center)}, yscale=2]
            \pgfmathsetmacro{\l}{1};
            \draw [->] (-2, 0) -- (5, 0);
            \draw [->] (0, -0.5) -- (0, 1.2);
            \draw [ultra thick] (-2, 0) -- (0, 0);
            \draw [domain=0:5, smooth, variable = \x, ultra thick] plot ({\x}, {1 - e^(-\l*\x)});
            \node [below] at (5, 0) {$x$};
            \node [left] at (0, 1.2) {$F_\xi(x)$};
            \node [below right] at (0, 0) {$0$};
            \node [left] at (0, 1) {$1$};
            \draw [dashed] (0, 1) -- (5, 1);
        \end{tikzpicture}
    \end{tabular}
\end{center}

Для знаходження основних числових характеристик знайдемо всі \emph{початкові моменти} експоненційного розподілу:
\begin{gather*}
    \E\xi^k = \int\limits_{-\infty}^{+\infty} x^k f_\xi(x)dx = \lambda \int\limits_{0}^{+\infty} x^k e^{-\lambda x}dx = \left[\; \lambda x = t \;\right]=
\frac{\lambda}{\lambda^k \cdot \lambda} \int\limits_{0}^{+\infty} t^k e^{-t} dt = \frac{\Gamma(k+1)}{\lambda^k} = \frac{k!}{\lambda^k}
\end{gather*}

\noindent\textbf{Числові характеристики:}
\begin{enumerate}
    \item $\E\xi = \frac{1}{\lambda}$.
    \item $\D\xi = \E\xi^2 - (\E\xi)^2 = \frac{2}{\lambda^2} - \frac{1}{\lambda^2} = \frac{1}{\lambda^2}$, $\sigma_\xi = \frac{1}{\lambda}$.
    \item $\Mo\xi = 0$.
    \item $\Me\xi = \frac{\ln2}{\lambda}$.
    \item $\As\xi = 2$ --- не залежить від $\lambda$, бо $\beta_3 = \alpha_3 - 3\alpha_2 \alpha_1 + 2 \alpha_1^3 = \frac{2}{\lambda^3}$.
    \item $\Ex\xi = 6$ --- теж не залежить від $\lambda$, бо $\beta_4 = \alpha_4 - 4\alpha_3 \alpha_1 + 6 \alpha_2 \alpha_1^2 - 3\alpha_1^4 = \frac{9}{\lambda^4}$.
\end{enumerate}

\begin{exercise}
    Дослідити характеристики <<зсунутого>> експоненційного розподілу
    зі щільністю
    $\mathrm{Exp}(\lambda, x_0)$ з щільністю $f_\xi(x) = \begin{cases}
        \lambda e^{-\lambda (x-x_0)}, & x \geq x_0 \\
        0, & x < x_0
    \end{cases}$.
\end{exercise}

Як і геометричний розподіл, експоненційний \textbf{<<не має пам'яті>>}.
Доведемо, що якщо проміжок часу $T$, розподілений за експоненційним законом,
тягнувся час $\tau$, то проміжок $T_1 = T - \tau$ розподілений так само --- 
за експоненційним законом з тим самим параметром:
\begin{gather*}
    F_{T_1}(t) = \P\left\{T_1 < t / T \geq \tau\right\} = 
    \frac{\P\left\{(T_1 < t) \cap (T \geq \tau)\right\}}{\P\left\{ T \geq \tau \right\}} =
    \frac{\P{\left\{\tau \leq T < t +\tau\right\}}}{1 - \P\left\{ T < \tau \right\}} = \\
    \frac{F_T(t+\tau) - F_T(\tau)}{1-F_T(\tau)} = 
    \frac{1-e^{-\lambda (t+\tau)} - (1 - e^{-\lambda \tau})}{1-(1-e^{-\lambda \tau})} =
    \frac{e^{-\lambda \tau}(1-e^{-\lambda t})}{e^{-\lambda \tau}} = 1-e^{-\lambda t} = F_T(t)
\end{gather*}

\noindent\textbf{Застосування:}
\begin{enumerate}
    \item Час між двома сусідніми подіями в потоці Пуассона має експоненційний розподіл.
    \begin{exercise}
        Довести цю властивість.
    \end{exercise}
    \item Як правило, час безвідмовної роботи приладу має експоненційний розподіл.
    
    Нехай в момент часу $t=0$ увімкнули прилад. Припустимо, що умовна ймовірність виходу з ладу
    приладу в інтервалі часу $\left[ t; t+\Delta t\right)$ за умови, що до часу $t$ він працював,
    пропорційна $\Delta t$ і рівна $\lambda\cdot \Delta t + o(\Delta t)$. Нехай $\xi$ --- час безвідмовної роботи,
    покладемо $Q(t) = \P\left\{\xi \geq t\right\}$.
    $Q(t+\Delta t) = \P\left\{\xi \geq t + \Delta t\right\} = \P\left\{\xi \geq t\right\} \cdot \P\left\{\xi \geq t+\Delta t / \xi \geq t\right\} = 
    Q(t) \cdot (1 - \lambda\cdot \Delta t + o(\Delta t))$.
    $\frac{Q(t+\Delta t) - Q(t)}{\Delta t} = -\lambda Q(t) + \frac{o(\Delta t)}{\Delta t} \cdot Q(t)$,
    при $\Delta t \rightarrow 0$ отримаємо $Q'(t) = -\lambda Q(t)$.
    Розв'язком цього диференціального рівняння буде $Q(t) = C\cdot e^{-\lambda t}$.
    Оскільки вважаємо, що в момент часу $t=0$ прилад працював, то $Q(0) = 1$ і $C=1$.

    Тоді $F_\xi(t) = \P\left\{ \xi < t\right\} = 1 - \P\left\{ \xi \geq t\right\} = 1 - Q(t) = \begin{cases}
        1 - e^{-\lambda t}, & t \geq 0 \\
        0, & t < 0
    \end{cases}$. Отже, $\xi \sim \mathrm{Exp}(\lambda)$.
\end{enumerate}

\begin{example}
    Яка ймовірність того, що людина проживе 100 років? Нехай $\xi$ --- час життя людини.
    Оскільки $\P\left\{\xi = 100\right\} = 0$, знайдемо $\P\left\{ \xi \geq 100\right\}$.
    $\P\left\{ \xi \geq 100\right\}$.
    Оскільки $\lambda = \frac{1}{\E\xi}$, то, прийнявши $\E\xi = 70$ (середня тривалість життя),
    отримаємо $\P\left\{ \xi \geq 100\right\} = e^{-100/70} \approx 0.24$.
\end{example}

\subsection{Гамма-розподіл}
\noindent\textbf{Означення:}
    НВВ $\xi$ розподілена за \emph{гамма-законом}, 
    якщо її щільність розподілу має вигляд 
    \begin{equation}
        f_\xi(x) = \begin{cases}
            \frac{\beta^\alpha}{\Gamma(\alpha)} x^{\alpha-1} e^{-\beta x}, & x \geq 0 \\
            0, & x < 0
        \end{cases}
    \end{equation}
\textbf{Коротке позначення:} $\xi \sim {\Gamma}(\alpha, \beta)$.
    $\alpha >0, \beta > 0$ --- параметри закону.
\begin{remark}
    Експоненційний розподіл $\mathrm{Exp}(\lambda)$ є частинним випадком гамма-розподілу, а саме ${\Gamma}(1, \lambda)$.
\end{remark}
\begin{samepage}
    \noindent \textbf{Крива розподілу:} приклад для $\alpha = 2, \beta=1$.
    \begin{center}
        \begin{tikzpicture}[yscale = 7, xscale = 0.5, baseline={(current bounding box.center)}]
            \pgfmathsetmacro{\a}{2};
            \pgfmathsetmacro{\b}{1};
            \pgfmathsetmacro{\s}{2};
            \draw [->] (-2, 0) -- (20, 0);
            \draw [->] (0, -0.05) -- (0, 0.4);
            \draw [ultra thick] (-2, 0) -- (0, 0);
            \draw [domain=0:19.7, smooth, variable = \x, ultra thick, samples=100] plot ({\x}, {\b^(\a)*(\x/\s)^(\a-1) * e^(-\x*\b/\s)});
            \node [below] at (20, 0) {$x$};
            \node [left] at (0, 0.4) {$f_\xi(x)$};
            \node [below left] at (0, 0) {$0$};
        \end{tikzpicture} 
    \end{center}
\end{samepage}

% \begin{center}
%     \begin{tikzpicture}[yscale = 5, xscale = 0.5, baseline={(current bounding box.center)}]
%         \pgfmathsetmacro{\a}{1};
%         \pgfmathsetmacro{\b}{0.5};
%         \draw [->] (-2, 0) -- (20, 0);
%         \draw [->] (0, -0.05) -- (0, 0.6);
%         \draw [ultra thick] [->] (-2, 0) -- (0, 0);
%         \draw [domain=0:19.7, smooth, variable = \x, ultra thick] plot ({\x}, {\b^(\a)*\x^(\a-1) * e^(-\x*\b)});
%         \node [below] at (20, 0) {$x$};
%         \node [left] at (0, 0.6) {$f_\xi(x)$};
%         \node [below right] at (0, 0) {$0$};
%         \node at (18, 0.375) {$\alpha = 1$};
%         \node at (18, 0.285) {$\beta = 1/2$};
%     \end{tikzpicture} 
% \end{center}

\noindent \textbf{Функція розподілу:}
\nopagebreak

\begin{tabular}{c c}
    \begin{tabular}{c}
        $
        F_\xi(x) = \begin{cases}
            0, & x \leq 0 \\
            \int\limits_0^x f_\xi(t)dt = \frac{\gamma(\alpha, \beta x)}{\Gamma(\alpha)}, & x>0 
        \end{cases}$
        \\
        де $\gamma(s, x) = \int\limits_0^x t^{s-1} e^{-t} dt$
    \end{tabular}
    &
    \begin{tikzpicture}[baseline={(current bounding box.center)}, yscale=0.9]
        \pgfmathsetmacro{\a}{2};
        \pgfmathsetmacro{\b}{3};
        \draw [->] (-2, 0) -- (5, 0);
        \draw [->] (0, -0.5) -- (0, 2.7);
        \draw [ultra thick] (-2, 0) -- (0, 0);
        \draw [dashed] (0, 2.3) -- (5, 2.3);
        \node [below] at (5, 0) {$x$};
        \node [left] at (0, 2.7) {$F_\xi(x)$};
        \node [left] at (0, 2.3) {$1$};
        \node [below right] at (0, 0) {$0$};
        \begin{axis}[width=0.4\textwidth,height=0.25\textwidth, samples=100, domain = 0:4, restrict y to domain = 0:1, axis x line=bottom,axis y line=left, clip=false, axis line style={draw=none}, ticks=none]
        \addplot [ultra thick] plot ({\x}, {(\x^\a) * (
            1/\a -
            \x/(1+\a) +
            \x^2/(2*(2+\a)) -
            \x^3/(6*(3+\a)) +
            \x^4/(24*(4+\a)) -
            \x^5/(120*(5+\a)) +
            \x^6/(720*(6+\a)) -
            \x^7/(5040*(7+\a)) +
            \x^8/(40320*(8+\a)) -
            \x^9/(362880*(9+\a)) +
            \x^10/(3628800*(10+\a)) -
            \x^11/(39916800*(11+\a))
        )});
        \end{axis}
    \end{tikzpicture}
\end{tabular}

Знайдемо всі \emph{початкові моменти} гамма-розподілу.
\begin{gather*}
    \E\xi^k = \int\limits_{-\infty}^{+\infty} x^k f_\xi(x)dx =
    \frac{\beta^\alpha}{\Gamma(\alpha)} \int\limits_0^{+\infty} x^k x^{\alpha-1} e^{-\beta x} dx =
    \left[\; \beta x = t \;\right] = \\
    = \frac{\beta^\alpha}{\Gamma(\alpha)} \cdot \frac{1}{\beta^k\cdot \beta^{\alpha}} \int\limits_0^{+\infty} t^{k+\alpha-1} e^{-t} dt =
    \frac{\Gamma(\alpha+k)}{\Gamma(\alpha) \cdot \beta^k}
\end{gather*}
Зокрема, $\E\xi = \frac{\Gamma(\alpha+1)}{\Gamma(\alpha) \cdot \beta} = \frac{\alpha}{\beta}$ та 
$\E\xi^2 = \frac{\Gamma(\alpha+2)}{\Gamma(\alpha) \cdot \beta^2}  = \frac{(\alpha+1)\alpha}{\beta^2}$.

\noindent\textbf{Числові характеристики:}
\begin{enumerate}
    \item $\E\xi = \frac{\alpha}{\beta}$.
    \item $\D\xi = \E\xi^2 - (\E\xi)^2 = \frac{(\alpha+1)\alpha}{\beta^2} - \frac{\alpha^2}{\beta^2} = \frac{\alpha}{\beta^2}$, $\sigma_\xi = \frac{\sqrt{\alpha}}{\beta}$.
    \item $\Mo\xi = \frac{\alpha-1}{\beta}$.
    \item $\Me\xi$ --- немає виразу у замкненій формі.
    \item $\As\xi = \frac{2}{\sqrt{\alpha}}$.
    \item $\Ex\xi = \frac{6}{\alpha}$.
\end{enumerate}

\noindent\textbf{Застосування:} гамма-розподіл застосовується для моделювання
складних потоків подій, в економіці, теорії масового обслуговування, логістиці.
У випадку натурального параметру $\alpha = k \in \mathbb{N}$ за законом
$\Gamma(k, \lambda)$ розподілений час очікування появи $k$-тої події
в процесі Пуассона з інтенсивністю $\lambda$. Таку версію гамма-розподілу іноді називають \emph{розподілом Ерланга}.
\begin{exercise}
    Нехай $T$ --- час очікування появи $k$-тої події
    в потоці Пуассона з інтенсивністю $\lambda$. Довести $T \sim \Gamma(k, \lambda)$.
\end{exercise} % https://towardsdatascience.com/gamma-distribution-intuition-derivation-and-examples-55f407423840

\subsection{Гауссівський (нормальний) розподіл}
\noindent\textbf{Означення:}
    НВВ $\xi$ розподілена за \emph{нормальним законом}, 
    якщо її щільність розподілу має вигляд 
    \begin{equation}
        f_\xi(x) = \frac{1}{\sqrt{2\pi}\sigma} e^{-\frac{(x-a)^2}{2\sigma^2}}
    \end{equation}
\textbf{Коротке позначення:} $\xi \sim \mathrm{N}(a, \sigma)$ або 
    $\xi \sim \mathrm{N}(a, \sigma^2)$.
    $a \in \mathbb{R}, \sigma > 0$ --- параметри закону.

    Розподіл $\mathrm{N}(0, 1)$ називають \emph{стандартним гауссівським розподілом}.

\begin{samepage}
    \noindent \textbf{Крива розподілу:}
    \begin{center}
        \begin{tikzpicture}[yscale = 4, xscale = 2]
            \pgfmathsetmacro{\a}{2.5};
            \pgfmathsetmacro{\s}{1}
            \draw [->] (-1, 0) -- (5, 0);
            \draw [->] (0, -0.2) -- (0, 0.5);
            \draw [dashed] (\a, 0) -- (\a, 0.398942280401/\s);
            \draw [dashed] (\a-\s, 0) -- (\a-\s, 0.241970724519/\s);
            \draw [dashed] (\a+\s, 0) -- (\a+\s, 0.241970724519/\s);
            \draw [domain=-1:5, smooth, variable = \x, ultra thick] plot ({\x}, {0.398942280401/\s * e^(-(\x-\a)^2/(2*\s^2))});
            \node [below] at (5, 0) {$x$};
            \node [below] at (\a, -0.025) {$a$};
            \node [below] at (\a-\s, 0) {$a - \sigma$};
            \node [below] at (\a+\s, 0) {$a + \sigma$};
            \node [below right] at (0, 0) {$0$};
            \node [left] at (0, 0.5) {$f_\xi(x)$};
        \end{tikzpicture}
    \end{center}
\end{samepage}

\noindent \textbf{Функція розподілу:}
$F_\xi(x) = \int\limits_{-\infty}^{x} f_\xi(t) dt = 
\frac{1}{\sqrt{2\pi}\sigma} \int\limits_{-\infty}^{x} 
e^{-\frac{(t-a)^2}{2\sigma^2}} dt$ не виражається в елементарних 
функціях.
Для зручного використання гауссівського закону в задачах введемо 
спеціальну функцію, \emph{функцію Лапласа} $\Phi(x)$, значення якої занесено до \hyperref[table:laplace]{таблиці}:
\begin{gather*}
    \Phi(x) = \frac{1}{\sqrt{2\pi}} 
    \int\limits_{0}^{x} e^{-\frac{t^2}{2}} dt
\end{gather*}
\begin{center}
    \begin{tabular}{c c}
        \begin{tikzpicture}[baseline={(current bounding box.center)}, yscale=3, 
            scale = 1]
            \fill [lightgray, domain=0:1, smooth, variable = \x] plot ({\x}, 
            {
                (0.3989422804) * e^(- (\x * \x / 2))
            }) -- (1, 0) -- (0, 0) -- (0, 0.3989422804);
            \draw [->] (-3, 0) -- (3, 0);
            \node [below] at (3, 0) {$t$};
            \draw [->] (0, -0.2) -- (0, 0.7);
            \draw [domain=-3:3, smooth, variable = \x, ultra thick] plot ({\x}, 
            {
                (0.3989422804) * e^(- (\x * \x / 2))
            });
            \node [below] at (1, 0) {$x$};
            \draw [dashed] (1, 0) -- (1, 0.25);
            \draw [->, thick] (1.5, 0.4) -- (0.7, 0.2);
            \node [below left] at (3.1, 0.5) {$S = \Phi(x)$};
            \draw [->] (-0.8, 0.5) -- (-0.495, 0.355);
            \node [left] at (-0.8, 0.5) {$\frac{1}{\sqrt{2\pi}}e^{-\frac{t^2}{2}}$};
        \end{tikzpicture} &
        \begin{tikzpicture}[baseline={(current bounding box.center)}, yscale=2, scale = 1.2]
            \draw [->] (-2, 0) -- (2, 0);
            \draw [->] (0, -0.6) -- (0, 0.6);
            \draw [domain=-2:2, smooth, variable = \x, ultra thick] plot ({\x}, 
            {
                (0.564189583547756) * (
                (\x / 1.414213562373095) 
                - ((\x / 1.414213562373095)^3 / 3) 
                + ((\x / 1.414213562373095)^5 / 10) 
                - ((\x / 1.414213562373095)^7 / 42)
                + ((\x / 1.414213562373095)^9 / 216) 
                - ((\x / 1.414213562373095)^11 / 1320)
                + ((\x / 1.414213562373095)^13 / 9360)
                )
            });
            \node [below] at (2, 0) {$x$};
            \node [above left] at (0, 0.5) {$\Phi(x)$};
            \node [below right] at (0, 0) {$0$};
            \draw [dashed] (-2, -0.5) -- (2, -0.5);
            \draw [dashed] (-2, 0.5) -- (2, 0.5);
            \node [below left] at (0, 0.5) {$0.5$};
            \node [above left] at (0, -0.5) {$-0.5$};
        \end{tikzpicture}
    \end{tabular}
\end{center}

\noindent
\emph{Властивості функції Лапласа: }
    \begin{enumerate}
        \item $\Phi(-x) = -\Phi(x)$ --- непарна.
        \item $\Phi(0) = 0$.
        \item $y = \pm \;0.5$ --- горизонтальні асимптоти
        \item $\forall x \geq 5: \Phi(x) \approx 0.5 $.
        \item $\forall x \leq -5: \Phi(x) \approx -0.5 $.
    \end{enumerate}

\noindent Виразимо функцію розподілу гауссівського закону через функцію Лапласа:
\begin{equation}
    F_\xi (x) = \frac{1}{\sqrt{2\pi}\sigma} \int\limits_{-\infty}^{x} 
    e^{-\frac{(t-a)^2}{2\sigma^2}} dt = 
    \left[\frac{t-a}{\sigma} = z\right] = 
    \frac{1}{\sqrt{2\pi}} \int\limits_{-\infty}^{\frac{x-a}{\sigma}} 
    e^{-\frac{z^2}{2}} dz = \frac{1}{2} + 
    \Phi\left(\frac{x-a}{\sigma}\right)
\end{equation}

Виявляється, що параметри $a$ та $\sigma$ мають простий ймовірнісний сенс.
Знайдемо спочатку математичне сподівання:
\begin{gather*}
    \E\xi = 
    \frac{1}{\sqrt{2\pi}\sigma}
    \int\limits_{-\infty}^{+\infty}x e^{-\frac{(x-a)^2}{2\sigma^2}} dx 
    = \left[ \frac{x-a}{\sigma \sqrt{2}} = t, \; dx = \sigma\sqrt{2}dt \right] = 
    \frac{1}{\sqrt{\pi}}\int\limits_{-\infty}^{+\infty} 
    (a + \sigma\sqrt{2}t)e^{-t^2} dt = \\
    =\frac{a}{\sqrt{\pi}}
    \underbrace{\int\limits_{-\infty}^{+\infty}e^{-t^2}dt}_{ = \sqrt{\pi}} \; + \; 
    \frac{\sigma \sqrt{2}}{\sqrt{\pi}} 
    \underbrace{\int\limits_{-\infty}^{+\infty} t e^{-t^2}dt}_{= 0}
    = a
\end{gather*}
Знайдемо тепер всі центральні моменти:
\begin{gather*}
    \beta_k = \E(\xi - \E\xi)^k = 
    \frac{1}{\sigma\sqrt{2\pi}}
    \int\limits_{-\infty}^{+\infty}(x-a)^k 
    e^{-\frac{(x-a)^2}{2\sigma^2}} dx = 
    \left[ \frac{x-a}{\sigma \sqrt{2}} = t\right] 
    =
    \frac{\sigma^k 2^{k/2}}{\sqrt{\pi}}
    \int\limits_{-\infty}^{+\infty}t^k 
    e^{-t^2} dt
\end{gather*}
Якщо $k = 2l + 1$, то $\beta_k = 0$, оскільки під інтегралом непарна функція.
Зокрема, $\beta_3 = 0$, тому $\As\xi = 0$.
Візьмемо $k = 2l$:
\begin{gather*}
    \beta_{2l} = 
    \frac{\sigma^{2l} 2^{l}}{\sqrt{\pi}}
    \int\limits_{-\infty}^{+\infty}t^{2l} 
    e^{-t^2} dt = \left[ t^2 = z, \; dt = \frac{1}{2} z^{-\frac{1}{2}}dz\right] = 
    2\frac{\sigma^{2l}2^l}{\sqrt{\pi}}
    \int\limits_{0}^{+\infty}z^l e^{-z}\cdot \frac{1}{2} z^{-\frac{1}{2}}
    dz = \\
    = \frac{\sigma^{2l}2^l}{\sqrt{\pi}}
    \int\limits_{0}^{+\infty}z^{l-\frac{1}{2}} e^{-z}
    dz = 
    \frac{\sigma^{2l}2^l}{\sqrt{\pi}}\cdot\Gamma\left(l+\frac{1}{2}\right) = 
    \frac{\sigma^{2l}2^l}{\sqrt{\pi}} \cdot \frac{(2l-1)!!}{2^l} \sqrt{\pi} = \sigma^{2l} (2l-1)!!
\end{gather*}
Зокрема, $\beta_2 = \D\xi = \frac{2 \sigma^2}{\sqrt{\pi}}\cdot\Gamma\left(\frac{3}{2}\right) = \sigma^2$,
$\beta_4 = \frac{\sigma^4\cdot2^2}{\sqrt{\pi}}\cdot \Gamma\left(\frac{5}{2}\right) = 3\sigma^4$.

\noindent \textbf{Числові характеристики:}
\begin{enumerate}
    \item $\E\xi = \Mo\xi = \Me\xi = a$.
    \item $\D\xi = \sigma^2$, $\sigma_\xi = \sigma$.
    \item $\As\xi = 0$.
    \item $\Ex\xi = \frac{\beta_4}{\sigma_\xi^4} - 3 = 
    \frac{3\sigma^4}{\sigma^4} - 3 = 0$. 
\end{enumerate}
\begin{remark}
    Доданок $-3$ у формулі ексцесу був введений саме для того, щоб нормальний розподіл мав нульовий ексцес,
    тому додатній ексцес означає, що крива розподілу біля точки максимуму є крутішою за криву гауссівського розподілу,
    а від'ємний --- що більш пласкою.
\end{remark}

\noindent\textbf{Застосування:} гауссівський закон є найважливішим в теорії ймовірностей,
оскільки може описати розподіл багатьох величин, що зустрічаються в природі.
Цьому сприяє, наприклад, симетричність розподілу відносно математичного
сподівання (значення якого також є модою і медіаною, тобто <<середнім>> у всіх можливих в цьому випадку сенсах) та наведене нижче \emph{<<правило $3\sigma$>>},
яке означає, що майже всі значення гауссівської випадкової величини
лежать на відстані не більше трьох середньоквадратичних 
відхилень від її математичного сподівання.
Ці властивості природно очікувати від деяких <<загальних>> характеристик, 
як-от зріст дорослої людини: інтуїтивно зрозуміло, що є деякий <<середній>> зріст (наприклад, 170 см),
а майже всі інші можливі значення знаходяться в деякому діапазоні навколо нього (наприклад, 155-185 см), окрім досить малої кількості
винятково малих чи великих. Важливість гауссівського розподілу ще пояснюється тим,
що за деяких умов він може наближено замінити розподіл суми достатньо великої кількості випадкових величин, 
навіть якщо вони мають різні закони розподілу --- про це йтиме мова у розділі \hyperref[ch:limit_theorems]{<<Граничні теореми теорії ймовірностей>>}.

\vspace{0.5em}
\noindent \textbf{Робочі формули для розв'язання задач:}
\begin{enumerate}
    \item $\P\left\{\xi \in \left<\alpha; \beta\right>\right\} 
    = F_\xi(\beta) - F_\xi(\alpha) = \Phi\left(\frac{\beta - a}{\sigma}\right)
    - \Phi\left(\frac{\alpha - a}{\sigma}\right)$.
    \item $\P\left\{\left| \xi - a \right| < \varepsilon\right\} = 
    \P\left\{\xi \in \left(a - \varepsilon, a + \varepsilon\right)\right\} = 
    \Phi\left(\frac{a + \varepsilon - a}{\sigma}\right)
    - \Phi\left(\frac{a - \varepsilon - a}{\sigma}\right) = 
    2\Phi(\frac{\varepsilon}{\sigma})$.
    \item \emph{<<Правило $3\sigma$>>}: 
    $\P\left\{\left| \xi - a \right| < 3\sigma\right\} = 
    \P\left\{\xi \in \left( a-3\sigma; a+3\sigma\right)\right\} = 2\Phi\left(3\right)
    \approx 0.9973$.
\end{enumerate}
    \chapter{Випадкові вектори}
        % !TEX root = ../main.tex
\section{Дискретні випадкові вектори}
\begin{definition}
    Вимірна функція $\Omega \rightarrow \mathbb{R}^n$, задана на ймовірнісному 
    просторі $\left\{\Omega, \mathcal{F}, P \right\}$, 
    називається \emph{випадковим вектором (системою випадкових величин)}.
    Позначається
$\vec{\xi} = \vec{\xi}(\omega) = 
\left(\xi_1(\omega), \xi_2(\omega), ... , \xi_n(\omega)\right)^{T}$.
\end{definition}

\begin{remark}
    Під вимірністю мається на увазі те, що 
    \begin{equation*}
        \forall \vec{x} \in \mathbb{R}^n: 
        A=\left\{\omega \in \Omega: \xi_1(\omega) < x_1, 
                                    \xi_2(\omega) < x_2,
                                    ... ,
                                    \xi_n(\omega) < x_n\right\}
        \in \mathcal{F}
    \end{equation*}
\end{remark}

\begin{definition}
    Випадковий вектор називається \emph{дискретним}, якщо всі його координати --- 
    дискретні випадкові величини.
\end{definition}
\begin{remark}
    Випадковий вектор можна трактувати як випадкову точку в $\mathbb{R}^n$.
    При $n = 2$: $\vec{\xi} = (\xi_1, \xi_2)^T$ --- випадкова точка на площині. 
\end{remark}

Закон розподілу двовимірного дискретного випадкового вектора задається 
\emph{таблицею розподілу}.


\hbox to \hsize{\hfil{
\begin{tabular}{c c}
    \begin{tabular}{|c|c|c|c|c|}
        \hline
        \diagbox{$\xi_2$}{$\xi_1$} & $x_1$ & $x_2$ & ... & $x_n$ \\
        \hline
        $y_1$ & $p_{11}$ & $p_{21}$ & ... & $p_{n1}$ \\
        \hline
        $y_2$ & $p_{12}$ & $p_{22}$ & ... & $p_{n2}$ \\
        \hline
        ... & ... & ... & $p_{ij}$ & ... \\
        \hline
        $y_m$ & $p_{1m}$ & $p_{2m}$ & ... & $p_{nm}$ \\
        \hline
    \end{tabular} &
    \begin{tikzpicture}[baseline={(current bounding box.center)}]
        \draw [->] (-0.5, 0) -- (5, 0); 
        \draw [->] (0, -0.5) -- (0, 2.5);
        %\draw [fill] (1, 1) circle [radius = 0.05];
        \foreach \i in {1,...,2}:
            %\draw [dashed] (\i, 0) -- (\i, 1.5);
            \foreach \j in {1,...,2}:
                \draw [fill] (\i, {\j/2}) circle [radius = 0.05];
        \foreach \k in {1,...,2} {
            \draw [fill] (4, {\k/2}) circle [radius = 0.05];
            \draw [fill] (\k, 2) circle [radius = 0.05];
            \draw [dashed] (\k, 0) -- (\k, 2);
            \node [below] at (\k, 0) {$x_\k$};
            \node [left] at (0, {\k/2}) {$y_\k$};
            \draw [dashed] (0, {\k/2}) -- (4, {\k/2});
        }
        \draw [fill] (4, 2) circle [radius = 0.05];
        \draw [dashed] (4, 0) -- (4, 2);
        \node [below] at (3, -0.1) {$\ldots$};
        \node [below] at (4, 0) {$x_n$};
        \node [below] at (5, 0) {$x$};
        \draw [dashed] (0, 2) -- (4, 2);
        \node [left] at (0, 1.6) {$\vdots$};
        \node [left] at (0, 2) {$y_m$};
        \node [left] at (0, 2.5) {$y$};
    \end{tikzpicture}
\end{tabular}
}\hfil}
$p_{ij} = P\left\{\xi_1 = x_i, \xi_2 = y_j\right\}$, $\sum\limits_{i,j} p_{ij} = 1$.

З таблиці розподілу можна обчислити ряди розподілу $\xi_1$ та $\xi_2$:

\hbox to \hsize{\hfil{
    \begin{tabular}{c c}
        \begin{tabular}{|c|c|c|c|c|}
            \hline
            $\xi_1$ & $x_1$ & $x_2$ & $...$ & $x_n$ \\
            \hline
            $p$ & $\sum\limits_{k=1}^m p_{1k}$ & $\sum\limits_{k=1}^m p_{2k}$ & $...$ & $\sum\limits_{k=1}^m p_{nk}$ \\
            \hline
        \end{tabular} &
        \begin{tabular}{|c|c|c|c|c|}
            \hline
            $\xi_2$ & $y_1$ & $y_2$ & $...$ & $y_m$ \\
            \hline
            $p$ & $\sum\limits_{k=1}^n p_{k1}$ & $\sum\limits_{k=1}^n p_{k2}$ & $...$ & $\sum\limits_{k=1}^n p_{km}$ \\
            \hline
        \end{tabular}
    \end{tabular}
}\hfil}

\begin{remark}
    Обчислити таблицю розподілу, знаючи ряди розподілу координат, можна лише у випадку їх незалежності.
\end{remark}

\subsection{Сумісна функція розподілу двовимірного випадкового вектора}
\begin{definition} 
    \emph{Сумісною функцією розподілу} двовимірного випадкового вектора $\vec{\xi}$ 
    називається $F_{\vec{\xi}}(x, y) = P\left\{\omega: \xi_1 < x, \xi_2 < y\right\}$, $x, y \in \mathbb{R}$.
\end{definition}

\noindent\textbf{Геометрична інтерпретація: }
Значення $F_{\vec{\xi}}(x, y)$ рівне імовірності 
потрапляння точки в заштриховану зону.

\hbox to \hsize{\hfil{
    \begin{tikzpicture}
        \draw [->] (-0.5, 0) -- (3, 0);
        \draw [->] (0, -0.5) -- (0, 3);
        \draw [fill] (2, 2) circle [radius=0.05];
        \node [above right] at (2, 2) {$(x, y)$};
        \path [pattern=north west lines, 
                pattern color=gray] (2, 2) rectangle (-0.55, -0.51);
        \draw [dashed] (-0.5, 2) -- (2, 2) -- (2, -0.5);
        \node [below] at (3, 0) {$x$};
        \node [above right] at (0.2, 0.2) {$F_{\vec{\xi}}(x, y)$};
        \node [left] at (0, 3) {$y$};
    \end{tikzpicture}
}\hfil}

\noindent\textbf{Властивості: }
\begin{enumerate}
    \item $D(F_{\vec{\xi}}) = \mathbb{R}^2$, $E(F_{\vec{\xi}}) = 
    \left<0; 1\right>$.
    \item Монотонно неспадна по кожній з координат.
    \begin{proof}
        $A = \left\{\xi_1 < x_2,\;\xi_2 < y \right\}$, 
        $B = \left\{x_2 \leq \xi_1 < x_1,\; \xi_2<y\right\}$, 
        $C = \left\{\xi_1 < x_1,\; \xi_2 < y\right\}$.

        $C = A \cup B \Rightarrow P(C) = P(A) + P(B)$
    
        $P(C) = F_{\vec{\xi}}(x_1, y)$,  
        $P(A) = F_{\vec{\xi}}(x_2, y)$

        $F_{\vec{\xi}}(x_1, y) = F_{\vec{\xi}}(x_2, y) + P(B)
        \Rightarrow F_{\vec{\xi}}(x_1, y) \geq  F_{\vec{\xi}}(x_2, y)$. 
        
        Для другої координати аналогічно.
    \end{proof}
    \item $\lim\limits_{x \to -\infty} F_{\vec{\xi}}(x, y) = 
           \lim\limits_{y \to -\infty} F_{\vec{\xi}}(x, y) = 
           \lim\limits_{x,y \to -\infty} F_{\vec{\xi}}(x, y) = 0$.
    \begin{proof}
        Строге доведення ґрунтується на теоремах про неперервність 
        ймовірності.

        $\lim\limits_{x \to -\infty} F_{\vec{\xi}}(x, y) = 
        \lim\limits_{x \to -\infty} P\left\{\xi_1<x,\;\xi_2<y\right\} 
        = [\text{використаємо теорему про неперервність}$ 
        $\text{ймовірності}] = P(\varnothing \cap \{\xi_2 < y\})
        = P(\varnothing) = 0$.
        Інші випадки --- аналогічно.
    \end{proof}
    \item \emph{<<Умови узгодженості>>}:
    $\lim\limits_{x \to +\infty} F_{\vec{\xi}}(x, y) = F_{\xi_2}(y)$, 
    $\lim\limits_{y \to +\infty} F_{\vec{\xi}}(x, y) = F_{\xi_1}(x)$.
    \begin{proof}
        $\lim\limits_{x \to +\infty} F_{\vec{\xi}}(x, y) = 
        \lim\limits_{x \to +\infty} 
        P\left(\xi_1 < x,\; \xi_2 < y\right) = 
        P\left(\xi_1 < +\infty,\;\xi_2<y\right) = $
        
        $= P\left(\xi_2<y\right) = F_{\xi_2}(y)$. Інша --- аналогічно.
    \end{proof}
    \item $\lim\limits_{x,y \to +\infty} F_{\vec{\xi}}(x, y) = 1$.
    \begin{proof}
        Випливає з умов узгодженості.
    \end{proof}
    \item Функція розподілу є неперервною зліва по кожному з аргументів: 
    
    $\lim\limits_{x \to x_0 - 0} F_{\vec{\xi}}(x, y) = F_{\vec{\xi}}(x_0, y)$, 
    $\lim\limits_{y \to y_0 - 0} F_{\vec{\xi}}(x, y) = F_{\vec{\xi}}(x, y_0)$.
    \begin{proof}
        Доведення аналогічне одновимірному випадку (твердж. \refeq{2_1_con}, 
        с. \pageref{2_1_con}).
    \end{proof}
    \item $\Pi = [a; b)\times [c; d)$. $P\left\{\vec{\xi} \in \Pi \right\} = \text{?}$
    
    \begin{tabular}{c p{9cm}}
        \begin{tikzpicture}[baseline={(current bounding box.north)}]
            \draw [->] (-1, 0) -- (3, 0);
            \draw [->] (0, -1) -- (0, 2);
            \draw [very thick] (0.5,1.5) -- (0.5,0.5) -- (2.5,0.5);
            \draw [dashed, very thick] (2.5,0.5) -- (2.5,1.5) -- (0.5,1.5);
            \fill [black!10!white] (0.5,0.5) rectangle (2.5,1.5);
            \draw [dashed] (0.5, 0.5) -- (-1, 0.5);
            \draw [dashed] (0.5, 0.5) -- (0.5, -1);
            \draw [dashed] (0.5, 1.5) -- (-1, 1.5);
            \draw [dashed] (2.5, 0.5) -- (2.5, -1);
            \node [below right] at (0.5, 0) {$a$};
            \node [below left] at (2.5, 0) {$b$};
            \node [above left] at (0, 0.5) {$c$};
            \node [below left] at (0, 1.5) {$d$};
            \node [below] at (3, 0) {$x$};
            \node [left] at (0, 2) {$y$};
            \node [below left] at (2.5, 1.5) {$\Pi$};
        \end{tikzpicture} &
        Введемо події $A = \left\{\xi_1 < a, \xi_2 < c\right\}$, \newline
        $B = \left\{\xi_1 \in [a;b), \xi_2 < c\right\}$,
        $C = \left\{\xi_1 < a, \xi_2 \in [c;d)\right\}$, \newline
        $D = \left\{\vec{\xi} \in \Pi\right\}$,
        $E = \left\{\xi_1 < b, \xi_2 < d\right\}$.
        $E = A \cup B \cup C \cup D$, причому $A$, $B$, $C$, $D$ --- попарно несумісні, тому
        $P(E) = P(A) + P(B) + P(C) + P(D)$.
        $P(E) = F(b, d)$, $P(A) = F(a, c)$, $P(B) = F(b,c) - F(a,c)$,
        $P(C) = F(a, d) - F(a, c)$. 
        
        Отже, $P\left\{\vec{\xi} \in \Pi\right\} = F(b,d) + F(a,c) - F(b,c) - F(a,d)$.
    \end{tabular}
\end{enumerate}

\subsection{Побудова функції розподілу дискретного випадкового вектора}

Нехай двовимірний випадковий вектор  задано таблицею розподілу:
\begin{tabular}{|c|c|c|}
    \hline
    \diagbox{$\xi_2$}{$\xi_1$} & $1$ & $2$ \\
    \hline
    $0$ & $0.3$ & $0.4$ \\
    \hline
    $1$ & $0.2$ & $0.1$ \\
    \hline
\end{tabular}

Значення функції розподілу також зручно звести в таблицю:

\begin{tabular}{|c|c|c|c|}
    \hline
    \diagbox[height=2em, width=6em]{$y$}{$x$} & $x\leq1$ & $1<x\leq2$ & $x> 2$ \\
    \hline
    $y\leq0$ & $0$ & $0$ & $0$ \\
    \hline
    $0<y\leq 1$ & $0$ & $0.3$ & $0.7$ \\
    \hline
    $y>1$ & $0$ & $0.5$ & $1$ \\
    \hline
\end{tabular}

Останній рядок таблиці --- значення $F_{\xi_1} (x)$, 
останній стовпчик --- значення $F_{\xi_2} (y)$.

\subsection{Сумісна функція розподілу \texorpdfstring{$n$}{n}-вимірного вектора}
\begin{definition} 
    \emph{Сумісною функцією розподілу} $n$-вимірного випадкового вектора $\vec{\xi}$ 
    називається $F_{\vec{\xi}}(\vec{x}) = P\left\{\omega: \xi_1 < x_1, \xi_2 < x_2, ..., \xi_n < x_n\right\}$, $\vec{x} \in \mathbb{R}^n$.
\end{definition}
\noindent\textbf{Властивості: }
\begin{enumerate}
    \item Є монотонно неспадною по кожній координаті.
    \item Є неперервною зліва по кожній координаті.
    \item $\forall k = 1,...,n: ( x_k \rightarrow -\infty) \Rightarrow ( F_{\vec{\xi}}(\vec{x}) \rightarrow 0)$.
    \item $\lim\limits_{x_1, ..., x_n \rightarrow +\infty} F_{\vec{\xi}}(\vec{x}) = 1$.
    \item \emph{<<Умови узгодженості>>}: $\forall k = 1,...,n: F_{\xi_k}(x) = F_{\vec{\xi}}(+\infty, ..., +\infty, x_k, +\infty, ..., +\infty)$,
    $\forall k<n: F_{\xi_1 \xi_2 ... \xi_k}(x_1, x_2, ..., x_k) = F_{\vec{\xi}}(x_1, x_2, ..., x_k, +\infty, ..., +\infty)$.
\end{enumerate}
\begin{exercise}
    Довести ці властивості.
\end{exercise}
        % !TEX root = ../main.tex
\subsection{Неперервні випадкові вектори}

\begin{definition}\index{випадковий вектор!неперервний}
    Вимірна функція $\vec{\xi}(\omega): \Omega \rightarrow \mathbb{R}^n$ називається 
    \emph{неперервним випадковим вектором}, якщо координати $\xi_i$ є 
    неперервними випадковими величинами для $i = 1,...,n$.
\end{definition}
Як наслідок, сумісна функція розподілу $F_{\vec{\xi}}\left(\vec{x}\right) = 
\P\left\{\omega:\xi_1(\omega)<x_1,...,\xi_n(\omega)<x_n\right\}$ є неперервною за кожним аргументом.
Введемо аналог щільності розподілу спочатку для двовимірних випадкових векторів.
\begin{definition}\index{щільність розподілу!сумісна}
    \emph{Щільністю розподілу (сумісною)} двовимірного випадкового вектора називається 
    подвійна границя
    \begin{equation}
        f_{\vec{\xi}}(x, y) = \lim_{\substack{\Delta x \to 0 \\ 
        \Delta y \to 0}} \frac{\P\left\{\vec{\xi} \in \left[x; x+\Delta x\right) \times \left[y; y+\Delta y\right)\right\}}
        {\Delta x \Delta y}
    \end{equation}
\end{definition}
\begin{remark}
    В означенні замість прямокутника можна брати будь-яку обмежену замкнену множину  
    $D$ і тоді
    \begin{equation*}
        f_{\vec{\xi}}(x, y) = \lim_{S(D) \to 0} 
        \frac{\P\left\{\vec{\xi} \in D\right\}}
        {S(D)}\text{, де } S(D) \text{ --- площа множини } D
    \end{equation*}
\end{remark}

Як і у випадку неперервних випадкових величин, можна отримати
\emph{зв'язок щільності розподілу з сумісною функцією розподілу}\index{функція розподілу!сумісна}:
\begin{gather*}
    \lim_{\substack{\Delta x \to 0 \\ 
\Delta y \to 0}} \frac{\P\left\{\vec{\xi} \in \Pi\right\}}
{\Delta x \Delta y} = 
\lim_{\substack{\Delta x \to 0 \\ \Delta y \to 0}} 
\left(
    \frac{F_{\vec{\xi}}(x+\Delta x, y+\Delta y) - F_{\vec{\xi}}(x, y+\Delta y)}
    {\Delta x \Delta y}
    -
    \frac{F_{\vec{\xi}}(x+\Delta x, y) - F_{\vec{\xi}}(x, y)}
    {\Delta x \Delta y}
\right) = \\
= \left[\; \text{формула Лагранжа про скінченні прирости}, \theta_1, \theta_2 \in (0, 1) \;\right] = \\
= \lim_{\substack{\Delta x \to 0 \\ \Delta y \to 0}} 
\left(
    \frac{\frac{\partial F}{\partial x}(x + \theta_1 \Delta x, y + \Delta y)\Delta x}
    {\Delta x \Delta y}
    -
    \frac{\frac{\partial F}{\partial x}(x + \theta_2 \Delta x, y)\Delta x}
    {\Delta x \Delta y}
\right) =  \\
= \left[\; \text{формула Лагранжа, }\theta_3, \theta_4 \in (0, 1)\;\right] 
= \lim_{\substack{\Delta x \to 0 \\ \Delta y \to 0}}
\frac{\frac{\partial^2 F}{\partial x \partial y}(x + \theta_3 \Delta x, 
y + \theta_4 \Delta y)\Delta y}
{\Delta y} = \frac{\partial^2 F_{\vec{\xi}}}{\partial x \partial y}(x, y)\end{gather*}

Остання рівність можлива, якщо друга похідна $\frac{\partial^2 F_{\vec{\xi}}}{\partial x \partial y}$
є неперервною.
Таким чином, якщо існує неперервна друга похідна $\frac{\partial^2 F_{\vec{\xi}}}{\partial x \partial y}$, то
\begin{equation}\label{eq:dens_r2}
    f_{\vec{\xi}}(x, y) = \frac{\partial^2 F_{\vec{\xi}}}{\partial x \partial y}(x, y)
\end{equation}

\begin{definition}\index{поверхня розподілу}
    Поверхня, що є графіком щільності двовимірного випадкового вектора, називається 
    \emph{поверхнею розподілу}.
\end{definition}
\begin{definition}
    Лінії, де $f_\xi(x, y) = const$, називаються \emph{лініями рівних ймовірностей}.
\end{definition}
\begin{definition}\index{щільність розподілу!маргінальна}
    Щільності розподілу окремих координат називаються 
    \emph{маргінальними щільностями}.
\end{definition}

\noindent\textbf{Властивості щільності розподілу:}
\begin{enumerate}
    \item $f_{\vec{\xi}}(x, y) \geq 0$.
    \item З формули \eqref{eq:dens_r2} $F_{\vec{\xi}}(x, y) 
    = \int\limits_{-\infty}^x \int\limits_{-\infty}^y f_{\vec{\xi}}(t, s) 
    dt ds $.
    \item \emph{Умова нормування:} \index{умова!нормування}: 
    $\int\limits_{-\infty}^{+\infty} \int\limits_{-\infty}^{\infty} f_{\vec{\xi}}(t, s) 
    dt ds = 1$ --- об'єм під поверхнею розподілу 
    дорівнює 1, що випливає з $\lim\limits_{x,y \to +\infty} F_{\vec{\xi}}(x, y) = 1$.
    \item З умов узгодженості $F_{\xi_1}(x) = \int\limits_{-\infty}^{x} \int\limits_{-\infty}^{+\infty} 
    f_{\vec{\xi}}(t, s) dt ds$,
    $F_{\xi_2}(y) = \int\limits_{-\infty}^{+\infty} \int\limits_{-\infty}^{y} 
    f_{\vec{\xi}}(t, s) dt ds$.
    \item З попередньої властивості диференціюванням можна отримати маргінальні щільності $f_{\xi_1}(x) = F'_{\xi_1}(x) 
    = \int\limits_{-\infty}^{+\infty} f_{\vec{\xi}}(x, s) ds $,
    $f_{\xi_2}(y) = F'_{\xi_2}(y) 
    = \int\limits_{-\infty}^{+\infty} f_{\vec{\xi}}(t, y) dt $.
    
    \item Якщо $D$ --- замкнена обмежена область в $\mathbb{R}^2$, то 
    $\P\left\{ \vec{\xi} \in D \right\} = \iint\limits_D f_{\vec{\xi}}(x, y) 
    dx dy$.
    \item Якщо координати випадкового вектора $\xi_1$ та $\xi_2$ незалежні, то
    $F_{\vec{\xi}}(x, y) = 
    F_{\xi_1}(x)\cdot F_{\xi_2}(y)$, і тоді
    $f_{\vec{\xi}}(x, y) = \frac{\partial^2 F_{\vec{\xi}}(x, y)}
    {\partial x \partial y} = \frac{\partial^2(F_{\xi_1}(x)F_{\xi_2}(y))}{\partial x \partial y} 
    = F_{\xi_1}^\prime (x)F_{\xi_2}^\prime (y) = 
    f_{\xi_1}(x)\cdot f_{\xi_2}(y)$.
\end{enumerate}
\vspace{1em}
Розглянемо важливий приклад неперервного випадкового вектора.
\begin{definition}\index{розподіл!рівномірний}
    Нехай $D$ --- замкнена обмежена область в $\mathbb{R}^2$,
    $S(D)$ --- її площа. 
    Вектор $\vec{\xi} = (\xi_1,\xi_2)^T$ називається 
    \emph{рівномірно розподіленим в області D} ($\vec{\xi} \sim \mathrm{U}(D)$), якщо 
    \begin{equation*}
        f_{\vec{\xi}}(x, y) = 
        \begin{cases}
            \frac{1}{S(D)},&(x, y) \in D \\
            0,&(x, y) \notin D
        \end{cases}
    \end{equation*}
\end{definition}
Зручною властивістю такого рівномірного розподілу є те, що
ймовірність потрапляння у підмножини $D$ пропорційна їх площі.
Більш строго, $\P\left\{\vec{\xi}\in A \right\} = \iint\limits_A f_{\vec{\xi}}(x, y) dx dy = 
\frac{S(A \cap D)}{S(D)}$.

На прикладі рівномірного розподілу нескладно розглянути приклад побудови
сумісної функції розподілу,
маргінальних щільностей та функцій розподілу.
\begin{example}
    Нехай $D$ --- трикутник з вершинами в точках $(0,0)$, $(1,0)$ та $(0,1)$,
    в якому рівномірно розподілений вектор $\vec{\xi}$:
    \begin{center}
        \begin{tabular}{c m{4cm}}
            $f_{\vec{\xi}}(x, y) = 
            \begin{cases}
                2, & (x, y) \in D \\
                0, & (x, y) \notin D
            \end{cases}
            $
            &
            \begin{tikzpicture}[scale = 1.5]
                \fill [lightgray] (0, 0) -- (1, 0) -- (0, 1);
                \draw [->] (0, -0.5) -- (0, 1.5);
                \draw [->] (-0.5, 0) -- (1.5, 0);
                \draw (1, 0) -- (0, 1);
                \node [below] at (1.5, 0) {$x$};
                \node [left] at (0, 1.5) {$y$};
                \node [above right] at (0.15, 0.15) {$D$};
                \node [right] at (0.6, 0.6) {$y = 1 - x$};
                \node [below] at (1, 0) {$1$};
                \node [left] at (0, 1) {$1$};
                \node [below left] at (0, 0) {$0$};  
            \end{tikzpicture}
        \end{tabular}
    \end{center}
    Знайдемо маргінальні щільності, функції розподілу та сумісну функцію розподілу.

    \begin{enumerate}
        \item $f_{\xi_1}(x) = \int\limits_{-\infty}^{+\infty} f_{\vec{\xi}}(x, y) dy = 
        \begin{cases}
            0 , &x\leq0 \text{ або } x>1\\
            \int\limits_0^{1-x} 2 dy = 2(1-x), & 0 < x \leq 1 
        \end{cases}$

        $f_{\xi_2}(y) = \int\limits_{-\infty}^{+\infty} f_{\vec{\xi}}(x, y) dx = 
        \begin{cases}
            0 , &y\leq0 \text{ або } y>1\\
            \int\limits_0^{1-y} 2 dx = 2(1-y), & 0 < y \leq 1 
        \end{cases}$
        \item Функції розподілу координат: 
        
        $F_{\xi_1}(x) = \int\limits_{-\infty}^{x} f_{\xi_1}(t) dt = \begin{cases}
            0, & x\leq 0 \\
            2\int\limits_0^x (1-t) dt = 2(x-\frac{x^2}{2}) = 2x - x^2, & 0<x\leq 1 \\
            1, x>1
        \end{cases}$
        
        $F_{\xi_2}(y) = \int\limits_{-\infty}^{y} f_{\xi_2}(t) dt = \begin{cases}
            0, & y\leq 0 \\
            2y - y^2, & 0<y\leq 1 \\
            1, y>1
        \end{cases}$
        \item Сумісна функція розподілу будується за допомогою розбиття площини $\mathbb{R}^2$ на області, в яких
        функція розподілу має однаковий вигляд, оскільки для рівномірного розподілу ймовірність потрапляння в певну область пропорційна її площі.

        \begin{tabular}{p{6cm} p{6cm}}
            \begin{tikzpicture}[scale = 1.5, baseline={(current bounding box.north)}]
                \fill [black!5] (-0.5, -0.5) -- (-0.5, 1.5) -- (0, 1.5) --
                               (0, 0) -- (1.5, 0) -- (1.5, -0.5) -- (-0.5, -0.5);
                \fill [black!10] (0, 0) -- (1, 0) -- (0, 1);
                \fill [black!15] (0, 1) -- (1, 0) -- (1, 1);
                \fill [black!25] (0, 1) rectangle (1, 1.5);
                \fill [black!20] (1, 0) rectangle (1.5, 1);
                \fill [black!30] (1, 1) rectangle (1.5, 1.5);
                \draw [->] (0, -0.5) -- (0, 1.5);
                \draw [->] (-0.5, 0) -- (1.5, 0);
                \draw (1, 0) -- (0, 1);
                \node [right] at (1.5, 0) {$t$};
                \node [above] at (0, 1.5) {$s$};
                \node [below left] at (0, 0) {$D_0$};
                \node [above right] at (0.15, 0.15) {$D_1$};
                \node [above right] at (0.5, 0.5) {$D_2$};
                \node at (0.5, 1.25) {$D_4$};
                \node at (1.25, 0.5) {$D_3$};
                \node at (1.25, 1.25) {$D_5$};
            \end{tikzpicture} 

            $(x, y) \in D_0$
            
            $D_0 = \left\{(x, y):\; x \leq 0 \lor y \leq 0\right\}$

            $F_{\vec{\xi}}(x, y) = 0$ 
            
            \vspace{5mm}
            
            $(x, y) \in D_1$
            
            $D_1 = \left\{(x, y):\; x \in \left[0, 1\right],
            y \in \left[0, 1-x\right]\right\}$

            $F_{\vec{\xi}}(x, y) = 2xy$&
            $(x, y) \in D_2$
            
            $D_2 = \left\{(x, y):\; x \in \left[0, 1\right],
            y \in \left[1-x, 1\right]\right\}$

            $F_{\vec{\xi}}(x, y) = 2(xy - \frac{1}{2}(x-1+y)(y-1+x))$

            \vspace{5mm}

            $(x, y) \in D_3$
            
            $D_3 = \left\{(x, y):\; x > 1,
            y \in \left[0, 1\right]\right\}$

            $F_{\vec{\xi}}(x, y) = 2\frac{1+1-y}{2}y = y(2-y)$
            
            \vspace{5mm}

            $(x, y) \in D_4$
            
            $D_4 = \left\{(x, y):\; y > 1,
            x \in \left[0, 1\right]\right\}$

            $F_{\vec{\xi}}(x, y) = 2\frac{1+1-x}{2}x = x(2-x)$
            
            \vspace{5mm}
            
            $(x, y) \in D_5$
            
            $D_5 = \left\{(x, y):\; x > 1 \land y > 1\right\}$

            $F_{\vec{\xi}}(x, y) = 1$\\
           
        \end{tabular}
    \end{enumerate}
\end{example}
\begin{remark}
    Перевірити правильність побудови функції розподілу можна за допомогою 
    умов узгодженості, перевірки точок стику та неперервності на лініях стику.
\end{remark}
\begin{exercise}
    Для яких областей $D$ координати випадкового вектора $\vec{\xi} \sim \mathrm{U}(D)$
    будуть незалежними? Які саме розподіли матимуть координати цього вектора?
\end{exercise}
Узагальнимо поняття щільності на випадок довільної розмірності.
\begin{samepage}
    \begin{definition}\index{щільність розподілу!сумісна}
        \emph{Щільністю розподілу (сумісною)} $n$-вимірного неперервного випадкового 
        вектора $\vec{\xi} = \left(\xi_1, \xi_2, ..., \xi_n\right)^T$ називається границя
        \begin{equation}
            f_{\vec{\xi}} (\vec{x}) = \lim_{\substack{\Delta x_i \to 0 \\
            \forall i = \overline{1,n}}} 
            \frac{\P\left\{\vec{\xi} \in 
            \left[x_1; x_1+\Delta x_1\right) \times ... \times 
            \left[x_n; x_n+\Delta x_n\right)\right\}}{\Delta x_1...\Delta x_n},
        \end{equation}
        Якщо існує неперервна похідна $\frac{\partial^n F_{\vec{\xi}}(\vec{x})}
        {\partial x_1 ... \partial x_n}$, то
        \begin{equation}\label{eq:dens_rn}
            f_{\vec{\xi}} (\vec{x}) = \frac{\partial^n F_{\vec{\xi}}}
            {\partial x_1 ... \partial x_n}(\vec{x})
        \end{equation}
    \end{definition}
\end{samepage}

\noindent \textbf{Властивості щільності:}
\begin{enumerate}
    \item $f_{\vec{\xi}}(\vec{x}) \geq 0$.
    \item З формули \eqref{eq:dens_rn} $F_{\vec{\xi}}(\vec{x}) = 
    \int\limits_{-\infty}^{x_1}...\int\limits_{-\infty}^{x_n}
    f_{\vec{\xi}} (\vec{t})dt_1...dt_n$.
    \item $\int\limits_{-\infty}^{+\infty}...\int\limits_{-\infty}^{+\infty}
    f_{\vec{\xi}} (\vec{x})dx_1...dx_n = 1$.
    \item Для кожного $I = \left\{i_1, i_2, ..., i_k\right\} \subset \{1,...,n\}$
    позначимо $\Pi = \left\{t_{i} < x_{i} , i \in I \text{ та } t_i \in \mathbb{R}, i \notin I\right\}$.
    Тоді $F_{\xi_{i_1} \xi_{i_2} ... \xi_{i_k}}(x_{i_1}, x_{i_2}, ..., x_{x_k}) = \int_{\Pi} f_{\vec{\xi}}(\vec{t}) d\vec{t}$.
    Зокрема, 
    $F_{\xi_i}(x_i) = \int\limits_{-\infty}^{+\infty}...
    \int\limits_{-\infty}^{x_i}...\int\limits_{-\infty}^{+\infty}
    f_{\vec{\xi}} (\vec{t})d\vec{t}$.

    \item Для кожного $I = \left\{i_1, i_2, ..., i_k\right\} \subset \{1,...,n\}$ $f_{\xi_{i_1}...\xi_{i_k}}(x_{i_1}, ..., x_{i_k})$ отримується
    інтегруванням $f_{\vec{\xi}}(\vec{x})$ по <<зайвим>> координатам від $-\infty$ до $+\infty$. Наприклад, 
    $$f_{\xi_1 ... \xi_k} (x_1, ..., x_k) = 
    \underbrace{
        \int\limits_{-\infty}^{+\infty} 
        ... 
        \int\limits_{-\infty}^{+\infty}
    }_{n-k} f_{\vec{\xi}}(x_1,...,x_k, t_{k+1}, ..., t_n) dt_{k+1}..dt_n$$
    \item Якщо $D$ --- замкнена обмежена область в $\mathbb{R}^n$, то
    $\P\left\{\vec{\xi} \in D\right\} = \int_D f_{\vec{\xi}}(\vec{x})
    dx_1 ... dx_n$.
    \item Якщо координати $\xi_1, ..., \xi_n$ незалежні у сукупності, то 
    $f_{\vec{\xi}} (x_1, ..., x_n) = f_{\xi_1}(x_1) \cdot ... \cdot f_{\xi_n}(x_n)$.
\end{enumerate}
\begin{exercise}
    Довести властивості щільності.
\end{exercise}
        % !TEX root = ../main.tex
\section{Числові характеристики випадкових векторів}

\subsection{Математичне сподівання випадкових векторів}
$\vec{\xi} = \left(\xi_1, ..., \xi_n\right)^T$
\begin{definition}
    \emph{Математичним сподіванням $E_{\vec{\xi}}$} 
    випадкового вектора $\vec{\xi}$ називається вектор 
    $\left(E{\xi_1}, ..., E{\xi_n}\right)^T$.
\end{definition}
\begin{remark}
    Математичне сподівання двовимірного випадкового вектора 
    називається \emph{центром розсіювання}.
\end{remark}

\noindent \textbf{Способи знаходження}

Знаходимо закони розподілу окремих координат, з цього - 
математичні сподівання координат і об'єднуємо їх в один вектор.

\begin{equation}
    E{\xi_i} = 
    \underbrace{
        \int\limits_{-\infty}^{+\infty} 
        ... 
        \int\limits_{-\infty}^{+\infty}
    }_{n} x_i f_{\vec{\xi}}(\vec{x})dx_i...dx_n 
    \;\;\;\;
    \forall i = \overline{1,n}
\end{equation}

\subsection{Мішані початкові та центральні 
            моменти випадкових векторів}
\begin{definition}
    \emph{Мішаним початковим моментом} порядку 
    $k_1+k_2+...+k_n$ 
    $\left(k_i \in \mathbb{N} \; \forall i = \overline{1,n}
    \right)$
    випадкового вектора 
    $\vec{\xi} = \left(\xi_1, ..., \xi_n\right)^T$
    називається число
    \begin{equation*}
        \alpha_{k_1+k_2+...+k_n} = 
        E{\xi_1^{k_1}...\xi_n^{k_n}}
    \end{equation*}
\end{definition}
\begin{remark}
    Зауважимо, що
    \begin{center}
        \begin{tabular}{c}
            $E\xi_1 = \alpha_{1+0+...+0}$ \\
            $E\xi_2 = \alpha_{0+1+...+0}$ \\
            $...$ \\
            $E\xi_n = \alpha_{0+0+...+n}$
        \end{tabular}
    \end{center}
\end{remark}
        % !TEX root = ../main.tex

\section{Умовні закони розподілу ВВ}

Розглядаємо випадок $n = 2$, $\vec{\xi} = \left(\xi_1, \xi_2\right)^T$.

\begin{definition}
    \emph{Умовним законом розподілу} $\xi_1$ називається закон розподілу 
    $\xi_1$ за умови того, що $\xi_2$ набула відповідного значення (ДВВ) 
    або потрапила в деякий проміжок (НВВ) (для $\xi_2$ аналогічно).
\end{definition}

\begin{definition}
    Універсальним умовним законом розподілу ВВ є \emph{умовна функція 
    розподілу}:
    \begin{equation*}
        F_{\xi_1}(x/y) = \P({\xi_1 < x}/{\xi_2 < y}) = 
        \frac{\P\left\{\xi_1 < x, \xi_2 < y\right\}}
        {\P\left\{\xi_2 < y\right\}} = 
        \frac{F_{\vec{\xi}}(x, y)}{F_{\xi_2}(y)}, F_{\xi_2}(y/x) = \frac{F_{\vec{\xi}}(x, y)}{F_{\xi_1}(x)}
    \end{equation*}
\end{definition}
\begin{remark}
    З означення випливає, що 
    $F_{\vec{\xi}}(x,y) = F_{\xi_1}(x)F_{\xi_2}(y/x)$
    та 
    $F_{\vec{\xi}}(x,y) = F_{\xi_2}(y)F_{\xi_1}(x/y)$.
\end{remark}
\subsection{Умовний закон розподілу дискретного випадкового вектора}
Знову розглядаємо випадок $n=2$ та $\vec{\xi} = (\xi_1, \xi_2)^T$.

Умовні розподіли координат задаються $\P\left\{\xi_1 = x_i / \xi_2 = y_j\right\} = 
\frac{\P\left\{\xi_1 = x_i , \xi_2 = y_j\right\}}
{\P\left\{\xi_2 = y_j\right\}}$ та \\ 
$\P\left\{\xi_2 = y_j / \xi_1 = x_i\right\} = 
\frac{\P\left\{\xi_1 = x_i , \xi_2 = y_j\right\}}
{\P\left\{\xi_1 = x_i\right\}}$.

\begin{example}
    Дискретний випадковий вектор має закон розподілу:

    \begin{tabular}{ccc}
        \begin{tabular}{|c|c|c|c|}
            \hline
            \diagbox{$\xi_2$}{$\xi_1$} & $-1$ & $0$ & $1$\\
            \hline
            0 & $0.1$ & $0.2$ & $0.1$ \\
            \hline
            $1$ & $0.2$ & $0.3$ & $0.1$ \\
            \hline
        \end{tabular}
        &
        \begin{tabular}{|c|c|c|c|}
            \hline
            $\xi_1$ & $-1$ & $0$ & $1$ \\
            \hline
            $p$ & $0.3$ & $0.5$ & $0.2$ \\
            \hline
        \end{tabular}
        &
        \begin{tabular}{|c|c|c|}
            \hline
            $\xi_2$ & $0$ & $1$ \\
            \hline
            $p$ & $0.4$ & $0.6$ \\
            \hline
        \end{tabular}
    \end{tabular}
\end{example}

\begin{tabular}{|c|c|c|c|}
    \hline
    $\xi_1$ & $-1$ & $0$ & $1$ \\
    \hline
    $\P\left\{\xi_1 / \xi_2 = 0\right\}$ & $1/4$ 
    & $2/4$ & $1/4$ \\
    \hline
    $\P\left\{\xi_1 / \xi_2 = 1\right\}$ & $2/6$ 
    & $3/6$ & $1/6$ \\
    \hline
\end{tabular}
--- умовний закон розподілу (умовний ряд розподілу) $\xi_1$.

\begin{tabular}{|c|c|c|}
    \hline
    $\xi_2$ & $0$ & $1$ \\
    \hline
    $\P\left\{\xi_2 / \xi_1 = -1\right\}$ & $1/3$ 
    & $2/3$ \\
    \hline
    $\P\left\{\xi_2 / \xi_1 = 0\right\}$ & $2/5$ 
    & $3/5$\\
    \hline
    $\P\left\{\xi_2 / \xi_1 = 1\right\}$ & $1/2$ 
    & $1/2$\\
    \hline
\end{tabular}
--- умовний закон розподілу (умовний ряд розподілу) $\xi_2$.

\subsection{Умовне математичне сподівання дискретного випадкового вектора}
\begin{definition}
    \emph{Умовним математичним сподіванням} випадкової величини $\xi_1$ 
    є математичне сподівання цієї випадкової величини за умови, що 
    $\xi_2$ набула певного значення.
    $$\E(\xi_1 / \xi_2 = y_j) =
\sum\limits_{i=1}^{n(\infty)}x_i 
\P\left\{\xi_1 = x_i / \xi_2 = y_j\right\}, 
\E(\xi_2 / \xi_1 = x_i) = \sum\limits_{j=1}^{n(\infty)}y_j 
\P\left\{\xi_2 = y_j / \xi_1 = x_i\right\}$$
\end{definition}

\begin{remark}
    Якщо $\xi_1$ та $\xi_2$ незалежні, то $\E(\xi_1 / \xi_2 = y_j) 
    = \E\xi_1$, $\E(\xi_2 / \xi_1 = x_i) = \E\xi_2$.
\end{remark}

Умовні математичні сподівання координат дискретного випадкового вектора є дискретними випадковими величинами, оскільки 
приймають декілька значень з певними ймовірностями, тому можна скласти 
їх закон розподілу.
\begin{example}
    Продовження попереднього прикладу:

    \begin{tabular}{c c}
        \begin{tabular}{|c|c|c|}
            \hline
            $\E(\xi_1 / \xi_2)$ & $-1/6$ & $0$ \\
            \hline
            $p$ & $0.6$ & $0.4$ \\
            \hline
        \end{tabular}
        &
        \begin{tabular}{|c|c|c|c|}
            \hline
            $\E(\xi_2 / \xi_1)$ & $1/2$ & $3/5$ & $2/3$ \\
            \hline
            $p$ & $0.2$ & $0.5$ & $0.3$ \\
            \hline
        \end{tabular}
    \end{tabular}
\end{example}

\subsection{Умовні закони розподілу неперервних випадкових величин}

У випадку $n = 2$, $\vec{\xi} = (\xi_1, \xi_2)^T$: 
\begin{equation*}
    \P\{x \leq \xi_1 < x + \Delta x / y \leq \xi_2 < y + \Delta y\} = 
    \frac{\P\{\vec{\xi} \in \Pi\}}{\P\{y \leq \xi_2 < y + \Delta y\}} = 
    \frac{f_{\vec{\xi}}(x,y)\Delta x \Delta y + 
    o(\sqrt{\Delta x^2 + \Delta y^2})}{f_{\xi_2}(y)\Delta y + o(\Delta y)}
\end{equation*}
\begin{definition}
    \emph{Умовною щільністю розподілу} називається функція 
    вигляду:
    \begin{equation*}
        f_{\xi_1}(x/y) = \frac{f_{\vec{\xi}}(x, y)}{f_{\xi_2}(y)}, f_{\xi_2}(y/x) = \frac{f_{\vec{\xi}}(x, y)}{f_{\xi_1}(x)}
    \end{equation*}
\end{definition}

\begin{remark}
    Графік умовної щільності розподілу можна інтерпретувати як лінію перетину 
    поверхні розподілу та площини $y = y_{\text{знач.}}$ або
    $x = x_{\text{знач.}}$ відповідно, нормовану на одиничну площу під нею.
\end{remark}

\noindent \textbf{Властивості умовної щільності:}
\begin{enumerate}
    \item $f_{\xi_1}(x / y) \geq 0, f_{\xi_2}(y / x) \geq 0$.
    \item $f_{\xi_1}(x/y) = \frac{f_{\vec{\xi}}(x,y)}
    {\int\limits_{-\infty}^{+\infty}f_{\vec{\xi}}(x,y)dx},
    f_{\xi_2}(y/x) = \frac{f_{\vec{\xi}}(x,y)}
    {\int\limits_{-\infty}^{+\infty}f_{\vec{\xi}}(x,y)dy}$.
    \item $\int\limits_{-\infty}^{+\infty} f_{\xi_1}(x / y) dx = 
    \int\limits_{-\infty}^{+\infty}\frac{f_{\vec{\xi}}(x,y)}
    {\int\limits_{-\infty}^{+\infty}f_{\vec{\xi}}(x,y)dx}dx = 1,
    \int\limits_{-\infty}^{+\infty} f_{\xi_2}(y / x) dy = 
    \int\limits_{-\infty}^{+\infty}\frac{f_{\vec{\xi}}(x,y)}
    {\int\limits_{-\infty}^{+\infty}f_{\vec{\xi}}(x,y)dy}dy = 1$.
\end{enumerate}
Властивості умовної щільності аналогічні властивостям звичайної (безумовної) щільності.

\subsection{Умовне математичне сподівання неперервного випадкового вектора}
\emph{Умовні математичні сподівання} координат НВВ визначаються через умовні щільності:

$$\E(\xi_1 / \xi_2 = y) = \int\limits_{-\infty}^{+\infty} xf_{\xi_1}(x/y)dx, \;
\E(\xi_2 / \xi_1 = x) = \int\limits_{-\infty}^{+\infty} yf_{\xi_2}(y/x)dy$$

\begin{definition}
    Функції $\varphi(y) = \E(\xi_1 / \xi_2 = y)$ та 
    $\psi(x) = \E(\xi_2 / \xi_1 = x)$ називаються \emph{лініями регресії}.
\end{definition}
\begin{example}
    \begin{tabular}{c m{4cm}}
        $f_{\vec{\xi}}(x, y) = 
        \begin{cases}
            2, & (x, y) \in D \\
            0, & (x, y) \notin D
        \end{cases}
        $
        &
        \begin{tikzpicture}[scale = 1.5]
            \draw [->] (0, -0.5) -- (0, 1.5);
            \draw [->] (-0.5, 0) -- (1.5, 0);
            \draw (0, 0) -- (1, 0) -- (1, 1) -- cycle;
            \draw [dashed] (0, 1) -- (1, 1);
            \fill [lightgray] (0, 0) -- (1, 0) -- (1, 1);
            \node [below] at (1.5, 0) {$x$};
            \node [left] at (0, 1.5) {$y$};
            \node [above right] at (0.45, 0.15) {$D$};
            \node [right] at (0, 0.75) {$y = x$};
            \node [below] at (1, 0) {$1$};
            \node [left] at (0, 1) {$1$};
            \node [below left] at (0, 0) {$0$};  
        \end{tikzpicture}
    \end{tabular}

    $f_{\xi_1}(x) = \begin{cases}
        \int\limits_0^x 2 dy = 2x, & x \in \left[ 0; 1\right] \\
        0, & x \notin \left[ 0; 1\right]
    \end{cases}$, 
    $f_{\xi_2}(y) = \begin{cases}
        \int\limits_y^1 2 dx = 2(1-y), & y \in \left[ 0; 1\right] \\
        0, & y \notin \left[ 0; 1\right]
    \end{cases}$.

    Запишемо умовні щільності: 

    $f_{\xi_1}(x/y) = \frac{f_{\vec{\xi}}(x, y)}{f_{\xi_2}(y)} = \begin{cases}
        \frac{1}{1-y}, & x \in \left[ y; 1\right], \\
        & y \in \left[ 0; 1\right)\\
        0, & \text{інакше}
    \end{cases}$,
    $f_{\xi_2}(y/x) = \frac{f_{\vec{\xi}}(x, y)}{f_{\xi_1}(x)} = \begin{cases}
        \frac{1}{x}, & y \in \left[ 0; x\right], \\
        & x \in \left( 0; 1\right] \\
        0, & \text{інакше}
    \end{cases}$.

    Умовні розподіли обох координат є рівномірними з параметрами $\left< y, 1\right>$ та $\left<0, x\right>$ відповідно.

    $\E\xi_1 = \int\limits_0^1 2x^2 dx = \frac{2}{3}$,
    $\E\xi_2 = \int\limits_0^1 2y(1-y) dy = \frac{1}{3}$.
    Знайдемо умовні математичні сподівання.

    $\E(\xi_1 / \xi_2 = y) = \int\limits_{-\infty}^{+\infty} x f_{\xi_1}(x/y)dx =
    \frac{1}{1-y} \int\limits_y^1 x dx = \frac{1}{1-y} \cdot \frac{1-y^2}{2} = \frac{1+y}{2}$, $y \in \left[ 0; 1\right)$.

    $\E(\xi_2 / \xi_1 = x) = \int\limits_{-\infty}^{+\infty} y f_{\xi_2}(y/x)dy = 
    \frac{1}{x} \int\limits_0^x y dy = \frac{1}{x} \cdot \frac{x^2}{2} = \frac{x}{2}$, $x \in \left( 0; 1\right]$.
\end{example}
\subsection{Формули повного математичного сподівання та дисперсії}
\noindent\textbf{Формула повного математичного сподівання.}
    $\E(\E(\xi_1 / \xi_2)) = \E\xi_1$, $\E(\E(\xi_2 / \xi_1)) = \E\xi_2$.
\begin{proof}
    Дискретний випадок:

    $\E(\E(\xi_1 / \xi_2)) = \sum\limits_{j = 1}^m 
    \left(
        \sum\limits_{i=1}^n x_i \P\{\xi_1 = x_i / \xi_2 = y_j\}
    \right) \P\{\xi_2 = y_j\} = 
    \sum\limits_{i=1}^n x_i \P\{\xi_1 = x_i\} = \E\xi_1$.

    Неперервний випадок:

    $\E(\E(\xi_1 / \xi_2)) = \int\limits_{-\infty}^{+\infty} 
    \left(
        \int\limits_{-\infty}^{+\infty} x f_{\xi_1}(x / y) dx
    \right) f_{\xi_2}(y) dy
    = \int\limits_{-\infty}^{+\infty} x f_{\xi_1}(x) dx = \E\xi_1$.

    Для $\xi_2$ --- аналогічно.
\end{proof}

%\begin{remark}
%    В англомовній літературі також використовується назва «Adam's law».
%\end{remark}
\begin{example}
    Продовження попередніх прикладів.
    \begin{enumerate}
        \item Дискретний випадок:
        
        $\E\xi_1 = -0.3 + 0.2 = -0.1$, $\E(\E(\xi_1 / \xi_2)) = -\frac{1}{6}
        \cdot0.6 = -0.1$.
    
        $\E\xi_2 = 0.6$, $\E(\E(\xi_2 / \xi_1)) = 0.3\cdot\frac{2}{3} + 
        \frac{1}{2}\cdot\frac{3}{5} + \frac{1}{2} \cdot\frac{2}{10} = 
        0.6$.
        \item Неперервний випадок:

        $\E(\E(\xi_1 / \xi_2)) = \int\limits_0^1 \frac{1+y}{2} \cdot 2(1-y) dy = \int\limits_0^1 (1-y^2) dy = \frac{2}{3} = \E\xi_1$,

        $\E(\E(\xi_2 / \xi_1)) = \int\limits_0^1 \frac{x}{2} \cdot 2x dx = \int\limits_0^1 x^2 dx = \frac{1}{3} = \E\xi_2$.
    \end{enumerate}
\end{example}


Крім умовного математичного сподівання розглядають
\emph{умовні дисперсії} --- міра розсіювання однієї випадкової величини 
за умови того, що інша набула певне значення.

\begin{definition}
    \emph{Умовною дисперсією} дискретної випадкової величини $\xi_1$ 
    є дисперсія цієї випадкової величини за умови, що 
    $\xi_2$ набула певного значення.
    $$\D(\xi_1 / \xi_2 = y_j) = 
    \E((\xi_1 - \E(\xi_1 / \xi_2 = y_j))^2/\xi_2 = y_j)$$
\end{definition}

\noindent\textbf{Формула повної дисперсії.}
    Нехай $\xi_1$, $\xi_2$ --- випадкові величини та $\D\xi_1 < +\infty$. 
    Тоді
    $\D\xi_1 = \E(\D(\xi_1/\xi_2)) + \D(\E(\xi_1 / \xi_2))$.

\begin{proof}
    $\D\xi_1 = \E\xi_1^2 - (\E \xi_1)^2 = 
    \E(\E(\xi_1^2 / \xi_2)) - (\E(\E(\xi_1/\xi_2)))^2 = 
    \E(\D(\xi_1/\xi_2) + (\E(\xi_1/\xi_2))^2) - (\E(\E(\xi_1/\xi_2)))^2 = 
    \E(\D(\xi_1 / \xi_2)) + (\E(\E(\xi_1/\xi_2)^2) - 
    (\E(\E(\xi_1 / \xi_2)))^2) = \E(\D(\xi_1/\xi_2)) + \D(\E(\xi_1/\xi_2))$.
\end{proof}


%\begin{remark}
%    В англомовній літературі також використовується назва \Eve's law».
%\end{remark}

\subsection{Випадок незалежних координат випадкового вектора}

Розглядаємо випадок $n=2$, $\vec{\xi} = \left(\xi_1, \xi_2\right)^T$.
Необхідною і достатньою умовою незалежності координат, як вже було розглянуто, є 
$F_{\vec{\xi}}(x, y) = F_{\xi_1}(x)\cdot F_{\xi_2}(y)$.

З цього випливає, що в разі незалежності координат $F_{\xi_1}(x/y) = F_{\xi_1}(x)$, 
$F_{\xi_2}(y/x) = F_{\xi_2}(y)$.

\noindent\textbf{Дискретний випадок: }

$\P\{\xi_1 = x_i, \xi_2 = y_j\} = \P\{\xi_1 = x_i\}\cdot \P\{\xi_2 = y_j\}$ 
$\forall \; i,j$.
З цього випливає, що:

\begin{enumerate}
    \item $\P\{\xi_1 = x_i / \xi_2 = y_j\} = 
    \P\{\xi_1 = x_i\}$.
    \item $\E(\xi_1 / \xi_2 = y_j) = \E\xi_1$,
    $\E(\xi_2 / \xi_1 = x_i) = \E\xi_2$.
\end{enumerate}

\noindent\textbf{Неперервний випадок: }

$f_{\vec{\xi}}(x, y) = f_{\xi_1}(x)\cdot f_{\xi_2}(y)$.
З цього випливає, що:

\begin{enumerate}
    \item $f_{\xi_1}(x/y) = f_{\xi_1}(x)$,    
    $f_{\xi_2}(y/x) = f_{\xi_2}(y)$.
    \item $\E(\xi_1 / \xi_2 = y) = \E\xi_1$, 
    $\E(\xi_2 / \xi_1 = x) = \E\xi_2$.
\end{enumerate}

Для $n$-вимірного випадку ($n > 2$):
$\forall i,j$: $F_{\xi_i\xi_j}(x_i, x_j) = F_{\xi_i}(x_i)\cdot F_{\xi_j}(x_j)$ --- лише
попарна незалежність.
Нагадаємо, що координати $\xi_1, \xi_2, ..., \xi_n$ є \emph{незалежними у сукупності} 
    тоді і тільки тоді, коли $F_{\vec{\xi}}(\vec{x}) = 
    \prod\limits_{k=1}^n F_{\xi_k}(x_k)$.
    З незалежності у сукупності випливає незалежність, а отже, некорельованість будь-якої пари координат.
    \chapter{Характеристичні функції. Гауссівські випадкові вектори}
        % !TEX root = ../main.tex
\section{Характеристичні функції випадкових величин}
\subsection{Поняття характеристичної функції}
\begin{definition}
    \emph{Характеристичною функцією} випадкової величини $\xi$
    називається комплекснозначна функція $\chi_\xi(t) = E\left(e^{it\xi}\right), t\in \mathbb{R}$.
    \begin{equation}\label{eq:char_func}
        \chi_\xi(t) = \int_{-\infty}^{+\infty} e^{itx} dF_\xi(x) = \begin{cases}
            \sum\limits_{k=1}^{n(\infty)} e^{itx_k} P\left\{\xi = x_k\right\}, & \xi \text{ --- ДВВ} \\
            \int\limits_{-\infty}^{+\infty} e^{itx} f_\xi(x)dx, & \xi \text{ --- НВВ}
        \end{cases}
    \end{equation}
\end{definition}
Інтеграл $\int\limits_{-\infty}^{+\infty} e^{itx} f_\xi(x)dx$ у курсі
гармонічного аналізу називається \emph{перетворенням Фур'є} функції $f_\xi(x)$.
Отже, у випадку НВВ $\xi$ $\chi_\xi(t)$ --- перетворення Фур'є щільності.
Також з гармонічного аналізу відомо, що за допомогою \emph{оберненого перетворення Фур'є}
можна відновити $f_\xi(x)$ за $\chi_\xi(t)$: $f_\xi(x) = \frac{1}{2\pi}\int\limits_{-\infty}^{+\infty} e^{-itx} \chi_\xi(t)dt$.

\subsection{Властивості характеристичної функції}
\begin{enumerate}
    \item $\chi_\xi(0) = 1$, оскільки $\int\limits_{-\infty}^{+\infty} f_\xi(x)dx = 1$.
    
    $\left|\chi_\xi(t)\right| \leq 1$, оскільки $\left|\int\limits_{-\infty}^{+\infty} e^{itx} dF_\xi(x)\right| \leq \int\limits_{-\infty}^{+\infty} \left|e^{itx}\right| dF_\xi(x) = \int\limits_{-\infty}^{+\infty} dF_\xi(x) = 1$.
    \item Характеристична функція є рівномірно неперервною.
    \begin{proof}
        $\left| \chi_\xi(t+h) - \chi_\xi(t) \right| = 
        \left| \int\limits_{-\infty}^{+\infty} \left( e^{i(t+h)x} - e^{itx}\right) dF_\xi(x) \right| \leq \\
        \leq \int\limits_{-\infty}^{+\infty} \left|e^{i(t+h)x} - e^{itx}\right| dF_\xi(x) = 
        \int\limits_{-\infty}^{+\infty} \left|e^{itx}\right|\left| e^{ihx} - 1\right| dF_\xi(x) = 
        \int\limits_{-\infty}^{+\infty} \left|e^{ihx} - 1\right| dF_\xi(x)$.

        $e^{ihx} - 1 = (\cos(hx) - 1) + i\sin(hx) = -2\sin^2(\frac{hx}{2}) + 2\sin(\frac{hx}{2})\cos(\frac{hx}{2})$, 
        
        $\left| e^{ihx} - 1\right| = 
        \sqrt{4\sin^4(\frac{hx}{2}) + 4\sin^2(\frac{hx}{2})\cos^2(\frac{hx}{2})} = 2\left| \sin(\frac{hx}{2})\right|$.

        $\int\limits_{-\infty}^{+\infty} \left|e^{ihx} - 1\right| dF_\xi(x) = 2 \int\limits_{-\infty}^{+\infty} \left| \sin(\frac{hx}{2})\right| dF_\xi(x) =
        \\\overset{A>0}{=} 2\cdot \left( \int\limits_{-\infty}^{-A}\left| \sin(\frac{hx}{2})\right| dF_\xi(x) + 
        \int\limits_{-A}^{A}\left| \sin(\frac{hx}{2})\right| dF_\xi(x) + 
        \int\limits_{A}^{+\infty}\left| \sin(\frac{hx}{2})\right| dF_\xi(x)\right)
        $.

        Інтеграл $\int\limits_{-A}^{A}\left| \sin(\frac{hx}{2})\right| dF_\xi(x)$
        можна зробити як завгодно малим за рахунок вибору $h$,
        а інші два --- за рахунок вибору $A$.
    \end{proof}
    \item $\eta = a\xi + b$ --- афінне перетворення $\xi$, $a, b \in \mathbb{R}$.

    $\chi_\eta(t) = E\left( e^{it\eta}\right) = E\left( e^{it(a\xi + b)}\right) = e^{itb}\cdot E\left( e^{ita\xi}\right) = e^{itb}\cdot \chi_\xi(at)$.
    \item Якщо $\xi_1, ..., \xi_n$ --- незалежні у сукупності, то
    $\chi_{\sum_{k=1}^n {\xi_k}} (t) = \prod\limits_{k=1}^n \chi_{\xi_k}(t)$.
    \begin{proof}
        $\chi_{\sum_{k=1}^n \xi_k}(t) = E\left( e^{it\sum_{k=1}^n \xi_k}\right) =
        E\left( \prod\limits_{k=1}^n e^{it\xi_k}\right) = \\
        = \left[e^{it\xi_k} \text{ --- теж незалежні у сукупності}\right] = 
        \prod\limits_{k=1}^n E\left( e^{it\xi_k}\right) = \prod\limits_{k=1}^n \chi_{\xi_k}(t)$.
    \end{proof}
    \item $\chi_\xi(t) = \int\limits_{-\infty}^{+\infty} e^{itx} dF_\xi(x)$.
    Похідні характеристичної функції $\chi_\xi^{(k)}(t) = \int\limits_{-\infty}^{+\infty} (ix)^k e^{itx} dF_\xi(x)$,
    причому $\chi_\xi^{(k)}(0) = \int\limits_{-\infty}^{+\infty} (ix)^k dF_\xi(x) = i^k \cdot E\xi^k$.
    
    Отже, $E\xi = -i \chi_\xi'(0)$, $D\xi = E\xi^2 - \left( E\xi\right)^2 = - \chi_\xi''(0) + \left( \chi_\xi'(0) \right)^2$.
    \item Зв'язок характеристичної функції та генератриси ДВВ.

    $\varphi_\xi(z) = \sum\limits_{k=0}^{\infty} P\left\{\xi = k\right\} z^k$,
    $\chi_\xi(t) = \sum\limits_{k=0}^{\infty} e^{itk} P\left\{\xi = x_k\right\}$. Отже, $\chi_\xi(t) = \varphi_\xi(e^{it})$.
    \item $\overline{\chi_\xi(t)} = E\left( e^{-it\xi}\right) = \chi_\xi(-t) = \chi_{-\xi} (t)$. 
    \item Для того, щоб характеристична функція була дійснозначною,
    необхідно і достатньо, щоб розподіл ВВ був симетричним відносно 0.
    \begin{proof}
        Розглянемо випадок НВВ. Нехай розподіл є симетричним відносно 0, тоді $f_\xi(x)$ --- парна.
        $\chi_\xi(t) = \int\limits_{-\infty}^{+\infty} e^{itx} f_\xi(x)dx = 
        \int\limits_{-\infty}^{+\infty} \cos(tx) f_\xi(x)dx +
        i\int\limits_{-\infty}^{+\infty} \sin(tx) f_\xi(x)dx = 
        \int\limits_{-\infty}^{+\infty} \cos(tx) f_\xi(x)dx = {Re}\chi_\xi(t)$, 
        інтеграл з синусом рівний 0, бо інтегрується непарна функція по симетричному проміжку.

        Нехай $\chi_\xi(t)$ --- дійснозначна, тоді $\chi_\xi(t) = \overline{\chi_\xi(t)} = \chi_\xi(-t)$.
        Отже, $\chi_\xi(t)$ --- парна, тоді з оберненого перетворення Фур'є $f_\xi(t)$ --- теж парна.
    \end{proof}
\end{enumerate}

\subsection{Необхідні умови того, що функція є характеристичною}
Нехай $\chi(t)$ --- деяка комплекснозначна функція дійсного аргументу.
Якщо вона є характеристичною функцією деякої ВВ, то для неї має виконуватися:
\begin{enumerate}
    \item $\chi(0) = 1, \left| \chi(t)\right| \leq 1$.
    \item $\chi(t)$ --- рівномірно неперервна.
    \item $\overline{\chi(t)} = \chi(-t)$.
\end{enumerate}
\begin{example}
    \begin{enumerate}
        \item $\chi(t) = \cos(t)$ --- може бути характеристичною.
        Оскільки $\cos(t) = \frac{1}{2}e^{it} + \frac{1}{2}e^{-it}$, то відповідна ВВ --- дискретна: 
        \begin{tabular}{c|c|c}
            $\xi$ & -1 & 1 \\
            \hline
            $p$ & $1/2$ & $1/2$
        \end{tabular}
        \item $\chi(t) = \sin(t)$ --- не може бути характеристичною, бо $\chi(0) = 0 \neq 1$.
        \item $\chi(t) = \cos^2(t)$ --- може бути характеристичною.
        $\cos(t) = \frac{1}{2} + \frac{1}{2}\cos(2t) = \frac{1}{2} + \frac{1}{4}e^{2it} + \frac{1}{4}e^{-2it}$, тому
        відповідна ВВ --- дискретна:
        \begin{tabular}{c|c|c|c}
            $\xi$ & -1 & 0 & 1 \\
            \hline
            $p$ & $1/4$ & $1/2$ & $1/4$
        \end{tabular}
        \item $\chi(t) = \frac{\alpha^2}{\alpha^2 + t^2}$ --- може бути характеристичною. 
        Знайдемо щільність розподілу відповідної НВВ за формулою $f_\xi(x) = \frac{1}{2\pi}\int\limits_{-\infty}^{+\infty} \frac{\alpha^2 \cdot e^{-itx}}{\alpha^2 + t^2} dt$.

        При $x<0$: $\int\limits_{-\infty}^{+\infty} \frac{e^{-itx}}{\alpha^2 + t^2} dt = 2\pi i \cdot \underset{z=\alpha i}{{Res}} \frac{e^{-izx}}{\alpha^2 + z^2} = 2\pi i \cdot \frac{e^{-i\cdot \alpha i\cdot x}}{2\alpha i} = \pi \cdot \frac{e^{\alpha x}}{\alpha}$.

        При $x>0$ аналогічно отримуємо $\pi \cdot \frac{e^{-\alpha x}}{\alpha}$.
        Остаточно $f_\xi(x) = \frac{1}{2\alpha}e^{-\alpha |x|}$ --- це щільність розподілу Лапласа.
    \end{enumerate}
\end{example}

\begin{exercise}
    Нехай $\chi(t)$ --- характеристична функція. Чи можуть бути характеристичними функції $\overline{\chi}$, $\chi^2$, $|\chi|^2$, ${Re}\chi$, $|\chi|$, ${Im}\chi$? 
\end{exercise}

Необхідні та достатні умови того, що функція є характеристичною, дає
\emph{теорема Бохнера-Хінчина}: $\chi(t)$ має бути неперервною, $\chi(0) = 1$ та 
невід'ємно визначеною: $$
    \forall \; t_1, t_2, ..., t_n \in \mathbb{R}, c_1, c_2, ..., c_n \in \mathbb{C}: 
    \sum\limits_{k=1}^n {\sum\limits_{m=1}^n \chi(t_k - t_m) c_k \overline{c_m}} \geq 0
$$

\subsection{Характеристичні функції деяких розподілів}
\begin{enumerate}
    \item $\xi \sim {Bin}(n,p)$ --- біноміальний закон.
    $\varphi_\xi(z) = (pz+q)^n \Rightarrow \chi_\xi(t) = \left( p e^{it} + q\right)^n$.
    \item $\xi \sim {Geom}(p)$ --- геометричний закон.
    $\varphi_\xi(z) = \frac{pz}{1-qz} \Rightarrow \chi_\xi(t) = \frac{p e^{it}}{1-q e{it}}$.
    \item $\xi \sim {Poiss}(a)$ --- закон Пуассона.
    $\varphi_\xi(z) = e^{a(z-1)} \Rightarrow \chi_\xi(t) = e^{a(e^{it}-1)}$.
    \item $\xi \sim {U}\left< a; b\right>$ --- рівномірний закон.

    $\chi_\xi(t) = \frac{1}{b-a} \int\limits_a^b e^{itx} dx = \frac{e^{itb} - e^{ita}}{it(b-a)}$.
    Зокрема, при $\xi \sim {U}\left< -a; a\right>$ $\chi_\xi(t) = \begin{cases}
        \frac{\sin(at)}{at}, & t \neq 0 \\
        1, & t = 0
    \end{cases}$.
    \item $\xi \sim {Exp}(\lambda)$ --- експоненційний закон.
    
    $\chi_\xi(t) = \lambda \int\limits_0^{+\infty} e^{itx} e^{-\lambda x} dx = \lambda \int\limits_0^{+\infty} e^{-(\lambda-it)x} dx = \frac{\lambda}{\lambda - it}$.
    \item $\xi \sim {N}(a, \sigma^2)$ --- нормальний закон. Розглянемо стандартний розподіл $\eta \sim {N}(0, 1)$,
    тоді $\xi = a + \sigma\cdot\eta$.

    $\chi_\eta(t) = \frac{1}{\sqrt{2\pi}} \int\limits_{-\infty}^{+\infty} e^{itx} e^{-x^2/2} dx = 
    \frac{1}{\sqrt{2\pi}} \int\limits_{-\infty}^{+\infty} e^{-x^2/2 + itx} dx = 
    \frac{1}{\sqrt{2\pi}} \int\limits_{-\infty}^{+\infty} e^{-\frac{1}{2}(x^2 - 2itx + i^2t^2) + \frac{1}{2}i^2t^2} dx = 
    \frac{1}{\sqrt{2\pi}} e^{-\frac{t^2}{2}} \int\limits_{-\infty}^{+\infty} e^{-\frac{(x-it)^2}{2}} dx = 
    \left[ x+it = u, dx = du\right] = e^{-\frac{t^2}{2}}$.
    Отже, $\chi_\xi(t) = e^{iat - \frac{\sigma^2 t^2}{2}}$.
\end{enumerate}

\begin{definition}
    \emph{Композицією законів розподілу} називається складання закону розподілу суми незалежних випадкових величин.
    
    Закон розподілу називається \emph{стійким по відношенню до операції додавання},
    якщо закон розподілу суми випадкових величин, розподілених за цим законом (в загальному випадку з різними параметрами),
    є таким самим.
\end{definition}
\begin{example}
    \begin{enumerate}
        \item Перевірити стійкість нормального розподілу. Нехай $\xi_1 \sim {N}(a_1, \sigma_1), ..., \xi_n \sim {N}(a_n, \sigma_n)$ ---
        незалежні у сукупності. 
    
        Знайдемо характеристичну функцію суми: $\chi_{\xi_1 + ... + \xi_n}(t) = \prod\limits_{k=1}^n \chi_{\xi_k}(t)=
        \prod\limits_{k=1}^n e^{ia_k t - \frac{\sigma_k^2 t^2}{2}} = 
        \exp\left\{i\cdot \left( \sum\limits_{k=1}^n a_k\right)\cdot t - \frac{t^2}{2}\cdot\left( \sum\limits_{k=1}^n \sigma_k^2\right)\right\}$ ---
        це характеристична функція нормального розподілу. Отже, $\sum\limits_{k=1}^n = \xi_1 + ... + \xi_n \sim {N}\left(\sum\limits_{k=1}^n a_k,  \sqrt{\sum\limits_{k=1}^n \sigma_k^2}\right)$.
        \item Перевірити стійкість біноміального розподілу. Нехай $\xi_1 \sim {Bin}(n_1, p_1)$, $\xi_2 \sim {Bin}(n_2, p_2)$ --- незалежні.
        $\chi_{\xi_1 + \xi_2} (t) = \left( p_1 e^{it} + q_1\right)^{n_1} \cdot \left( p_2 e^{it} + q_2\right)^{n_2} \overset{?}{=} \left(P e^{it} + Q\right)^N$.

        Взагалі кажучи, біноміальний розподіл не є стійким по відношенню до  операції додавання.
        Його називають \emph{умовно стійким} при $p_1 = p_2$, тоді
        $\chi_{\xi_1 + \xi_2} (t) = \left( p e^{it} + q\right)^{n_1 + n_2}$ і $\xi_1 + \xi_2 \sim {Bin}(n_1+n_2, p)$.
    \end{enumerate}
\end{example} 
        % !TEX root = ../main.tex

\section{Багатовимірний нормальний розподіл}
\subsection{Виведення характеристичної функції та щільності}

Розглянемо випадковий вектор $\vec{\xi} = (\xi_1, \xi_2, ..., \xi_n)^T$, $\xi_k \sim \mathrm{N}(a_k^o, \sigma_k)$ для $k=1,...,n$, координати якого незалежні у сукупності.
За властивостями характеристичної функції та щільності $\chi_{\vec{\xi}}(\vec{t}) = \prod\limits_{k=1}^n \chi_{\xi_k}(t_k)$,
$f_{\vec{\xi}}(\vec{x}) = \prod\limits_{k=1}^n f_{\xi_k}(x_k)$. Розпишемо детальніше:
\begin{gather*}
    \chi_{\vec{\xi}}(\vec{t}) = \prod\limits_{k=1}^n e^{i a_k^o t_k - \frac{\sigma_k^2 t_k^2}{2}} = \exp\left\{i(\vec{a^o}, \vec{t}) - \frac{1}{2}\sum\limits_{k=1}^n \sigma_k^2 t_k^2\right\}
    \\
    f_{\vec{\xi}}(\vec{x}) = \prod\limits_{k=1}^n \frac{1}{\sqrt{2\pi}\sigma_k} e^{-\frac{(x_k-a_k^o)^2}{2\sigma_k^2}} = \frac{1}{(2\pi)^{n/2}\prod_{k=1}^n \sigma_k} \exp \left\{ -\frac{1}{2} \sum_{k=1}^n \frac{(x_k-a_k^o)^2}{\sigma_k^2}\right\}
\end{gather*}

Введемо матрицю $D = \begin{pmatrix}
    \sigma_1^2 & 0 & \cdots & 0 \\
    0 & \sigma_2^2 & \cdots & 0 \\
    \vdots & \vdots & \ddots & \vdots \\
    0 & 0 & \cdots & \sigma_n^2
\end{pmatrix}$, $D^{-1} = \begin{pmatrix}
    1/\sigma_1^2 & 0 & \cdots & 0 \\
    0 & 1/\sigma_2^2 & \cdots & 0 \\
    \vdots & \vdots & \ddots & \vdots \\
    0 & 0 & \cdots & 1/\sigma_n^2
\end{pmatrix}$.

\noindent Тоді отримаємо
\begin{gather}
    \chi_{\vec{\xi}}(\vec{t}) = \exp\left\{i(\vec{a^o}, \vec{t}) - \frac{1}{2}(D\vec{t}, \vec{t})\right\}
    \\
    f_{\vec{\xi}}(\vec{x}) = \frac{1}{(2\pi)^{n/2} \sqrt{{\det{D}}}} \exp \left\{ -\frac{1}{2} \left( D^{-1}(\vec{x} - \vec{a^o}), (\vec{x} - \vec{a^o})\right)\right\}
\end{gather}
Зауважимо, що $D$ є кореляційною матрицею.

Тепер розглянемо вектор $\vec{\eta} = A\vec{\xi}$, де $A$ --- невироджена матриця. За властивістю 
$\chi_{\vec{\eta}}(\vec{t}) = \chi_{\vec{\xi}}(A^{*}\vec{t})$. Маємо
\begin{equation}
    \begin{gathered}
        \chi_{\vec{\eta}}(\vec{t}) = \exp\left\{i(\vec{a^o}, A^{*}\vec{t}) - \frac{1}{2}(DA^{*}\vec{t}, A^{*}\vec{t})\right\} =
    \exp\left\{i(A\vec{a^o}, \vec{t}) - \frac{1}{2}(ADA^{*}\vec{t}, \vec{t})\right\} = \\
    = \exp\left\{i(\vec{a}, \vec{t}) - \frac{1}{2}(K\vec{t}, \vec{t})\right\}
    \end{gathered}
\end{equation}
Доведемо, що $K = ADA^{*}$ --- симетрична й додатно визначена, тому її можна вважати кореляційною матрицею.
\begin{enumerate}
    \item $K^{*} = \left( ADA^{*}\right)^{*} = \left( A^{*}\right)^{*}DA^{*} = ADA^{*} = K$.
    \item $\forall \;\vec{u} \in \mathbb{R}^n$ $\left( K \vec{u}, \vec{u}\right) = \left( ADA^{*} \vec{u}, \vec{u}\right) =
    \left( DA^{*} \vec{u}, A^{*}\vec{u}\right) = \left[ A^{*}\vec{u} = \vec{v}\right] = \left( D\vec{v}, \vec{v}\right) > 0$ для $\vec{u}\neq \vec{0}$.
\end{enumerate}

Нехай в $n$-вимірному евклідовому просторі задано вектор $\vec{a}$ та симетричну додатно визначену матрицю $K$.
Існує ортогональне перетворення $U$, яке дає можливість записати $K=UDU^{*}$, причому на діагоналі $D$ стоять строго додатні
власні числа $K$. Тоді функцію вигляду $\exp\left\{i(\vec{a}, \vec{t}) - \frac{1}{2}(K\vec{t}, \vec{t})\right\}$ 
можна розглядати як характеристичну функцію випадкового вектора $\vec{\eta} = U \vec{\xi}$, де $\vec{\xi}$ ---
вектор, координати якого незалежні у сукупності та мають нормальний розподіл.

\begin{definition}
    $n$\emph{-вимірним гауссівським вектором} $\vec{\xi}$ називається випадковий вектор,
    характеристична функція якого має вигляд $\chi_{\vec{\xi}}(\vec{t}) = \exp\left\{i(\vec{a}, \vec{t}) - \frac{1}{2}(K\vec{t}, \vec{t})\right\}$,
    де $K$ --- кореляційна матриця, а $\vec{a} = \left( \E\xi_1, \E\xi_2, ..., \E\xi_n\right)^T$ --- центр розсіювання.
\end{definition}
\noindent \textbf{Позначення:} $\vec{\xi} \sim \mathrm{N}( \vec{a}, K)$. $\mathrm{N}( \vec{0}, {I})$ --- стандартний нормальний розподіл.
\begin{exercise}
    Перевірити, що $\vec{a}$ дійсно є центром розсіювання.
\end{exercise}
Знайдемо щільність сумісну розподілу такого вектору:
\begin{gather*}
    \P\left\{\vec{\eta} \in C \subset \mathbb{R}^n\right\} = \P\left\{U\vec{\xi}\in C\right\} = 
    \P\left\{\vec{\xi}\in U^{-1}C\right\} = 
    \int\limits_{U^{-1}C} f_{\vec{\xi}}(\vec{x}) d\vec{x} = \\
    \int\limits_{U^{-1}C} \frac{1}{(2\pi)^{n/2} \sqrt{{\det{D}}}} \exp \left\{ -\frac{1}{2} \left( D^{-1}(\vec{x} - \vec{a^o}), (\vec{x} - \vec{a^o})\right)\right\} d\vec{x} =
    \left[ \vec{x} - \vec{a^o} = U^{-1}(\vec{y} - \vec{a}) = \right. \\ \left. = U^{*}(\vec{y} - \vec{a}), d\vec{x} = d\vec{y}\;\right] = 
    \frac{1}{(2\pi)^{n/2} \sqrt{{\det{D}}}} \int\limits_{C} \exp \left\{ -\frac{1}{2} \left( D^{-1}(U^{*}\vec{y} - U^{*}\vec{a}), (U^{*}\vec{y} - U^{*}\vec{a})\right)\right\} d\vec{y} = \\
    = \left[ UD^{-1}U^{*} = K^{-1}\right] =
    \frac{1}{(2\pi)^{n/2} \sqrt{{\det{K}}}} \int\limits_{C} \exp \left\{ -\frac{1}{2} \left( K^{-1}(\vec{y} - \vec{a}), (\vec{y} - \vec{a})\right)\right\} d\vec{y}
\end{gather*}
Оскільки $\P\left\{\vec{\eta} \in C\right\} = \int\limits_{C} f_{\vec{\eta}}(\vec{y}) d\vec{y}$,
отримуємо \textbf{щільність розподілу} гауссівського вектора $\vec{\xi} \sim \mathrm{N}(\vec{a}, K)$:
\begin{equation}
    f_{\vec{\xi}}(\vec{x}) = \frac{1}{(2\pi)^{n/2} \sqrt{{\det{K}}}} \exp \left\{ -\frac{1}{2} \left( K^{-1}(\vec{x} - \vec{a}), (\vec{x} - \vec{a})\right)\right\}
\end{equation}
\begin{remark}
    Щільність розподілу існує тоді і лише тоді, коли $K$ невироджена, тобто є додатно визначеною.
\end{remark}

\subsection{Властивості гауссівських векторів}
\begin{enumerate}
    \item Якщо $\vec{\xi}$ --- гауссівський вектор, то всі його координати гауссівські,
    а будь-яка підсистема теж є гауссівським вектором.
    \begin{proof}
        $\chi_{\vec{\xi}}(0,0,...,t_j, ..., 0) = \chi_{\xi_j}(t_j) = 
        e^{i a_j t_j - \frac{1}{2}\sigma_j^2 t_j^2} \Rightarrow \xi_j \sim \mathrm{N}(a_j, \sigma_j^2)$.

        Аналогічно для будь-якої підсистеми $\vec{\eta} = (\xi_{k_1}, \xi_{k_2}, ..., \xi_{k_n})^T$.
    \end{proof}
    \suspend{enumerate}
    \begin{remark}
        Обернене твердження, взагалі кажучи, не має місця.

        Розглянемо випадковий вектор із щільністю
        $$ f_{\vec{\xi}}(x, y) = \frac{1}{2\pi} \left( 
            \left( \sqrt{2} e^{-x^2/2} - e^{-x^2}\right)e^{-y^2} + 
            \left( \sqrt{2} e^{-y^2/2} - e^{-y^2}\right)e^{-x^2}\right)$$

        Очевидно, це не щільність нормального розподілу. Знайдемо щільності розподілу координат $\xi_1$ та $\xi_2$.
        \begin{gather*}
            f_{\xi_1}(x) = \int\limits_{-\infty}^{+\infty}f_{\vec{\xi}}(x, y) dy =
            \frac{\sqrt{2}}{2\pi} e^{-x^2/2}\int\limits_{-\infty}^{+\infty}e^{-y^2}dy -
            \frac{1}{2\pi}e^{-x^2}\int\limits_{-\infty}^{+\infty}e^{-y^2}dy \; +\\
            + \; \frac{\sqrt{2}}{2\pi}e^{-x^2}\int\limits_{-\infty}^{+\infty} e^{-y^2/2}dy - 
            \frac{1}{2\pi} e^{-x^2}\int\limits_{-\infty}^{+\infty}e^{-y^2}dy =
            \frac{\sqrt{2}}{2\pi} e^{-x^2/2}\cdot \sqrt{\pi} - \frac{1}{2\pi}e^{-x^2} \cdot \sqrt{\pi} \; + \\
            + \; \frac{\sqrt{2}}{2\pi}e^{-x^2} \cdot \sqrt{2\pi} - \frac{1}{2\pi}e^{-x^2}\cdot \sqrt{\pi} = 
            \frac{1}{\sqrt{2\pi}}e^{-x^2/2} - \frac{1}{2\sqrt{\pi}}e^{-x^2} + \frac{1}{\sqrt{\pi}}e^{-x^2} - \frac{1}{2\sqrt{\pi}}e^{-x^2} \; = \\ 
            \\ = \; \frac{1}{\sqrt{2\pi}}e^{-x^2/2} \Rightarrow \xi_1 \sim \mathrm{N}(0, 1).
        \end{gather*}
        Аналогічно $\xi_2 \sim \mathrm{N}(0, 1)$. Отже, координати вектора мають нормальний розподіл, а сам вектор --- ні,
        оскільки його щільність не є гауссівською.
    \end{remark}
    \resume{enumerate}
    \item Для гауссівського випадкового вектора поняття незалежності та некорельованості координат є еквівалентними.
    \begin{proof}
        Було доведено, що з незалежності координат випливає ії некорельованість.

        Якщо координати некорельовані, то кореляційна матриця $K = \mathrm{diag}(\sigma_1^2, \sigma_2^2, ..., \sigma_n^2)$.
        Тоді $\left( K \vec{t}, \vec{t}\right) = \sum_{k=1}^n \sigma_k^2 t_k^2$.
        Тоді $\chi_{\vec{\xi}}(\vec{t}) = \exp\left\{i(\vec{a}, \vec{t}) - \frac{1}{2}\left( K \vec{t}, \vec{t}\right)\right\} = \prod\limits_{k=1}^n e^{i a_k t_k - \frac{1}{2}\sigma_k^2 t_k^2} = \prod\limits_{k=1}^n \chi_{\xi_k}(t_k)$
        і координати незалежні.
    \end{proof}
    \item В $n$-вимірному евклідовому просторі $\mathbb{R}^n$ завжди можна перейти до
    ортонормованого базису з власних векторів матриці $K$, в якому $K$ приймає діагональний вигляд.
    Отже, в базисі з власних векторів матриці $K$ координати відповідного гауссівського вектора стають незалежними.
    \item \emph{Афінне перетворення гауссівських векторів}.

    $\vec{\xi} \sim \mathrm{N}(\vec{a}, K)$. $A : \mathbb{R}^n \rightarrow \mathbb{R}^m$ --- лінійний оператор, заданий матрицею, $\vec{b} \in \mathbb{R}^m$, $\vec{\eta} = A\vec{\xi} + \vec{b}$.
    За властивістю характеристичної функції $\chi_{\vec{\eta}}(\vec{t}) = e^{i(\vec{b}, \vec{t})}\cdot\chi_{\vec{\xi}}(A^{*}\vec{t}) =
    \exp\left\{i(\vec{b}, \vec{t})\right\}\cdot \exp\left\{i(\vec{a}, A^{*}\vec{t}) - \frac{1}{2}\left( K A^{*}\vec{t}, A^{*}\vec{t}\right)\right\} =
    \exp\left\{i(A\vec{a} + \vec{b}, \vec{t}) - \frac{1}{2}\left( A K A^{*}\vec{t}, \vec{t}\right)\right\}$. 
    
    Отже, $\vec{\eta} \sim \mathrm{N}(A\vec{a} + \vec{b}, A K A^{*})$.
\end{enumerate}
\begin{example}
    \begin{enumerate}
        \item Задано вектор $\vec{\xi} \sim \mathrm{N}(\vec{a}, K)$. Знайти розподіл $\eta = \xi_1 + \xi_2 + ... + \xi_n$.

        $\eta = \begin{pmatrix}
            1 & 1 & \cdots & 1
        \end{pmatrix}\cdot
        \begin{pmatrix}
            \xi_1 & \xi_2 & \cdots & \xi_n
        \end{pmatrix}^T = A\vec{\xi}$.
        $\E\eta = A\vec{a} = \E\xi_1 + \E\xi_2 + ... + \E\xi_n = \sum\limits_{k=1}^n \E\xi_k$.
        $\D\eta = A K A^{*} = \begin{pmatrix}
            1 & 1 & \cdots & 1
        \end{pmatrix}
        \cdot K \cdot \begin{pmatrix}
            1 & 1 & \cdots & 1
        \end{pmatrix}^T$. Якщо $K = \mathrm{diag}(\sigma_1^2, \sigma_2^2, ..., \sigma_n^2)$, то $\D\eta = \sum\limits_{k=1}^n \D\xi_k$,
        а в загальному випадку у цій сумі ще будуть доданки вигляду $2\K\xi_i\xi_j$.
        \item $\vec{\xi} = \begin{pmatrix}
            \xi_1 \\ \xi_2
        \end{pmatrix} \sim \mathrm{N}\left( \begin{pmatrix}
            0 \\ 1
        \end{pmatrix}, \begin{pmatrix}
            1 & 2 \\
            2 & 6
        \end{pmatrix}\right) = \mathrm{N}(\vec{a}, K)$. Знайти розподіл, характеристичну функцію, щільність розподілу 
        $\vec{\eta} = (3\xi_1-2\xi_2+1, \xi_1+3\xi_2)^T$ та розподіл координати $\eta_1$.

        Запишемо $\vec{\eta}$ у вигляді $\vec{\eta} = A\vec{\xi} + \vec{b}$: $\vec{\eta} = \begin{pmatrix}
            3 & -2 \\
            1 & 3 
        \end{pmatrix}\begin{pmatrix}
            \xi_1 \\ \xi_2
        \end{pmatrix} + \begin{pmatrix}
            1 \\ 0 
        \end{pmatrix}$. Тоді $\vec{\eta} \sim \mathrm{N}(\vec{d}, B)$,
        $\vec{d} = A\vec{a} + \vec{b} = \begin{pmatrix}
            3 & -2 \\
            1 & 3 \\
        \end{pmatrix}\begin{pmatrix}
            0 \\ 1
        \end{pmatrix} + \begin{pmatrix}
            1 \\ 0 
        \end{pmatrix} = \begin{pmatrix}
            -2 \\ 3 
        \end{pmatrix} + \begin{pmatrix}
            1 \\ 0
        \end{pmatrix} = \begin{pmatrix}
            -1 \\ 3
        \end{pmatrix} = \begin{pmatrix}
            \E\eta_1 \\ \E\eta_2
        \end{pmatrix}$, $B=A K A^{*} = \begin{pmatrix}
            3 & -2 \\
            1 & 3 \\
        \end{pmatrix}\begin{pmatrix}
            1 & 2 \\
            2 & 6
        \end{pmatrix}\begin{pmatrix}
            3 & 1 \\
            -2 & 3
        \end{pmatrix} = \begin{pmatrix}
            9 & -19 \\
            -19 & 67
        \end{pmatrix}$.

        $\chi_{\vec{\eta}}(t_1, t_2) = \exp\left\{i(\vec{d}, \vec{t}) - \frac{1}{2}\left( B\vec{t},\vec{t}\right)\right\} = 
        \exp\left\{i(-1 t_1 + 3 t_2) - \frac{1}{2}(9t_1^2 - 38 t_1 t_2 + 67 t_2^2)\right\}$.

        $B^{-1} = \begin{pmatrix}
            67/142 & 19/142 \\
            19/142 & 9/142
        \end{pmatrix}$, $\det{B} = 142$.

        $f_{\vec{\eta}}(x_1, x_2) = \frac{1}{2\pi \cdot \sqrt{142}} \cdot \exp\left\{-\frac{1}{2}\left( B^{-1}(\vec{x} - \vec{d}), (\vec{x} - \vec{d}))\right)\right\} = $

        $ = \frac{1}{2\pi \cdot \sqrt{142}} \cdot \exp\left\{-\frac{1}{2}\left( \frac{67}{142}(x_1+1)^2 + \frac{38}{142}(x_1+1)(x_2-3) + \frac{9}{142}(x_2-3)^2\right)\right\}$.

        Знайдемо розподіл $\eta_1$ за допомогою характеристичної функції: 
        $\chi_{\eta_1}(t) = \chi_{\vec{\eta}}(t, 0) = $

        $= \exp\left\{-i t - \frac{1}{2}\cdot 9t^2\right\} \Rightarrow
        \eta_1 \sim \mathrm{N}(-1, 9)$.
    \end{enumerate}
\end{example}

\subsection{Вироджені гауссівські вектори}
Розглянемо випадок $n$-вимірного нормального розподілу $\vec{\xi} \sim \mathrm{N}(\vec{a}, K)$ з виродженою матрицею $K$,
у якої $\mathrm{rang}K = m < n$. Як вже було сказано, щільності у такого розподілу не існує. Існує ортогональне перетворення $U$ таке, що $K = UDU^{*}$,
$D = \mathrm{diag}(\lambda_1, \lambda_2, ..., \lambda_m, 0, ..., 0)$. 
Знайдемо розподіл вектора $\vec{\eta} = U^{*}\vec{\xi}$ за властивістю характеристичної функції:
\begin{gather*}
    \chi_{\vec{\eta}} (\vec{t}) = \chi_{\vec{\xi}} (U^{**}\vec{t}) = \chi_{\vec{\xi}} (U\vec{t}) = 
\exp\left\{i(\vec{a}, U\vec{t}) - \frac{1}{2}(KU\vec{t}, U\vec{t})\right\} = 
\exp\left\{i(U^{*}\vec{a}, \vec{t}) - \frac{1}{2}(U^{*}KU\vec{t}, \vec{t})\right\} = \\
= \left[ U^{*}\vec{a} = \vec{b}, U^{*}KU = D\right] = 
\exp\left\{i(\vec{b}, \vec{t}) - \frac{1}{2}(D\vec{t}, \vec{t})\right\} = 
\exp\left\{i\sum\limits_{k=1}^n b_k t_k - \frac{1}{2}\sum\limits_{k=1}^m \lambda_k t_k^2\right\} = \\
= \exp\left\{i\sum\limits_{k=m+1}^n b_k t_k\right\} \cdot
\exp\left\{i\sum\limits_{k=1}^m b_k t_k - \frac{1}{2}\sum\limits_{k=1}^m \lambda_k t_k^2\right\}
\end{gather*}

Перший множник --- це характеристична функція сталого вектору, а другий --- характеристична функція гауссівського вектору меншої розмірності.

Позначимо $\vec{c} = (0, ..., 0, b_{m+1}, ..., b_{n})^T$,
$\vec{d} = (b_1, ..., b_m, 0, ..., 0)^T$,
$\vec{\zeta} = (\zeta_1, ..., \zeta_m, 0, ..., 0)^T$, $\zeta_k \sim {N}(0, 1)$ та незалежні у сукупності,
$\Lambda = \mathrm{diag}(\sqrt{\lambda_1}, \sqrt{\lambda_2}, ..., \sqrt{\lambda_m}, 1, ..., 1)$.

Тоді $\vec{\eta} = \vec{c} + \left(\Lambda \vec{\zeta} + \vec{d} \right)$. 
Оскільки $\vec{\eta} = U^{*}\vec{\xi}$, то $\vec{\xi} = U\vec{\eta} = U\Lambda \vec{\zeta} + U(\vec{c} + \vec{d}) = U\Lambda \vec{\zeta} + \vec{a}$.
Позначимо $U\Lambda = L$. Остаточно: $\vec{\xi} = L\vec{\zeta} + \vec{a}$.

Що тепер можна сказати про розподіл такого вектора? Оскільки перші $m$ координат $\vec{\zeta}$ можуть приймати довільні дійсні значення, то
можливі значення вектора $L\vec{\zeta}$ --- це точки з лінійної оболонки векторів $\left\{u_1, u_2, ..., u_m\right\}$ (це власні вектори, що відповідають ненульовим власним числам $K$).
Оскільки власні вектори, що відповідають нульовому власному числу, утворюють базис $\mathrm{Ker}K$, то лінійна оболонка $\left\{u_1, u_2, ..., u_m\right\}$ --- це $\mathrm{Im}K$.


\textbf{Висновок:} $n$-вимірний гауссівський вектор з виродженою кореляційною матрицею, що має ранг $m<n$, можна представити
у вигляді суми свого вектора математичного сподівання та вектора, перші $m$ координат є незалежними у сукупності та мають стандартний нормальний розподіл, 
а інші координати нульові, помноженого на невироджену матрицю. 

Усі значення $n$-вимірного гауссівського вектора $\vec{\xi} \sim \mathrm{N}(\vec{a}, K)$ з виродженою кореляційною матрицею, що має ранг $m<n$,
зосереджені на $m$-вимірному підпросторі $\mathbb{R}^n$ --- $\mathrm{Im}K + \vec{a}$.

\subsection{Нормальний розподіл на площині} 
Розглянемо двовимірний гауссівський вектор $\vec{\xi} = (\xi_1, \xi_2)^T \sim \mathrm{N}(\vec{a}, K)$.
Нехай $\vec{a} = \begin{pmatrix}
    a_1 \\ a_2
\end{pmatrix}$, $K = \begin{pmatrix}
    \sigma_1^2 & \rho \sigma_1 \sigma_2 \\
    \rho \sigma_1 \sigma_2 & \sigma_2^2
\end{pmatrix}$, $\det{K} = \sigma_1^2 \sigma_2^2 - \rho^2 \sigma_1^2 \sigma_2^2 = (1-\rho^2)\sigma_1^2\sigma_2^2$, 
$K^{-1} = \frac{1}{(1-\rho^2)\sigma_1^2\sigma_2^2}\begin{pmatrix}
    \sigma_2^2 & -\rho \sigma_1 \sigma_2 \\
    -\rho \sigma_1 \sigma_2 & \sigma_1^2
\end{pmatrix}$.
Окремо запишемо щільність:
\begin{equation}
    \begin{gathered}
        f_{\vec{\xi}}(x,y) = \frac{1}{2\pi\sigma_1\sigma_2\sqrt{1-\rho^2}} \exp\left\{-\frac{1}{2}\cdot
        \frac{\left( \sigma_2^2(x-a_1)^2 - 2\rho\sigma_1\sigma_2(x-a_1)(y-a_2) + \sigma_1^2(y-a_2)^2\right)}{\sigma_1^2\sigma_2^2(1-\rho^2)}
        \right\} = \\
        = \frac{1}{2\pi\sigma_1\sigma_2\sqrt{1-\rho^2}} \exp\left\{
            -\frac{1}{2(1-\rho^2)}\cdot
            \left(\frac{(x-a_1)^2}{\sigma_1^2} -
            2\rho\frac{x-a_1}{\sigma_1}\frac{y-a_2}{\sigma_2} +
            \frac{(y-a_2)^2}{\sigma_2^2}
            \right) 
        \right\}
    \end{gathered}
\end{equation}
При $\rho = 0$ отримуємо $f_{\vec{\xi}}(x,y) = \frac{1}{\sqrt{2\pi}\sigma_1} e^{-\frac{(x-a_1)^2}{2\sigma_1^2}} \cdot \frac{1}{\sqrt{2\pi}\sigma_2} e^{-\frac{(y-a_2)^2}{2\sigma_2^2}} = f_{\xi_1}(x)\cdot f_{\xi_2}(y)$.
\vspace{0.5em}

Поверхня, утворена графіком щільності,
називається \emph{поверхнею} або \emph{палаткою Гаусса}.
 
\hbox to \hsize{\hfil{
    \includegraphics[scale=0.45]{bivar}
}\hfil}
Лінії рівня цієї поверхні задають \emph{еліпси розсіювання}:
$$f_{\vec{\xi}}(x,y) = C \Leftrightarrow \frac{(x-a_1)^2}{\sigma_1^2} -
2\rho\frac{x-a_1}{\sigma_1}\frac{y-a_2}{\sigma_2} +
\frac{(y-a_2)^2}{\sigma_2^2} = \lambda^2$$

\begin{tabular}{c p{7.8cm}}
\begin{tikzpicture}[baseline={(current bounding box.north)}]
    \draw [->] (-0.5, 0) -- (5, 0);
    \draw [->] (0, -0.5) -- (0, 3);
    \draw (2, 1.5) circle [x radius=1, y radius=0.5, rotate=40];
    \draw (2, 1.5) circle [x radius=1.3, y radius=0.7, rotate=40];
    \draw (2, 1.5) circle [x radius=1.5, y radius=0.9, rotate=40];
    \draw [fill] (2, 1.5) circle [radius = 0.05];
    \draw [<-] (3.5, 2.824) -- (0.3, 0);
    \draw [<-] (1, 2.64) -- (3, 0.38);
    \draw [dashed] (2, 1.5) -- (0, 1.5);
    \draw [dashed] (2, 1.5) -- (2, 0);
    \node [below] at (5, 0) {$x$};
    \node [left] at (0, 2.8) {$y$};
    \node [below] at (2, 0) {$a_1$};
    \node [left] at (0, 1.5) {$a_2$};
    \draw (0.6, 0) arc (0:40:0.3);
    \node [above right] at (0.55, 0) {$\alpha$};
\end{tikzpicture}&
$\mathrm{tg}2\alpha = \frac{2\rho\sigma_1\sigma_2}{\sigma_1^2 - \sigma_2^2}$\newline
Осі еліпсів називаються \emph{головними осями розсіювання}.\newline
Якщо $\rho = 0$, то головні осі розсіювання паралельні осям координат.
\end{tabular}

Позначимо $\mathcal{E}_\lambda = \left\{(x;y) \in \mathbb{R}^2: \frac{(x-a_1)^2}{\sigma_1^2} -
2\rho\frac{x-a_1}{\sigma_1}\frac{y-a_2}{\sigma_2} +
\frac{(y-a_2)^2}{\sigma_2^2} \leq \lambda^2\right\}$ та
знайдемо ймовірність потрапляння в цю область у випадку $\rho = 0$:

\begin{gather*}
    \P\left\{\vec{\xi} \in \mathcal{E}_\lambda\right\} = \iint\limits_{\mathcal{E}_\lambda} f_{\vec{\xi}}(x,y) dx dy = 
    \frac{1}{2\pi\sigma_1\sigma_2} \iint\limits_{\mathcal{E}_\lambda} \exp\left\{-\frac{1}{2}\left( 
        \frac{(x-a_1)^2}{\sigma_1^2} + 
        \frac{(y-a_2)^2}{\sigma_2^2}
    \right)\right\} dx dy = \\
    \left[ \begin{gathered}
        x-a_1 = \lambda\sigma_1 r \cos\varphi \\ 
        y-a_2 = \lambda\sigma_2 r \sin\varphi \\
        |\mathcal{J}| = \lambda^2 \sigma_1 \sigma_2 r
    \end{gathered}\right] = 
    \frac{\lambda^2}{2\pi} \int\limits_0^{2\pi}  d\varphi
    \int\limits_0^1 r e^{-\frac{\lambda^2r^2}{2}} dr = 
    \int\limits_0^1 e^{-\frac{\lambda^2r^2}{2}} d\left(\frac{\lambda^2r^2}{2} \right) = 
    1 - e^{-\frac{\lambda^2}{2}}
\end{gather*}

\begin{exercise}
    Довести, що у випадку $\rho > 0$ $\P\left\{\vec{\xi} \in\mathcal{E}_\lambda\right\} = 
    1 - e^{-\frac{\lambda^2}{2(1-\rho^2)}}$.
\end{exercise}

\begin{example}
    $\vec{\xi} \sim \mathrm{N}\left( \begin{pmatrix}
        -1 \\ 1
    \end{pmatrix}, \begin{pmatrix}
        1 & 2 \\
        2 & 16
    \end{pmatrix}\right)$. Знайти рівняння еліпса розсіювання, для якого \\
    $\P\left\{\vec{\xi} \in \mathcal{E}_\lambda\right\} = 0.93$. З кореляційної матриці $\rho = 0.5$.
    Розв'яжемо відносно $\lambda^2$ рівняння $1 - e^{-\frac{\lambda^2}{2(1-0.5^2)}} = 0.93
    \Leftrightarrow e^{-\frac{\lambda^2}{1.5}} = 0.07 \Rightarrow \lambda^2 \approx 4$. Тому рівняння еліпса
    $\frac{(x+1)^2}{1} - \frac{x+1}{1} \frac{y-1}{4} + \frac{(y-1)^2}{16} = 4$ або 
    $\frac{(x+1)^2}{4} - \frac{(x+1)(y-1)}{16} + \frac{(y-1)^2}{64} = 1$.
\end{example}

Розглянемо гауссівський вектор $\vec{\xi} = (\xi_1, \xi_2)$ з незалежними координатами.
Для нього $F_{\vec{\xi}}(x,y) = F_{\xi_1}(x) F_{\xi_2}(y) = 
\left(\frac{1}{2} + \Phi\left( \frac{x-a_1}{\sigma_1}\right) \right)
\left(\frac{1}{2} + \Phi\left( \frac{y-a_2}{\sigma_2}\right) \right)$.
Тоді ймовірність потрапляння в прямокутник $\Pi = \left\{(x;y)\in\mathbb{R}^2 : \alpha \leq x < \beta, \gamma \leq y < \delta\right\}$
дорівнює $$\left( \Phi\left( \frac{\beta-a_1}{\sigma_1}\right) - \Phi\left( \frac{\alpha-a_1}{\sigma_1}\right)\right) \cdot
\left( \Phi\left( \frac{\delta-a_2}{\sigma_2}\right) - \Phi\left( \frac{\gamma-a_2}{\sigma_2}\right)\right)$$
\begin{exercise}
    Довести формулу для $\P\left\{\vec{\xi}\in\Pi\right\}$.
\end{exercise}
Для гауссівського вектору з незалежними координатами також має місце <<правило $3\sigma$>>:
\begin{gather*}
    \P\left\{\vec{\xi} \in (a_1-3\sigma_1; a_1+3\sigma_1)\times(a_2-3\sigma_2; a_2+3\sigma_2)\right\} = \\
    = \P\left\{\xi_1 \in(a_1-3\sigma_1; a_1+3\sigma_1)\right\}\cdot \P\left\{\xi_2 \in(a_2-3\sigma_2; a_2+3\sigma_2)\right\}
    \approx 0.9973^2 \approx 0.9946
\end{gather*}

\subsection{Колове розсіювання}
\begin{definition}
    Двовимірний гауссівський вектор має \emph{колове розсіювання}, якщо
    $\vec{a} = \vec{0}, \; K = \begin{pmatrix}
        \sigma^2 & 0 \\
        0 & \sigma^2
    \end{pmatrix}$. В цьому випадку $f_{\vec{\xi}}(x, y) = \frac{1}{2\pi\sigma^2} \exp\left\{-\frac{x^2+y^2}{2\sigma^2}\right\}$, 
    а еліпси розсіювання стають колами.
\end{definition}
Знайдемо ймовірність потрапляння такого $\vec{\xi}$ в коло $K_R = \left\{(x; y) \in \mathbb{R}^2 : x^2 + y^2 \leq R^2\right\}$:
\begin{gather*}
    \P\left\{\vec{\xi} \in K_R\right\} = \frac{1}{2\pi\sigma^2} \iint\limits_{K_R} e^{-\frac{x^2+y^2}{2\sigma^2}} dx dy = 
    \left[ \begin{gathered}
        x = \sigma r\cos\varphi \\
        y = \sigma r\sin\varphi \\
        |\mathcal{J}| = \sigma^2 r
    \end{gathered}\right] =
    \frac{1}{2\pi} \int\limits_0^{2\pi} d\varphi \int\limits_0^{R/\sigma} r e^{-\frac{r^2}{2}} dr = \\
    = \int\limits_0^{R/\sigma} e^{-\frac{r^2}{2}} d \left( \frac{r^2}{2}\right) = 
    1 - e^{-\frac{R^2}{2\sigma^2}}
\end{gather*}
Таким чином, тепер відомий розподіл норми $\vec{\xi}$. $\eta = \Vert \vec{\xi} \Vert = \sqrt{\xi_1^2 + \xi_2^2}$ --- випадкова величина,
для якої $F_{\eta}(R) = \P\left\{\Vert \vec{\xi} \Vert < R\right\} = \begin{cases}
    1 - e^{-\frac{R^2}{2\sigma^2}}, & R > 0 \\
    0, & R \leq 0
\end{cases}$ --- це \emph{розподіл Релея}.

\begin{exercise}
    Знайти основні числові характеристики розподілу Релея.
\end{exercise}

\subsection{Умовний гауссівський розподіл на площині}
Знайдемо одну з умовних щільностей розподілу:
\begin{gather*}
    f_{\xi_1}(x/\xi_2 = y) = \frac{f_{\vec{\xi}}(x,y)}{f_{\xi_2}(y)} = 
    \frac{\frac{1}{2\pi\sigma_1\sigma_2\sqrt{1-\rho^2}} \exp\left\{
        -\frac{1}{2(1-\rho^2)}\cdot
        \left(\frac{(x-a_1)^2}{\sigma_1^2} -
        2\rho\frac{x-a_1}{\sigma_1}\frac{y-a_2}{\sigma_2} +
        \frac{(y-a_2)^2}{\sigma_2^2}
        \right)\right\} }{
            \frac{1}{\sqrt{2\pi}\sigma_2} \exp\left\{-\frac{1}{2\sigma_2^2}(y-a_2)^2\right\}
        } = \\
        \frac{1}{\sqrt{2\pi}\sigma_1\sqrt{1-\rho^2}} \exp\left\{
            -\frac{1}{2(1-\rho^2)}
        \left(\frac{(x-a_1)^2}{\sigma_1^2} -
        2\rho\frac{(x-a_1)(y-a_2)}{\sigma_1\sigma_2} +
        \frac{(y-a_2)^2}{\sigma_2^2}
        \right) +
        \frac{(y-a_2)^2}{2\sigma_2^2}
        \right\} = \\
        = \left [ \frac{(y-a_2)^2}{\sigma_2^2} - \frac{(1-\rho^2)(y-a_2)^2}{\sigma_2^2} = \rho^2 \frac{(y-a_2)^2}{\sigma_2^2}\right]=\\
        = \frac{1}{\sqrt{2\pi}\sigma_1\sqrt{1-\rho^2}} \exp\left\{-\frac{1}{2(1-\rho^2)}\left( \frac{x-a_1}{\sigma_1} - \rho\cdot\frac{y-a_2}{\sigma_2}\right)^2\right\} = \\
        = \frac{1}{\sqrt{2\pi}\sigma_1\sqrt{1-\rho^2}} \exp\left\{-\frac{1}{2\sigma_1^2(1-\rho^2)}\left( x- \left( a_1 + \frac{\rho\sigma_1(y-a_2)}{\sigma_2}\right)\right)^2\right\}
\end{gather*}
Аналогічно $f_{\xi_2}(y/\xi_1 = x) = \frac{1}{\sqrt{2\pi}\sigma_2\sqrt{1-\rho^2}} \exp\left\{-\frac{1}{2\sigma_2^2(1-\rho^2)}\left( y- \left( a_2 + \frac{\rho\sigma_2(x-a_1)}{\sigma_1}\right)\right)^2\right\}$.
\vspace{0.5em}

\noindentОтже, обидві умовні \emph{щільності є щільностями нормального розподілу}.

$\E\left( \xi_1 / \xi_2 = y\right) = a_1 + \frac{\rho\sigma_1(y-a_2)}{\sigma_2}$, 
$\E\left( \xi_2 / \xi_1 = x\right) = a_2 + \frac{\rho\sigma_2(x-a_1)}{\sigma_1}$. Лінії регресії --- прямі,

\noindent
$\rho\frac{\sigma_1}{\sigma_2} = \frac{K\xi_1\xi_2}{D\xi_2}$ та $\rho\frac{\sigma_2}{\sigma_1}  = \frac{K\xi_1\xi_2}{D\xi_1}$ --- відповідні кутові коефіцієнти прямих регресії.

$\D\left( \xi_1 / \xi_2 = y\right) = \sigma_1^2(1-\rho^2)$, 
$\D\left( \xi_2 / \xi_1 = x\right) = \sigma_2^2(1-\rho^2)$. Умовні дисперсії є сталими, ця властивість називається \emph{гомоскедастичністю}.
    \chapter{Функції випадкових аргументів}
        % !TEX root = ../main.tex

\section{Функції одного випадкового аргументу}
\subsection{Функції від дискретного випадкового аргументу}
Нехай $\xi$ --- ДВВ з рядом розподілу 
\begin{tabular}{|c|c|c|c|c|c|}
    \hline
    $\xi$ & $x_1$ & $x_2$ & $...$ & $x_n$ & $...$ \\
    \hline
    $p$ & $p_1$ & $p_2$ & $...$ & $p_n$ & $...$ \\
    \hline
\end{tabular}, а $\varphi$ --- деяка вимірна числова функція, 
область визначення якої містить можливі значення $\xi$.
Можна розглядати випадкову величину $\eta = \varphi(\xi)$.

Очевидно, що $\eta$ --- теж ДВВ, що приймає значення $y_k = \varphi(x_k)$, 
$k = 1, 2, ...$ з ймовірностями $P(\eta = y_k) = P(\xi = x_k) = p_k$.
Однак, можливо, що $y_k = \varphi(x_k) = \varphi(x_s) = ... = \varphi(x_r)$.
Тоді $P(\eta = y_k) = p_k + p_s + ... + p_r$.

\begin{example}
    \begin{tabular}{|c|c|c|c|c|c|}
        \hline
        $\xi$ & $-2$ & $-1$ & $0$ & $1$ & $2$ \\
        \hline
        $p$ & $1/9$ & $2/9$ & $1/9$ & $2/9$ & $3/9$ \\
        \hline
    \end{tabular}. Знайти закон розподілу $\eta = \xi^2$.
    
    З ряду розподілу $\xi$ видно, що $\xi^2$ приймає значення $0, 1, 4$ з
    ймовірностями $1/9$, $2/9 + 2/9$ та $1/9 + 3/9$.
    Отже, маємо ряд розподілу $\eta$: \begin{tabular}{|c|c|c|c|}
        \hline
        $\eta$ & $0$ & $1$ & $4$ \\
        \hline
        $p$ & $1/9$ & $4/9$ & $4/9$ \\
        \hline
    \end{tabular}
\end{example}

З алгоритму побудови ряду розподілу функції від дискретного
випадкового аргументу $E\eta^k = \sum\limits_{i=1}^{n(\infty)} \varphi^k(x_i) P(\xi = x_i)$.

\subsection{Розподіл монотонної функції від неперервного випадкового аргументу}
Нехай $\xi$ --- НВВ, $f_\xi(x)$ --- її щільність розподілу, а $\varphi$ --- монотонна неперервна числова функція.
Розглянемо два випадки.

\begin{enumerate}
    \item $\eta = \varphi(\xi)$, де $\varphi$ --- монотонно зростаюча.

    \begin{tabular}{c p{8.8cm}}
        \begin{tikzpicture}[xscale = 0.7, yscale = 0.3, baseline={(current bounding box.north)}]
            \draw [->] (-3, 0) -- (3, 0);
            \draw [->] (0, -0.5) -- (0, 5.5);
            \draw [domain=-3:3, smooth, variable = \x, ultra thick] plot ({\x}, {e^(\x/2});
            \node [below] at (3, 0) {$x$};
            \node [left] at (0, 5.5) {$y$};
        \end{tikzpicture} &
        $F_\eta (y) = P\left\{ \eta < y\right\} = P\left\{ \varphi(\xi) < y\right\}$.

        Нехай $y = \varphi(x)$, тоді з монотонності $\varphi$ маємо рівність подій $\left\{ \varphi(\xi) < y\right\} = \left\{ \xi < x\right\}$.
        Тому $F_\eta (y) = F_\xi (x) = \int\limits_{-\infty}^x f_\xi(t) dt$. $\varphi$ має обернену, $x = \varphi^{-1} (y)$.
    \end{tabular}

    Отже, $F_\eta (y) = \int\limits_{-\infty}^{\varphi^{-1} (y)} f_\xi(t) dt$,
    $f_\eta(y) = F^\prime_\eta (y) = f_\xi\left(\varphi^{-1} (y)\right) \cdot \left(\varphi^{-1} (y) \right)^{\prime}$.
    \item $\eta = \varphi(\xi)$, де $\varphi$ --- монотонно спадна.
    
    \begin{tabular}{c p{8.8cm}}
        \begin{tikzpicture}[xscale = 0.7, yscale = 0.3, baseline={(current bounding box.north)}]
            \draw [->] (-3, 0) -- (3, 0);
            \draw [->] (0, -0.5) -- (0, 5.5);
            \draw [domain=-3:3, smooth, variable = \x, ultra thick] plot ({\x}, {e^(-\x/2});
            \node [below] at (3, 0) {$x$};
            \node [left] at (0, 5.5) {$y$};
        \end{tikzpicture} &
        $F_\eta (y) = P\left\{ \eta < y\right\} = P\left\{ \varphi(\xi) < y\right\}$.

        Нехай $y = \varphi(x)$, тоді з монотонності $\varphi$ маємо рівність подій $\left\{ \varphi(\xi) < y\right\} = \left\{ \xi > x\right\}$.
        Тому $F_\eta (y) = 1 - F_\xi (x) = 1 - \int\limits_{-\infty}^x f_\xi(t) dt$. $\varphi$ має обернену, $x = \varphi^{-1} (y)$.
    \end{tabular}

    Отже, $F_\eta (y) = 1 - \int\limits_{-\infty}^{\varphi^{-1} (y)} f_\xi(t) dt$,
    $f_\eta(y) = F^\prime_\eta (y) = - f_\xi\left(\varphi^{-1} (y)\right) \cdot \left(\varphi^{-1} (y) \right)^{\prime}$.
\end{enumerate}

\noindent Остаточно, для монотонної неперервної функції $\varphi$ має місце формула:
$$f_\eta(y) = f_\xi\left(\varphi^{-1} (y)\right) \cdot \left|\left(\varphi^{-1} (y) \right)^{\prime}\right|, \; \eta = \varphi(\xi)$$
При цьому обов'язково треба зазначити допустимі значення $y$.

\begin{remark}
    Взагалі кажучи, на функцію $\varphi$, окрім неперервності, треба накладати й умову диференційовності $\varphi^{-1}$ майже скрізь. 
\end{remark}

\begin{example}
    $\xi \sim \mathrm{U}(-\frac{\pi}{2}; \frac{\pi}{2})$. Знайти розподіл $\eta = \sin \xi$.

    \noindent $f_\xi(x) = \begin{cases}
        \frac{1}{\pi}, & |x| \leq \frac{\pi}{2} \\
        0, & |x| > \frac{\pi}{2}
    \end{cases}$, $\varphi(x) = \sin x$, $\varphi^{-1} (y) = \arcsin(y)$, $\left(\varphi^{-1} (y)\right)^{\prime} = \frac{1}{\sqrt{1-y^2}}$,
    $\eta$ може приймати значення від $-1$ до $1$. 
    Отже, $f_\eta(y) = \begin{cases}
        \frac{1}{\pi\sqrt{1-y^2}}, & |y| \leq 1 \\
        0, & |y| > 1
    \end{cases}$ --- це <<закон арксинуса>>.
\end{example}

\begin{example}
    Нехай $\xi$ --- довільна НВВ, $F_\xi (x)$ --- її функція розподілу. 
    Знайти закон розподілу $\eta = F_\xi (\xi)$.

    \noindent $F_\eta (y) = P\{\eta < y\} = P\{F_\xi(\xi) < y\} = P\{\xi < F_\xi^{-1}(y)\} = F_\xi(F_\xi^{-1}(y)) = \begin{cases}
        0, & y \leq 0 \\
        y, & 0 < y \leq 1 \\
        1, & y > 1
    \end{cases}$.

    \noindent Отже, $F_\xi (\xi) \sim U\left<0;1\right>$. Ще раз зауважимо, що результат має місце для довільної НВВ $\xi$.
\end{example}

\begin{remark}
    Якщо $\varphi$ не є неперервною, то $\eta = \varphi(\xi)$ може не бути НВВ. Наведемо декілька прикладів.
    \begin{enumerate}
        \item Нехай $\xi \sim \mathrm{Exp}(1)$. Знайдемо розподіл $\eta = \left[ \xi\right]$, де $\left[ \cdot\right]$ --- ціла частина.
        Ціла частина приймає значення $0, 1, 2, ...$, знайдемо відповідні ймовірності. 
        
        $P\left\{ \left[ \xi\right] = n\right\} = P\left\{ \xi \in [n; n+1)\right\} = \int\limits_n^{n+1} e^{-x} dx = e^{-n} - e^{-(n+1)} = e^{-n}(1 - e^{-1})$, $n = 0, 1, 2, ...$

        Отже, $\eta = \left[ \xi\right] \sim \mathrm{Geom}(1-e^{-1})$.

        За цим прикладом можна зробити висновок: якщо $\varphi$ --- кусково стала, а $\xi$ --- НВВ, то $\varphi(\xi)$ --- ДВВ.
        \item Нехай $\xi \sim \mathrm{Exp}(1)$. Знайдемо розподіл $\eta = \left\{ \xi\right\} = \xi - \left[ \xi\right]$. 
        В цьому випадку $\varphi$ є кусково неперервною функцією та приймає значення з інтервалу $[0; 1)$.
        
        Для $0< y < 1$ $F_\eta (y) = P\left\{\eta < y\right\} = P\left\{\eta \in [0; y)\right\} = P\left\{\xi \in [n; n+y), n = 0, 1, 2, ...\right\}$.

        Для фіксованого $n$ $P\left\{\xi \in [n; n+y)\right\} = \int\limits_n^{n+y} e^{-x} dx = e^{-n}(1 - e^{-y})$. 
        Отже, $P\left\{\eta \in [0; y)\right\} = \sum\limits_{n=0}^{\infty} P\left\{\xi \in [n; n+y)\right\} = (1 - e^{-y}) \sum\limits_{n=0}^{\infty} e^{-n} = \frac{1 - e^{-y}}{1 - e^{-1}}$. Отримали функцію розподілу $\eta$:
        $F_\eta (y) = \begin{cases}
            0, & y \leq 0 \\
            \frac{1 - e^{-y}}{1 - e^{-1}}, & 0 < y \leq 1 \\
            1, & y > 1
        \end{cases}$. Таким чином, $\eta$ --- НВВ.
    \end{enumerate}
\end{remark}

\subsection{Розподіл немонотонної функції від неперервного випадкового аргументу}
Нехай $\xi$ --- НВВ, $f_\xi(x)$ --- її щільність розподілу, а $\varphi$ --- немонотонна числова функція.
Тоді область можливих значень $\xi$ можна розбити на проміжки, на яких $\varphi$ буде монотонною. 
Тоді, скориставшись результатом для монотонної функції, отримаємо формулу для щільності $\eta = \varphi(\xi)$:
$$ f_\eta (y) = \sum\limits_{k=1}^m f_\xi\left(\varphi_k^{-1} (y)\right) \cdot \left|\left(\varphi_k^{-1} (y) \right)^{\prime}\right|$$
де $m$ --- кількість проміжків монотонності, а $\varphi_k^{-1}$ --- відповідні обернені функції.

\begin{example}
    \begin{enumerate}
        \item $\xi \sim \mathrm{U}(-\frac{\pi}{2}, \frac{\pi}{2})$, знайти розподіл $\eta = \cos\xi$.

        На проміжках $[-\frac{\pi}{2}; 0]$ та $[0; \frac{\pi}{2}]$ $\cos x$ є монотонною функцією, 
        відповідні обернені --- $\varphi_1^{-1} (y) = -\arccos y$, $\varphi_2^{-1} (y) = \arccos y$.

        Тоді $f_\eta (y) = \frac{1}{\pi} \cdot \left| - \frac{1}{\sqrt{1-y^2}}\right| + \frac{1}{\pi} \cdot \left|\frac{1}{\sqrt{1-y^2}}\right| = \begin{cases}
            \frac{2}{\pi} \frac{1}{\sqrt{1-y^2}}, & y \in [0; 1] \\
            0, & y \notin [0; 1]
        \end{cases}$.
        \item $\xi \sim \mathrm{N}(a, \sigma)$, знайти розподіл $\eta = \xi^2$.

        На проміжках $(-\infty; 0]$ та $[0; +\infty)$ $x^2$ є монотонною функцією, 
        відповідні обернені --- $\varphi_1^{-1} (y) = -\sqrt{y}$, $\varphi_2^{-1} (y) = \sqrt{y}$.
        Щільність розподілу $\xi$ --- $f_\xi (x) = \frac{1}{\sqrt{2\pi}\sigma} e^{-\frac{(x-a)^2}{2\sigma^2}}$.

        Тоді $f_\eta (y) = \begin{cases}
            \frac{1}{\sqrt{2\pi}\sigma} e^{-\frac{(-\sqrt{y}-a)^2}{2\sigma^2}} \cdot \frac{1}{2\sqrt{y}} + 
        \frac{1}{\sqrt{2\pi}\sigma} e^{-\frac{(\sqrt{y}-a)^2}{2\sigma^2}} \cdot \frac{1}{2\sqrt{y}}, & y > 0 \\
        0, & y \leq 0
        \end{cases}$.

        При $\xi \sim \mathrm{N}(0, 1)$ $f_\eta(y) = \begin{cases}
            \frac{1}{\sqrt{2\pi y}} e^{-y/2}, y > 0 \\
            0, y \leq 0
        \end{cases}$, що означає $\xi^2 \sim \Gamma(\frac{1}{2}, \frac{1}{2})$.
    \end{enumerate}
\end{example}

\subsection{Числові характеристики функції неперервного випадкового аргументу}
\noindent\textbf{Твердження.} $E (\varphi(\xi))^k = \int\limits_{-\infty}^{+\infty} \varphi^k(x) f_\xi(x) dx$ для всіх $k \in \mathbb{N}$.
\begin{proof}
    Достатньо довести для монотонно зростаючої $\varphi$.

    \noindent$\eta = \varphi(\xi)$, $E\eta^k = \int\limits_{-\infty}^{+\infty} y^k f_\eta(y) dy = \int\limits_{-\infty}^{+\infty} y^k f_\xi (\varphi^{-1} (y)) (\varphi^{-1} (y))^\prime dy =
    \int\limits_{-\infty}^{+\infty} y^k f_\xi(\varphi^{-1} (y)) d \varphi^{-1} (y) = \left[ \varphi^{-1} (y) = x, y = \varphi(x)\right] = \int\limits_{-\infty}^{+\infty} \varphi^k(x) f_\xi(x) dx$.
    У випадку монотонно спадної $\varphi$ під інтегралом отримаємо $-(\varphi^{-1} (y))^\prime$, а інтеграл після заміни змінної буде від $+\infty$ до $-\infty$.
    У випадку немонотонної $\varphi$ треба буде скористатися адитивністю інтеграла.
\end{proof}

\begin{remark}
    Для знаходження числових характеристик функції неперервного випадкового аргументу розподіл самої функції знаходити не потрібно.
\end{remark}

\begin{exercise}
    Дослідити, за яких умов на $\varphi$ ця формула має місце.
\end{exercise}
        % !TEX root = ../main.tex

\section{Функції кількох випадкових аргументів}
\subsection{Випадок довільної функції}
Нехай $\varphi : \mathbb{R}^n \to \mathbb{R}$ --- задана числова функція.

Якщо $\vec{\xi} = \left(\xi_1, ..., \xi_n\right)$ --- дискретний випадковий вектор, тоді $\eta = \varphi(\vec{\xi})$ --- ДВВ.
Побудову закону розподілу $\eta$ доцільно розглянути на прикладі.
\begin{example}
    $\vec{\xi} = \left( \xi_1, \xi_2\right)$ задано таблицею розподілу:
    \begin{tabular}{|c|c|c|c|}
        \hline
        \diagbox{$\xi_2$}{$\xi_1$} & $0$ & $1$ & $2$ \\
        \hline
        $-1$ & $0.1$ & $0.2$ & $0.3$ \\
        \hline
        $1$ & $0.2$ & $0.1$ & $0.1$ \\
        \hline
    \end{tabular}.

    \noindentЗнайти закони розподілу $\eta_1 = \xi_1 \xi_2$ та $\eta_2 = \xi_1 + \xi_2$.
    Для цього треба визначити значення, які приймають ці величини, та обчислити відповідні ймовірності.

    \begin{tabular}{|c|c|c|c|c|c|}
        \hline
        $\eta_1$ & $-2$ & $-1$ & $0$ & $1$ & $2$ \\
        \hline
        $p$ & $0.3$ & $0.2$ & $0.3$ & $0.1$ & $0.1$ \\
        \hline
    \end{tabular}
    \begin{tabular}{|c|c|c|c|c|c|}
        \hline
        $\eta_2$ & $-1$ & $0$ & $1$ & $2$ & $3$ \\
        \hline
        $p$ & $0.1$ & $0.2$ & $0.5$ & $0.1$ & $0.1$ \\
        \hline
    \end{tabular}
\end{example}

Якщо $\vec{\xi} = \left(\xi_1, ..., \xi_n\right)$ --- неперервний випадковий вектор
із щільністю $f_{\vec{\xi}} (\vec{x})$, то можна знайти функцію розподілу $\eta = \varphi(\vec{\xi})$.
$$F_\eta (y) = P \left\{ \eta < y\right\} = P \left\{ \xi \in D_y\right\} = \underset{D_y}{\int ... \int} f_{\vec{\xi}} (\vec{x}) d \vec{x}, \text{ де }D_y = \left\{\vec{x} \in \mathbb{R}^n : \varphi(\vec{x}) < y\right\}$$

Розглянемо тепер взаємно однозначне гладке перетворення $\vec{\psi} : \mathbb{R}^n \to \mathbb{R}^n$ та
знайдемо щільність розподілу $\vec{\eta} = \vec{\psi} (\vec{\xi})$. Для множини $D \subset \mathbb{R}^n$
$P\left\{ \vec{\psi} (\vec{\xi}) \in D\right\} = P\left\{ \vec{\xi} \in \vec{\psi}^{-1}(D)\right\} = \int_{\vec{\psi}^{-1}(D)} f_{\vec{\xi}} (\vec{x}) d\vec{x} = 
\left[ \text{заміна }\vec{y} = \vec{\psi}(\vec{x})\right] = \int_D f_{\vec{\xi}} (\vec{\psi}^{-1}(\vec{y})) \left| \mathcal{J}^{-1} (\vec{\psi}^{-1}(\vec{y})\right| d\vec{y}$,
де $\mathcal{J} (\vec{x})$ --- якобіан $\vec{\psi}$. Отже,
$$f_{\vec{\eta}} (\vec{y}) = f_{\vec{\xi}} (\vec{\psi}^{-1}(\vec{y})) \left| \mathcal{J}^{-1} (\vec{\psi}^{-1}(\vec{y})\right|$$

\begin{example}
    Нехай $A$ --- невироджена матриця розміру $n \times n$, $\vec{b} \in \mathbb{R}^n$ --- деякий вектор, $\vec{\xi}$ --- неперервний випадковий вектор.
    Знайти щільність розподілу $\vec{\eta} = A \vec{\xi} + \vec{b}$.

    Тут $\vec{y} = \vec{\psi}(\vec{x}) = A \vec{x} + \vec{b}$, тому $\vec{\psi}^{-1} (\vec{y}) = A^{-1} (\vec{y} - \vec{b})$. Якобіан $\vec{\psi}$ рівний $\left| \det A\right|$. 
    Отже,
    $f_{\vec{\eta}}(\vec{y}) = \frac{1}{\left| \det A\right|} f_{\vec{\xi}}\left(A^{-1} (\vec{y} - \vec{b})\right)$
\end{example}

\subsection{Закон розподілу добутку двох НВВ}

\subsection{Закон розподілу частки двох НВВ}

\subsection{Закон розподілу суми двох НВВ}
        % !TEX root = ../main.tex
\section{Деякі нерівності}
\subsection{Нерівність Єнсена}

\noindent\textbf{Твердження.} Нехай $\xi$ --- деяка випадкова величина
з $E|\xi| < \infty$, а $\varphi(x)$ --- опукла функція. Тоді $\varphi(E\xi) \leq E\varphi(\xi)$.
\begin{proof}
    Оскільки $\varphi(x)$ --- опукла, то $\forall \; y \in \mathbb{R} \;\exists \; C(y): \forall \; x \in \mathbb{R}: \varphi(x) - \varphi(y) \geq c(y)(x-y)$.
    Покладемо $y = E\xi$, $x = \xi$ (оскільки $\xi$ приймає дійсні значення), і, взявши з обох сторін нерівності математичне сподівання,
    отримаємо $E\varphi(\xi) \geq \varphi(E\xi)$. 
\end{proof}
\begin{remark}
    Для увігнутої $\varphi(x)$ нерівність виконується в іншу сторону: $E\varphi(\xi) \leq \varphi(E\xi)$.
\end{remark}
\begin{example}
    Нехай $0 < s < t$. Розглянемо $\varphi(x) = |x|^{\frac{t}{s}}$, яка є опуклою. Скористаємося нерівністю
    Єнсена для $\eta = \left| \xi\right|^s$:
    $\varphi(E\eta) \leq E\varphi(\eta) \Leftrightarrow \left( E \left| \xi\right|^s\right)^{\frac{t}{s}} \leq E \left| \xi\right|^t 
    \Leftrightarrow ( E \left| \xi\right|^s)^{\frac{1}{s}} \leq ( E \left| \xi\right|^t)^{\frac{1}{t}}$.

    Маємо важливий \textbf{наслідок}: якщо у випадкової величини $\xi$ існує скінченний абсолютний момент $E\left|\xi\right|^m$ ($m\in \mathbb{N}$), то
    $$E \left| \xi\right| \leq (E \left| \xi\right|^2)^{\frac{1}{2}} \leq (E \left| \xi\right|^3)^{\frac{1}{3}} \leq ... \leq ( E \left| \xi\right|^m)^{\frac{1}{m}}$$

    Тобто, існування скінченного моменту $E\left|\xi\right|^m$ гарантує існування як початкових, так і центральних моментів нижчих порядків
    (оскільки $\left| E\xi \right|^k \leq E\left|\xi\right|^k$).
\end{example}

\subsection{Нерівність Гельдера}
\noindent\textbf{Твердження.} Нехай $1 < p < \infty$, $\frac{1}{p} + \frac{1}{q} = 1$. Якщо для випадкових величин $\xi$ та $\eta$ 
$E\left| \xi\right|^p$ та $E\left| \eta\right|^q$ скінченні, то $E\left| \xi \eta\right|$ теж скінченне, причому
$E\left| \xi \eta\right| \leq \left( E\left| \xi\right|^p\right)^{\frac{1}{p}} \cdot \left( E\left| \eta\right|^q\right)^{\frac{1}{q}}$.
\begin{proof}
    Якщо $E\left| \xi\right|^p = 0$ або $E\left| \eta\right|^q = 0$, то нерівність, очевидно, виконується.
    Нехай $E\left| \xi\right|^p >0$ та $E\left| \eta\right|^q > 0$. Позначимо $\xi_0 = \frac{\left| \xi \right|}{\left( E\left| \xi\right|^p\right)^{\frac{1}{p}}}$,
    $\eta_0 = \frac{\left| \eta \right|}{\left( E\left| \eta\right|^q\right)^{\frac{1}{q}}}$, причому $E\xi_0^p = E\eta_0^q = 1$.
    З опуклості функції $f(x) = -\ln x$ маємо $xy = \exp\{\ln xy\}= \exp\{\frac{\ln x^p}{p} + \frac{\ln y^q}{q}\} \leq
    \exp\{ \ln \left( \frac{x^p}{p} + \frac{y^q}{q}\right)\} = \frac{x^p}{p} + \frac{y^q}{q}$, тому
    $E\left( \xi_0 \eta_0 \right) \leq \frac{1}{p}E\xi_0^p + \frac{1}{q}E\eta_0^q = \frac{1}{p} + \frac{1}{q} = 1$,
    звідки $E\left| \xi \eta\right| \leq \left( E\left| \xi\right|^p\right)^{\frac{1}{p}} \cdot \left( E\left| \eta\right|^q\right)^{\frac{1}{q}}$.
\end{proof}
\begin{remark}
    При $p = q = 2$ отримуємо вже знайому нерівність Коші-Буняковського:
    
    \noindent$\left| E \xi \eta \right| \leq E\left| \xi \eta \right| \leq \sqrt{E\xi^2}\cdot\sqrt{E\eta^2}$.
\end{remark}

\subsection{Нерівність Мінковського}
\noindent\textbf{Твердження.} Нехай для випадкових величин $\xi$ та $\eta$ і $1\leq p < \infty$ маємо скінченні
$E\left| \xi\right|^p$ та $E\left| \eta\right|^p$. Тоді
$\left(E\left| \xi + \eta\right|^p\right)^{\frac{1}{p}} \leq \left(E\left| \xi \right|^p\right)^{\frac{1}{p}}
+\left(E\left| \eta \right|^p\right)^{\frac{1}{p}}$.
\begin{proof}
    Для $p=1$ нерівність є наслідком нерівності $\left| x+y\right| \leq |x| + |y|$ для дійсних чисел.
    Нехай $p > 1$, тоді $\left| \xi + \eta\right|^p = |\xi + \eta| \cdot |\xi + \eta|^{p-1} \leq |\xi| \cdot |\xi + \eta|^{p-1} +
    |\eta| \cdot |\xi + \eta|^{p-1}$.
    Покладемо $q = \frac{p}{p-1}$. Тоді $\frac{1}{p} + \frac{1}{q} = 1$, $E\left(|\xi + \eta|^{p-1} \right)^{q} = E|\xi+ \eta|^p$ скінченне, бо 
    для дійсних чисел виконується нерівність $|x+y|^p \leq C(p) \cdot (|x|^p + |y|^p)$.
    Тоді з нерівності Гельдера: $E\left( |\xi|\cdot |\xi + \eta|^{p-1}\right) \leq
    \left( E|\xi|^p\right)^{\frac{1}{p}} \cdot \left( E |\xi + \eta|^{(p-1)q}\right)^{\frac{1}{q}} = 
    \left( E|\xi|^p\right)^{\frac{1}{p}} \cdot \left( E |\xi + \eta|^{p}\right)^{\frac{1}{q}}$ і,
    аналогічно, 
    $E\left( |\eta|\cdot |\xi + \eta|^{p-1}\right) \leq 
    \left( E|\eta|^p\right)^{\frac{1}{p}} \cdot \left( E |\xi + \eta|^{p}\right)^{\frac{1}{q}}$.
    Отримуємо $E\left| \xi + \eta\right|^p \leq \left( E |\xi + \eta|^{p}\right)^{\frac{1}{q}} \cdot 
    \left(\left(E|\xi|^p\right)^{\frac{1}{p}} + \left(E|\eta|^p\right)^{\frac{1}{p}}\right)$.
    
    \noindentУ випадку $E\left| \xi + \eta\right|^p = 0$ виконання нерівності очевидне, а якщо $E\left| \xi + \eta\right|^p > 0$, то, поділивши на 
    $\left( E |\xi + \eta|^{p}\right)^{\frac{1}{q}}$, отримаємо
    $\left(E\left| \xi + \eta\right|^p\right)^{1 - \frac{1}{q}} \leq  
    \left(E|\xi|^p\right)^{\frac{1}{p}} + \left(E|\eta|^p\right)^{\frac{1}{p}}$.
    Залишилося зауважити, що $1 - \frac{1}{q} = \frac{1}{p}$.
\end{proof}
    \chapter{Основні розподіли математичної статистики}
        % !TEX root = ../main.tex

У цьому розділі буде наведено виведення законів розподілу, що застосовуються
в задачах математичної статистики, та їх числових характеристик.

\section{Розподіл \texorpdfstring{$\chi^2$}{x2} (Пірсона)}
\noindent\textbf{Означення:}
    нехай $\xi_k \sim \mathrm{N}(a_k, \sigma_k)$, $k= 1,..., n$ --- незалежні у сукупності.
    Тоді $\xi = \sum\limits_{k=1}^n \left( \frac{\xi_k - a_k}{\sigma_k}\right)^2$ має
    розподіл \emph{$\chi^2$ (хі-квадрат, Пірсона) з $n$ ступенями вільності}.

    \noindent$\mathring{\xi}_{k} = \frac{\xi_k - a_k}{\sigma_k} \sim \mathrm{N}(0, 1)$, тому можна ще записати
    $\xi = \sum\limits_{k=1}^n (\mathring{\xi}_{k})^2$.

\noindent\textbf{Коротке позначення:} $\xi \sim \chi_n^2$, $n\in\mathbb{N}$ --- кількість ступенів вільності.

\noindent\textbf{Щільність розподілу:}
відомо, що якщо $\eta \sim \mathrm{N}(0, 1)$, то $\eta^2 \sim \Gamma\left(\frac{1}{2}, \frac{1}{2}\right)$.
Гамма-розподіл стійкий при $\beta_1 = \beta_2 = ... = \beta_n$, $\mathring{\xi}_{k}$ незалежні у сукупності,
тому $\xi = \sum\limits_{k=1}^n (\mathring{\xi}_{k})^2 \sim \Gamma\left(\frac{n}{2}, \frac{1}{2}\right) = \chi_n^2$.
\begin{equation*}
    f_{\chi_n^2}(x) = \begin{cases}
        \frac{1}{2^{\frac{n}{2}} \Gamma\left(\frac{n}{2}\right)} x^{\frac{n}{2}-1} e^{-\frac{x}{2}}, & x \geq 0 \\
        0, & x < 0
    \end{cases}
\end{equation*}

\noindent \textbf{Крива розподілу:} графіки для різних значень $n$.

\begin{tikzpicture}[yscale = 12, xscale = 0.5, baseline={(current bounding box.center)}]
    \pgfmathsetmacro{\a}{2};
    \pgfmathsetmacro{\b}{0.5};
    \pgfmathsetmacro{\c}{3};
    \pgfmathsetmacro{\d}{4};

    \draw [->] (-2, 0) -- (18.3, 0);
    \draw [->] (0, -0.05) -- (0, 0.2);
    \draw [thick] (-2, 0) -- (0, 0);
    \draw [domain=0:18, smooth, variable = \x, thick] plot ({\x}, {\b^(\a)*\x^(\a-1)/factorial(\a-1) * e^(-\x*\b)});
    \draw [domain=0:18, smooth, variable = \x, thick] plot ({\x}, {\b^(\c)*\x^(\c-1)/factorial(\c-1) * e^(-\x*\b)});
    \draw [domain=0:18, smooth, variable = \x, thick] plot ({\x}, {\b^(\d)*\x^(\d-1)/factorial(\d-1) * e^(-\x*\b)});
    \node [below] at (18.2, 0) {$x$};
    \node [left] at (0, 0.2) {$f_{\chi_n^2}(x)$};
    \node [below left] at (0, 0) {$0$};
\end{tikzpicture}

\noindent\textbf{Числові характеристики:}
\begin{enumerate}
    \item $E\chi_n^2 = \frac{n/2}{1/2} = n$.
    \item $D\chi_n^2 = \frac{n/2}{1/4} = 2n$.
\end{enumerate}

\section{Розподіл \texorpdfstring{$\chi$}{x}}
\noindent\textbf{Означення:} нехай випадкова величина $\xi$ має розподіл $\chi_n^2$.
Тоді $\eta = \sqrt{\xi}$ має розподіл
\emph{$\chi$ (хі) з $n$ ступенями вільності}.

\noindent\textbf{Коротке позначення:} $\eta \sim \chi_n$, $n\in\mathbb{N}$ --- кількість ступенів вільності.

\noindent\textbf{Щільність розподілу:} скористаємося формулою для визначення щільності розподілу функції від
випадкової величини. $\eta = \sqrt{\xi}$, тому позначимо $\varphi(x) = \sqrt{x}$, $\varphi^{-1}(y) = y^2$,
$\left( \varphi^{-1}(y) \right)^{\prime} = 2y$.
\begin{equation*}
    f_{\chi_n}(y) = f_{\chi_n^2}\left(\varphi^{-1} (y)\right) \cdot \left|\left(\varphi^{-1} (y) \right)^{\prime}\right| =
f_{\chi_n^2}(y^2) \cdot 2y = 
\begin{cases}
    \frac{1}{2^{\frac{n}{2} - 1} \Gamma\left(\frac{n}{2}\right)} y^{n-1} e^{-\frac{y^2}{2}}, & y \geq 0 \\
    0, & y < 0
\end{cases}
\end{equation*}
\noindent\textbf{Числові характеристики:}
\begin{enumerate}
    \item $E\chi_n = \sqrt{2} \cdot \frac{\Gamma\left(\frac{n+1}{2}\right)}{\Gamma\left(\frac{n}{2}\right)}$.
    \item $D\chi_n = n - \left( E\chi_n \right)^2$.
\end{enumerate}

\begin{remark}
    Нескладно помітити, що $\chi_2$ --- це розподіл Релея.
\end{remark} 

\noindent Знайдемо ще розподіл $\zeta = \frac{\chi_n}{\sqrt{n}}$. $\varphi(y) = \frac{y}{\sqrt{n}}$,
$\varphi^{-1}(z) = z \sqrt{n}$, 
$\left( \varphi^{-1}(z) \right)^{\prime} = \sqrt{n}$.

\begin{equation*}
    f_{\frac{\chi_n}{\sqrt{n}}}(z) = f_{\chi_n}\left(\varphi^{-1} (z)\right) \cdot \left|\left(\varphi^{-1} (y) \right)^{\prime}\right| =
f_{\chi_n}(z \sqrt{n}) \cdot \sqrt{n} = 
\begin{cases}
    \frac{n^{\frac{n}{2}}}{2^{\frac{n}{2} - 1} \Gamma\left(\frac{n}{2}\right)} z^{n-1} e^{-\frac{nz^2}{2}}, & z \geq 0 \\
    0, & z < 0
\end{cases}
\end{equation*}
\begin{exercise}
    Записати числові характеристики випадкової величини, що має розподіл $\frac{\chi_n}{\sqrt{n}}$.
\end{exercise}

\section{Розподіл Стьюдента (\texorpdfstring{$t$}{t}-розподіл)}
\noindent\textbf{Означення:} якщо $\xi \sim \mathrm{N}(0, 1)$ та $\eta \sim \frac{\chi_n}{\sqrt{n}}$ незалежні,
то $\zeta = \frac{\xi}{\eta} = \frac{\xi}{\chi_n /\sqrt{n}}$ має \emph{розподіл Стьюдента з $n$ ступенями вільності}.

\noindent\textbf{Коротке позначення:} $\zeta \sim \mathrm{St}_n$, $n\in\mathbb{N}$ --- кількість ступенів вільності.

\noindent\textbf{Щільність розподілу:} скористаємося формулою для визначення щільності розподілу частки
двох незалежних НВВ.
\begin{gather*}
    f_{\mathrm{St}_n} (z) = \int\limits_0^{+\infty} x f_{\xi}(z x) f_{\frac{\chi_n}{\sqrt{n}}} (x) dx =
    \frac{1}{\sqrt{2\pi}} \cdot \frac{n^{\frac{n}{2}}}{2^{\frac{n}{2} - 1} \Gamma\left(\frac{n}{2}\right)}
    \int\limits_0^{+\infty} x e^{-\frac{z^2 x^2}{2}} x^{n-1} e^{-\frac{nx^2}{2}} dx = \\
    = \frac{n^{\frac{n}{2}}}{\sqrt{\pi} 2^{\frac{n-1}{2}} \Gamma\left(\frac{n}{2}\right)}
    \int\limits_0^{+\infty} x^n e^{-\frac{x^2}{2}(z^2+n)} dx = 
    \left[ \frac{x^2}{2} (z^2 + n) = t, x = \frac{\sqrt{2} \sqrt{t}}{\sqrt{z^2 + n}},
    dx = \frac{1}{\sqrt{2}\sqrt{z^2 + n}} \frac{dt}{\sqrt{t}}\right] = \\
    = \frac{n^{\frac{n}{2}}}{\sqrt{\pi} 2^{\frac{n-1}{2}} \Gamma\left(\frac{n}{2}\right)} \cdot
    \frac{2^{\frac{n}{2}}}{(z^2 + n)^{\frac{n}{2}}} \frac{1}{2^{\frac{1}{2}}(z^2 + n)^{\frac{n}{2}}}
    \int\limits_0^{+\infty} t^{\frac{n-1}{2}} e^{-t} dt = 
    \frac{n^{\frac{n}{2}} \Gamma\left(\frac{n+1}{2}\right)}{\sqrt{\pi} \Gamma\left(\frac{n}{2}\right)} \cdot
    \frac{1}{(z^2 + n)^{\frac{n+1}{2}}}, \; z \in \mathbb{R}
\end{gather*}
\noindent \textbf{Крива розподілу:} графіки для різних значень $n$, називаються \emph{кривими Стьюдента}.
Вони схожі на криву гауссівського розподілу, але повільніше прямують до 0 на нескінченності.

\begin{tikzpicture}[yscale = 6, xscale = 1.3, baseline={(current bounding box.center)}]
    \pgfmathsetmacro{\a}{1}; % n = 2
    \pgfmathsetmacro{\b}{12}; % n = 4
    \pgfmathsetmacro{\c}{0.318309886184}; % n = 1

    \draw [->] (-5, 0) -- (5, 0);
    \draw [->] (0, -0.05) -- (0, 0.4);
    \draw [domain=-5:5, smooth, variable = \x, thick] plot ({\x}, {\a/((2+(\x)^2)^((2+1)/2))});
    \draw [domain=-5:5, smooth, variable = \x, thick] plot ({\x}, {\b/((4+(\x)^2)^((4+1)/2))});
    \draw [domain=-5:5, smooth, variable = \x, thick] plot ({\x}, {\c/((1+(\x)^2)^((1+1)/2))});
    \node [below] at (5.2, 0) {$x$};
    \node [left] at (0, 0.4) {$f_{\mathrm{St}_n}(x)$};
    \node [below left] at (0, 0) {$0$};
\end{tikzpicture}

\noindent\textbf{Числові характеристики:}
\begin{enumerate}
    \item $E\mathrm{St}_n = 0$.
    \item $D\mathrm{St}_n = \frac{n}{n-2}$, $n>2$.
\end{enumerate}

\begin{remark}
    Нескладно помітити, що $\mathrm{St}_1$ --- це розподіл Коші.
\end{remark}

\section{Розподіл Фішера-Снедекора (\texorpdfstring{$F$}{F}-розподіл)}
\noindent\textbf{Означення:} випадкова величина $\eta = \frac{\chi_{n_1}^2/n_1}{\chi_{n_2}^2/n_2}$, чисельник
та знаменник якої незалежні, має \emph{розподіл Фішера-Снедекора з $n_1$, $n_2$ ступенями вільності}.

\noindent\textbf{Коротке позначення:} $\eta \sim \mathrm{F}(n_1, n_2)$, $n_1, n_2\in\mathbb{N}$ --- кількість ступенів вільності.

\noindent\textbf{Щільність розподілу:} скористаємося формулою для визначення щільності розподілу частки
двох незалежних НВВ. Нагадаємо, що 
$f_{\chi_n^2}(x) = \begin{cases}
    \frac{1}{2^{\frac{n}{2}} \Gamma\left(\frac{n}{2}\right)} x^{\frac{n}{2}-1} e^{-\frac{x}{2}}, & x \geq 0 \\
    0, & x < 0
\end{cases}$.

\noindentТоді $f_{\chi_n^2/n}(y) = f_{\chi_n^2}(n y) \cdot n = 
\begin{cases}
    \frac{n^{\frac{n}{2}}}{2^{\frac{n}{2}} \Gamma\left(\frac{n}{2}\right)} y^{\frac{n}{2}-1} e^{-\frac{ny}{2}}, & y \geq 0 \\
    0, & y < 0
\end{cases}$.

\begin{gather*}
    f_{\mathrm{F}(n_1, n_2)} (z) = \int\limits_0^{+\infty} x f_{\chi_{n_1}^2/n_1}(zx) f_{\chi_{n_2}^2/n_2}(x) dx = \\
    = \frac{n_1^{\frac{n_1}{2}} n_2^{\frac{n_2}{2}}}{2^{\frac{n_1+n_2}{2}} \Gamma\left(\frac{n_1}{2}\right) \Gamma\left(\frac{n_2}{2}\right)}
    \int\limits_0^{+\infty} x z^{\frac{n_1}{2} - 1} x^{\frac{n_1}{2} - 1} e^{-\frac{n_1 zx}{2}} x^{\frac{n_2}{2} - 1} e^{-\frac{n_2 x}{2}} dx = \\
    = \frac{n_1^{\frac{n_1}{2}} n_2^{\frac{n_2}{2}}}{2^{\frac{n_1+n_2}{2}} \Gamma\left(\frac{n_1}{2}\right) \Gamma\left(\frac{n_2}{2}\right)} \cdot
    z^{\frac{n_1}{2} - 1} \int\limits_0^{+\infty} x^{\frac{n_1 + n_2}{2} - 1} e^{-\frac{x}{2}(n_1 z + n_2)} dx =
    \left[ \frac{x}{2}(n_1 z + n_2) = t, x = \frac{2t}{n_1 z + n_2}\right] = \\
    = \frac{n_1^{\frac{n_1}{2}} n_2^{\frac{n_2}{2}}}{2^{\frac{n_1+n_2}{2}} \Gamma\left(\frac{n_1}{2}\right) \Gamma\left(\frac{n_2}{2}\right)} \cdot
    z^{\frac{n_1}{2} - 1} \cdot 2^{\frac{n_1+n_2}{2}} \cdot \frac{1}{(n_1 z + n_2)^{\frac{n_1+n_2}{2}}}
    \int\limits_0^{+\infty} t^{\frac{n_1+n_2}{2} - 1} e^{-t} dt = \\
    = n_1^{\frac{n_1}{2}} n_2^{\frac{n_2}{2}} \frac{\Gamma\left(\frac{n_1+n_2}{2}\right)}{\Gamma\left(\frac{n_1}{2}\right) \Gamma\left(\frac{n_2}{2}\right)} \cdot
    \frac{z^{\frac{n_1}{2} - 1}}{(n_1 z + n_2)^{\frac{n_1+n_2}{2}}}, \; z \geq 0 \text{ та } 0 \text{ інакше}.
\end{gather*}

\noindent \textbf{Крива розподілу:} графіки для різних значень $n_1$, $n_2$, називаються \emph{кривими Фішера}.

\begin{center}
    \begin{tikzpicture}[yscale = 6, xscale = 3, baseline={(current bounding box.center)}]
        \pgfmathsetmacro{\a}{3.30797337253}; % n1 = 3, n2 = 1
        \pgfmathsetmacro{\b}{64}; % n1 = 4, n2 = 2
        \pgfmathsetmacro{\c}{68.7549354157}; % n = 1
    
        \draw [->] (-0.5, 0) -- (4, 0);
        \draw [->] (0, -0.05) -- (0, 0.7);
        \draw [domain=0:4, smooth, variable = \x, thick, samples = 200] plot ({\x}, {\a*((\x)^(3/2 - 1))/(3*\x + 1)^((3+1)/2)});
        \draw [domain=0:4, smooth, variable = \x, thick, samples = 400] plot ({\x}, {\b*((\x)^(4/2 - 1))/(4*\x + 2)^((4+2)/2)});
        \draw [domain=0:4, smooth, variable = \x, thick, samples = 200] plot ({\x}, {\c*((\x)^(3/2 - 1))/(3*\x + 3)^((3+3)/2)});
        \node [below] at (4, 0) {$x$};
        \draw [thick] (-0.5, 0) -- (0, 0);
        \node [left] at (0, 0.7) {$f_{\mathrm{F}(n_1, n_2)}(x)$};
        \node [below left] at (0, 0) {$0$};
    \end{tikzpicture} 
\end{center}

\noindent\textbf{Числові характеристики:}
\begin{enumerate}
    \item $E\mathrm{F}(n_1, n_2) = \frac{n_2}{n_2 - 2}$, $n_2 > 2$.
    \item $D\mathrm{F}(n_1, n_2) = \frac{2 n_2^2 (n_1 + n_2 - 2)}{n_1 (n_2 - 2)^2 (n_2 -4)}$, $n_2>4$.
\end{enumerate}

\begin{remark}
    Якщо $\eta \sim \mathrm{F}(n_1, n_2)$, то $\frac{1}{\eta} \sim \mathrm{F}(n_2, n_1)$.
\end{remark}
    \chapter{Граничні теореми теорії ймовірностей}
        % !TEX root = ../main.tex
\section{Послідовності випадкових величин}
\subsection{Нерівності Маркова та Чебишова}
\begin{theorem*}[нерівність Маркова]
    Нехай модуль випадкової величини $\xi$ має скінченне математичне сподівання: $E|\xi| < +\infty$.
    Тоді 
    \begin{gather}\label{Markov_ineq}
        \forall \; \varepsilon >0 : P\left\{ |\xi| \geq \varepsilon\right\} \leq \frac{E|\xi|}{\varepsilon}
    \end{gather}
\end{theorem*}
\begin{proof}
    Запишемо випадкову величину $|\xi|$ через події-індикатори: 
    $|\xi| = |\xi|\cdot I\left\{|\xi| \geq \varepsilon\right\} + |\xi|\cdot I\left\{|\xi| < \varepsilon\right\} \geq
    \varepsilon\cdot I\left\{|\xi| \geq \varepsilon\right\}$. Звідси $E|\xi| \geq E \left( \varepsilon\cdot I\left\{|\xi| \geq \varepsilon\right\}\right) =
    \varepsilon \cdot P\left\{ |\xi| \geq \varepsilon\right\}$.
\end{proof}
\begin{remark}
    Еквівалентною нерівністю є $P\left\{ |\xi| < \varepsilon\right\} \geq 1 - \frac{E|\xi|}{\varepsilon}$.
\end{remark}

\begin{example}
    Багатьма спостереженнями з'ясовано, що середня кількість
    сонячних дні у Києві складає 220. Оцінити ймовірність того, що
    сонячних днів за рік буде не менше 300.

    \noindentПозначимо $\xi$ кількість сонячних днів. За умовою $\xi$ невід'ємна та $E\xi = 220$,
    тому за нерівністю Маркова $P\left\{ \xi \geq 300\right\} \leq \frac{220}{300} = \frac{11}{15}$.
\end{example}

\begin{theorem*}[нерівність Чебишова]
    Нехай випадкова величина $\xi$ має скінченні математичне сподівання та дисперсію.
Тоді
\begin{gather}\label{Cheb_ineq}
    \forall \; \varepsilon >0 : P\left\{ \left|\xi - E\xi\right| \geq \varepsilon\right\} \leq \frac{D\xi}{\varepsilon^2}
\end{gather}
\end{theorem*} 
\begin{proof}
    $P\left\{ |\xi - E\xi| \geq \varepsilon\right\} = P\left\{ (\xi - E\xi)^2 \geq \varepsilon^2\right\}$.
    Застосуємо нерівність Маркова:
    
    \noindent $P\left\{ (\xi - E\xi)^2 \geq \varepsilon^2\right\} \leq \frac{E(\xi - E\xi)^2}{\varepsilon^2} = \frac{D\xi}{\varepsilon^2}$.
\end{proof}
\begin{remark}
    Еквівалентною нерівністю є $P\left\{ \left|\xi - E\xi\right| < \varepsilon\right\} \geq 1 - \frac{D\xi}{\varepsilon^2}$.
\end{remark}
\begin{example}
    Отримаємо <<правило $3 \sigma$>> для довільної випадкової величини зі скінченними математичним сподівання та дисперсією.
    $P\left\{ |\xi - E\xi| < 3 \sigma\right\} \geq 1 - \frac{D\xi}{9 \sigma^2} = 1 - \frac{\sigma^2}{9 \sigma^2} = \frac{8}{9}$.
\end{example}
Розглянемо застосування \emph{нерівності Чебишова в схемі Бернуллі}. Нехай $\xi \sim \mathrm{Bin}(n, p)$, $E\xi = np$, $D\xi = npq$.
Відношення $\frac{\xi}{n}$ називається відносною частотою появи успіху або частістю. $E\left( \frac{\xi}{n}\right) = p$, 
$D\left( \frac{\xi}{n}\right) = \frac{pq}{n}$. З нерівності Чебишова 
$P\left\{ \left|\frac{\xi}{n} - p\right| \geq \varepsilon\right\} \leq \frac{pq}{n \epsilon^2}$.

\subsection{Послідовності випадкових величин}
Розглядаємо фіксований ймовірнісний простір $\left\{ \Omega, \mathcal{F}, P\right\}$ та
послідовність випадкових величин $\left\{ \xi_n (\omega)\right\}_{n=1}^{\infty}$.
Якщо для кожного $n \in \mathbb{N}$ події $\xi_1, \xi_2, ..., \xi_n$ \emph{незалежні у сукупності},
то послідовність $\left\{ \xi_n (\omega)\right\}_{n=1}^{\infty}$ називається \emph{послідовністю незалежних випадкових величин}.

Послідовності випадкових величин можна задавати різними способами:
\begin{enumerate}
    \item Нехай $\xi$ --- деяка випадкова величина, можна задати $\xi_n = f_n(\xi)$, де $f_n$ --- деяка числова функція. 
    Наприклад: $\xi_n = \xi^n$, $\xi_n = \cos (n\xi)$.
    \item $n$ може входити як параметр закону розподілу $\xi_n$. Наприклад, $\xi_n \sim \mathrm{Exp}(n)$, $\xi_n \sim \mathrm{N}(0, \frac{1}{n})$.
    \item Для послідовностей ДВВ $n$ може входити як в значення, що приймає $\xi_n$, так і у відповідні ймовірності. Наприклад:
    \begin{tabular}{|c|c|c|c|}
        \hline
        $\xi_n$ & $-\sqrt{n}$ & $0$ & $\sqrt{n}$ \\
        \hline
        $p$ & $1/n$ & $1 - 2/n$ & $1/n$ \\
        \hline
    \end{tabular}
\end{enumerate}
В курсі функціонального аналізу вводяться різні види збіжності послідовності вимірних функцій та
зв'язок між цими видами збіжності.

\subsection{Види збіжності послідовності випадкових величин}
Нагадаємо класичне означення границі числової послідовності. Число $a$ називають границею послідовності 
$\left\{ a_n\right\}_{n=1}^{\infty}$, якщо
$$ \forall \; \varepsilon > 0 \; \exists \; N(\varepsilon) \in \mathbb{N}: \forall \; n \geq N(\varepsilon) \; \left| a_n - a\right| < \varepsilon$$
Оскільки послідовність випадкових величин $\left\{ \xi_n\right\}_{n=1}^{\infty}$ є послідовністю функцій з $\Omega$ в $\mathbb{R}$, то це означення
не є застосовним, бо $\left| \xi_n - \xi\right| < \varepsilon$ є випадковою подією, що виконується, взагалі кажучи, не для всіх елементарних подій
$\omega \in \Omega$. ших. Тому ми маємо ввести інше означення границі (та збіжності) послідовності
випадкових величин. Виявляється, що таких означень можна запропонувати декілька, причому вони
не є еквівалентними одне одному.
\vspace{1em}
\begin{enumerate}
    \item \textbf{Збіжність майже напевно (сильна збіжність, збіжність з ймовірністю 1).}
    \noindent$\left\{ \xi_n\right\}_{n=1}^{\infty}$ збігається до $\xi$ \emph{майже напевно}, якщо
    $P\left\{ \omega: \underset{n\to\infty}{\lim} \xi_n(\omega) = \xi(\omega)\right\} = 1$. 
    Це еквівалентно умові $P\left\{ \omega: \underset{n\to\infty}{\lim} \xi_n(\omega) \neq \xi(\omega)\right\} = 0$.

    Позначення: $\xi_n \overset{P1}{\longrightarrow} \xi, n \to \infty$ або $\xi_n \overset{\text{м.н.}}{\longrightarrow} \xi, n \to \infty$.
    \begin{exercise}
        Довести ще одне еквівалентне визначення збіжності майже напевно: $\xi_n \overset{P1}{\longrightarrow} \xi, n \to \infty$, якщо
        $\forall \; \varepsilon > 0: P\left( \bigcap\limits_{n=1}^{\infty} \bigcup\limits_{k = n}^{\infty}
        \left\{ \omega : \left| \xi_k(\omega) - \xi(\omega)\right| > \varepsilon\right\}\right) = 0$.
    \end{exercise}

    Це найбільш природний з інтуїтивної точки зору вид збіжності. Зауважимо, однак, що він має доволі дивні властивості. Наприклад, можна навести приклад
    послідовності $\xi_n$, що не збігається до деякої $\xi$, але будь-яка її підпослідовність $\xi_{n_k}$ містить свою підпослідовність
    $\xi_{n_{k_l}}$, яка все ж таки збігається до $\xi$.
    
    \textbf{Лема Бореля-Кантеллі.} Якщо для послідовності подій $\left\{A_n \right\}_{n=1}^{\infty}$ ряд
    $\sum\limits_{n=1}^{\infty} P(A_n)$ збігається, то $P\left( \bigcap\limits_{n=1}^{\infty} \bigcup\limits_{k = n}^{\infty} A_k\right) = 0$.
    Це означає, що ймовірність того, що відбудеться нескінченна кількість цих подій, є нульовою.
    \begin{proof}
        Послідовність подій $B_n = \bigcup\limits_{k = n}^{\infty} A_k$ монотонно спадна, тому за теоремою неперервності \ref{th:2}
        $P\left( \bigcap\limits_{n=1}^{\infty} B_n\right) = \underset{n\to\infty}{\lim} P(B_n)$.
        $P(B_n) = P\left( \bigcup\limits_{k = n}^{\infty} A_k\right) \leq \sum\limits_{k=n}^{\infty}P(A_n) \to 0$ при $n\to \infty$ 
        зі збіжності ряду. Тому $\underset{n\to\infty}{\lim} P(B_n) = P\left( \bigcap\limits_{n=1}^{\infty} \bigcup\limits_{k = n}^{\infty} A_k\right) = 0$.
    \end{proof}
    Застосуванням цієї леми до послідовності подій $A_n = \left\{ \omega : \left| \xi_n(\omega) - \xi(\omega)\right| > \varepsilon\right\}$ отримаємо зручну для використання
    ознаку збіжності майже напевно: якщо для будь-якого $\varepsilon >0$ ряд
    $\sum\limits_{n=1}^{\infty} P\left\{\left| \xi_n - \xi\right| > \varepsilon\right\}$ збігається, то 
    $\xi_n \overset{P1}{\longrightarrow} \xi, n \to \infty$.
    \begin{example}
        Довести, що послідовність $\xi_n \sim \mathrm{U}(-\frac{1}{n}, \frac{1}{n})$ при $n\to\infty$ збігається до 0 майже напевно.
        
        Візьмемо довільне $\varepsilon > 0$ та знайдемо $P\left\{ |\xi_n| > \varepsilon\right\}$. Для $\varepsilon \geq 1$
        ця ймовірність, очевидно, рівна 0. В іншому випадку, для кожного $\varepsilon \in (0; 1)$ можна знайти такий номер $N$,
        для якого $\varepsilon$ буде більше за $\frac{1}{n}$ при $n \geq N$. Тому для будь-якого $\varepsilon >0$ ймовірності
        $P\left\{ |\xi_n| > \varepsilon\right\}$ рівні 0, починаючи з якогось $n$. Отже, для будь-якого $\varepsilon >0$ ряд 
        $\sum\limits_{n=1}^{\infty} P\left\{\left| \xi_n \right| > \varepsilon\right\}$ збігається і 
        $\xi_n \overset{P1}{\longrightarrow} 0, n \to \infty$.
    \end{example}
    \item \textbf{Збіжність за ймовірністю.}
    \noindent$\left\{ \xi_n\right\}_{n=1}^{\infty}$ збігається до $\xi$ \emph{за ймовірністю}, якщо 
    $\forall \; \varepsilon > 0: \underset{n \to \infty}{\lim} P\left\{|\xi_n - \xi| \geq \varepsilon\right\}= 0$.
    Це еквівалентно умові $\forall \; \varepsilon > 0: \underset{n \to \infty}{\lim} P\left\{|\xi_n - \xi| \leq \varepsilon\right\}= 1$.
    Позначення: $\xi_n \overset{P}{\longrightarrow} \xi, n \to \infty$.
    \begin{example}
        Нехай $\xi_n$ --- послідовність ДВВ: 
        \begin{tabular}{|c|c|c|}
            \hline
            $\xi_n$ & $0$ & $n^7$ \\
            \hline
            $p$ & $1-1/n$ & $1/n$ \\
            \hline
        \end{tabular}.
        Перевірити збіжність $\xi_n \overset{P}{\longrightarrow} 0, n \to \infty$.
        
        \noindentДля $\varepsilon >0$ $P\left\{|\xi_n -0| \geq \varepsilon\right\} = P\left\{ \xi_n \geq \varepsilon\right\}$.
        $\forall \; \varepsilon >0 \; \exists \; N: \forall n\geq N:n^7 > \varepsilon$, тому з якогось номера
        $P\left\{ \xi_n \geq \varepsilon\right\} = P\left\{ \xi_n = n^7 \right\} = \frac{1}{n} \to 0, n\to\infty$, тому $\xi_n \overset{P}{\longrightarrow} 0, n \to \infty$.
    \end{example}
    \item \textbf{Збіжність в середньому.}
    \noindent$\left\{ \xi_n\right\}_{n=1}^{\infty}$ збігається до $\xi$ \emph{в середньому},
    якщо $\underset{n \to \infty}{\lim} E|\xi_n - \xi| = 0$.
    Позначення: $\xi_n \overset{\text{С}}{\longrightarrow} \xi, n \to \infty$.
    \item \textbf{Збіжність в середньому квадратичному.}
    \noindent$\left\{ \xi_n\right\}_{n=1}^{\infty}$ збігається до $\xi$ \emph{в середньому квадратичному},
    якщо $\underset{n \to \infty}{\lim} E(\xi_n - \xi)^2 = 0$.
    
    \noindentПозначення: $\xi_n \overset{\text{СК}}{\longrightarrow} \xi, n \to \infty$.
    \item \textbf{Збіжність за розподілом (слабка збіжність, збіжність в основному).}

    \noindent$\left\{ \xi_n\right\}_{n=1}^{\infty}$ збігається до $\xi$ \emph{за розподілом}, якщо функціональна послідовність
    $\left\{ F_{\xi_n} (x)\right\}_{n=1}^{\infty}$ збігається до $F_{\xi}(x)$ в точках її неперервності.

    Позначення: $\xi_n \overset{F}{\longrightarrow} \xi, n \to \infty$.

    Ця збіжність за характером відрізняється від інших тим, що не враховує залежність або незалежність $\xi_n$.
    Є також \emph{еквівалентне означення збіжності} за розподілом: якщо для будь-якої обмеженої неперервної функції $\varphi : \mathbb{R} \to \mathbb{R}$
    $\underset{n \to \infty}{\lim} E \varphi(\xi_n) = E\varphi(\xi)$.
    Доведення еквівалентності цих двох означень виходить за рамки курсу: його ідея полягає в тому, що $F_{\xi}(x) = P\left\{ \xi < x\right\} = E 1_{(-\infty, x)}(\xi)$,
    де $1_{(-\infty, x)}$ --- індикатор множини $(-\infty, x)$, і такі функції-індикатори можна наблизити з будь-якою заданою точністю неперервними функціями та навпаки.
    \begin{example}
        Нехай $\xi_n \sim \mathrm{Exp}(\frac{1}{n})$. Перевірити $\xi_n \overset{F}{\to} 0, n \to \infty$.
        Тут під 0 розуміється ДВВ, що приймає значення 0 з ймовірністю 1.

        $F_{\xi_n}(x) = \begin{cases}
            1 - e^{\frac{x}{n}}, & x > 0 \\
            0, & x \leq 0
        \end{cases}$. Видно, що при $n \to \infty$ $F_{\xi_n}(x) \to F_0(x) = \begin{cases}
            1, & x > 0 \\
            0, & x \leq 0
        \end{cases}
        $ --- функція розподілу 0.
    \end{example}
\end{enumerate}

\begin{remark}
    Збіжності в середньому та середньому квадратичному є частковими випадками \emph{збіжності порядку $k$}, для якої
$\underset{n \to \infty}{\lim} E|\xi_n - \xi|^k = 0$. Оскільки 
для $0 < s < t$ має місце
$( E \left| \xi\right|^s)^{\frac{1}{s}} \leq ( E \left| \xi\right|^t)^{\frac{1}{t}}$
то збіжність порядку $k$ гарантує збіжність порядків менше $k$.
\end{remark}
\begin{proposition*}
     Нехай послідовність $\left\{ \xi_n\right\}_{n=1}^{\infty}$ збігається за ймовірністю до двох випадкових
    величин $\xi$ та $\eta$. Тоді $P\left\{ \xi = \eta\right\} = 1$.
\end{proposition*}
\begin{proof}
    Для будь-якого $\varepsilon > 0$ : $\left\{ |\xi - \eta| > \varepsilon\right\} \subset 
    \left\{ |\xi - \xi_n| > \frac{\varepsilon}{2}\right\} \cup \left\{ |\eta - \xi_n| > \frac{\varepsilon}{2}\right\}$, тому \\
    $P\left\{ |\xi - \eta| > \varepsilon\right\} \leq P\left\{ |\xi - \xi_n| > \frac{\varepsilon}{2}\right\} + 
    P\left\{ |\eta - \xi_n| > \frac{\varepsilon}{2}\right\} \to 0, n\to\infty$.
    Отже, $P\left\{ |\xi - \eta| > \varepsilon\right\} = 0$, і за довільністю $\varepsilon > 0$ маємо $P\left\{ \xi \neq \eta\right\} = 0$,
    або ж $P\left\{ \xi = \eta\right\} = 1$.
\end{proof}
\begin{exercise}
    Довести, що це твердження виконується для збіжностей з ймовірністю 1, в середньому та середньому квадратичному, але не виконується
    для збіжності за розподілом.
\end{exercise}
Можна довести, що усім збіжностям, крім збіжності за розподілом, притаманні відомі арифметичні властивості: збіжність суми $\xi_n + \eta_n$ та добутку
$\xi_n \eta_n$ до $\xi + \eta$ та $\xi \eta$ відповідно за умови збіжностей $\xi_n$ до $\xi$ та $\eta_n$ до $\eta$. Також, $\varphi(\xi_n)$ збігається
до $\varphi(\xi)$ за умови неперервності $\varphi$.

В курсі функціонального аналізу встановлюється зв'язок між розглянутими видами збіжності:
\vspace{5 pt}
\begin{center}
\begin{tikzpicture}
\label{scheme}
\node[block] (a) {Збіжність з ймовірністю 1 \\ ($P1$)};
\node[block, above right = 0cm and 2cm of a] (b) {Збіжність в середньому\\ квадратичному ($\text{СК}$)};
\node[block, below right = 0cm and 2cm of a]   (c) {Збіжність в середньому \\ ($\text{С}$)};
\node[block, below right = 2cm and -1.5cm of a]   (d) {Збіжність за ймовірністю \\ ($P$)};
\node[block, below = 1cm of d]   (e){Збіжність за розподілом \\ ($F$)};
\node[below left = 1cm and -2.5 cm of a]   (1){\circled 1};
\node[below right = 0.25cm and -1.6 cm of d]   (2){\circled 2};

\draw[line] (a.south) -- ([xshift=-0.5cm]d.north);
\draw[line] (b.south) -- (c.north);
\draw[line] (c.south) -- ([xshift=0.5cm]d.north);
\draw[line] ([xshift=-0.5cm]d.south) -- ([xshift=-0.5cm]e.north);
\draw[line,dashed] ([xshift=-1.5cm]d.north) -- ([xshift=-1cm]a.south);
\draw[line,dashdotted] ([xshift=0.5cm]e.north) -- ([xshift=0.5cm]d.south);
\end{tikzpicture}
\end{center}

Штрихова лінія \circled 1 означає, що у кожній послідовності $\left\{ \xi_n\right\}_{n=1}^{\infty}$, що збігається за ймовірністю, 
міститься підпослідовність $\left\{ \xi_{n_k}\right\}_{k=1}^{\infty}$, що збігається з імовірністю 1. 
Штрих-пунктирна лінія \circled 2 показує, що відповідний перехід справедливий, коли гранична випадкова величина $\xi$ є константою.
Доведемо деякі з цих тверджень.

\begin{proposition*} 
    Зі збіжності в середньому квадратичному випливає збіжність в середньому ($\text{СК} \Rightarrow \text{С}$).
\end{proposition*}
\begin{proof}
    $E|\xi_n -\xi| \leq \sqrt{E(\xi_n - \xi)^2}$, як було сказано вище, тому з $E(\xi_n - \xi)^2 \to 0$
    випливає $E|\xi_n - \xi| \to 0, n\to\infty$.
\end{proof}
\begin{proposition*}
    Зі збіжності в середньому випливає збіжність за ймовірністю.
\end{proposition*}
\begin{proof}
    З нерівності Маркова (\ref{Markov_ineq})
    $\forall \; \varepsilon > 0: P\left\{|\xi_n - \xi| \geq \varepsilon\right\} \leq \frac{E|\xi_n - \xi|}{\varepsilon}$, 
    тому з $E|\xi_n - \xi| \to 0$ маємо $P\left\{|\xi_n - \xi| \geq \varepsilon\right\} \to 0, n\to\infty$.
\end{proof}
\begin{proposition*}
     Зі збіжності за розподілом \emph{до константи} випливає збіжність за ймовірністю.
\end{proposition*}
\begin{proof}
    Нехай $\xi_n \overset{F}{\longrightarrow} c$, $c \in \mathbb{R}$. Візьмемо $\varepsilon > 0$.
    $P\left\{ |\xi_n - c| \leq \varepsilon\right\} = P\left\{ c - \varepsilon \leq \xi_n \leq c + \varepsilon \right\} \geq
    P\left\{ c - \varepsilon \leq \xi_n < c + \varepsilon \right\} = F_{\xi_n}(c+\varepsilon) - F_{\xi_n}(c-\varepsilon)$, тому
    $\underset{n \to \infty}{\lim} P\left\{ |\xi_n - c| \leq \varepsilon\right\} \geq \underset{n \to \infty}{\lim}  F_{\xi_n}(c+\varepsilon) -
    \underset{n \to \infty}{\lim}  F_{\xi_n}(c-\varepsilon) = F_c(c+\varepsilon) - F_c(c-\varepsilon) = 1$, звідки $\xi_n \overset{P}{\longrightarrow} c$.
\end{proof}
\begin{example}
    Нехай $\xi_n$ --- послідовність ДВВ: 
        \begin{tabular}{|c|c|c|c|}
            \hline
            $\xi_n$ & $-n$ & $0$ & $2n$ \\
            \hline
            $p$ & $1/{2n}$ & $1 - 1/n$ & $1/{2n}$ \\
            \hline
        \end{tabular}.

    Зі збільшенням $n$ $\xi_n$ все з більшою ймовірністю набувають значення 1. 
    Водночас, інші два можливі значення ($-n$ та $2n$) розбігаються до нескінченності. 
    Ці два процеси спричиняють протилежні ефекти --- перший наближає $\xi_n$ до 1, а другий --- віддаляє. 
    Тому з точки зору одних видів збіжності, більш чутливих до <<аномальних викидів>>, послідовність $\xi_n$ не буде мати границі, 
    а з точки зору інших, менш чутливих, границя дорівнюватиме 1.
    Дійсно: 
    \begin{gather*}
        E\left|\xi_n -1 \right| = \frac{1}{2n} |-n-1|  + \left( 1 - \frac{1}{n}\right)|1-1| + \frac{1}{2n}|2n-1| \to \frac{3}{2} \\
        E\left(\xi_n -1 \right)^2 = \frac{1}{2n} (-n-1)^2  + \left( 1 - \frac{1}{n}\right)(1-1)^2 + \frac{1}{2n}(2n-1)^2 \to \infty
    \end{gather*}
    Отже, збіжності до 1 ні в середньому, ні в середньому квадратичному немає. Зауважимо, що якби <<викиди>> прямували до нескінченності повільніше або відповідні
    ймовірності прямували до нуля швидше, то ці збіжності могли б бути: наприклад, якщо замість $n$ та $-2n$ $\xi_n$ набували значення $\sqrt{n}$ та $2\sqrt{n}$
    з ймовірностями $\frac{1}{2n^2}$, то послідовність збігалася б до 1 і в середньому, і в середньому квадратичному.
    З іншого боку, для будь-якого $\varepsilon>0$ значення $-n$ та $2n$ рано чи пізно вийдуть за межі відрізку $[1-\varepsilon;1+\varepsilon]$, і тому
    $\xi_n \overset{P}{\longrightarrow} 1$. Тепер стає зрозуміло, чому перевіряли збіжність в середньому та середньому квадратичному лише до 1: якби
    якась з цих границь існувала, то вона була б і границею за ймовірністю.
\end{example}
\begin{remark}Для збіжності $\xi_n \overset{\text{СК}}{\longrightarrow} C$ ($C$ --- стала) необхідно і достатньо, щоб
$\underset{n \to \infty}{\lim} E\xi_n = C$ та $\underset{n \to \infty}{\lim} D\xi_n = 0$.
$\underset{n \to \infty}{\lim} E\xi_n = C \Leftrightarrow \underset{n \to \infty}{\lim} (E\xi_n - EC) = 0$,
$E(\xi_n - C)^2 = D\xi_n + (E(\xi_n -C))^2$,
звідки отримуємо твердження при $n \to \infty$, оскільки $D\xi_n \geq 0$ та $(E(\xi_n -C))^2 \geq 0$.
Зауважимо, що ці дві умови є достатніми для збіжності $\xi_n \overset{P}{\longrightarrow} C$.
\end{remark}
\begin{example}
    Нехай $\xi_n \sim \mathrm{N}\left(5 + \frac{1}{n}, \sigma = \frac{1}{n}\right)$. Перевірити $\xi_n \overset{P}{\longrightarrow} 5$.
    
    \noindent Оскільки $E\xi_n = 5+\frac{1}{n} \to 5$, $D\xi_n = \frac{1}{n^2} \to 0$ при $n \to \infty$,
    то $\xi_n \overset{\text{СК}}{\longrightarrow} 5 \Rightarrow \xi_n \overset{P}{\longrightarrow} 5$.
\end{example}

Наведемо без доведення важливу теорему, що стосується збіжності за розподілом.

\begin{theorem*}[теорема неперевності Леві]
    Збіжність за розподілом послідовності випадкових величин $\left\{ \xi_n\right\}_{n=1}^{\infty}$ еквівалентна поточковій збіжності їх
    характеристичних функцій:
    $$ \xi_n \overset{F}{\longrightarrow} \xi \Longleftrightarrow \chi_{\xi_n}(t) \to \chi_\xi (t) \; \forall \; t\in \mathbb{R}$$
\end{theorem*}
        % !TEX root = ../main.tex
\section{Закон великих чисел}
\emph{Закон великих чисел (ЗВЧ)} --- загальна назва низки теорем та фактів, які встановлюють умови, 
за яких середнє арифметичне випадкових величин зі зростанням кількості доданків втрачає 
свою <<випадковість>> і може бути передбачено з наперед заданою точністю.

\subsection{Теорема Чебишова (ЗВЧ у формі Чебишова).}

\begin{theorem*}
    Нехай $\left\{ \xi_n\right\}_{n=1}^{\infty}$ --- послідовність незалежних випадкових величин, 
    таких, що існують скінченні $E\xi_k = a_k$ та $D\xi_k = \sigma_k^2$ для кожного $k \in \mathbb{N}$,
    причому дисперсії рівномірно 
    обмежені: тобто $\exists\; C < +\infty : D\xi_k \leq C \;\forall\; k \in \mathbb{N}$. 
    Тоді:
    \begin{gather}
        \forall \; \varepsilon > 0 \lim_{n \rightarrow \infty} P\left\{\left|
            \frac{1}{n}\sum\limits_{k=1}^n \xi_k - \frac{1}{n}\sum\limits_{k=1}^n E\xi_k 
        \right| \geq \varepsilon\right\} = 0 \text{ (у випадку} < \varepsilon\text{ --- рівна 1)}
    \end{gather}
    Це означає
    \begin{gather}
        \frac{1}{n}\sum\limits_{k=1}^n \xi_k - \frac{1}{n}\sum\limits_{k=1}^n E\xi_k \overset{\mathrm{P}}{\longrightarrow} 0, \; n \to \infty
    \end{gather}
\end{theorem*}
\begin{proof}
    Позначимо $\eta_n = \frac{1}{n}\sum\limits_{k=1}^n \xi_k$, тоді $E\eta_n = E\left(\frac{1}{n}\sum\limits_{k=1}^n \xi_k \right) = \frac{1}{n}\sum\limits_{k=1}^n E\xi_k$,

    \noindent$D\eta_n = D\left( \frac{1}{n}\sum\limits_{k=1}^n \xi_k\right) = \left[\xi_k \text{ --- незалежні} \right] = \frac{1}{n^2}\sum\limits_{k=1}^n D\xi_k \leq \frac{C}{n}$.
    Тепер скористаємося нерівністю Чебишова (\ref{Cheb_ineq}): 
    $\forall \; \varepsilon >0 : P\left\{ \left|\eta_n - E\eta_n \right| \geq \varepsilon\right\} \leq \frac{D\eta_n}{\varepsilon^2} \leq \frac{C}{n \varepsilon^2} \to 0, n\to\infty$
    --- що і треба було довести.
\end{proof}
\noindent\textbf{Наслідок.}
    Припустимо, що всі ВВ $\xi_n$ розподілені однаково і виконуються всі умови теореми. Тоді 
    $\frac{1}{n}\sum\limits_{k=1}^n E\xi_k = a$ і $\frac{1}{n}\sum\limits_{k=1}^n \xi_k 
    \overset{\mathrm{P}}{\longrightarrow} a, n\to\infty$. Розглянемо природну інтерпретацію цього факту.
    Нехай \emph{невипадкову} величину $a$ може бути виміряно за допомогою деякого пристрою. 
    Внаслідок наявності похибок вимірювання цей пристрій вимірює не точне значення $a$, 
    а лише деяку \emph{випадкову} величину $\xi$, в якомусь сенсі близьку до $a$. 
    Будемо, однак, вважати, що в середньому пристрій повертає правильний результат: 
    $E\xi=a$ (іноді в прикладних науках це називають відсутністю систематичних похибок). 
    Як в цій ситуації отримати якомога більш точне значення $a$? Отриманий наслідок теореми Чебишова обґрунтовує
    інтуїтивну відповідь на це питання: провести декілька вимірювань і обчислити їх середнє арифметичне.

\begin{remark}
    Для однаково розподілених $\xi_n$ припущення щодо дисперсії, виявляється, є зайвим. Для спрощення розглянемо випадок $E\xi_n = 0$.
    Нехай $\chi(t)$ --- характеристична функція розподілу всіх $\xi_n$, тоді для $\eta_n = \frac{1}{n}\sum\limits_{k=1}^n \xi_k$ 
    маємо характеристичну функцію $\chi_{\eta_n} (t) = \chi^n \left(\frac{t}{n}\right)$. 
    Тепер $\left| \chi_{\eta_n} (t) - 1\right| = \left| \chi^n \left(\frac{t}{n}\right) - 1\right| \leq n \left| \chi \left(\frac{t}{n}\right) - 1\right|$.
    Нерівність отримано з простого факту: для $z, w \in \mathbb{C}$ з $|z| \leq 1$ та $|w|\leq 1$
    $\left|z^n - w^n\right| = \left|z - w\right| \cdot \left|z^{n-1} + z^{n-2}w + ... + z w^{n-2} + w^{n-1}\right| \leq n \cdot \left|z - w\right|$.
    Оскільки $1$ --- це характеристична функція нульової випадкової величини, а \\
    \noindent$ \underset{n \to \infty}{\lim} n \left(\chi(\frac{t}{n}) - 1\right) = t \cdot \chi'(0) = \left[ E\xi_n = 0\right]=0$,
    то за теоремою Леві $\xi_n \overset{\mathrm{F}}{\longrightarrow} 0$, а тому й $\xi_n \overset{\mathrm{P}}{\longrightarrow} 0$. Залишилося зауважити, що у випадку
    $E\xi_n = a \neq 0$ можемо розглядати послідовність $\xi_n - a$, і $\xi_n - a \overset{\mathrm{P}}{\longrightarrow} 0 \Leftrightarrow \xi_n \overset{\mathrm{P}}{\longrightarrow} a$. 
\end{remark}
Наведемо ще один варіант формулювання закону великих чисел.
\begin{theorem*}[теорема Маркова]
    Нехай $\left\{ \xi_n\right\}_{n=1}^{\infty}$ --- послідовність випадкових величин (можливо, залежних), 
    таких, що існують скінченні $E\xi_k = a_k$ та $D\xi_k = \sigma_k^2$ для кожного $k \in \mathbb{N}$,
    причому $D \left( \sum\limits_{k=1}^n \xi_k\right) = o(n^2)$ при $n\to\infty$.
    Тоді
    $
        \frac{1}{n}\sum\limits_{k=1}^n \xi_k - \frac{1}{n}\sum\limits_{k=1}^n E\xi_k \overset{\mathrm{P}}{\longrightarrow} 0, \; n \to \infty
    $.
\end{theorem*}
\begin{exercise}
    Довести теорему Бернуллі.
\end{exercise}
\begin{example}
    \begin{enumerate}
        \item Скільки вимірювань величини $a$ треба провести, щоб з ймовірністю $0.998$ стверджувати, що похибка середнього арифметичного результатів
        вимірювань не перевищує $\frac{1}{100}$, якщо середньоквадратичне відхилення кожного вимірювання $\sigma = 0.03$?
        Позначимо $\eta_n = \frac{1}{n}\sum\limits_{k=1}^n \xi_k$, де $\xi_k$ --- результат $k$-того випробування.
        $D\eta_n = \frac{1}{n^2} \sum\limits_{k=1}^n D\xi_k = \frac{1}{n^2} \cdot n\sigma^2 = \frac{\sigma^2}{n}$.
        $P\{\left|\eta_n - a \right| \leq \underbrace{0.01}_{\varepsilon}\} \geq 1 - \frac{1}{0.01^2}\cdot D\eta_n =
        1 - \frac{0.03^2}{n\cdot 0.01^2} = 0.998 \Rightarrow \frac{9}{n} = 0.002$, тому $n = 4500$.
        \item Нехай $\xi_n$ --- послідовність випадкових величин, що мають розподіл $\mathrm{U}[-1; 1]$. Знайти границю за
        ймовірністю послідовності $\eta_n = \frac{1}{n}\sum\limits_{k=1}^n e^{\xi_k}$.

        За ЗВЧ $\eta_n \overset{\mathrm{P}}{\longrightarrow} E e^\xi$, де $\xi \sim \mathrm{U}[-1; 1]$. 
        $E e^\xi = \frac{1}{2}\int\limits_{-1}^1 e^x dx = \sinh{1}$, тому $\eta_n \overset{\mathrm{P}}{\longrightarrow} \sinh{1}$.
    \end{enumerate}
\end{example}
\begin{exercise}
    Знайти границю за ймовірністю послідовності $\eta_n = \sqrt[^n]{\xi_1 \cdot ... \cdot \xi_n}$, де $\xi_n$ незалежні та мають розподіл $\mathrm{U}[0;1]$.
\end{exercise}


\subsection{Посилений закон великих чисел}
\emph{Посилений закон великих чисел} --- це загальна назва ЗВЧ, що стосуються збіжності з ймовірністю 1. Наведемо без доведення приклад такого закону.
\begin{theorem*}[теорема Колмогорова]
    Нехай $\left\{ \xi_n\right\}_{n=1}^{\infty}$ --- послідовність незалежних однаково розподілених випадкових величин, що мають скінченне
    математичне сподівання $a$. Тоді без припущень щодо дисперсії
    \begin{gather}\label{Kolm_theor}
        \frac{1}{n}\sum\limits_{k=1}^n \xi_k \overset{\mathrm{P1}}{\longrightarrow} a, \; n \to \infty
    \end{gather}
\end{theorem*}

\subsection{Закон великих чисел у схемі Бернуллі}
Розглянемо частковий випадок ЗВЧ Чебишова для схеми Бернуллі.
\begin{theorem*}[теорема Бернуллі]
    Нехай $\xi_n$ задає кількість успіхів в схемі Бернуллі з $n$ випробуваннями зі сталою ймовірністю успіху $p$, $\xi_n \sim \mathrm{Bin}(n, p)$. Тоді 
    $\frac{\xi_n}{n}  \overset{\mathrm{P}}{\longrightarrow} p, n\to\infty$, тобто
    \begin{gather}
        \forall \; \varepsilon >0 : \lim_{n \rightarrow \infty} P\left\{ \left| \frac{\xi_n}{n} - p\right| \geq \varepsilon\right\} = 0
    \end{gather}
\end{theorem*}
\begin{proof}
    Зведемо до теореми Чебишова. $\xi_n = \sum\limits_{k=1}^n \eta_k$, де $\eta_k$ --- індикатор появи успіху в $k$-тому випробуванні,
    $\eta_k \sim \mathrm{Bin}(1, p)$. $\eta_k$ --- незалежні та однаково розподілені, $E\eta_k = p$ тому за наслідком ЗВЧ Чебишова для однаково
    розподілених ВВ
    $\frac{\xi_n}{n} = \frac{1}{n} \sum\limits_{k=1}^n \eta_k \overset{\mathrm{P}}{\longrightarrow} p, n\to\infty$.
\end{proof}
\begin{remark}
    Для практичного використання цієї теореми важливою є оцінка збіжності $\frac{\xi_n}{n}$ до $p$:
    $$P\left\{ \left| \frac{\xi_n}{n} - p\right| \geq \varepsilon\right\} \leq \frac{D\eta_k}{n \varepsilon^2} = \frac{p(1-p)}{n \varepsilon^2} \leq \frac{1}{4n \varepsilon^2}$$
    Остання нерівність пояснюється тим, що максимальне значення функції $f(t) = t(1-t) = -\left(t - \frac{1}{2}\right)^2 + \frac{1}{4}$ на відрізку $[0;1]$ дорівнює $\frac{1}{4}$.
\end{remark}
\begin{example}
    Оцінити ймовірність того, що при $10^4$ підкиданнях симетричної монети частість випадіння герба відхилиться від $\frac{1}{2}$ на $0.01$ і більше.

    Нехай $\xi_n$ задає кількість гербів, що випали за $n$ підкидань. Тоді
    $P\left\{ \left| \frac{\xi_{10000}}{10000} - \frac{1}{2}\right| \geq 0.01\right\} \leq \frac{1}{4\cdot 10^4 \cdot 0.01^2} = \frac{1}{4}$.
\end{example}
Наведемо узагальнення теореми Маркова.
\begin{theorem*}
    Нехай $\xi_n$ задає кількість успіхів в схемі Бернуллі з $n$ випробуваннями та ймовірністю успіху $p_n$, $\xi_n \sim \mathrm{Bin}(n, p_n)$.
    Тоді 
    $\frac{\xi_n}{n} - \frac{1}{n}\sum\limits_{k=1}^n p_k \overset{\mathrm{P}}{\longrightarrow} 0, \; n \to \infty$.
\end{theorem*}
\begin{exercise}
    Довести це узагальнення.
\end{exercise}

\subsection{Методи Монте-Карло}
\emph{Методом (або методами) Монте-Карло} називають широкий клас підходів, що дозволяють наближено розв'язувати детерміновані (<<невипадкові>>) задачі ймовірнісними методами. 
Історично одним з перших застосувань такого підходу було наближене обчислення числа $\pi$ за допомогою голки --- задача Бюффона. Проілюструємо метод Монте-Карло задачі наближеного інтегрування.

Нехай $g\:[a,b]\to[0,+\infty)$ --- деяка неперервна невід'ємна функція, для якої потрібно наближено обчислити значення $\int\limits_a^b g(x) dx$. Запропуємо два способи такого обчислення.

Спочатку розглянемо послідовність випадкових величин $\xi_n$, що мають спільний розподіл $\mathrm{U}[a; b]$.
$Eg(\xi_1) = \frac{1}{b-a} \int\limits_a^b g(x) dx$, тому $\int\limits_a^b g(x) dx = (b-a)\cdot E g(\xi_1)$. З теореми Колмогорова (\ref{Kolm_theor})
$\frac{1}{n}\sum\limits_{k=1}^n g(\xi_k) \overset{\mathrm{P1}}{\longrightarrow} Eg(\xi_1), n\to\infty$, тому при достатньо великих $n$ виконується наближена рівність:
$$\int\limits_a^b g(x) dx\approx\frac{b-a}{n}\sum\limits_{k=1}^ng(\xi_k)$$

Другий спосіб розв'язання цієї задачі заснований на інтерпретації інтеграла як площі під графіком функції. 
Оскільки $g$ є неперервною функцією на відрізку, вона також є обмеженою: існує таке $M>0$, що $g(x)\leq M$ для будь-якого $x\in[a,b]$.
Тому, кидаючи випадкові точки в прямокутник $\Pi=[a;b]\times[0;M]$, ми будемо потрапляти у підграфік $G$ функції $g$ якраз з ймовірністю
$$p=\frac{\text{площа $G$}}{\text{площа $\Pi$}}=
\frac{\int\limits_a^b g(x) dx}{M(b-a)}$$
\begin{center}
    \begin{tikzpicture}
    [thick,dot/.style = {
        draw,
        fill = black,
        circle,
        inner sep = 0pt,
        minimum size = 3 pt}]
    \draw[->] (-0.3,0) -- (7,0) coordinate[label = {below:$x$}];
    \draw[->] (0,-0.3) -- (0,5) coordinate[label = {right:$y$}];
    \draw plot[smooth] coordinates {(1,2) (2.5,4) (3.5,4) (5,1)}
    node[above left = 2.5 cm and 2.7 cm] {$g(x)$};
    \fill[gray!20] (1,0.01) -- plot[smooth] coordinates {(1,2) (2.5,4) (3.5,4) (5,1)} -- (5,0.01);
    \draw[dashed] (1,0) node[below] {$a$} -- (1,4.5) -- (5,4.5) node[below left] {$\Pi$} -- (5,0) node[below] {$b$};
    \draw[loosely dotted] (1,4.5) -- (0,4.5) node[left] {$M$};
    \node at (4,2) {$G$};
    \node[dot] at (2,2) (p) {};
    \node[below of = p, node distance = 0.3 cm]
    {\scriptsize $(\xi_1,\eta_1)$};
    \node[dot] at (4.5,3.5) (p) {};
    \node[below of = p, node distance = 0.3 cm]
    {\scriptsize $(\xi_2,\eta_2)$};
    \node[dot] at (3.5,1) (p) {};
    \node[below of = p, node distance = 0.3 cm]
    {\scriptsize $(\xi_3,\eta_3)$};
    \end{tikzpicture}
\end{center}
Для формалізації введемо дві незалежні послідовності незалежних випадкових величин $\xi_n$ та $\eta_n$, де 
$\xi_n \sim \mathrm{U}[a;b]$, $\eta_n \sim \mathrm{U}[0; M]$ для всіх $n\in\mathbb{N}$. Розглянемо схему Бернуллі, де 
<<успіхом>> в $n$-тому випробуванні будемо вважати потрапляння точки $(\xi_n, \eta_n)$ в область $G$, причому його ймовірність рівна
заданому вище $p$. Знову за теоремою Колмогорова відношення кількості успіхів до загальної кількості проведених випробувань з ймовірністю 1
прямує до $p$.
Тому при достатньо великих $n$ виконується наближена рівність
$$\int\limits_a^b g(x) dx\approx M \cdot(b-a)\cdot \nu_n$$
де $\nu_n$ --- відношення кількості успіхів до загальної кількості проведених випробувань.
Зрозуміло, що такий підхід узагальнюється на кратні інтеграли.
        % !TEX root = ../main.tex
\section{Центральна гранична теорема}
Як було показано вище, для послідовності незалежних ВВ 
$\left\{ \xi_n\right\}_{n=1}^\infty$ зі скінченними математичними сподіваннями та обмеженими в сукупності
дисперсіями $\frac{1}{n}\sum\limits_{k=1}^n (\xi_k - \E\xi_k) \overset{\mathrm{P}}{\longrightarrow} 0$.
Виявляється, що якщо знаменник цих сум буде прямувати до $0$ повільніше, то границя (хоча й в слабшому сенсі) вже не буде нульовою.
\subsection{Теорема Ляпунова}
\begin{theorem*}[теорема Ляпунова]
    Нехай $\left\{ \xi_n\right\}_{n=1}^\infty$ --- послідовність незалежних випадкових величин таких, 
    що існують скінченні $\E\xi_k = a_k$, $\D\xi_k = \sigma_k^2$ та
    $m_k = \E\left|\xi_k - a_k\right|^3$ і виконується умова Ляпунова:
    \begin{gather*}
        \lim_{n \rightarrow \infty} \frac{\sum\limits_{k=1}^n m_k}{\left(
            \sum\limits_{k=1}^n \sigma_k^2
        \right)^{3/2}} = 0
    \end{gather*}
    Тоді рівномірно відносно $x$:
    \begin{gather}
        \lim_{n \rightarrow \infty} \P \left\{
            \frac{\sum\limits_{k=1}^n (\xi_k - a_k)}
            {\sqrt{\sum\limits_{k=1}^n \sigma_k^2}}
            < x
        \right\} = F_{\mathrm{N}(0, 1)}(x) = \Phi(x) + \frac{1}{2} = 
        \frac{1}{\sqrt{2\pi}} \int\limits_0^x e^{-\frac{t^2}{2}} dt + \frac{1}{2}
    \end{gather}
    або:
    \begin{gather*}
        \frac{\sum\limits_{k=1}^n (\xi_k - a_k)}
        {\sqrt{\sum\limits_{k=1}^n \sigma_k^2}}
        \overset{\mathrm{F}}{\longrightarrow} \xi \sim \mathrm{N}(0, 1),  \; n\to \infty
    \end{gather*}
\end{theorem*}
\begin{proof}
    Позначимо $\eta_n = \frac{\sum\limits_{k=1}^n (\xi_k - a_k)}
    {\sqrt{\sum\limits_{k=1}^n \sigma_k^2}}$, $\xi_k - a_k = \mathring{\xi_k}$ ($\E\mathring{\xi_k} = 0)$, 
    $B_n = \sqrt{\sum\limits_{k=1}^n \sigma_k^2}$. 
    Тоді $\eta_n = \frac{1}{B_n}\sum\limits_{k=1}^n \mathring{\xi_k}$.
    $\E\eta_n = 0$, $\D\eta_n = \frac{1}{B_n^2}\sum\limits_{k=1}^n \D\mathring{\xi_k} = 
    \left[\D\xi_k = \D\mathring{\xi_k}\right] = 1$.

    Доведення ґрунтується на теоремі Леві (\ref{levi_theor}): збіжність за розподілом послідовності 
    випадкових величин $\left\{ \xi_n\right\}_{n=1}^{\infty}$ еквівалентна поточковій збіжності їх
    характеристичних функцій.

    Нехай $\chi_k(t)$ --- характеристична функція $\mathring{\xi_k}$, а $F_k(x)$ --- 
    функція розподілу $\mathring{\xi_k}$.
    За властивостями характеристичних функцій $\chi_{\eta_n}(t) = 
    \prod\limits_{k=1}^n \chi_k(\frac{t}{B_n})$. 
    Доведемо, що $\chi_{\eta_n}(t)\to e^{-t^2/2} = 
    \chi_{\mathrm{N}(0, 1)}(t)$ при $ n \to \infty$.
    \begin{gather*}\chi_k\left(\frac{t}{B_n}\right) = \int\limits_{-\infty}^{+\infty} e^{i\frac{t}{B_n}x} dF_k(x) = 
    \int\limits_{-\infty}^{+\infty} \left(1 + i\frac{t}{B_n}x - \frac{t^2}{2B_n^2}x^2 - 
    i\frac{t^3}{6B_n^3}x^3 + ...\right) dF_k(x) = \\ = \left[\theta_k \in (0; 1)\right]
    = 1 + \frac{it}{B_n}\E\mathring{\xi_k} - \frac{t^2}{2B_n^2}
    \E\mathring{\xi_k}^2 + \theta_k\frac{|t|^3}{6B_n^3}\E|\mathring{\xi_k}|^3 = 
    1 - \frac{t^2}{2B_n^2}\sigma_k^2 + \theta_k\frac{|t|^3}{6B_n^3}m_k \\
    \ln\chi_{\eta_n}(t) = \sum\limits_{k=1}^n \ln \chi_k \left(\frac{t}{B_n}\right) = 
    \sum\limits_{k=1}^n \ln\left(1 - \frac{t^2}{2B_n^2}\sigma_k^2 + 
    \theta_k\frac{|t|^3}{6B_n^3}m_k\right)
    \end{gather*}

    З умови Ляпунова $\underset{n \rightarrow \infty}{\lim} \frac{\sum\limits_{k=1}^n m_k}{B_n^3} = 0$, 
    тому з відомої властивості $\ln(1 + \alpha_n) \sim \alpha_n$, де $\alpha_n$ --- нескінченно мала при 
    $n \rightarrow \infty$, маємо $\ln\chi_{\eta_n}(t) \sim \sum\limits_{k=1}^n\left(
        -\frac{t^2\sigma^2_k}{2B_n^2} + \theta_k\frac{|t|^3}{6B_n^3}m_k
    \right) 
    = -\frac{t^2}{2}\frac{\sum\limits_{k=1}^n\sigma_k^2}{B_n^2} + \sum\limits_{k=1}^n \theta_k\frac{|t|^3}{6B_n^3}m_k= -\frac{t^2}{2} 
    + \sum\limits_{k=1}^n \theta_k\frac{|t|^3}{6B_n^3}m_k \to -\frac{t^2}{2} 
    $ при $n \rightarrow \infty$.
    
    Таким чином, $\chi_{\eta_n}(t) \to e^{-\frac{t^2}{2}}$ при $n \rightarrow \infty$ і 
    з теореми Леві маємо, що $\eta_n \overset{\mathrm{F}}{\longrightarrow} \xi \sim \mathrm{N}(0, 1)$.
\end{proof}
\begin{remark}
    Можна довести, що для послідовності неперервних випадкових величин також має місце збіжність щільностей розподілу $\eta_n$ до $\frac{1}{\sqrt{2\pi}} e^{-\frac{x^2}{2}}$
    --- щільності розподілу $\mathrm{N}(0, 1)$.
\end{remark}
\noindent\textbf{Наслідок.} Нехай всі $\xi_n$ \emph{однаково розподілені}, $\E\xi_k = a$, $\D\xi_k = \sigma^2$,
$m_k = \E\left| \xi_k - a\right|^3 = m$. Тоді умова Ляпунова виконується автоматично:
$$\underset{n \rightarrow \infty}{\lim} \frac{\sum\limits_{k=1}^n m_k}{\left(
    \sum\limits_{k=1}^n \sigma_k^2
\right)^{3/2}} = \underset{n \rightarrow \infty}{\lim} \frac{n \cdot m}{(n\cdot \sigma^2)^{3/2}} = 0$$
В цьому випадку
$$\frac{\sum\limits_{k=1}^n (\xi_k - a_k)}
{\sqrt{\sum\limits_{k=1}^n \sigma_k^2}}
= \frac{1}{\sqrt{n}} \sum\limits_{k=1}^n \left( \frac{\xi_k - a}{\sigma}\right) \overset{\mathrm{F}}{\longrightarrow}\xi \sim \mathrm{N}(0, 1),
\;n \to \infty$$
Це твердження означає, що яким би не був розподіл $\xi_n$, $\zeta_n  = \xi_1 + ... + \xi_n$ матиме <<приблизно нормальний>> розподіл $\mathrm{N}(n a, n\sigma^2)$.
Проілюструємо це на прикладі рівномірного розподілу. На рисунку нижче зображено графіки щільностей розподілів
$\mathrm{U}[-1; 1]$ та $\mathrm{N}\left(0, \frac{1}{3}\right)$. Зауважимо, що $\frac{1}{3}$ --- це дисперсія $\mathrm{U}[-1; 1]$.
\begin{center}
    \begin{tikzpicture}[yscale = 5, xscale = 0.6]
        \pgfmathsetmacro{\a}{0};
        \pgfmathsetmacro{\s}{0.57735}
        \draw [->] (-5, 0) -- (5, 0);
        \draw [->] (0, -0.2) -- (0, 0.75);
        \draw [domain=-4.9:4.9, smooth, variable = \x, thick, gray] plot ({\x}, {0.398942280401/\s * e^(-(\x-\a)^2/(2*\s^2))});
        \node [below] at (5, 0) {$x$};
        \node [below right] at (0, 0) {$_0$};
        \node [below right] at (2, 0) {$_2$};
        \node [below left] at (-2, 0) {$_{-2}$};
        \node [below right] at (4, 0) {$_4$};
        \node [below left] at (-4, 0) {$_{-4}$};
        
        \draw [thick] (-1, 0.5) -- (1, 0.5);
        \draw [dashed] (-1, 0) -- (-1, 0.5);
        \draw [dashed] (1, 0) -- (1, 0.5);
        \draw [thick] (-5, 0) -- (-1, 0);
        \draw [thick] (1, 0) -- (4.9, 0);
    \end{tikzpicture}
\end{center}
Розглянемо тепер щільності розподілу сум двох (зліва) та трьох (справа) незалежних ВВ, кожна з яких має розподіл $\mathrm{U}[-1; 1]$, і щільності
нормальних розподілів $\mathrm{N}\left(0, \frac{2}{3}\right)$ та $\mathrm{N}\left(0, 1\right)$ відповідно.

\begin{center}
    \begin{tabular}{c c}
        \begin{tikzpicture}[yscale = 5, xscale = 0.6]
            \pgfmathsetmacro{\a}{0};
            \pgfmathsetmacro{\s}{0.8165}
            \draw [->] (-5, 0) -- (5, 0);
            \draw [->] (0, -0.2) -- (0, 0.55);
            \draw [domain=-4.9:4.9, smooth, variable = \x, thick, gray] plot ({\x}, {0.398942280401/\s * e^(-(\x-\a)^2/(2*\s^2))});
            \node [below] at (5, 0) {$x$};
            \node [below right] at (0, 0) {$_0$};
            \node [below right] at (2, 0) {$_2$};
            \node [below left] at (-2, 0) {$_{-2}$};
            \node [below right] at (4, 0) {$_4$};
            \node [below left] at (-4, 0) {$_{-4}$};
            
            \draw [domain=-2:0, smooth, variable = \x, thick] plot ({\x}, {0.5 + \x/4});
            \draw [domain=0:2, smooth, variable = \x, thick] plot ({\x}, {0.5 - \x/4});
            \draw [thick] (-5, 0) -- (-2, 0);
            \draw [thick] (2, 0) -- (4.9, 0);
        \end{tikzpicture} &
        \begin{tikzpicture}[yscale = 6, xscale = 0.6]
            \pgfmathsetmacro{\a}{0};
            \pgfmathsetmacro{\s}{1}
            \draw [->] (-5, 0) -- (5, 0);
            \draw [->] (0, -0.17) -- (0, 0.45);
            \draw [domain=-4.9:4.9, smooth, variable = \x, thick, gray] plot ({\x}, {0.398942280401/\s * e^(-(\x-\a)^2/(2*\s^2))});
            \node [below] at (5, 0) {$x$};
            \node [below right] at (0, 0) {$_0$};
            \node [below right] at (2, 0) {$_2$};
            \node [below left] at (-2, 0) {$_{-2}$};
            \node [below right] at (4, 0) {$_4$};
            \node [below left] at (-4, 0) {$_{-4}$};
            
            \draw [domain=-3:-1, smooth, variable = \x, thick] plot ({\x}, {(3+\x)^2/16});
            \draw [domain=-1:0, smooth, variable = \x, thick] plot ({\x}, {3/8 - (\x)^2/8});
            \draw [domain=0:1, smooth, variable = \x, thick] plot ({\x}, {3/8 - \x^2/8});
            \draw [domain=1:3, smooth, variable = \x, thick] plot ({\x}, {(-3+\x)^2/16});
            \draw [thick] (-5, 0) -- (-3, 0);
            \draw [thick] (3, 0) -- (4.9, 0);
        \end{tikzpicture}
    \end{tabular}
\end{center}

Видно, що графік щільності суми трьох ВВ дуже схожий на графік щільності нормального розподілу $\mathrm{N}\left(0, 1\right)$.
\begin{exercise}
    Записати в явному вигляді щільності розподілу $\zeta_2 = \xi_1 + \xi_2$ та $\zeta_3 = \xi_1 + \xi_2 + \xi_3$, де
    $\xi_k \sim \mathrm{U}[-1; 1], k = 1,2,3$.
\end{exercise}
Варто пояснити практичне значення умови Ляпунова $\underset{n \rightarrow \infty}{\lim} \frac{\sum\limits_{k=1}^n \E\left| \xi_k - \E\xi_k\right|^3}{\left(
    \sum\limits_{k=1}^n \D\xi_k
\right)^{3/2}} = 0$ в доведенні теореми. Вона означає, що дисперсії величин у послідовності мають бути приблизно однакового порядку. 
Можна навести приклад з життя: якщо в багатоквартирному будинку не буде квартири, що використовує значно більше електроенергії, ніж інші,
то сумарний розподіл кількості використаної електроенергії буде приблизно нормальним.

\subsection{Умова Ліндеберга}
    Замість умови Ляпунова в доведенні однойменної теореми можна вимагати виконання умови Ліндеберга:
    $$\forall \; \varepsilon > 0 : \underset{n\to\infty}{\lim} 
    \frac{1}{B_n^2}\sum\limits_{k=1}^n \E \left( \left( \xi_k - a_k\right)^2 \cdot 1_{\left\{|\xi_k - a_k| > \varepsilon \cdot B_n\right\}}\right) = 0$$
    Тут $a_k = \E\xi_k$, $B_n = \sqrt{\sum\limits_{k=1}^n \D\xi_k}$, $1_A$ --- індикатор події $A$.
    Ймовірнісний зміст цієї умови такий: нехай $A_k = \left\{ \left|\xi_k - a_k \right| > \varepsilon \cdot B_n \right\}$, тоді
    \begin{gather*}
        \P \left( \bigcup\limits_{k=1}^n A_k\right) \leq \sum\limits_{k=1}^n \P(A_k) = \sum\limits_{k=1}^n \int_{|x-a_k| > \varepsilon \cdot B_n} dF_k(x) \leq
    \sum\limits_{k=1}^n \int_{|x-a_k| > \varepsilon \cdot B_n} \frac{(x-a_k)^2}{\varepsilon^2 B_n^2} dF_k(x) = \\
    = \frac{1}{\varepsilon^2} \cdot \frac{1}{B_n^2}\sum\limits_{k=1}^n \E \left( \left( \xi_k - a_k\right)^2 \cdot 1_{\left\{|\xi_k - a_k| > \varepsilon \cdot B_n\right\}}\right) 
    \to 0, \; n\to\infty
    \end{gather*}
    Це означає, що кожний доданок $\frac{\xi_k - a_k}{B_n}$ має рівномірно малий внесок в суму $\frac{1}{B_n}\sum\limits_{k=1}^n (\xi_k - a_k)$.
Повне доведення ЦГТ з умовою Ліндеберга виходить за рамки курсу.
        % !TEX root = ../main.tex
\section{Застосування ЦГТ до схеми Бернуллі}
\subsection{Інтегральна теорема Муавра-Лапласа}
\begin{theorem*}
    Нехай $\xi_n$ задає кількість успіхів в схемі Бернуллі з ймовірністю успіху $p$, тобто $\xi_n \sim \mathrm{Bin}(n, p)$,
    $q = 1-p$.
    Тоді рівномірно відносно $x$:
    \begin{gather}
        \lim_{n \rightarrow \infty} P \left\{
            \frac{\xi_n - np}{\sqrt{npq}}
            < x
        \right\} = F_{\mathrm{N}(0, 1)}(x) = \Phi(x) + \frac{1}{2} = 
        \frac{1}{\sqrt{2\pi}} \int\limits_0^x e^{-\frac{t^2}{2}} dt + \frac{1}{2}
    \end{gather}
\end{theorem*}
\begin{proof}
    $\xi_n = \sum\limits_{k=1}^n \eta_k$, де $\eta_k$ --- індикатор появи успіху в $k$-тому випробуванні,
    $\eta_k \sim \mathrm{Bin}(1, p)$. $\eta_k$ --- незалежні та однаково розподілені, $E\eta_k = p$, $D\eta_k = pq$,
    тому до них можна застосувати ЦГТ:
    \begin{gather*}
        \sum\limits_{k=1}^n \left( \eta_k - E\eta_k\right) = \xi_n - np, \; \sum\limits_{k=1}^n D\eta_k = n p q \\
        \frac{\sum\limits_{k=1}^n \left( \eta_k - E\eta_k\right)}{\sqrt{\sum\limits_{k=1}^n D\eta_k}} =
         \frac{\xi_n - np}{\sqrt{n p q}} \overset{\mathrm{F}}{\longrightarrow} \xi \sim \mathrm{N}(0, 1), \; n\to \infty
    \end{gather*}
\end{proof}
\noindent\textbf{Робочі формули.} При великих $n$:
\begin{enumerate}
    \item $P\left\{ \frac{\xi_n - np}{\sqrt{n p q}} < x\right\} \approx \Phi(x) + \frac{1}{2}$.
    \item $\forall \; a, b \in \mathbb{R}, \; a<b : P\left\{a < \frac{\xi_n - np}{\sqrt{n p q}} < b\right\} \approx \Phi(b) - \Phi(a)$.
    \item $\forall \; m_1, m_2 \in \mathbb{R}, m_1<m_2 : P\left\{ m_1 < \xi_n < m_2\right\} = 
    P\left\{\frac{m_1 - np}{\sqrt{n p q}} < \frac{\xi_n - np}{\sqrt{n p q}} < \frac{m_2 - np}{\sqrt{n p q}}\right\} \approx$

    $\approx \Phi\left(\frac{m_2 - np}{\sqrt{n p q}}\right) - \Phi\left(\frac{m_1 - np}{\sqrt{n p q}}\right)$.
    \item $\forall \; \varepsilon > 0 : P\left\{ \left|\frac{\xi_n}{n} - np\right| < \varepsilon\right\} = 
    P\left\{-\varepsilon < \frac{\xi_n - np}{n} < \varepsilon\right\} =
    P\left\{-\varepsilon \sqrt{\frac{n}{pq}} < \frac{\xi_n - np}{\sqrt{n p q}} < \varepsilon \sqrt{\frac{n}{pq}}\right\} \approx$
    $\approx \Phi\left( \varepsilon \sqrt{\frac{n}{pq}}\right) - \Phi\left( -\varepsilon \sqrt{\frac{n}{pq}}\right) = 2\Phi\left( \varepsilon \sqrt{\frac{n}{pq}}\right)$.
\end{enumerate}
\begin{remark}
    В усіх наведених наближених рівностях будь-яку строгу нерівність можна замінити на нестрогу. На практиці формулі 3 іноді застосовують \emph{<<поправку на неперервність>>},
    використовуючи $m_2 + \frac{1}{2}$ та $m_1 - \frac{1}{2}$ замість $m_2$ та $m_1$ відповідно. Емпірично з'ясовано, що це дозволяє підвищити точність наближення.
    Також важливо зауважити, що на практиці ці формули застосовують при $n \geq 50$ та $npq \geq 10$ (як і для наступних формул, 
    різні джерела можуть давати інші межі, наприклад $npq \geq 20$).
\end{remark}
\begin{example}
    Стрілець робить 100 пострілів по мішені з ймовірністю влучення $0.7$, $\xi$ --- кількість влучень. Знайти наближено 
    $P\left\{ 67 \leq \xi \leq 72\right\}$ та порівняти з точним значенням.
    
    \noindent Маємо $np = 70$, $npq = 100 \cdot 0.7 \cdot 0.3 = 21 \geq 10$, $P\left\{ 67 \leq \xi \leq 72\right\} \approx 
    \Phi\left( \frac{2}{\sqrt{21}}\right) - \Phi\left(-\frac{3}{\sqrt{21}}\right) \approx 0.4124$.
    З поправкою на неперервність отримаємо 
    $P\left\{ 67 \leq \xi \leq 72\right\} \approx 
    \Phi\left( \frac{2.5}{\sqrt{21}}\right) - \Phi\left(-\frac{3.5}{\sqrt{21}}\right) \approx 0.4848$.
    Підрахунок за точною формулою $\sum\limits_{k=67}^{72} C_{100}^{k} 0.7^k 0.3^{100-k}$ дає значення $0.4829$.
\end{example}

\subsection{Локальна теорема Муавра-Лапласа}
\begin{theorem*}
    Нехай проводиться $n$ незалежних випробувань з ймовірністю успіху $p$, $q = 1-p$, причому $n$ досить велике.
    Тоді ймовірність отримання $m$ успіхів 
    \begin{gather}
        P_n(m) \approx \frac{1}{\sqrt{2\pi n p q}} \exp \left\{ - \frac{(m - np)^2}{2npq}\right\}
    \end{gather}
\end{theorem*}
\begin{proof}
    Нехай $\xi$ задає кількість успіхів у заданій схемі Бернуллі, тоді за інтегральною теоремою Муавра-Лапласа
    $P_n(m) = P\left\{ m-\frac{1}{2} < \xi < m+\frac{1}{2}\right\} \approx 
    \Phi\left(\frac{m+\frac{1}{2}- np}{\sqrt{n p q}}\right) - \Phi\left(\frac{m-\frac{1}{2} - np}{\sqrt{n p q}}\right)$.
    За означенням похідної $\Phi(x + \varepsilon) - \Phi(x - \varepsilon) \approx 2 \varepsilon \Phi'(x)$ при малих $\varepsilon$,
    причому $\Phi'(x) = \frac{1}{\sqrt{2\pi}} \exp \left\{ - \frac{x^2}{2}\right\}$.
    Звідси 
    $\Phi\left(\frac{m+\frac{1}{2}- np}{\sqrt{n p q}}\right) - \Phi\left(\frac{m-\frac{1}{2} - np}{\sqrt{n p q}}\right) \approx
    \frac{1}{\sqrt{2\pi n p q}} \exp \left\{ - \frac{(m - np)^2}{2npq}\right\}$.
\end{proof}
\begin{remark}
    Так само, як інтегральна теорема, локальна теорема Муавра-Лапласа на практиці застосовується при $n \geq 50$ та $npq \geq 10$ (або $npq \geq 20$).
\end{remark}
\begin{example}
    Стрілець робить $100$ пострілів по мішені з ймовірністю влучення $0.7$, $\xi$ --- кількість влучень. Знайти наближено
    $P\left\{ \xi = 69\right\}$ та порівняти з точним значенням.

    \noindent Маємо $npq = 21$, $P\left\{ \xi = 69\right\} \approx \frac{1}{\sqrt{42\pi}} \exp\left\{ -\frac{1}{42}\right\} \approx 0.085$.
    Підрахунок за точною формулою $C_{100}^{69} 0.7^{69} 0.3^{31}$ дає значення $0.084$.
\end{example}

\subsection{Гранична теорема Пуассона}
Розглянемо граничу теорему для схеми Бернуллі, яка застосовується для наближених обчислень, коли умова $npq \geq 10$ 
для застосування локальної теореми Муавра-Лапласа не виконується.
\begin{theorem*}
    Задано нескінченну серію схем Бернуллі: перша складається з одного випробування $B_1$ з імовірністю успіху $p_1$, 
    друга --- з двох випробувань $B_2^1$ та $B_2^2$ з імовірністю успіху $p_2$, 
    третя --- з трьох випробувань $B_3^1$, $B_3^2$ та $B_3^3$ з імовірністю успіху $p_3$ і так далі.
    Випадкова величина $\xi_n \sim \mathrm{Bin}(n, p_n)$ --- кількість успіхів в $n$-тій схемі Бернуллі.
    Існує таке $a > 0$, що $\underset{n\to\infty}{\lim} n p_n = a$.
    Тоді для будь-якого цілого $m \geq 0$:
    \begin{gather}
        P\left\{\xi_n = m\right\} \to \frac{a^m}{m!} e^{-a}, \; n\to \infty
    \end{gather}
\end{theorem*}
\begin{proof}
    Нехай $\xi \sim \mathrm{Poiss}(a)$. Покажемо $\xi_n \overset{\mathrm{F}}{\longrightarrow} \xi, n\to\infty$, скориставшись твердженням теореми Леві (\ref{levi_theor}).
    Характеристичні функції $\xi_n$ та $\xi$ рівні, відповідно, $\chi_{\xi_n}(t) = \left(p_n e^{it} + 1 - p_n \right)^n$ та $\chi_{\xi}(t) = e^{a\left( e^{it} - 1\right)}$.
    З умови $\underset{n\to\infty}{\lim} n p_n = a$ та нерівності $\left| z^n - w^n\right| \leq n\cdot\left| z - w\right|$ для $z, w \in \mathbb{C}$, $|z|\leq1$, $|w|\leq1$:
    $$
    \left| \left(p_n e^{it} + 1 - p_n \right)^n - \left(\frac{a}{n} e^{it} + 1 - \frac{a}{n} \right)^n\right| \leq
    n \cdot \left| p_n - \frac{a}{n}\right| \left| e^{it} - 1\right| \to 0, \; n\to\infty
    $$
    Тому маємо
    $$
    \underset{n\to\infty}{\lim} \chi_{\xi_n}(t) = \underset{n\to\infty}{\lim} \left(\frac{a}{n} e^{it} + 1 - \frac{a}{n} \right)^n = 
    \underset{n\to\infty}{\lim} \left(1 + \frac{a}{n} \left(e^{it} - 1\right) \right)^n = e^{a\left( e^{it} - 1\right)} = \chi_{\xi}(t)
    $$
    Отже, $\xi_n \overset{\mathrm{F}}{\longrightarrow} \xi, n\to\infty$. Функція розподілу $\xi$ неперервна в будь-якій нецілій точці, тому для цілих $m \geq 0$:
    \begin{gather*}
        P\left\{ m - \frac{1}{2} \leq \xi_n < m + \frac{1}{2}\right\} = F_{\xi_n}\left( m + \frac{1}{2}\right) -
        F_{\xi_n}\left( m - \frac{1}{2}\right) \to \\ \to F_{\xi}\left( m + \frac{1}{2}\right) -
        F_{\xi}\left( m - \frac{1}{2}\right) = P\left\{ m - \frac{1}{2} \leq \xi < m + \frac{1}{2}\right\}
        P\left\{ \xi = m\right\} = \frac{a^m}{m!} e^{-a}, \; n\to \infty
    \end{gather*}
    що і треба було довести.
\end{proof}
\begin{exercise}
    Довести цю теорему без використання характеристичних функцій, показавши
    $$C_n^m p_n^m (1-p_n)^{n-m} \to \frac{a^m}{m!}e^{-a}, n\to\infty$$
\end{exercise}
На практиці цією теоремою користуються, якщо $n \geq 50$ та $np \leq 10$ 
(або $n$ --- достатньо велике, $p$ --- достатньо мале, причому $1 < np < 20$):
ймовірність отримати $m$ успіхів у схемі Бернуллі з $n$ випробуваннями та ймовірністю успіху $p$
$P_n(m) \approx \frac{(np)^m}{m!} e^{-np}, \; m = 0, ..., n$.
\begin{example}
    Стрілець робить $100$ пострілів по мішені. 
    Ймовірність влучення при одному пострілі становить $0.98$. 
    Знайти наближено ймовірність того, що вдалих пострілів буде не більше $97$, та порівняти з точним значенням.
    
    \noindentЩоб звести задачу до використання теореми Пуассона, шукатимемо ймовірність, що промахів буде щонайменше $3$ (ймовірність
    промаху --- $0.02$). $\xi$ --- кількість промахів,
    $P\left\{ \xi \geq 3\right\} = 1 - P\left\{ \xi \leq 2\right\} = 
    1 - p_{100}(0) - p_{100}(1) - p_{100}(2)
    \approx 1 - \left( \frac{2^0}{0!} + \frac{2^1}{1!} + \frac{2^2}{2!}\right)e^{-2} = 1 - 5e^{-2} \approx 0.32332$.
    Підрахунок за точною формулою $1 - \sum\limits_{k=0}^2 C_{100}^k 0.02^k 0.98^{100-k}$ дасть значення $0.32331$. Нескладно
    перевірити, що застосування локальної теореми Муавра-Лапласа дало б менш точну відповідь $0.36049$.
\end{example}
        % !TEX root = ../main.tex
\section{Збіжність та граничні теореми для послідовностей випадкових векторів}
В цьому розділі наведемо без доведення деякі означення та факти, що стосуються збіжності
послідовностей, закону великих чисел та центральної граничної теореми для випадкових векторів.
Розглядаємо послідовності випадкових векторів $\left\{ \vec{\xi}_n (\omega)\right\}_{n=1}^{\infty}$ на фіксованому
ймовірносному просторі.
\subsection{Збіжність випадкових векторів}
На послідовності випадкових векторів цілком природно переносяться означення збіжності за ймовірністю, за розподілом та з ймовірністю 1.
\begin{enumerate}
    \item Збіжність з ймовірністю 1: $\left(\vec{\xi}_n \overset{\mathrm{P1}}{\longrightarrow} \vec{\xi}, n \to \infty\right) \Leftrightarrow 
    \left(\P\left\{ \omega: \underset{n\to\infty}{\lim} \vec{\xi}_n(\omega) = \vec{\xi}(\omega)\right\} = 1\right)$.
    \item Збіжність за ймовірністю: $\left(\vec{\xi}_n \overset{\mathrm{P}}{\longrightarrow} \vec{\xi}, n \to\infty\right) \Leftrightarrow
    \left(\forall \; \varepsilon > 0: \underset{n \to \infty}{\lim} \P\left\{\left\Vert\vec{\xi}_n - \vec{\xi}\right\Vert \geq \varepsilon\right\}= 0\right)$.
    \item Збіжність за розподілом: 
    
    $\left(\vec{\xi}_n \overset{\mathrm{F}}{\longrightarrow} \vec{\xi}, n \to\infty\right) \Leftrightarrow
    \left(\underset{n\to\infty}{\lim}F_{\vec{\xi}_n}(\vec{x}) = F_{\vec{\xi}}(\vec{x}) \text{ для точок неперервності } F_{\vec{\xi}}\right)$.
\end{enumerate}
Зв'язки між цими видами збіжності такі ж, як у випадку випадкових величин.

\subsection{Граничні теореми}
Для простоти розглядаємо лише випадок \emph{незалежних однаково розподілених} $\vec{\xi}_n$.
\begin{theorem*}[посилений закон великих чисел]
    Нехай $\left\{ \vec{\xi}_n\right\}_{n=1}^{\infty}$ --- послідовність незалежних однаково розподілених випадкових векторів, що мають скінченне
    математичне сподівання $\vec{a}$ (тобто $\Vert \vec{a} \Vert < \infty$). Тоді 
    $$
    \frac{1}{n} \sum\limits_{k=1}^n \vec{\xi}_k \overset{\mathrm{P1}}{\longrightarrow} \vec{a}, \; n \to \infty
    $$
\end{theorem*}
\begin{theorem*}[центральна гранична теорема]
    Нехай $\left\{ \vec{\xi}_n\right\}_{n=1}^{\infty}$ --- послідовність незалежних однаково розподілених випадкових векторів, що мають скінченні
    математичне сподівання $\vec{a}$ та кореляційну матрицю $K$. Тоді
    $$
    \sqrt{n}\left( \frac{1}{n} \sum\limits_{k=1}^n \vec{\xi}_k - \vec{a}\right) = 
    \frac{1}{\sqrt{n}} \sum\limits_{k=1}^n \left( \vec{\xi_k} - \vec{a}\right) \overset{\mathrm{F}}{\longrightarrow} \vec{\eta} \sim \mathrm{N}\left(\vec{0}, K\right), \; n\to\infty
    $$
\end{theorem*}
    \part{Математична статистика}
    \chapter{Вибірка, точкове та інтервальне оцінювання}
        % !TEX root = ../../main.tex
\section{Вибірка та її характеристики}
\subsection{Поняття вибірки}
Математична статистика вивчає випадкові величини за певними дослідними даними, які отримано в ході експерименту.
Коротко кажучи, методами математичної статистики можна оцінювати (але не визначати точно) числові характеристики 
та розподіл деякої випадкової величини,
якщо відомо деякий набір значень, яких ця величина набула в ході досліду. При цьому, зазвичай, апріорних відомостей про
цю випадкову величину майже немає.
\begin{definition}
    \emph{Генеральною сукупністю (ГС)} називають як випадкову величину $\xi$, що досліджується, так і множину всіх її можливих значень.
\end{definition}
\begin{definition}
    \emph{Випадковою вибіркою обсягу $n$} називається випадковий вектор $\left( \xi_1, ..., \xi_n\right)$, координати якого є однаково
    розподіленими, як ГС, і незалежними в сукупності. Підмножина $G\subseteq \mathbb{R}^n$, що складається з усіх можливих значень визначеної вище
    випадкової вибірки, називається $\emph{вибірковим простором}$, а вектори $\vec{x} \in G$ --- \emph{реалізаціями вибірки}.
    \emph{Конкретною реалізацією вибірки} інколи називають саме ту реалізацію, з якою працюють після проведення експерименту.
\end{definition}
\begin{example}
    Нехай $\xi \sim \mathrm{Bin}(N, p)$, $N = 3$. В цьому випадку $G = \left\{0, 1, 2, 3\right\}^{\times n}$ --- множина
    $n$-вимірних векторів, всі координати яких набувають лише значень $0$, $1$, $2$ та $3$.
\end{example}
\begin{remark}
    В прикладній статистиці зазвичай не користуються таким різноманітним набором означень. Там під вибіркою розуміють як і будь-які 
    отримані внаслідок експерименту спостереження, так і процес їх отримання. Варто навести два означення, які, хоч і не будуть застосовуватися далі
    в курсі, але зустрічаються в прикладній статистиці. \emph{Репрезентативна вибірка} --- така, яка має всі властивості генеральної сукупності. Інакше кажучи, всі можливі значення 
    (або проміжки значень) ГС мають однакову ймовірність потрапити до конкретної реалізації вибірки.
    \emph{Стратифікована вибірка} --- така, що гарантує збереження пропорцій, наявних у ГС.
    Наприклад, якщо мова йде про результати якогось опитування, то репрезентативна вибірка має містити результати всіх категорій населення, а 
    стратифікована ще й має містити їх в тих пропорціях, які ці категорії складають в усьому населенні країни.
    На практиці поняття репрезентативної та стратифікованої вибірки залежать від ГС, структуру та природу якої дослідник попередньо вивчає,
    та від мети самого дослідження: наприклад, населення всієї країни може грати роль ГС у багатьох статистичних дослідженнях, 
    але його поділ на категорії може відрізнятися.
\end{remark}
\subsection{Розподіл випадкової вибірки}
Нехай $\vec{\xi} = \left( \xi_1, ..., \xi_n\right)$ --- випадкова вибірка, а $F_{\xi}(x)$ --- функція розподілу ГС. Тоді для 
$\vec{x} \in \mathbb{R}^n$ визначено функцію розподілу 
$F_{\vec{\xi}}(\vec{x}) = P\left\{\xi_1 < x_1, ..., \xi_n < x_n \right\} = \prod\limits_{k=1}^n F_{\xi}(x_k)$. Для досліджень вибірок, однак,
зручно користуватися іншою функцією.
\begin{definition}
    \emph{Функцією правдоподібності} випадкової вибірки обсягу $n$ з ГС $\xi$ називається 
    \begin{gather}
        \mathcal{L}(\vec{x}) = \prod\limits_{k=1}^n P\left\{ \xi = x_k\right\}, \text{ якщо } \xi \text{ --- ДВВ} \\
        \mathcal{L}(\vec{x}) = \prod\limits_{k=1}^n f_{\xi}(x_k) \text{ якщо } \xi \text{ --- НВВ}
    \end{gather}
\end{definition}
Часто аргументами функції правдоподібності вважають параметри закону розподілу ГС. В такому випадку при фіксованому значенні
$\vec{x}$ ця функція фактично показує, як в залежності від параметрів розподілу змінюється ймовірність отримати саме таку реалізацію
вибірки --- $\left(x_1, ..., x_n \right)$.
Отримаємо функції правдоподібності для основних законів розподілу. В подальшому будуть більш корисними не самі функції правдоподібності,
а їх логарифми.
\begin{enumerate}
    \item $\xi \sim \mathrm{Bin}(N,p)$ --- біноміальний розподіл, $\forall \; k \in \mathbb{N} \; x_k \in \left\{0, 1, ..., N \right\}$:
    \begin{gather*}
        \mathcal{L}_{\mathrm{Bin}}(\vec{x}, N, p) = \prod\limits_{k=1}^n C_N^{x_k} p^{x_k} (1-p)^{N - x_k} = 
        \prod\limits_{k=1}^n C_N^{x_k} \cdot p^{\sum_{k=1}^n x_k} \cdot (1-p)^{N\cdot n - \sum_{k=1}^n x_k} \\
        \ln \mathcal{L}_{\mathrm{Bin}}(\vec{x}, N, p) = \sum\limits_{k=1}^n \ln C_N^{x_k} + \ln p \cdot \sum\limits_{k=1}^n x_k + 
        \ln{(1-p)} \cdot\left( N\cdot n - \sum\limits_{k=1}^n x_k\right)
    \end{gather*}
    \item $\xi \sim \mathrm{Geom}(p)$ --- геометричний розподіл, $\forall \; k \in \mathbb{N} \; x_k \in \left\{1, 2, 3, ...\right\}$. 
    \begin{gather*}
        \mathcal{L}_{\mathrm{Geom}}(\vec{x}, p) = \prod\limits_{k=1}^n p (1-p)^{x_k - 1} = p^n \cdot (1-p)^{\sum_{k=1}^n x_k - n} \\
        \ln \mathcal{L}_{\mathrm{Geom}}(\vec{x}, p) = n \ln p + \ln{(1-p)} \cdot\left(\sum\limits_{k=1}^n x_k - n\right)
    \end{gather*}
    \item $\xi \sim \mathrm{Pas}(a)$ --- розподіл Паскаля, $\forall \; k \in \mathbb{N} \; x_k \in \left\{0, 1, 2, 3, ...\right\}$. 
    \begin{gather*}
        \mathcal{L}_{\mathrm{Pas}}(\vec{x}, a) = \prod\limits_{k=1}^n \frac{a^{x_k}}{(1+a)^{x_k + 1}} = 
        \frac{a^{\sum_{k=1}^n x_k}}{(1+a)^{n + \sum_{k=1}^n x_k}} \\
        \ln \mathcal{L}_{\mathrm{Pas}}(\vec{x}, a) = \ln a \cdot \sum\limits_{k=1}^n x_k - \ln{(1+a)} \cdot \left(n+ \sum\limits_{k=1}^n x_k\right) 
    \end{gather*}
    \item $\xi \sim \mathrm{Poiss}(a)$ --- розподіл Пуассона, $\forall \; k \in \mathbb{N} \; x_k \in \left\{0, 1, 2, 3, ...\right\}$. 
    \begin{gather*}
        \mathcal{L}_{\mathrm{Poiss}}(\vec{x}, a) = \prod\limits_{k=1}^n \frac{a^{x_k}}{(x_k)!} e^{-a} = e^{-na} \cdot \frac{a^{\sum\limits_{k=1}^n x_k}}{\prod\limits_{k=1}^n x_k !} \\
        \ln \mathcal{L}_{\mathrm{Poiss}}(\vec{x}, a) = -n a + \ln a \cdot \sum\limits_{k=1}^n x_k - \sum\limits_{k=1}^n \ln{x_k!}
    \end{gather*}
    \item $\xi \sim \mathrm{U}\left< a; b\right>$ --- рівномірний розподіл, $\forall \; k \in \mathbb{N} \; x_k \in \left< a; b\right>$:
    \begin{gather*}
        \mathcal{L}_{\mathrm{U}}(\vec{x}, a, b) = \prod\limits_{k=1}^n \frac{1}{b-a} = \frac{1}{(b-a)^n} \\
        \ln\mathcal{L}_{\mathrm{U}}(\vec{x}, a, b) = -n \ln{(b-a)}
    \end{gather*} 
    \item $\xi \sim \mathrm{Exp}(\lambda, b)$ --- експоненційний розподіл зі зсувом, $\forall \; k \in \mathbb{N} \; x_k \geq b$:
    \begin{gather*}
        \mathcal{L}_{\mathrm{Exp}}(\vec{x}, \lambda, b) = \prod\limits_{k=1}^n \lambda e^{-\lambda (x_k-b)} = \lambda^n e^{-\lambda\left(\sum\limits_{k=1}^n x_k + n b\right)} \\
        \ln \mathcal{L}_{\mathrm{Exp}}(\vec{x}, \lambda, b) = n \ln \lambda - \lambda\left(\sum\limits_{k=1}^n x_k + n b\right)
    \end{gather*}
    \item $\xi \sim \mathrm{N}(a, \sigma^2)$ --- нормальний розподіл, $\forall \; k \in \mathbb{N} \; x_k \in \mathbb{R}$:
    \begin{gather*}
        \mathcal{L}_{\mathrm{Exp}}(\vec{x}, a, \sigma) = \prod\limits_{k=1}^n \frac{1}{\sqrt{2\pi}\sigma} \exp\left\{-\frac{(x_k - a)^2}{2\sigma^2}\right\} \\
        \ln \mathcal{L}_{\mathrm{Exp}}(\vec{x}, a, \sigma) = \frac{1}{(2\pi)^{\frac{n}{2}} \sigma^n} \cdot \exp\left\{-\frac{1}{2\sigma^2}\sum\limits_{k=1}^n (x_k - a)^2\right\}
    \end{gather*}
\end{enumerate}
        % !TEX root = ../../main.tex
\section{Точкові оцінки}
Нехай $\xi$ --- ГС, а $F_{\xi}(x, \theta_1, \theta_2, ..., \theta_k)$ --- її функція розподілу, де
$\theta_1, \theta_2, ..., \theta_k$ --- набір параметрів. Тип самої функцію розподілу у випадку точкового оцінювання
вважається відомим, невідомими є параметри. Наприклад, $\lambda$ в експоненційному законі чи $a$ та $\sigma$ в нормальному.
\begin{definition}
    \emph{Точковою оцінкою} $\theta^*$ невідомого параметру $\theta$ називається деяка статистика
    $\theta^*(\xi_1, ..., \xi_n)$, значення якої на конкретній
    реалізації вибірки приймається за наближене значення $\theta$.
\end{definition}
Зрозуміло, що точкова оцінка $\theta^*$, на відміну від параметру $\theta$, є випадковою величиною, 
яка залежить від закону розподілу $\xi$ та обсягу вибірки. Безумовно, можна ввести багато функцій від результатів спостережень, 
які можна брати в якості $\theta^*$. Наприклад, якщо параметр $\theta$ є математичним сподіванням $\xi$, 
то за оцінку математичного сподівання за результатами спостережень можна взяти середнє арифметичне, моду, медіану, 
півсуму найбільшого та найменшого значень вибірки тощо. 
Отже, яку статистику краще обрати? Назвати <<найкращою>> оцінку ту, яка найбільш близька до істинного значення оцінюваного параметру, 
неможливо, оскільки точкова оцінка --- випадкова величина. Таким чином, робити висновки про якість оцінки варто не по її конкретним значенням, 
а по її розподілу. В зв'язку з цим розглянемо вимоги, що висувають до точкових оцінок.

\subsection{Незміщені точкові оцінки}
 \begin{definition}
    Точкова оцінка $\theta^*$ параметру $\theta$ називається \emph{незміщеною}, якщо
    \begin{gather}\label{estim_unbiased}
        E\theta^*(\xi_1, ..., \xi_n) = \theta \text{ для всіх } n\in\mathbb{N}
    \end{gather} 
    і \emph{асимптотично незміщеною}, якщо $\underset{n\to\infty}{\lim} E\theta^*(\xi_1, ..., \xi_n) = \theta$.
 \end{definition}
 \begin{example} 
    Розглянемо деякі важливі незміщені оцінки.
    \begin{enumerate}
        \item Вибіркове середнє --- незміщена оцінка математичного сподівання:
        $\overline{\xi} = \frac{1}{n}\sum\limits_{k=1}^n \xi_k$, $E\overline{\xi} = \frac{1}{n}\sum\limits_{k=1}^n E\xi_k = \frac{1}{n} \cdot{n} \cdot{E\xi} = E\xi$,
        оскільки всі $\xi_k$ однаково розподілені.
        \item Вибіркова дисперсія --- незміщена оцінка дисперсії у випадку відомого математичного сподівання $E\xi$:
        $D^*\xi = \frac{1}{n}\sum\limits_{k=1}^n \left(\xi_k - E\xi \right)^2$, 
        $E\left( D^* \xi\right) = \frac{1}{n}\sum\limits_{k=1}^n E\left(\xi_k - E\xi \right)^2 = D\xi$ знову через однаковий розподіл $\xi_k$.
        \item Дослідимо вибіркову дисперсію, але у випадку невідомого математичного сподівання. 
        \begin{gather*}
            D^*\xi = \frac{1}{n}\sum\limits_{k=1}^n \left(\xi_k - \overline{\xi} \right)^2 = 
            \frac{1}{n}\sum\limits_{k=1}^n \left((\xi_k - E\xi) + (\overline{\xi} - E\xi) \right)^2 =  \\
            = \frac{1}{n}\sum\limits_{k=1}^n \left(\xi_k - E\xi \right)^2 - 2\left(\overline{\xi} - E\xi\right)\cdot 
            \underbrace{\frac{1}{n}\sum\limits_{k=1}^n \left(\xi_k - E\xi\right)}_{\overline{\xi} - E\xi} + \left(\overline{\xi} - E\xi\right)^2 = \\
            = \frac{1}{n}\sum\limits_{k=1}^n \left(\xi_k - E\xi \right)^2 - \left(\overline{\xi} - E\xi\right)^2
        \end{gather*}
        Перший доданок --- це формула для вибіркової дисперсії при відомому математичному сподіванні, а 
        $E\xi = E\overline{\xi}$, тому
        \begin{gather*}
            E\left( D^* \xi\right) = D\xi - E\left(\overline{\xi} - E\xi\right)^2= D\xi - D\overline{\xi} \\
            D\overline{\xi} = D\left(\frac{1}{n}\sum\limits_{k=1}^n \xi_k \right) = \frac{1}{n^2} \sum\limits_{k=1}^n D\xi_k = \frac{1}{n} D\xi
        \end{gather*}
        Дві останні рівності одержані через незалежність та однаковий розподіл $\xi_k$. Отже, маємо $E\left( D^* \xi\right) = \left( 1- \frac{1}{n}\right) D\xi$,
        тому ця оцінка є лише асимптотично незміщеною. Проте, оцінка $D^{**}\xi = \frac{n}{n-1} D^*\xi$ буде незміщеною.
        Таким чином, якщо $E\xi$ невідоме, то \emph{виправлена вибіркова дисперсія} 
        $D^{**}\xi = \frac{1}{n-1} \sum\limits_{k=1}^n \left(\xi_k - \overline{\xi} \right)^2$ є незміщеною оцінкою дисперсії.
        \item Нехай $\xi \sim \mathrm{U}\left< a; b\right>$, перевіримо незміщеність $a^* = \underset{1\leq k \leq n}{\min}\xi_k$.
        Знайдемо $Ea^*$. Як відомо, 
        $f_{\min}(x) = n \left(1-F_{\xi}(x)\right)^{n-1} f_{\xi}(x) = n\left( 1- \frac{x-a}{b-a}\right)^{n-1}\frac{1}{b-a} = n\cdot\frac{(b-x)^{n-1}}{(b-a)^n}$, якщо
        $x \in \left< a; b\right>$, та $0$ інакше.
        \begin{gather*}
            Ea^* = \frac{n}{(b-a)^n} \int_a^b x(b-x)^{n-1} dx = \left[ b-x = t \right] = 
            \frac{n}{(b-a)^n} \int_0^{b-a} (b-t)t^{n-1} dt = \\
            = \frac{n}{(b-a)^n} \left.\left( \frac{bt^n}{n} - \frac{t^{n+1}}{n+1}\right)\right|_0^{b-a} = 
            \frac{n}{(b-a)^n} \left(\frac{b(b-a)^n}{n} - \frac{(b-a)^{n+1}}{n+1}\right) = \\
            = b - \frac{n(b-a)}{n+1} = \frac{bn + b - nb + na}{n+1} = a \cdot \frac{n}{n+1} + \frac{b}{n+1} \neq a
         \end{gather*}
         Але $\underset{n\to\infty}{\lim} Ea^* = a$, тому $a^*$ є асимптотично незміщеною оцінкою.
    \end{enumerate}
 \end{example}
 \begin{exercise}
     Перевірити, що $b^* = \underset{1\leq k \leq n}{\max}\xi_k$ є асимптотично незміщеною оцінкою для параметра $b$ у випадку $\xi \sim \mathrm{U}\left< a; b\right>$.
     Користуючись вже дослідженими оцінками $a^*$ та $b^*$, знайти незміщені оцінки для параметрів $a$ і $b$ (підказка: це будуть деякі лінійні комбінації $a^*$ та $b^*$).
 \end{exercise}
        % !TEX root = ../../main.tex
\section{Інтервальне оцінювання}
Нехай $\xi$ --- ГС, $\theta$ --- якийсь параметр її
розподілу.
Задача \emph{інтервального оцінювання} $\theta$ --- це пошук за заданим \emph{рівнем надійності} 
$\gamma$ статистик
$\theta^*_1(\vec{\xi})$, $\theta^*_2(\vec{\xi})$ таких, що:
\begin{gather*}
    \P\left\{\theta \in (\theta^*_1, \theta^*_2)\right\} = \gamma \Leftrightarrow
    \P\left\{ \theta^*_1 < \theta < \theta^*_2\right\} = \gamma
\end{gather*}
Іноді в цих співвідношеннях пишуть $\geq \gamma$, оскільки у випадку дискретного розподілу цих статистик рівності можуть не мати сенсу.

\begin{definition}
    $(\theta^*_1, \theta^*_2)$ називається \emph{довірчим інтервалом з рівнем надійності $\gamma$}.
    Більш точно --- маємо справу з послідовністю таких інтервалів, що залежать від обсягу вибірки.
\end{definition}
\begin{remark}
    Оскільки межі інтервалу є випадковими, то запис $\theta \in (\theta^*_1, \theta^*_2)$ правильно читати не як
    <<$\theta$ потрапляє в інтервал $(\theta^*_1, \theta^*_2)$>>, а як
    <<інтервал $(\theta^*_1, \theta^*_2)$ накриває $\theta$>>.
\end{remark}
Зручно шукати довірчі інтервали конкретного вигляду, наприклад,
\emph{симетричні} відносно $\theta$ з умови $\P\left\{|\theta - \theta^*| < \varepsilon\right\} = \gamma$, де $\varepsilon$ --- \emph{точність} 
довірчого інтервалу, або \emph{однобічні} з умов $\P\{\theta > \theta^*\} = \gamma$ чи $\P\{\theta < \theta^*\} = \gamma$.
У випадку симетричного довірчого інтервалу зазвичай одразу обирається сама статистика $\theta^*$ (наприклад, якась <<хороша>> точкова оцінка).
В такому разі пошук $\varepsilon$ --- це пошук ширини довірчого інтервалу, що забезпечує заданий рівень надійності.

Якщо розглядається конкретна реалізація вибірки, то можна обчислити межі знайденого інтервалу як значення відповідних статистик.

\subsection{Побудова довірчих інтервалів для гаусcівської ГС}
Нехай ГС $\xi \sim \mathrm{N}(a, \sigma^2)$. Розглянемо чотири випадки побудови довірчих інтервалів
для параметрів $a$ та $\sigma^2$ (нагадаємо, що це математичне сподівання та дисперсія).

\noindent\textbf{Довірчий інтервал для математичного сподівання при відомій дисперсії.}

Шукатимемо симетричний довірчий інтервал $\P\left\{|a - a^*| < \varepsilon\right\} = \gamma$, де в якості
оцінки $a^*$ буде вибіркове середнє $\overline{\xi}$. З властивостей незалежних гаусcівських ВВ
маємо $\sum\limits_{k=1}^n \xi_k \sim \mathrm{N}(na, n\sigma^2)$, тому $\overline{\xi} \sim \mathrm{N}\left(a, \frac{\sigma^2}{n}\right)$.
Отже,
$
    \P\{|a - a^*| < \varepsilon\} = \P \{a - \varepsilon < \overline{\xi} < a + \varepsilon\} = 
    2\Phi\left(\frac{\varepsilon\sqrt{n}}{\sigma}\right) = \gamma
$, де $\Phi(x)$ --- функція Лапласа. З таблиці її значень знайдемо відповідне значення $\varepsilon$ і отримаємо шуканий
довірчий інтервал $\left(\overline{\xi}-\varepsilon, \overline{\xi}+\varepsilon \right)$.

\noindent\textbf{Довірчий інтервал для математичного сподівання при невідомій дисперсії.}

Шукатимемо симетричний довірчий інтервал $\P\left\{|a - a^*| < \varepsilon\right\} = \gamma$. В умовах попереднього прикладу
$\sqrt{n} \cdot \frac{\overline{\xi} - a}{\sigma} \sim \mathrm{N}(0, 1)$, але $\sigma$ тепер невідоме. Розглянемо статистику
$\D^* \xi = \frac{1}{n}\sum\limits_{k=1}^n (\xi_k - \overline{\xi})^2 = \frac{1}{n} \sum\limits_{k=1}^n (\xi_k - a)^2 - (\overline{\xi} - a)^2$.
$\frac{n \D^* \xi}{\sigma^2} = \sum\limits_{k=1}^n \left(\frac{\xi_k - a}{\sigma} \right)^2 - 
\left(\sqrt{n} \cdot \frac{\overline{\xi} - a}{\sigma} \right)^2$. В теоремі нижче буде доведено незалежність доданків,
що дасть право сказати, що $\frac{n \D^* \xi}{\sigma^2} \sim \chi_{n-1}^2$. Як наслідок,
$$ 
\frac{\sqrt{n}\cdot\frac{\overline{\xi} - a}{\sigma}}{\sqrt{\frac{1}{n-1} \cdot \frac{n \D^* \xi}{\sigma^2} }} = 
\frac{\sqrt{n}\cdot(\overline{\xi} - a)}{\sqrt{\D^{**}\xi}} \sim \mathrm{St}_{n-1}
$$

Тепер з рівності $\P\left\{ \frac{\sqrt{n}\cdot|\overline{\xi} - a|}{\sqrt{\D^{**}\xi}} < t_{\gamma}\right\} = \gamma$ знайдемо
значення $t_{\gamma}$. Шуканим значенням $\varepsilon$ буде $\varepsilon = \frac{t_{\gamma}}{\sqrt{n}} \cdot \sqrt{(\D^{**}\xi)_{\text{зн}}}$.
\begin{theorem*}[теорема Фішера]
    Статистики стандартної гаусcівської ГС $\overline{\xi}$ та $\D^*\xi$ --- незалежні випадкові величини.
\end{theorem*}
\begin{proof}
    $\vec{\xi}$ --- випадкова вибірка, $\vec{\xi} \sim \mathrm{N}(\vec{0}, I)$. Нехай $C$ --- деяка ортогональна матриця,
    тоді $\vec{\eta} = C \vec{\xi}$ теж має розподіл $\mathrm{N}(\vec{0}, I)$, причому 
    $\Vert \vec{\eta} \Vert = \Vert \vec{\xi} \Vert$. Розглянемо матрицю 
    $$
    C = \begin{pmatrix}
        c_{1,1} & c_{1,2} & \ldots & c_{1,n} \\
        c_{2,1} & c_{2,2} & \ldots & c_{2,n} \\
        \vdots & \vdots & \ddots & \vdots \\
        c_{n-1,1} & c_{n-1,2} & \ldots & c_{n-1,n} \\
        1/\sqrt{n} & 1/\sqrt{n} & \ldots & 1/\sqrt{n}
    \end{pmatrix}
    $$
    перші $n-1$ рядків якої --- це елементи ортонормованого базису $\text{л.о.}\left\{
    \begin{pmatrix}
        \frac{1}{\sqrt{n}} & ... & \frac{1}{\sqrt{n}}
    \end{pmatrix}^{T}\right\}^{\perp}$.
    $\vec{\eta} = C\vec{\xi}$, $\eta_n = \frac{1}{\sqrt{n}} \left(\xi_1 + \xi_2 + ... + \xi_n \right) = \sqrt{n} \cdot \overline{\xi}$.
    Вище у прикладі побудови довірчого інтервалу було показано, що
    $\D^* \xi = \frac{1}{n} \sum\limits_{k=1}^n (\xi_k - a)^2 - (\overline{\xi} - a)^2$. В умовах теореми ця рівність спрощується до
    $\D^* \xi = \frac{1}{n} \sum\limits_{k=1}^n \xi_k^2 - \overline{\xi}^2 = \frac{1}{n} \Vert \vec{\xi} \Vert^2 - \frac{1}{n}\eta_n^2 = 
    \frac{1}{n} \Vert \vec{\eta} \Vert^2 - \frac{1}{n}\eta_n^2 = \frac{1}{n} \sum\limits_{k=1}^n \eta_k^2 - \frac{1}{n}\eta_n^2 = 
    \frac{1}{n} \sum\limits_{k=1}^{n-1} \eta_k^2$. Таким чином, $\D^* \xi$ залежить від перших $n-1$ координат $\vec{\eta}$, а отже ---
    не залежить від $\overline{\xi} = \frac{1}{\sqrt{n}} \eta_n$.
\end{proof}
\begin{example}
    Побудувати 95\% довірчий інтервал для математичного сподівання гаусcівської ГС, якщо $\overline{x} = 2$, $n=25$,
    а дисперсія відома і рівна $6$, а потім --- якщо невідома і $(\D^{**}\xi)_{\text{зн}} = 5.78$.
    \begin{enumerate}
        \item $\D\xi = 6$. За умовою $\gamma = 0.95$, тому $\varepsilon$ шукаємо з 
        $2\Phi\left(\frac{\varepsilon\sqrt{n}}{\sigma}\right) = \gamma$. $\Phi\left(\varepsilon\cdot\frac{5}{\sqrt{6}}\right) = 0.475$, звідки
        $\varepsilon = \frac{\sqrt{6}}{5} \cdot 1.96 \approx 0.96$. Отже, шуканий довірчий інтервал --- $(1.04, 2.96)$.
        \item $(\D^{**}\xi)_{\text{зн}} = 5.78$. Спочатку знайдемо $t_{\gamma} = 2.064$, звідки $\varepsilon = \frac{2.064}{\sqrt{25}}\cdot \sqrt{5.78} \approx 2.386$.
        Отже, шуканий довірчий інтервал --- $(-0.064, 4.064)$.
    \end{enumerate}
\end{example}

\label{normal_variance_conf_interv}
\noindent\textbf{Довірчий інтервал для дисперсії при відомому математичному сподіванні.}

В якості точкової оцінки дисперсії візьмемо $\D^* \xi = \frac{1}{n} \sum\limits_{k=1}^n (\xi_k - a)^2$, тоді
$\frac{n \D^* \xi}{\sigma^2} = \sum\limits_{k=1}^n \left(\frac{\xi_k - a}{\sigma} \right)^2 \sim \chi^2_n$.
Шукатимемо довірчий інтервал з умови $\P\left\{ t_1 < \frac{n \D^* \xi}{\sigma^2} < t_2\right\} = \gamma$, де $t_1$ та 
$t_2$ задовольняють $\P\left\{\frac{n \D^* \xi}{\sigma^2} > t_1 \right\} = \frac{1 + \gamma}{2}$ та
$\P\left\{\frac{n \D^* \xi}{\sigma^2}\geq t_2 \right\} = \frac{1 - \gamma}{2}$. Шуканим довірчим інтервалом буде
$\left( \frac{n}{t_2}(\D^{*}\xi)_{\text{зн}}, \frac{n}{t_1}(\D^{*}\xi)_{\text{зн}} \right)$.

\noindent\textbf{Довірчий інтервал для дисперсії при невідомому математичному сподіванні.}

Як було показано раніше, для $\D^* \xi = \frac{1}{n} \sum\limits_{k=1}^n (\xi_k - \overline{\xi})^2$ статистика 
$\frac{n \D^* \xi}{\sigma^2}$ має розподіл $\chi^2_{n-1}$, тому $\frac{(n-1) \D^{**} \xi}{\sigma^2} \sim \chi^2_{n-1}$.
Як і в попередньому випадку, шукаємо $t_1$ та $t_2$ з умов 
$\P\left\{\frac{(n-1) \D^{**} \xi}{\sigma^2} > t_1 \right\} = \frac{1 + \gamma}{2}$ та
$\P\left\{\frac{(n-1) \D^{**} \xi}{\sigma^2}\geq t_2 \right\} = \frac{1 - \gamma}{2}$.
Шуканий довірчий інтервал ---
$\left( \frac{n-1}{t_2}(\D^{**}\xi)_{\text{зн}}, \frac{n-1}{t_1}(\D^{**}\xi)_{\text{зн}} \right)$.

\subsection{Наближені довірчі інтервали}
Якщо статистика $\theta^*$, що оцінює невідомий параметр, є асимптотично нормальною, то розподіл 
$\frac{\theta^* - \theta}{\sqrt{\D\theta^*}}$ можна вважати приблизно рівним $\mathrm{N}(0, 1)$ (звісно, при достатньо великих $n$).
В такому випадку довірчий інтервал можна будувати з рівності
$\P\left\{\frac{|\theta^* - \theta|}{\sqrt{\D\theta^*}} < t_{\gamma}\right\} = \gamma$, де $t_{\gamma}$
знаходимо з таблиці значень функції Лапласа. Щоб тепер отримати довірчий інтервал, треба розв'язати відносно
$\theta$ нерівність $\frac{|\theta^* - \theta|}{\sqrt{\D\theta^*}} < t_{\gamma}$.
\begin{example}
    Побудуємо наближений довірчий інтервал для $a$ у випадку $\xi \sim \mathrm{Poiss}(a)$. Вважатимемо, що значення $t_\gamma$
    вже знайдемо з рівності 
    $\P\left\{\frac{|a^* - a|}{\sqrt{\D a^*}} < t_{\gamma}\right\} = \gamma$, де $a^* = \overline{\xi}$. Розв'яжемо нерівність:
    \begin{gather*}
        \frac{|a^* - a|}{\sqrt{\D a^*}} < t_{\gamma} \Leftrightarrow
        \frac{(a^* - a)^2}{\D a^*} < t_{\gamma}^2, \; \D a^* = \D \overline{\xi} = \frac{1}{n}\D\xi = \frac{1}{n}{a} \\
        (a^* - a)^2 < a \cdot \frac{t_{\gamma}^2}{n} \Leftrightarrow (a^*)^2 - 2a^* a + a^2 < a \cdot \frac{t_{\gamma}^2}{n}
        \Leftrightarrow a^2 - 2a \left( a^* + \frac{t_{\gamma}^2}{2n}\right) + (a^*)^2 < 0
    \end{gather*}
    Позначимо $a^* + \frac{t_{\gamma}^2}{2n} = b$:
    \begin{gather*}
        a^2 - 2ab + (a^*)^2 < 0, \; D/4 = b^2 - a^2 (a^*)^2 \Rightarrow
        a \in \left(b - \sqrt{D/4}, b + \sqrt{D/4} \right) \\
        b \pm \sqrt{D/4} =  a^* + \frac{t_{\gamma}^2}{2n} \pm 
        \sqrt{\left( a^* + \frac{t_{\gamma}^2}{2n}\right)^2  - (a^*)^2} = 
        a^* + \frac{t_{\gamma}^2}{2n} \pm 
        \sqrt{a^* \cdot\frac{t_{\gamma}^2}{n} + \frac{t_{\gamma}^4}{4n^2}}
    \end{gather*}
    Отже, шуканий довірчий інтервал:
    $$ 
    \left(\overline{\xi} + \frac{t_{\gamma}^2}{2n} -
    \sqrt{\overline{\xi} \cdot\frac{t_{\gamma}^2}{n} + \frac{t_{\gamma}^4}{4n^2}},
    \overline{\xi} + \frac{t_{\gamma}^2}{2n} +
    \sqrt{\overline{\xi} \cdot\frac{t_{\gamma}^2}{n} + \frac{t_{\gamma}^4}{4n^2}}
    \right)
    $$
\end{example}

\subsection{Довірчий інтервал для ймовірності появи події}
За точкову оцінку ймовірності $p$ появи події $A$
в схемі Бернуллі беруть частість $p^* = \frac{m}{n}$, де $n$ ---
загальна кількість незалежних випробувань,
$m$ --- кількість появ події $A$ в цих випробуваннях
Задамо рівень надійності $\gamma$
і знайдемо такі величини $p_1$ та $p_2$, щоб
виконувалось співвідношення $\P\left\{ p_1 < p < p_2\right\} = \gamma$.
Інтервал $(p_1, p_2)$ буде шуканим довірчим інтервалом. Розглянемо два випадки.

\noindent\textbf{Кількість випробувань досить велика.} 

В цьому випадку розподіл величини $m$ в силу граничної теореми Муавра-Лапласа 
можна наближено замінити $\mathrm{N}(np, \sqrt{npq})$, тому
розподіл $p^* = \frac{m}{n}$ приблизно рівний $\mathrm{N}\left(p, \sqrt{\frac{pq}{n}} \right)$.
Таким чином, статистика $\frac{(p^* - p)\sqrt{n}}{\sqrt{pq}}$ має наближено розподіл $\mathrm{N}(0, 1)$.
Користуючись таблицею значень функції Лапласа, для заданої довірчої ймовірності $\gamma$ знайдемо таке $t_{\gamma}$,
при якому 
$\P\left\{\frac{|p^* - p|\sqrt{n}}{\sqrt{pq}} < t_{\gamma}\right\} = \gamma$. Розв'яжемо відносно $p$ нерівність під знаком ймовірності:
\begin{gather*}
    \frac{|p^* - p|\sqrt{n}}{\sqrt{pq}} < t_{\gamma} \Leftrightarrow
    \frac{(p^* - p)^2 n}{p(1-p)} < t^2_{\gamma} \Leftrightarrow
    (p^*)^2 - 2p^* p + p^2 < \left(p - p^2\right)\cdot \frac{t^2_{\gamma}}{n} \Leftrightarrow \\
    \Leftrightarrow
    \left(1 + \frac{t^2_{\gamma}}{n}\right)p^2 - \left(2p^* + \frac{t^2_{\gamma}}{n}\right)p + (p^*)^2 < 0 \Leftrightarrow \\
    \Leftrightarrow
    p \in (p_1, p_2), \; p_{1, 2} = \frac{\left(p^* + \frac{t^2_{\gamma}}{2n}\right) \pm t_{\gamma} \cdot \sqrt{\frac{p^*(1-p^*)}{n} + \frac{t^2_{\gamma}}{4n^2}}}{1 + \frac{t^2_{\gamma}}{n}}
\end{gather*}

Цей результат має геометричну інтерпретацію. Розглянемо систему координат, по осі абсцис якої відкладаємо частість $p^*$, а по осі ординат ---
ймовірність $p$. Повернемося до нерівності, яку вже розв'язали:
\begin{gather*}
    \left(1 + \frac{t^2_{\gamma}}{n}\right)p^2 - 2p^* p + (p^*)^2 - \frac{t^2_{\gamma}}{n}p < 0
\end{gather*}
\begin{tabular}{c p{10.4cm}}
    \begin{tikzpicture}[xscale=2.5, yscale=2.5, baseline={(current bounding box.north)}]
        \fill [gray!20] (0.47, 0.475) circle [x radius=0.7, y radius=0.2, rotate=40];
        \draw [thick] (0.47, 0.475) circle [x radius=0.7, y radius=0.2, rotate=40];
        \draw [->] (0, 0) -- (1.2, 0);
        \draw [->] (0, 0) -- (0, 1.2);
        \draw [dashed] (0, 0.95) -- (0.95, 0.95);
        \draw [dashed] (0.95, 0) -- (0.95, 0.95);
        \node [below] at (1, 0) {$1$};
        \node [left] at (0, 1) {$1$};
        \node [below] at (1.2, 0) {$p^*$};
        \node [left] at (0, 1.2) {$p$};
        \node [below left] at (0, 0) {$0$};
        \draw [dashed] (0.6, 0) -- (0.6, 0.82);
        \draw [dashed] (0.6, 0.82) -- (0, 0.82);
        \draw [dashed] (0.6, 0.325) -- (0, 0.325);
        \node [left] at (0, 0.325) {$p_1$};
        \node [left] at (0, 0.82) {$p_2$};
        \node [below] at (0.6, 0) {$p^*_{\text{зн.}}$};
    \end{tikzpicture} &
    Ця нерівність задає внутрішню частину деякого еліпса.
    Таким чином, довірчим інтервалом для $p$ при відомому значенні частості $p^*_{\text{зн.}}$ буде множина точок всередині цього еліпса з абсцисою,
    що дорівнює $p^*_{\text{зн.}}$. Важливо зауважити, що навіть якщо $p^*_{\text{зн.}} = 1$, то немає підстав казати, що справжнє значення $p$ дорівнює 1.
    В цьому випадку отримаємо довірчий інтервал
    $\left(\frac{1}{1+t_{\gamma}^2/n}, 1\right)$. Аналогічно, при $p^*_{\text{зн.}} = 0$ отримаємо довірчий інтервал
    $\left(0, \frac{t_{\gamma}^2/n}{1+t_{\gamma}^2/n}\right)$.
\end{tabular}
Якщо обсяг вибірки $n$ доволі великий, то величиною $\frac{t_{\gamma}^2}{n}$ можна знехтувати. 
Тоді межі довірчого інтервалу набувають наближених значень 
$$p_{1, 2} \approx p^* \pm t_{\gamma} \sqrt{\frac{p^*(1-p^*)}{n}}$$
\begin{example}
    Деяка подія в серії з $n=100$ незалежних випробувань відбулась $m=78$ разів. Побудувати довірчий інтервал для
    ймовірності $p$ появи цієї події з надійністю $\gamma = 0.9$.
    
    За умовою $p^*_{\text{зн.}} = 0.78$, а відповідне значення $t_{\gamma}$ з таблиці функції Лапласа дорівнює $1.65$.
    Отже,
    \begin{gather*}
        p_{1, 2} = \frac{0.78 + \frac{1.65^2}{200} \pm 1.65 \sqrt{\frac{0.78\cdot 0.22}{100} + \frac{1.65^2}{4\cdot 100^2}}}{1+\frac{1.65^2}{100}} \Rightarrow
        p_1 \approx 0.7047, \; p_2 \approx 0.8404
    \end{gather*}
    Якщо обчислити межі довірчого інтервалу за наближеними формулами для великих $n$, отримаємо
    \begin{gather*}
        p_{1, 2} \approx 0.78 \pm 1.65\sqrt{\frac{0.78\cdot 0.22}{100}} \Rightarrow p_1 \approx 0.7116, \; p_2 \approx 0.8483
    \end{gather*}
\end{example}

\noindent\textbf{Кількість випробувань мала.} 
В цьому випадку граничними теоремами скористатися не вийде. Згадаємо формулу для ймовірності появи події $A$ в схемі Бернуллі
з $n$ іспитами $k$ разів: 
$\P\left\{ \xi = k\right\} = C_n^k p^k (1-p)^{n-k}$, де $\xi\sim \mathrm{Bin}(n, p)$. 
Задамо рівень надійності $\gamma$ та знайдемо такі $p_1$ та $p_2$, що $\P\left\{p_1 < p < p_2 \right\} = \gamma$.
Приймемо без доведення, що $p_1$ --- розв'язок рівняння 
$\sum\limits_{k=0}^{m-1} C_n^k p_1^k (1-p_1)^{n-k} = \frac{1+\gamma}{2}$, а $p_2$ --- розв'язок рівняння
$\sum\limits_{k=0}^{m} C_n^k p_2^k (1-p_2)^{n-k} = \frac{1-\gamma}{2}$.
В цих формулах $m$ є конкретним числом (а не випадковою величиною), кількістю випробувань, в яких сталася подія $A$.
Існують спеціальні таблиці для знаходження значень $p_1$ та $p_2$, що задовольняють цим рівнянням.

\subsection{Довірчі інтервали для параметрів рівномірної ГС}
Як відомо,  асимптотично незміщеними та конзистентними оцінками параметрів $a$ та $b$ рівномірної ГС є, відповідно,
$a^* = \underset{1\leq k \leq n}{\min}\xi_k$ та $b^* = \underset{1\leq k \leq n}{\max}\xi_k$. З огляду на характер цих параметрів,
шукатимемо довірчі інтервали з рівностей
$\P\left\{a\leq a^* \leq a+\varepsilon_1 \right\} \geq \gamma$ та $\P\left\{b-\varepsilon_2\leq b^* \leq b\right\} \geq \gamma$,
де $\gamma$ --- заданий рівень надійності.
Користуючись щільностями розподілу $a^*$ та $b^*$, обчислимо ці ймовірності:
\begin{gather*}
    \P\left\{a\leq a^* \leq a+\varepsilon_1 \right\} = \int_{a}^{a+\varepsilon_1} \frac{n}{(b-a)^n} (b-x)^{n-1} dx = 1 - \left(1 - \frac{\varepsilon_1}{b-a}\right)^n \\
    \P\left\{b-\varepsilon_2\leq b^* \leq b\right\} = \int_{b-\varepsilon_2}^b \frac{n}{(b-a)^n} (x-a)^{n-1} dx = 1 - \left(1 - \frac{\varepsilon_2}{b-a}\right)^n
\end{gather*}
Для різниці $b-a$ з умов $a^* \leq a+\varepsilon_1$ та $b^* \geq b-\varepsilon_2$ маємо оцінку
$b-a \leq \underset{1\leq k \leq n}{\max}\xi_k + \varepsilon_2 - \underset{1\leq k \leq n}{\min}\xi_k + \varepsilon_1$.
Далі для зручності позначатимемо $\underset{1\leq k \leq n}{\max}\xi_k = M$, $\underset{1\leq k \leq n}{\min}\xi_k = m$.
Отримуємо систему:
\begin{gather*}
    \begin{cases}
        1 - \left(1 - \frac{\varepsilon_1}{b-a}\right)^n \geq 1 - \left(1 - \frac{\varepsilon_1}{M - m + \varepsilon_2 + \varepsilon_1}\right)^n = \gamma \\
        1 - \left(1 - \frac{\varepsilon_2}{b-a}\right)^n \geq 1 - \left(1 - \frac{\varepsilon_2}{M - m + \varepsilon_2 + \varepsilon_1}\right)^n = \gamma
    \end{cases}
\end{gather*}
Розв'яжемо її відносно $\varepsilon_1$ та $\varepsilon_2$.
\begin{gather*}
    \begin{cases}
        \left(1 - \frac{\varepsilon_1}{M - m + \varepsilon_2 + \varepsilon_1}\right)^n = 1 - \gamma \\
        \left(1 - \frac{\varepsilon_2}{M - m + \varepsilon_2 + \varepsilon_1}\right)^n = 1 - \gamma
    \end{cases} \Leftrightarrow
    \begin{cases}
        1 - \frac{\varepsilon_1}{M - m + \varepsilon_2 + \varepsilon_1} = \sqrt[\leftroot{-3}\uproot{3}n]{1-\gamma} \\
        1 - \frac{\varepsilon_2}{M - m + \varepsilon_2 + \varepsilon_1} = \sqrt[\leftroot{-3}\uproot{3}n]{1-\gamma}
    \end{cases} \Leftrightarrow \\
    \Leftrightarrow
    \begin{cases}
        \varepsilon_1 = \left(M - m + \varepsilon_2 + \varepsilon_1\right)\left(1 - \sqrt[\leftroot{-3}\uproot{3}n]{1-\gamma}\right) \\
        \varepsilon_2 = \left(M - m + \varepsilon_2 + \varepsilon_1\right)\left(1 - \sqrt[\leftroot{-3}\uproot{3}n]{1-\gamma}\right)
    \end{cases}
\end{gather*}
Отже, $\varepsilon_1 = \varepsilon_2$, тому далі розв'язуємо лише одне рівняння:
\begin{gather*}
    \varepsilon_1 = \left(M - m + 2\varepsilon_1\right) \left(1 - \sqrt[\leftroot{-3}\uproot{3}n]{1-\gamma}\right) \Leftrightarrow
    \left( 2\sqrt[\leftroot{-3}\uproot{3}n]{1-\gamma} - 2 + 1\right)\varepsilon_1 = \left(M - m\right)\left(1 - \sqrt[\leftroot{-3}\uproot{3}n]{1-\gamma}\right)
\end{gather*}
Таким чином, $\varepsilon_1 = \varepsilon_2 = \frac{\left(M - m\right)\left(1 - \sqrt[\leftroot{-3}\uproot{3}n]{1-\gamma}\right)}{2\sqrt[\leftroot{-3}\uproot{3}n]{1-\gamma} - 1}$.
Оскільки $\varepsilon_1, \varepsilon_2 > 0$, то на $\gamma$ треба накладати вимогу $\gamma < 1 - 2^{-n}$.
Шукані довірчі інтервали матимуть вигляд $\left(a^* - \varepsilon_1, a^*\right)$ та $\left(b^*, b^* + \varepsilon_2\right)$.
    \chapter{Перевірка статистичних гіпотез}
        % !TEX root = ../../main.tex
\section{Поняття статистичних гіпотез та їх перевірки}
\subsection{Статистичні гіпотези. Помилки першого та другого роду}
Інформація, що отримана при обробці вибірки з деякої генеральної 
сукупності, може бути використана для отримання висновків про всю 
генеральну сукупність. Подібні висновки називають \emph{статистичними}. 
Завдяки їх ймовірнісному характеру завжди можна знайти ймовірність того, 
що прийняте рішення буде помилковим, тобто оцінити ризик того чи 
іншого прийнятого рішення.
\begin{definition}
    \emph{Статистичною непараметричною гіпотезою} $H$
    називається припущення про вигляд розподілу генеральної сукупності, яке 
    перевіряється за вибіркою. Часто розподіл генеральної сукупності відомий і за вибіркою треба 
    перевірити припущення щодо значень параметрів цього розподілу. Такі 
    статистичні гіпотези називають \emph{параметричними}. 
\end{definition}
Гіпотези бувають прості та складні. Гіпотеза називається \emph{простою}, 
якщо вона однозначно визначає розподіл генеральної сукупності, в 
іншому випадку гіпотеза називається \emph{складною}. Наприклад, простою 
гіпотезою є припущення, що ГС має нормальний розподіл з параметрами $0$ та $1$. 
Якщо висувається гіпотеза про те, що ГС має нормальний розподіл з параметрами $1$ та 
$\sigma$, де $\sigma \in (1, 2)$, то ця гіпотеза є складною. Теж саме стосується гіпотез про значення невідомого параметру розподілу ГС:
гіпотеза виду $H: \theta = 1$ є простою, а $H: \theta \in (1, 2)$ --- складною.
\begin{definition}
    Гіпотеза, що перевіряється, називається \emph{нульовою (основною) гіпотезою} й позначається $H_0$.
    Решта гіпотез називається \emph{альтернативними (конкуруючими)} відносно нульової гіпотези й позначаються
    $H_1$, $H_2$ тощо.
\end{definition}
Наприклад, якщо перевіряється проста гіпотеза про рівність параметра $\theta$ деякому значенню $\theta_0$,
тобто, $H_0 : \theta = \theta_0$, то в якості альтернативної гіпотези 
можна розглядати одну з таких: $H_1: \theta \neq \theta_0$, $H_2: \theta > \theta_0$,
$H_3: \theta < \theta_0$, $H_4: \theta = \theta_1$ (де $\theta_1 \neq \theta_0$).

Зазначимо, що математична статистика не дає ніяких рекомендацій
щодо вибору нульової та альтернативної гіпотези, Цей вибір повністю 
визначається дослідником і залежить від поставленої задачі.
\begin{definition}
    Правило, за яким приймається чи відхиляється гіпотеза на 
    основі вибірки, називається \emph{статистичним критерієм} для перевірки 
    гіпотези $H$. Якщо перевіряється гіпотеза про належність розподілу 
    генеральної сукупності до якогось класу розподілів (нормального, 
    рівномірного, Пуассона тощо), то зазначене правило називається \emph{критерієм згоди}.
\end{definition}
Оскільки висновок щодо прийняття або відхилення гіпотези 
приймається за реалізацією вибірки, то вибране рішення може бути 
помилковим. Розрізняють типи таких помилок.

\begin{definition}
    Помилка, яка полягає в тому, що правильна гіпотеза $H_0$, 
    згідно з вибраним критерієм, відхиляється, називається \emph{помилкою першого роду} --- це так званий <<хибно позитивний висновок>>. 
    \emph{Помилка другого роду} відбувається тоді, коли справджується 
    деяка альтернативна гіпотеза, але приймається основна гіпотеза $H_0$ --- це так званий <<хибно негативний висновок>>.
\end{definition}
% https://cdn-images-1.medium.com/max/1600/0*icEkiqBkLk38GPwM.jpg
Ці помилки істотно різні за своєю суттю. Проілюструємо відмінність 
між названими помилками на прикладі перевірки медичного препарату на дієвість.
В ролі ГС тут $\xi$ --- кількість випадків, коли препарат не подіяв. Висунемо дві гіпотези:
$H_0$ --- <<препарат недієвий>> та $H_1$ --- <<препарат дієвий>>. Помилка першого роду тут полягає в тому, що препарат дійсно не є дієвим,
але ця гіпотеза відхиляється (і це є тим самим хибно позитивним висновком, що препарат дієвий), 
а помилка другого роду --- в тому, що насправді дієвий препарат таким не вважають (це хибно негативний висновок про те, що препарат недієвий).
Наслідки помилки першого роду є випуск у виробництво недієвого (а може, й шкідливого) препарату, а помилки другого роду --- <<непомічання>> 
його дієвості та проведення наступних спроб вдосконалити препарат. Отже, помилка першого роду –-- це помилка, якої важливіше 
уникнути.

Слід зазначити, що статистичний критерій, за яким перевіряється певна 
гіпотеза, не відповідає на питання, правильна гіпотеза чи ні. За критерієм ми 
лише вирішуємо, чи суперечать вибіркові дані висунутій гіпотезі, чи ні. Висновок <<дані 
суперечать гіпотезі>> вважаються більш вагомим, ніж <<дані не суперечать гіпотезі>>.

\subsection{Методика перевірки статистичних гіпотез}
Нехай $\vec{\xi} = \left(\xi_1, \xi_2 ,..., \xi_n\right)$ --- випадкова вибірка, $\xi$ --- ГС, а $G\subseteq \mathbb{R}^n$ --- вибірковий простір.
В залежності від задачі формулюємо основну гіпотезу $H_0$ та альтернативну $H_1$. Сформулюємо загальний принцип побудови 
критеріїв перевірки гіпотез за певною реалізацією вибірки $\vec{x} = \left(x_1, x_2, ..., x_n\right)$.
Критерій задають за допомогою \emph{критичної множини} $W$, що є підмножиною вибіркового простору.
Множину $\overline{W} = G\setminus W$ назвемо \emph{областю прийняття гіпотези}. Рішення приймають так:
\begin{enumerate}
    \item Якщо вибірка $\vec{x} = \left(x_1, x_2, ..., x_n\right)$ належить критичній множині $W$, то вважають, що \emph{дані суперечать основній гіпотезі}, тобто $H_0$ відхиляють.
    \item Якщо вибірка $\vec{x} = \left(x_1, x_2, ..., x_n\right)$ належить області прийняття гіпотези $G\setminus W$, то відхиляють альтернативну гіпотезу й приймають основну $H_0$. В цьому
    випадку вважають, що \emph{дані не суперечать основній гіпотезі}.
\end{enumerate}

Знайдемо ймовірності помилок першого та другого роду. Нехай гіпотези $H_i, i = 0, 1$ полягають в тому, що щільність розподілу $\xi$ дорівнює $f_i(x)$,
якщо $\xi$ --- неперервна, або $\P\left\{ \xi_ = x_k\right\} = p_i(x_k)$, якщо $\xi$ --- дискретна.

\textbf{Ймовірність помилки першого роду} відносно нульової гіпотези $H_0$ дорівнює ймовірності того, що $\vec{\xi}$ потрапить в критичну множину $W$, тобто,
$\P\left\{ \vec{\xi} \in W / H_0\right\} = \alpha$. Величина $\alpha$ називається \emph{рівнем значущості критерію}. Якщо 
$\mathcal{L}_{H_0}(\vec{x})$ --- функція правдоподібності, побудована згідно висунутій гіпотезі $H_0$, то ймовірність помилки першого роду дорівнює
$\int_W \mathcal{L}_{H_0}(\vec{x}) d\vec{x}$. Рівень значущості, як правило, задається.

\textbf{Ймовірність помилки другого роду} відносно нульової гіпотези $H_0$ дорівнює ймовірності того, що $\vec{\xi}$ потрапить в $\overline{W}$,
якщо насправді вірна гіпотеза $H_1$, тобто, $\P\left\{ \vec{\xi} \in \overline{W} / H_1\right\} = \beta$. Якщо 
$\mathcal{L}_{H_1}(\vec{x})$ --- функція правдоподібності, побудована згідно альтернативній гіпотезі $H_1$, то ймовірність помилки другого роду дорівнює
$\int_{\overline{W}} \mathcal{L}_{H_1}(\vec{x}) d\vec{x}$.

Одночасно взяти якомога малими $\alpha$ та $\beta$ неможливо, тому критерій будують таким, щоб величина $1-\beta$, яка називається
\emph{потужністю критерію}, була найбільшою при заданому рівні значущості $\alpha$. Потужність критерію --- це ймовірність відхилити 
основну гіпотезу, якщо вона є хибною.

\section{Критерій Пірсона (\texorpdfstring{$\chi^2$}{x2}) та критерій Колмогорова}
Перейдемо до найважливішої задачі математичної статистики: перевірки гіпотези про закон розподілу генеральної сукупності за певною 
реалізацію вибірки. Припущення про вид закону генеральної сукупності варто висувати тільки після первинної обробки статистичних даних. 
Параметри розподілу, як правило, невідомі, тому їх замінюємо на найкращі точкові оцінки. Очевидно, що між теоретичними та емпіричними 
розподілами існують розходження. Важливо зрозуміти, чи пояснюються ці розходження тільки випадковими обставинами (наприклад, обмеженість 
кількості спостережень), чи суттєвими (наприклад, гіпотетичний закон підібрано невдало).

\subsection{Критерій Пірсона та алгоритм його використання}
Висунемо гіпотезу $H_0: \xi \text{ \emph{має функцію розподілу} } F(x, \theta_1, \theta_2, ..., \theta_m)$.
Згідно з цією гіпотезою $\xi$ може приймати значення з множини $X$, яку розіб'ємо на $r$ підмножин $X_i$, що попарно не перетинаються:
$X = \bigcup\limits_{i=1}^r X_i$ (рекомендації щодо вибору цих множин та їх кількості розглянемо пізніше). 
В припущенні, що гіпотеза $H_0$ справджується, можемо обчислити ймовірності
$p_i = \P\left\{\xi \in X_i / H_0 \right\}$, причому $\sum\limits_{i=1}^r p_i = 1$. Якщо $H_0$ справджується, то частості
$\frac{n_i}{n}$, де $n_i = \sum\limits_{k=1}^n 1{\left\{\xi_k \in X_i \right\}}$ (кількість значень, що потрапили в $X_i$),
мають прямувати за ймовірністю до $p_i$. Критерій Пірсона з'ясовує, чи можна вважати розходження між теоретичними ймовірностями $p_i$
та практичними значеннями $\frac{n_i}{n}$ випадковими, а не систематичними. 

Розглядається статистика 
\begin{gather}
    \eta = \sum_{i=1}^r \frac{n}{p_i}\left(\frac{n_i}{n} - p_i \right)^2 = \sum_{i=1}^r \frac{\left(n_i - np_i\right)^2}{np_i}
\end{gather}

За \emph{теоремою Пірсона}, доведення якої буде наведено далі, якщо складна гіпотеза $H_0$ про закон розподілу генеральної сукупності 
справджується, то статистика $\eta$ прямує за розподілом до розподілу $\chi^2_{r-s-1}$, де $s$ --- кількість невідомих параметрів гіпотетичного закону розподілу,
які оцінюємо, а $r$ --- кількість множин, за якими рахувалися теоретичні ймовірності $p_i$.

\textbf{Алгоритм використання критерію Пірсона.}
\begin{enumerate}
    \item Після висування гіпотези про закон розподілу $\xi$ розбиваємо множину $X$ можливих значень $\xi$ на $r$ підмножин, що попарно не перетинаються.
    \item Згідно висунутій гіпотезі обчислюємо $p_i = \P\left\{\xi \in X_i / H_0 \right\}$, перевіряємо $\sum\limits_{i=1}^r p_i = 1$.
    Якщо $r\geq 20$, то потрібно, щоб виконувалося $n p_i \geq 5, i = 1, ..., r$, а якщо $r<20$ --- $n p_i \geq 10, i = 1, ... r$. Це пов'язано з тим, що
    заміна розподілу $\eta$ на $\chi^2_{r-s-1}$ є наближеною. Якщо ці умови не виконуються, то сусідні підмножини об'єднують, зменшуючи $r$ та збільшуючи відповідні $p_i$.
    \item Обчислюємо $\eta_{\text{зн.}}$ (з, можливо, новим значенням $r$) та порівнюємо з \emph{критичним значенням} $t_{\text{кр.}}$, яке для заданого рівня значущості $\alpha$
    шукається як значення $t_{\alpha, r-s-1}$ з таблиці на ст. \pageref{tabel:chi_2}. Формально, $t_{\text{кр.}}$ --- це квантиль рівня $1-\alpha$ для розподілу $\chi^2_{r-s-1}$.
    \item Критична область в критерії Пірсона є \emph{правосторонньою}: якщо $\eta_{\text{зн.}} < t_{\text{кр.}}$, то дані не суперечать висунутій гіпотезі $H_0$, а якщо
    $\eta_{\text{зн.}} \geq t_{\text{кр.}}$, то суперечать на рівні значущості $\alpha$. Це означає, що відхилення частостей від теоретичних ймовірностей $p_i$ не можна вважати випадковим.
\end{enumerate}

Наведемо декілька прикладів застосування критерію Пірсона.
\begin{example}
    \begin{enumerate}
        \item Спостереження ГС наведені в інтервальному варіаційному ряді. Перевірити гіпотезу про нормальний розподіл ГС
        на рівні значущості $\alpha = 0.05$.
        \begin{center}
            \begin{tabular}{|c|c|c|c|c|}
                \hline
                інтервал & $[-4; 0)$ & $[0; 2)$ & $[2; 4)$ & $[4; 6]$ \\
                \hline
                $n_i$ & $20$ & $40$ & $30$ & $10$ \\
                \hline
            \end{tabular}
        \end{center}
            За реалізацією вибірки знайдемо значення найкращих 
            оцінок параметрів гауссівського розподілу: вибіркового середнього $\overline{x} = 1.4$ та виправленої вибіркової
            дисперсії $D^{**}_{\text{зн.}} = 4.48$. Висунемо гіпотезу $H_0:\xi \sim \mathrm{N}(1.4, 4.48)$. Оскільки $\xi$ за цим припущенням може приймати довільні дійсні значення, то розіб'ємо
            дійсну вісь на інтервали $(-\infty; -4)$, $[-4; 0)$, $[0; 2)$, $[2; 4)$, $[2; 4)$ та $(6; +\infty)$, для яких обчислимо теоретичні ймовірності $p_i$:
            \begin{center}
                \begin{tabular}{|c|c|c|c|c|c|c|}
                    \hline
                    $X_i$ & $(-\infty; -4)$ & $[-4; 0)$ & $[0; 2)$ & $[2; 4)$ & $[4; 6]$ & $(6; +\infty)$\\
                    \hline
                    $p_i$ & $0.005$ & $0.249$ & $0.3575$ & $0.2787$ & $0.0948$ & $0.015$ \\
                    \hline
                    $n p_i$ & $0.5$ & $24.9$ & $35.75$ & $27.87$ & $9.48$ & $1.5$ \\
                    \hline
                \end{tabular}
            \end{center}
            Умова $np_i \geq 10$ не виконується для $(-\infty; -4)$ та $(6; +\infty)$, тому об'єднаємо їх з сусідніми інтервалами:
            \begin{center}
                \begin{tabular}{|c|c|c|c|c|}
                    \hline
                    $X_i$ & $(-\infty; 0)$ & $[0; 2)$ & $[2; 4)$ & $[4; +\infty)$ \\
                    \hline
                    $p_i$ & $0.254$ & $0.3575$ & $0.2787$ & $0.1098$ \\
                    \hline
                    $n p_i$ & $25.4$ & $35.75$ & $27.87$ & $10.98$ \\
                    \hline
                    $n_i$ & $20$ & $40$ & $30$ & $10$ \\
                    \hline
                    $n_i - np_i$ & $-5.4$ & $4.25$ & $2.13$ & $-0.98$ \\
                    \hline
                \end{tabular}
            \end{center}
            Тепер можемо обчислити $\eta_{\text{зн.}} \approx 1.904$. За таблицею розподілу $\chi^2$ знаходимо $t_{\text{кр.}}$ для $4-2-1 = 1$ степені свободи,
            воно рівне $3.84$. Отже, $\eta_{\text{зн.}} < t_{\text{кр.}}$, тому на рівні значущості $0.05$ дані не суперечать гіпотезі про розподіл ГС $\mathrm{N}(1.4, 4.48)$.
            \item Серед 2020 дводітних сімей 527 мають двох хлопчиків, 476 --- двох дівчаток, а у решти 1017 сімей діти різної статі.
            Чи можна на рівні значущості $0.05$ вважати кількість хлопчиків у сім'ї, яка має двох дітей, є 
            біноміально розподіленою випадковою величиною?
            
            Розглянемо випадкову величину $\xi$, яка для кожної дводітної сім'ї, набуває значення $i = 0, 1, 2$, якщо в сім'ї $i$ хлопчиків,
            та висунемо гіпотезу $H_0: \xi \sim \mathrm{Bin}(2, p)$. Параметр $p$ розподілу невідомий, але його точковою оцінкою є
            $p^* = \frac{1}{2} \overline{\xi}$, $p^*_{\text{зн.}} = \frac{0\cdot 476 + 1\cdot 1017 + 2\cdot 527}{2\cdot 2020} \approx 0.513$.
            Вибірковий простір складається з трьох елементів, $X = \left\{0, 1, 2 \right\}$, тому розділимо його на три множини:
            \begin{center}
                \begin{tabular}{|c|c|c|c|}
                    \hline
                    $X_i$ & $\{0\}$ & $\{1\}$ & $\{2\}$ \\
                    \hline
                    $p_i$ & $0.2371$ & $0.4997$ & $0.2632$ \\
                    \hline
                    $n p_i$ & $478.942$ & $1009.394$ & $531.664$ \\
                    \hline
                    $n_i$ & $476$ & $1017$ & $527$ \\
                    \hline
                    $n_i - np_i$ & $-2.942$ & $7.606$ & $-4.664$ \\
                    \hline
                \end{tabular}
            \end{center}
            Обчислимо $\eta_{\text{зн.}} = \frac{(-2.942)^2}{478.942} + \frac{7.606^2}{1009.394} + \frac{(-4.664)^2}{531.664} \approx 0.116$.
            $r = 3$, $s=1$, за таблицею знаходимо $t_{\text{кр.}} = 3.84$.
            Отже, $\eta_{\text{зн.}} < t_{\text{кр.}}$, тому на рівні значущості $0.05$ дані не суперечать гіпотезі про розподіл $\xi \sim \mathrm{Bin}(2, 0.513)$.
    \end{enumerate}
\end{example}

Окремим, більш простішим, випадком застосування критерію $\chi^2$ є його \emph{застосування до схеми Бернуллі}, тобто про перевірці гіпотези про 
ймовірність появи деякої події в послідовності незалежних випробувань. Нехай у послідовності $n$ незалежних випробувань подія $A$ відбулася $m$
разів. Потрібно на рівні значущості $\alpha$ перевірити гіпотезу, що $\P(A) = p$. Дані 
можна розглядати як вибірку з $n$ значень випадкової величини $\xi$, що є індикатором події $A$ у випробуванні.
В цьому випадку статистика $\eta$ запишеться як 
$\eta = \frac{(m-np)^2}{np} + \frac{(n-m-nq)^2}{nq}$, де $q = 1-p$, $m$ --- кількість появ події $A$ у $n$ випробуваннях.
Можна дещо перетворити цю статистику:
$$
\eta = \frac{(m-np)^2}{np} + \frac{(n(1-q) - m)^2}{nq} = \frac{(m - np)^2}{n}\left(\frac{1}{p} + \frac{1}{q}\right) = 
\frac{(m - np)^2}{n p q} = \left( \frac{m - np}{\sqrt{npq}}\right)^2
$$
Згідно ЦГТ $\frac{m - np}{\sqrt{npq}}$ прямує за розподілом до $\mathrm{N}(0, 1)$, тому $\eta$ прямує в такому ж сенсі до $\chi^2_1$.
Далі перевірка основної гіпотези проводиться аналогічно загальному випадку.
\begin{example}
    При $n=4040$ підкиданнях монети <<герб>> випав $m=2048$ разів. Чи узгоджується на рівні значущості $0.1$ з цими даними гіпотеза, що монета симетрична?

    $\eta_{\text{зн.}} = \frac{\left(2048 - 4040/2\right)^2}{4040/2} + \frac{\left(1992 - 4040/2\right)^2}{4040/2} \approx 0.776$.
    За таблицею знайдемо $t_{\text{кр.}} = 2.71$, тому на рівні значущості $0.1$ дані не суперечать гіпотезі про ймовірність випадання <<герба>> $\frac{1}{2}$.
\end{example}

\subsection{Доведення теореми Пірсона}
Нагадаємо <<підготовчу роботу>> для застосування критерію Пірсона.
Висуваємо гіпотезу $H_0: \xi \text{ \emph{має функцію розподілу} } F(x, \theta_1, \theta_2, ..., \theta_m)$.
Згідно з цією гіпотезою $\xi$ може приймати значення з множини $X$, яку розіб'ємо на $r$ підмножин $X_i$, що попарно не перетинаються:
$X = \bigcup\limits_{i=1}^r X_i$, та обчислимо ймовірності $p_i = \P\left\{\xi \in X_i / H_0 \right\}, i = 1, ..., r$.

\begin{gather*}
    \eta = \sum_{i=1}^r \frac{n}{p_i}\left(\frac{n_i}{n} - p_i \right)^2 = \sum_{i=1}^r \frac{\left(n_i - np_i\right)^2}{np_i}
\end{gather*}
\begin{theorem*}[теорема Пірсона]
    В позначеннях, введених вище:
    \begin{enumerate}
        \item Якщо \textbf{проста} гіпотеза $H_0$ щодо закону розподілу ГС справджується,
        то статистика $\eta$ прямує за розподілом до розподілу $\chi^2_{r-1}$ (розподіл $\chi^2$ з $r-1$ ступенями вільності) при $n\to\infty$.
        \item Якщо \textbf{складна} гіпотеза $H_0$ щодо закону розподілу ГС справджується,
        то статистика $\eta$ прямує за розподілом до розподілу $\chi^2_{r-s-1}$ (розподіл $\chi^2$ з $r-s-1$ ступенями вільності) при $n\to\infty$. Тут $s$ --- кількість
        невідомих параметрів розподілу, які оцінюється.
    \end{enumerate}    
\end{theorem*}
Перед доведенням цієї теореми нагадаємо \emph{ЦГТ для випадкових векторів}: нехай $\left\{ \vec{\xi}_n\right\}_{n=1}^{\infty}$ --- послідовність незалежних однаково розподілених випадкових векторів, що мають скінченні
математичне сподівання $\vec{a}$ та кореляційну матрицю $K$. Тоді
\begin{gather}\label{th:clt_vect}
    \sqrt{n}\left( \frac{1}{n} \sum\limits_{k=1}^n \vec{\xi}_k - \vec{a}\right) = 
    \frac{1}{\sqrt{n}} \sum\limits_{k=1}^n \left( \vec{\xi_k} - \vec{a}\right) \overset{\mathrm{F}}{\longrightarrow} \vec{\eta} \sim \mathrm{N}\left(\vec{0}, K\right), \; n\to\infty
\end{gather}
Наслідком ЦГТ є ${\Vert \vec{\eta}_n \Vert}^2 \overset{\mathrm{F}}{\longrightarrow} {\Vert \vec{\eta} \Vert}^2, n\to\infty$, де
$\vec{\eta}_n = \frac{1}{\sqrt{n}} \sum\limits_{k=1}^n \left( \vec{\xi_k} - \vec{a}\right)$, $\vec{\eta} \sim \mathrm{N}\left(\vec{0}, K\right)$.

Також зауважимо, що у випадку $r=2$ теорему вже фактично було доведено, коли розглядалося застосування критерію Пірсона до схеми Бернуллі.
\begin{proof}
    З огляду на технічну складність, проведемо доведення лише для випадку простої гіпотези. 
    План доведення: покажемо, що статистика $\eta$ є квадратом норми деякого вектора, що за ЦГТ збігається за розподілом до $r-1$-вимірного
    стандартного гауссівського вектора, квадрат норми якого, в свою чергу, має розподіл $\chi^2_{r-1}$, звідки отримаємо твердження теореми за наслідком ЦГТ.
    
    Нехай $H_0$ справджується, гіпотетичну область значень $\xi$ розбиваємо на $r>2$
    підмножин, що попарно не перетинаються: $X = \bigcup\limits_{i=1}^r X_i$, $p_i = \P\left\{\xi \in X_i / H_0 \right\}, i=1,...,r$.
    $\vec{\xi} = \left(\xi_1, \xi_2, ..., \xi_n\right)$ --- випадкова вибірка, введемо індикатори 
    $I_{X_i}(\xi_j) = \begin{cases}
        1, & \xi_j \in X_i \\
        0, & \xi_j \notin X_i
    \end{cases}$.
    Оскільки розподіл усіх $\xi_j$ такий самий, як у $\xi$, то $\E\left(I_{X_i}(\xi_j)\right) = p_i$,
    $\D\left(I_{X_i}(\xi_j)\right) = p_i (1-p_i) = p_i q_i$.
    
    Введемо $r$-вимірні вектори:
    \begin{gather*}
        \overrightarrow{\mu^{(1)}} = \begin{pmatrix}
            \frac{I_{X_1}(\xi_1) - p_1}{\sqrt{p_1}} \\
            \frac{I_{X_2}(\xi_1) - p_2}{\sqrt{p_2}} \\
            \cdots \\
            \frac{I_{X_r}(\xi_1) - p_r}{\sqrt{p_r}}
        \end{pmatrix}, \;
        \overrightarrow{\mu^{(2)}} = \begin{pmatrix}
            \frac{I_{X_1}(\xi_2) - p_1}{\sqrt{p_1}} \\
            \frac{I_{X_2}(\xi_2) - p_2}{\sqrt{p_2}} \\
            \cdots \\
            \frac{I_{X_r}(\xi_2) - p_r}{\sqrt{p_r}}
        \end{pmatrix}, \;
        ..., \;
        \overrightarrow{\mu^{(n)}} = \begin{pmatrix}
            \frac{I_{X_1}(\xi_n) - p_1}{\sqrt{p_1}} \\
            \frac{I_{X_2}(\xi_n) - p_2}{\sqrt{p_2}} \\
            \cdots \\
            \frac{I_{X_r}(\xi_n) - p_r}{\sqrt{p_r}}
        \end{pmatrix}
    \end{gather*}
    Внаслідок однакового розподілу та незалежності усіх $\xi_j$ всі $\overrightarrow{\mu^{(j)}}$ теж мають однаковий розподіл та незалежні.
    Оскільки $\E\left(I_{X_i}(\xi_j)\right) = p_i$, то $\E \overrightarrow{\mu^{(1)}} = \E \overrightarrow{\mu^{(2)}} = ... = \E \overrightarrow{\mu^{(n)}} = \vec{0}$.
    Знайдемо кореляційну матрицю $\K$ вектора $\overrightarrow{\mu^{(1)}}$:
    \begin{gather*}
        {cov}\left(\frac{I_{X_i}(\xi_1) - p_i}{\sqrt{p_i}}, \frac{I_{X_j}(\xi_1) - p_j}{\sqrt{p_j}}\right) = 
        \frac{1}{\sqrt{p_i p_j}} \E\left(\left(I_{X_i}(\xi_1) - p_i\right)\left(I_{X_j}(\xi_1) - p_j\right)\right) = \\
        = \frac{1}{\sqrt{p_i p_j}} \left(\E\left(I_{X_i}(\xi_1) I_{X_j}(\xi_1)\right) - p_i p_j\right) = 
        \begin{cases}
            \frac{1}{p_i} \D\left(I_{X_i}(\xi_1)\right) = 1 - p_i, & i = j \\
            -\sqrt{p_i p_j}, & i\neq j
        \end{cases}
    \end{gather*}
    Випадок $i\neq j$ тут пояснюється тим, що множини $X_i$ та $X_j$ не перетинаються.
    \begin{gather*}
        \K = \begin{pmatrix}
            1 - p_1 & -\sqrt{p_1 p_2} & \cdots & -\sqrt{p_1 p_r} \\
            -\sqrt{p_2 p_1} & 1 - p_2 & \cdots & -\sqrt{p_2 p_r} \\
            \vdots & \vdots & \ddots & \vdots \\
            -\sqrt{p_r p_1} & -\sqrt{p_r p_2} & \cdots & 1 - p_r
        \end{pmatrix} = \mathbb{I}_r - 
        \begin{pmatrix}
            -\sqrt{p_1} \\
            -\sqrt{p_2} \\
            \vdots \\
            -\sqrt{p_r}
        \end{pmatrix} 
        \begin{pmatrix}
            -\sqrt{p_1} & -\sqrt{p_2} & \cdots & -\sqrt{p_r}
        \end{pmatrix}
    \end{gather*}
    Тут $\mathbb{I}_r$ --- одична матриця розмірності $r\times r$.
    Виявляється, що координати $\overrightarrow{\mu^{(1)}}$ лінійно залежні:
    \begin{gather*}
        \sum_{i=1}^r \sqrt{p_i} \overrightarrow{\mu^{(1)}_i} = 
        \sum_{i=1}^r \left(I_{X_i}(\xi_1) - p_i\right) = 
        \sum_{i=1}^r I_{X_i}(\xi_1) - 1 = 0 \text{ з ймовірністю } 1
    \end{gather*}
    Це означає, що матриця $\K$ вироджена і має ранг $n-1$: одночасно більше ніж одну координату виразити через інші неможливо,
    оскільки $X_1, X_2, ..., X_r$ попарно не перетинаються. 
    
    Отже, можна знайти таку ортогональну матрицю $U$, що
    вектор $\widehat{\mu^{(1)}} = U \overrightarrow{\mu^{(1)}}$ матиме нульову останню координату. З міркувань про лінійну залежність координат вихідного вектора
    бачимо, що такою матрицею може бути та, останній рядок якої дорівнює 
    $\begin{pmatrix}
        \sqrt{p_1} & \sqrt{p_2} & \cdots & \sqrt{p_r}
    \end{pmatrix}$ (його норма, очевидно, рівна $1$), а інші складаються з ортонормованого базису ортогонального доповнення до лінійної оболонки цього рядка.
    Покажемо, що якими б не були перші $r-1$ рядків матриці $U$, кореляційна матриця $\widehat{\K} = U \K U^{T}$ вектора $\widehat{\mu^{(1)}}$ матиме
    вигляд $\begin{pmatrix}
        \mathbb{I}_{r-1} & 0 \\
        0 & 0 
    \end{pmatrix}$.

    $U = \left(u_{ij}\right)_{i,j = 1, ... n}$, її ортогональність означає, що для всіх $m \neq r$, $l \neq m$
    \begin{gather*}
        \sum\limits_{j=1}^r u_{mj} u_{rj} = \sum\limits_{j=1}^r u_{mj} \sqrt{p_j} = 0, \;
        \sum\limits_{j=1}^r u_{mj}^2 = 1, \;  \sum\limits_{j=1}^r u_{mj} u_{lj} = 0
    \end{gather*}
    Позначимо $i,j$-тий елементи матриць $\K$ та $\widehat{\K}$ через $\sigma_{ij}$ та $\widehat{\sigma}_{ij}$, тоді
    \begin{gather*}
        \widehat{\sigma}_{ml} = \sum\limits_{i=1}^r \left(\sum\limits_{j=1}^r u_{mj}\sigma_{ji}\right)u_{li} = 
        \sum\limits_{i=1}^r \left(\left(\sum\limits_{i\neq j}
            -u_{mj} \sqrt{p_i p_j}\right) + u_{mi}(1-p_i)
        \right) u_{li} = \\
        = \sum\limits_{i=1}^r \left(
            \sqrt{p_i}\left(
                -\sum\limits_{j\neq i} u_{mj}\sqrt{p_j} - u_{mi}\sqrt{p_i}
            \right) + u_{mi}
        \right) u_{li} = \\
        = \sum\limits_{i=1}^r u_{li} \cdot
        \begin{cases}
            u_{mi}, & m\neq r \\
            0, & m = r
        \end{cases} = 
        \begin{cases}
            1, & m \neq r \text{ та } m = l \\
            0, & m = r \text{ або } m \neq l
        \end{cases} \Longrightarrow \widehat{\K} = \begin{pmatrix}
            \mathbb{I}_{r-1} & 0 \\
            0 & 0 
        \end{pmatrix}
    \end{gather*}
    Розглянемо $r$-вимірний випадковий вектор 
    $\overrightarrow{\eta^{(n)}} = \frac{1}{\sqrt{n}} \sum\limits_{i=1}^n \overrightarrow{\mu^{(i)}}$. Його перша координата дорівнює
    \begin{gather*}
        \overrightarrow{\eta^{(n)}_1} = \frac{1}{\sqrt{n}}\left(
            \frac{I_{X_1}(\xi_1) - p_1}{\sqrt{p_1}} +
            \frac{I_{X_1}(\xi_2) - p_1}{\sqrt{p_1}} + ...
            + \frac{I_{X_1}(\xi_n) - p_1}{\sqrt{p_1}}
        \right) = \frac{n_1 - n p_1}{\sqrt{n p_1}}
    \end{gather*}
    Аналогічно
    \begin{gather*}
        \overrightarrow{\eta^{(n)}_2} = \frac{n_2 - n p_2}{\sqrt{n p_2}},
        \overrightarrow{\eta^{(n)}_3} = \frac{n_3 - n p_3}{\sqrt{n p_3}}, ..., 
        \overrightarrow{\eta^{(n)}_r} = \frac{n_r - n p_r}{\sqrt{n p_r}}
    \end{gather*}
    Застосуємо до цього вектора матрицю $U$:
    \begin{gather*}
        \widehat{\eta^{(n)}} = U \overrightarrow{\eta^{(n)}} = \frac{1}{\sqrt{n}}
        \sum\limits_{i=1}^n U \overrightarrow{\mu^{(i)}} = 
        \frac{1}{\sqrt{n}} \sum\limits_{i=1}^n \widehat{\mu^{(i)}}
    \end{gather*}
    Оскільки усі $\overrightarrow{\mu^{(i)}}$ незалежні та однаково розподілені, то усі $\widehat{\mu^{(i)}}$ --- теж, причому
    математичні сподівання $\E \widehat{\mu^{(i)}} = U \E \overrightarrow{\mu^{(i)}} = \vec{0}$,
    а кореляційні матриці дорівнюють $\widehat{K} = \begin{pmatrix}
        \mathbb{I}_{r-1} & 0 \\
        0 & 0 
    \end{pmatrix}$.

    Отже, за ЦГТ для випадкових векторів (\ref{th:clt_vect})
    $\widehat{\eta^{(n)}} \overset{\mathrm{F}}{\longrightarrow} \vec{\eta} \sim \mathrm{N}\left(\vec{0}, \widehat{\K}\right), \; n\to\infty$, 
    а за наслідком з неї
    ${\Big\Vert \widehat{\eta^{(n)}}\Big\Vert}^2 \overset{\mathrm{F}}{\longrightarrow} {\Vert \vec{\eta} \Vert}^2 \sim \chi^2_{r-1}$.
    Залишилося зауважити, що ортогональне перетворення зберігає норму, тому
    $
        {\Big\Vert \widehat{\eta^{(n)}}\Big\Vert}^2 = {\Big\Vert \overrightarrow{\eta^{(n)}}\Big\Vert}^2 = 
        \sum\limits_{i=1}^r \frac{\left(n_i - n p_i\right)^2}{n p_i}
    $ збігається за розподілом до $\chi^2_{r-1}$, що і треба було довести.
\end{proof}

\begin{remark}
    Можна навести нестроге пояснення того, що у випадку складної гіпотези граничним розподілом статистики $\eta$
    є $\chi^2_{r-s-1}$, а не $\chi^2_{r-1}$. Кількість ступенів свободи у випадку простої гіпотези дорівнює $r-1$
    через те, що одну з координат кожного вектора $\overrightarrow{\mu^{(j)}}$ можна було лінійно виразити через інші, але дві чи більше
    координат одночасно так виразити вже неможливо --- це, власне, пояснює термін <<степені свободи>> в цьому випадку: значення однієї фіксованої координати
    цих векторів <<автоматично>> визначається, коли відомі значення інших $r-1$. У випадку складної гіпотези, коли $s$ невідомих параметрів розподілу оцінюються,
    додається ще $s$ подібних обмежень, що зменшує кількість ступенів свободи у граничного розподілу. Дещо подібне відбувалося при побудові довірчих інтервалів
    для дисперсії гауссівської ГС (ст. \pageref{normal_variance_conf_interv}): там використання вибіркового середнього замість точного значення математичного сподівання
    теж зменшувало на $1$ кількість ступенів свободи у розподілі вибіркової дисперсії.
\end{remark}

\subsection{Критерій незалежності \texorpdfstring{$\chi^2$}{x2}}
Наведемо без доведення ще один напрям застосування розподілу $\chi^2$ для перевірки статистичних гіпотез.
Нехай є $n$ спостережень двох випадкових величин $\xi$ та $\eta$, що позначимо відповідно
$\left(x_1, x_2, ..., x_n\right)$ та $\left(y_1, y_2 ,... ,y_n\right)$. Перевірятимемо гіпотезу
$H_0 : \xi \text{\emph{ та }} \eta \text{\emph{ незалежні}}$. 
Як побачимо далі, розподіли $\xi$ та $\eta$ знати непотрібно, тому будемо вважати, що вони приймають скінченну кількість значень
$\left\{X_1, X_2, ... , X_l\right\}$ та $\left\{Y_1, Y_2, ... , Y_k\right\}$ --- інакше спостереження можна згрупувати в деякі інтервали.

Якщо гіпотеза $H_0$ справджується, то $\P\left\{\xi = X_i, \eta = Y_j \right\} = \P\left\{\xi = X_i\right\} \cdot \P\left\{\eta = Y_j\right\}$.
Позначимо $\nu_{ij}$ кількість таких спостережень обох величин, що одночасно спостереження $\xi$ дорівнює $X_i$ і
спостереження $\eta$ дорівнює $Y_j$, $\nu_i^{\xi}$ --- кількість спостережень $\xi$, рівних $X_i$, $\nu_j^{\eta}$ -- кількість спостережень $\eta$, рівних $Y_j$.
Оскільки за ЗВЧ $\frac{\nu_i^\xi}{n} \overset{\mathrm{P}}{\longrightarrow} \P\left\{\xi = X_i\right\}$, 
$\frac{\nu_j^\eta}{n} \overset{\mathrm{P}}{\longrightarrow} \P\left\{\eta = Y_j\right\}$ та 
$\frac{\nu_{ij}}{n} \overset{\mathrm{P}}{\longrightarrow} \P\left\{\xi = X_i, \eta = Y_j \right\}$ при $n\to\infty$,
то вже можна здогадатися, що критерій незалежності базуватиметься на різниці між
$\frac{\nu_{ij}}{n}$ та $\frac{\nu_i^\xi}{n} \cdot \frac{\nu_j^\eta}{n}$. Має місце теорема:
\begin{theorem*}
    Якщо наведена гіпотеза $H_0$ справджується, то статистика
    \begin{gather}
        \zeta = \sum\limits_{i=1}^{l} \sum\limits_{j=1}^{k} \frac{\left(\nu_{ij} - \nu_i^\xi\cdot \nu_j^\eta / n\right)^2}{\nu_i^\xi\cdot \nu_j^\eta / n}
    \end{gather}
    прямує за розподілом до розподілу $\chi^2_{(l-1)(k-1)}$ при $n\to\infty$.
\end{theorem*}

Усі спостереження вносять до так званої \emph{таблиці спряженості}, застосування якої розглянемо на прикладі. Прийняття чи відхилення основної гіпотези
відбувається на основі порівняння значення статистики критерію з критичним значенням, як і випадку звичайного критерію $\chi^2$.

\begin{example}
    Проведено 300 спостережень одночасно над випадковими величинами $\xi$ та $\eta$, які набувають значень $1, 2$ та $1, 2, 3$ відповідно.
    \begin{center}
        \begin{tabular}{|c|c|c|c|c|}
            \hline
            \diagbox{$\xi$}{$\eta$} & 1 & 2 & 3 & $\nu_i^{\xi}$ \\
            \hline
            1 & 32 & 68 & 50 & 150 \\
            \hline
            2 & 40 & 70 & 40 & 150\\
            \hline
            $\nu_j^{\eta}$ & 72 & 138 & 90 & 300\\
            \hline
        \end{tabular}
    \end{center}
    Перевірити за критерієм $\chi^2$ гіпотезу про незалежність $\xi$ та $\eta$ на рівні значущості $0.01$.

    Спочатку знайдемо величини $m_{ij} = \nu_i^\xi\cdot \nu_j^\eta / n$:
    \begin{gather*}
        m_{1 1} = m_{2 1} = \frac{150 \cdot 72}{300} = 36, \; m_{1 2} = m_{2 2} = \frac{150 \cdot 138}{300} = 69, \;
        m_{1 3} = m_{2 3} = \frac{150 \cdot 90}{300} = 45
    \end{gather*}
    Обчислимо квадрати відхилень $\left(\nu_{ij} - m_{ij}\right)^2$:
    \begin{gather*}
        m_{1 1} = (32 - 36)^2 = 16, \; m_{1 2} = (68 - 69)^2 = 1, \; m_{1 3} = (50 - 45)^2 = 25 \\
        m_{2 1} = (40 - 36)^2 = 16, \; m_{2 2} = (70 - 69)^2 = 1, \; m_{2 3} = (40 - 45)^2 = 25
    \end{gather*}
    Значення статистика критерію:
    \begin{gather*}
        \zeta_{\text{зн.}} = 2\cdot \left(\frac{16}{36} + \frac{1}{69} + \frac{25}{45}\right) \approx 2.029
    \end{gather*}
    Кількість ступенів вільності дорівнює $2$, тому за таблицею знаходимо $t_{\text{кр.}} = 9.21$. Отже,
    $\zeta_{\text{зн.}} < t_{\text{кр.}}$ і на рівні значущості $0.01$ дані не суперечать гіпотезі про незалежність $\xi$ та $\eta$.
\end{example}

\subsection{Критерій згоди Колмогорова}
Розглянемо критерій згоди, що використовується для перевірки \emph{простих} гіпотез щодо \emph{неперервного} розподілу ГС: наприклад,
$H_0 : \xi \sim \mathrm{N}(0, 1)$ чи $H_0 : \xi \sim \mathrm{U}\left<-1, 1\right>$.
Вводиться статистика $D_n = \underset{x \in \mathbb{R}}{\sup} \left| F_n^*(x) - F_\xi(x)\right|$, де $F_n^*(x)$ --- емпірична функція розподілу.
На тому, що $F_n^*(x)$ з ймовірністю 1 прямує до $F_\xi(x)$ при $n\to\infty$, базується наступна теорема:

\begin{theorem*}[теорема Колмогорова]
    Нехай $\vec{\xi} = \left(\xi_1, \xi_2, ..., \xi_n\right)$ --- вибірка з неперервної ГС $\xi$ з функцією розподілу $F_{\xi}(x)$, 
    $F_n^*(x)$ --- побудована за вибіркою емпірична функція розподілу. Тоді
    \begin{gather*}
        \underset{n\to\infty}{\lim} \P\left\{ \sqrt{n} \cdot \underset{x \in \mathbb{R}}{\sup} \left| F_n^*(x) - F_\xi(x)\right| < \lambda\right\} = 
        K(\lambda), \;\lambda > 0, 
        \text{ де } K(\lambda) = \sum_{j=-\infty}^{+\infty} (-1)^j e^{-2 j^2 \lambda^2}
    \end{gather*}
\end{theorem*}
Доводити цю теорему не будемо. Зауважимо, що особливістю статистики $D_n$ є те, що її закон розподілу однаковий для всіх неперервних ГС і
залежить лише від обсягу вибірки: якщо зробити у ній заміну $x = F_{\xi}^{-1}(y)$, отримаємо
$D_n = \underset{y \in [0; 1]}{\sup} \left| F_n^*(F_{\xi}^{-1}(y)) - y\right|$. Оскільки випадкові величини $\eta_k = F_{\xi}(\xi_k)$ утворюють
вибірку з розподілу $\mathrm{U}\left<0, 1\right>$, то
\begin{gather*}
    F_n^*(F_{\xi}^{-1}(y)) = \frac{1}{n} \sum_{k=1}^n 1{\left\{ \xi_k < F_{\xi}^{-1}(y)\right\}} = 
    \frac{1}{n} \sum_{k=1}^n 1{\left\{ \eta_k < y\right\}}, \;
    1{\left\{ \eta_k < y\right\}} \sim \mathrm{Bin}(1, y) \text{ при } y \in [0; 1]
\end{gather*}
$\eta_k$ незалежні, тому $\sum\limits_{k=1}^n 1{\left\{ \eta_k < y\right\}} \sim \mathrm{Bin}(n, y)$. Подальші перетворення цієї випадкової величини
мають невипадковий характер, тому, дійсно, розподіл $D_n$ не залежить від розподілу $\xi$.
Таким чином, якщо $H_0$ справджується, то значення $D_n$ буде досить малим. Значення $K(\lambda)$ знаходять з відповідної таблиці.

\textbf{Алгоритм використання критерію Колмогорова.}
\begin{enumerate}
    \item Висуваємо просту гіпотезу $H_0$ про неперервний закон розподілу ГС $\xi$.
    \item За реалізацією вибірки будуємо емпіричну функцію розподілу $F_n^*(x)$.
    \item Знаходимо значення статистики $D_n$ та $\lambda = \sqrt{n} \cdot D_n$.
    \item За заданим рівнем значущості $\alpha$ знаходимо таке $\lambda_\alpha$, що
    $\P\left\{\sqrt{n}\cdot D_n \leq \lambda_\alpha / H_0\right\} = 1 - K(\lambda_\alpha) = \alpha$. 
    \item Якщо $\lambda_{\text{зн.}}$ буде більшим за критичне значення $\lambda_\alpha$, то на рівні значущості $\alpha$
    дослідні дані суперечать гіпотезі $H_0$. Інакше вважають, що гіпотезу $H_0$ можна прийняти.
\end{enumerate}

Наведемо таблицю значень $\lambda_\alpha$ для деяких $\alpha$:
\begin{center}
    \begin{tabular}{|c|c|c|c|c|}
 \hline 
$\alpha$ & 0.1 & 0.05 & 0.025 & 0.01 \\
 \hline 
$\lambda_{\alpha}$ & 1.224 & 1.358 & 1.480 & 1.628 \\
 \hline 
\end{tabular}
\end{center}

\begin{example}
    Нехай реалізацію вибірки подано у вигляді інтервального 
    варіаційного ряду:
    \begin{center}
        \begin{tabular}{*{11}{|c}|}
            \hline
            інтервал & $[0,1)$ & $[1,2)$ & $[2,3)$ & $[3,4)$ & $[4,5)$ & $[5,6)$ & $[6,7)$ & $[7,8)$ & $[8,9)$ & $[9,10]$ \\
            \hline
            $n_i$ & 35 & 16 & 15 & 17 & 17 & 19 & 11 & 16 & 30 & 24 \\
            \hline
        \end{tabular}
    \end{center}
    На рівні значущості $\alpha = 0.05$ перевірити гіпотезу $H_0: \xi \sim \mathrm{U}\left<0, 10\right>$.

    За інтервальним варіаційним рядом можемо обчислити значення емпіричної та теоретичної функції розподілу в правих кінцях інтервалів,
    а також різниці між ними:
    \begin{center}
        \begin{tabular}{*{11}{|c}|}
            \hline
            $x$ & 1 & 2 & 3 & 4 & 5 & 6 & 7 & 8 & 9 & 10 \\
            \hline
            $F_n^*(x)$ & $0.175$ & $0.255$ & $0.33$ & $0.415$ & $0.5$ & $0.595$ & $0.65$ & $0.73$ & $0.88$ & $1$ \\
            \hline
            $F_\xi(x)$ & $0.1$ & $0.2$ & $0.3$ & $0.4$ & $0.5$ & $0.6$ & $0.7$ & $0.8$ & $0.9$ & $1$ \\
            \hline
            $F_n^*(x) - F_\xi(x)$ & $0.07$5 & $0.055$ & $0.03$ & $0.015$ & $0$ & $-0.004$ & $-0.05$ & $-0.07$ & $-0.02$ & $0$ \\
            \hline
        \end{tabular}
    \end{center}
    $\left(D_n\right)_{\text{зн.}} = 0.075$, тому $\lambda_{\text{зн.}} = \sqrt{200} \cdot 0.075 \approx 1.06$.
    За таблицею $\lambda_{0.05} = 1.358$. Отже, на рівні значущості $0.05$ дані не суперечать гіпотезі про рівномірний
    розподіл $\xi$ на інтервалі $\left<0, 10\right>$.
\end{example}
    \chapter{Елементи регресійного аналізу}
    \chapter*{Таблиці значень деяких функцій}
    \markboth{Таблиці значень деяких функцій}{Таблиці значень деяких функцій}
    \addcontentsline{toc}{chapter}{Таблиці значень деяких функцій}
    \section*{Таблиця значень функції Лапласа}
        \addcontentsline{toc}{section}{Функція Лапласа}
        % !TEX root = ../main.tex

\begin{center}
    \begin{tabular}{c c}
        $
            \Phi(x) = \frac{1}{\sqrt{2\pi}} 
            \int\limits_{0}^{x} e^{-\frac{t^2}{2}} dt
        $
        &
        \begin{tikzpicture}[baseline={(current bounding box.center)}, yscale=3, 
            scale = 1]
            \fill [lightgray, domain=0:1, smooth, variable = \x] plot ({\x}, 
            {
                (0.3989422804) * e^(- (\x * \x / 2))
            }) -- (1, 0) -- (0, 0) -- (0, 0.3989422804);
            \draw [->] (-3, 0) -- (3, 0);
            \draw [->] (0, -0.2) -- (0, 0.7);
            \draw [domain=-3:3, smooth, variable = \x, ultra thick] plot ({\x}, 
            {
                (0.3989422804) * e^(- (\x * \x / 2))
            });
            \node [below] at (1, 0) {$x$};
            \draw [dashed] (1, 0) -- (1, 0.25);
            \draw [->, thick] (1.5, 0.4) -- (0.7, 0.2);
            \node [right] at (1.5, 0.4) {площа дорівнює $\Phi(x)$};
            \draw [->] (-0.8, 0.5) -- (-0.495, 0.355);
            \node [left] at (-0.8, 0.5) {$\frac{1}{\sqrt{2\pi}}e^{-\frac{t^2}{2}}$};
        \end{tikzpicture}
    \end{tabular}
\end{center}
В таблиці наведено лише дробову частину усіх значень, оскільки усі цілі частини рівні 0:
наприклад, .48745 треба читати як 0.48745.
\begin{center}
    \small
    \noindent \begin{tabular}{|c|c|c|c|c|c|c|c|c|c|c|}
\hline
$\pmb{x}$ & \textbf{ 0.00 } & \textbf{ 0.01 } & \textbf{ 0.02 } & \textbf{ 0.03 } & \textbf{ 0.04 } & \textbf{ 0.05 } & \textbf{ 0.06 } & \textbf{ 0.07 } & \textbf{ 0.08 } & \textbf{ 0.09 } \\ 
 \hline 
\textbf{ 0.00 } & .00000 & .00399 & .00798 & .01197 & .01595 & .01994 & .02392 & .02790 & .03188 & .03586\\ 
\hline
\textbf{ 0.10 } & .03983 & .04380 & .04776 & .05172 & .05567 & .05962 & .06356 & .06749 & .07142 & .07535\\ 
\hline
\textbf{ 0.20 } & .07926 & .08317 & .08706 & .09095 & .09483 & .09871 & .10257 & .10642 & .11026 & .11409\\ 
\hline
\textbf{ 0.30 } & .11791 & .12172 & .12552 & .12930 & .13307 & .13683 & .14058 & .14431 & .14803 & .15173\\ 
\hline
\textbf{ 0.40 } & .15542 & .15910 & .16276 & .16640 & .17003 & .17364 & .17724 & .18082 & .18439 & .18793\\ 
\hline
\textbf{ 0.50 } & .19146 & .19497 & .19847 & .20194 & .20540 & .20884 & .21226 & .21566 & .21904 & .22240\\ 
\hline
\textbf{ 0.60 } & .22575 & .22907 & .23237 & .23565 & .23891 & .24215 & .24537 & .24857 & .25175 & .25490\\ 
\hline
\textbf{ 0.70 } & .25804 & .26115 & .26424 & .26730 & .27035 & .27337 & .27637 & .27935 & .28230 & .28524\\ 
\hline
\textbf{ 0.80 } & .28814 & .29103 & .29389 & .29673 & .29955 & .30234 & .30511 & .30785 & .31057 & .31327\\ 
\hline
\textbf{ 0.90 } & .31594 & .31859 & .32121 & .32381 & .32639 & .32894 & .33147 & .33398 & .33646 & .33891\\ 
\hline
\textbf{ 1.00 } & .34134 & .34375 & .34614 & .34849 & .35083 & .35314 & .35543 & .35769 & .35993 & .36214\\ 
\hline
\textbf{ 1.10 } & .36433 & .36650 & .36864 & .37076 & .37286 & .37493 & .37698 & .37900 & .38100 & .38298\\ 
\hline
\textbf{ 1.20 } & .38493 & .38686 & .38877 & .39065 & .39251 & .39435 & .39617 & .39796 & .39973 & .40147\\ 
\hline
\textbf{ 1.30 } & .40320 & .40490 & .40658 & .40824 & .40988 & .41149 & .41309 & .41466 & .41621 & .41774\\ 
\hline
\textbf{ 1.40 } & .41924 & .42073 & .42220 & .42364 & .42507 & .42647 & .42785 & .42922 & .43056 & .43189\\ 
\hline
\textbf{ 1.50 } & .43319 & .43448 & .43574 & .43699 & .43822 & .43943 & .44062 & .44179 & .44295 & .44408\\ 
\hline
\textbf{ 1.60 } & .44520 & .44630 & .44738 & .44845 & .44950 & .45053 & .45154 & .45254 & .45352 & .45449\\ 
\hline
\textbf{ 1.70 } & .45543 & .45637 & .45728 & .45818 & .45907 & .45994 & .46080 & .46164 & .46246 & .46327\\ 
\hline
\textbf{ 1.80 } & .46407 & .46485 & .46562 & .46638 & .46712 & .46784 & .46856 & .46926 & .46995 & .47062\\ 
\hline
\textbf{ 1.90 } & .47128 & .47193 & .47257 & .47320 & .47381 & .47441 & .47500 & .47558 & .47615 & .47670\\ 
\hline
\textbf{ 2.00 } & .47725 & .47778 & .47831 & .47882 & .47932 & .47982 & .48030 & .48077 & .48124 & .48169\\ 
\hline
\textbf{ 2.10 } & .48214 & .48257 & .48300 & .48341 & .48382 & .48422 & .48461 & .48500 & .48537 & .48574\\ 
\hline
\textbf{ 2.20 } & .48610 & .48645 & .48679 & .48713 & .48745 & .48778 & .48809 & .48840 & .48870 & .48899\\ 
\hline
\textbf{ 2.30 } & .48928 & .48956 & .48983 & .49010 & .49036 & .49061 & .49086 & .49111 & .49134 & .49158\\ 
\hline
\textbf{ 2.40 } & .49180 & .49202 & .49224 & .49245 & .49266 & .49286 & .49305 & .49324 & .49343 & .49361\\ 
\hline
\textbf{ 2.50 } & .49379 & .49396 & .49413 & .49430 & .49446 & .49461 & .49477 & .49492 & .49506 & .49520\\ 
\hline
\textbf{ 2.60 } & .49534 & .49547 & .49560 & .49573 & .49585 & .49598 & .49609 & .49621 & .49632 & .49643\\ 
\hline
\textbf{ 2.70 } & .49653 & .49664 & .49674 & .49683 & .49693 & .49702 & .49711 & .49720 & .49728 & .49736\\ 
\hline
\textbf{ 2.80 } & .49744 & .49752 & .49760 & .49767 & .49774 & .49781 & .49788 & .49795 & .49801 & .49807\\ 
\hline
\textbf{ 2.90 } & .49813 & .49819 & .49825 & .49831 & .49836 & .49841 & .49846 & .49851 & .49856 & .49861\\ 
\hline
\end{tabular} 

\noindent \begin{tabular}{|c|c|c|c|c|c|c|c|c|c|c|}
\hline
$\pmb{x}$ & \textbf{ 0.00 } & \textbf{ 0.01 } & \textbf{ 0.02 } & \textbf{ 0.03 } & \textbf{ 0.04 } & \textbf{ 0.05 } & \textbf{ 0.06 } & \textbf{ 0.07 } & \textbf{ 0.08 } & \textbf{ 0.09 } \\ 
 \hline 
\textbf{ 3.00 } & .49865 & .49869 & .49874 & .49878 & .49882 & .49886 & .49889 & .49893 & .49896 & .49900\\ 
\hline
\textbf{ 3.10 } & .49903 & .49906 & .49910 & .49913 & .49916 & .49918 & .49921 & .49924 & .49926 & .49929\\ 
\hline
\textbf{ 3.20 } & .49931 & .49934 & .49936 & .49938 & .49940 & .49942 & .49944 & .49946 & .49948 & .49950\\ 
\hline
\textbf{ 3.30 } & .49952 & .49953 & .49955 & .49957 & .49958 & .49960 & .49961 & .49962 & .49964 & .49965\\ 
\hline
\textbf{ 3.40 } & .49966 & .49968 & .49969 & .49970 & .49971 & .49972 & .49973 & .49974 & .49975 & .49976\\ 
\hline
\textbf{ 3.50 } & .49977 & .49978 & .49978 & .49979 & .49980 & .49981 & .49981 & .49982 & .49983 & .49983\\ 
\hline
\textbf{ 3.60 } & .49984 & .49985 & .49985 & .49986 & .49986 & .49987 & .49987 & .49988 & .49988 & .49989\\ 
\hline
\textbf{ 3.70 } & .49989 & .49990 & .49990 & .49990 & .49991 & .49991 & .49992 & .49992 & .49992 & .49992\\ 
\hline
\textbf{ 3.80 } & .49993 & .49993 & .49993 & .49994 & .49994 & .49994 & .49994 & .49995 & .49995 & .49995\\ 
\hline
\textbf{ 3.90 } & .49995 & .49995 & .49996 & .49996 & .49996 & .49996 & .49996 & .49996 & .49997 & .49997\\ 
\hline
\textbf{ 4.00 } & .49997 & .49997 & .49997 & .49997 & .49997 & .49997 & .49998 & .49998 & .49998 & .49998\\ 
\hline
\textbf{ 4.10 } & .49998 & .49998 & .49998 & .49998 & .49998 & .49998 & .49998 & .49998 & .49999 & .49999\\ 
\hline
\textbf{ 4.20 } & .49999 & .49999 & .49999 & .49999 & .49999 & .49999 & .49999 & .49999 & .49999 & .49999\\ 
\hline
\textbf{ 4.30 } & .49999 & .49999 & .49999 & .49999 & .49999 & .49999 & .49999 & .49999 & .49999 & .49999\\ 
\hline
\textbf{ 4.40 } & .49999 & .49999 & .50000 & .50000 & .50000 & .50000 & .50000 & .50000 & .50000 & .50000\\ 
\hline
\textbf{ 4.50 } & .50000 & .50000 & .50000 & .50000 & .50000 & .50000 & .50000 & .50000 & .50000 & .50000\\ 
\hline
\textbf{ 4.60 } & .50000 & .50000 & .50000 & .50000 & .50000 & .50000 & .50000 & .50000 & .50000 & .50000\\ 
\hline
\textbf{ 4.70 } & .50000 & .50000 & .50000 & .50000 & .50000 & .50000 & .50000 & .50000 & .50000 & .50000\\ 
\hline
\textbf{ 4.80 } & .50000 & .50000 & .50000 & .50000 & .50000 & .50000 & .50000 & .50000 & .50000 & .50000\\ 
\hline
\textbf{ 4.90 } & .50000 & .50000 & .50000 & .50000 & .50000 & .50000 & .50000 & .50000 & .50000 & .50000\\ 
\hline
\end{tabular}
\end{center}\newpage
    \section*{Деякі квантилі розподілу $\chi^2_{n}$}
        \addcontentsline{toc}{section}{Деякі квантилі розподілу \texorpdfstring{$\chi^2_n$}{x2n}}
        % !TEX root = ../main.tex

Значення функції $\chi^2_{\alpha,n}$ визначаються з рівняння 
$ \int\limits_{\chi^2_{\alpha,n}}^{+\infty} f(x) dx = \alpha$
де $f(x)$ --- щільність розподілу $\chi^2_{n}$ (хі-квадрат з $n$ ступенями вільності).

\begin{center}
    \begin{tikzpicture}[yscale = 14, xscale = 0.5, baseline={(current bounding box.center)}]
        \pgfmathsetmacro{\b}{0.5};
        \pgfmathsetmacro{\c}{3};
        \pgfmathsetmacro{\p}{7};
    
        \fill [lightgray, domain=\p:18.3, smooth, variable = \x] plot ({\x}, 
        {
            \b^(\c)*\x^(\c-1)/factorial(\c-1) * e^(-\x*\b)
        }) -- (18.3, 0) -- (\p, 0) -- (\p, {\b^(\c)*\p^(\c-1)/factorial(\c-1) * e^(-\p*\b)});
        \draw [->] (-2, 0) -- (18.3, 0);
        \draw [->] (0, -0.05) -- (0, 0.15);
        \draw [ultra thick] (-2, 0) -- (0, 0);
        \draw [domain=0:18, smooth, variable = \x, ultra thick] plot ({\x}, {\b^(\c)*\x^(\c-1)/factorial(\c-1) * e^(-\x*\b)});
        \node [below] at (18.2, 0) {$x$};
        \node [left] at (0, 0.15) {$f(x)$};
        \node [below left] at (0, 0) {$0$};
        \node [below] at (\p, 0) {$\chi^2_{\alpha,n}$};
        \draw [dashed] (\p, 0) -- (\p, {\b^(\c)*\p^(\c-1)/factorial(\c-1) * e^(-\p*\b)});
        \draw [->, thick] (\p+3, 0.1) -- (\p+1, 0.03);
        \node [right] at (\p+3, 0.1) {площа дорівнює $\alpha$};
    \end{tikzpicture}
\end{center}
\begin{center}
    \noindent\begin{tabular}{*{9}{|c}|}
    \hline
    \diagbox{$\pmb{n}$}{$\pmb{\alpha}$} & \textbf{0.990} & \textbf{0.975} & \textbf{0.950} & \textbf{0.900} & \textbf{0.100} & \textbf{0.050} & \textbf{0.025} & \textbf{0.010}\\
    \hline 
\textbf{1} & 0.00 &0.00 &0.00 &0.02 &2.71 &3.84 &5.02 &6.63 \\
\hline 
\textbf{2} & 0.02 &0.05 &0.10 &0.21 &4.61 &5.99 &7.38 &9.21 \\
\hline 
\textbf{3} & 0.11 &0.22 &0.35 &0.58 &6.25 &7.81 &9.35 &11.34 \\
\hline 
\textbf{4} & 0.30 &0.48 &0.71 &1.06 &7.78 &9.49 &11.14 &13.28 \\
\hline 
\textbf{5} & 0.55 &0.83 &1.15 &1.61 &9.24 &11.07 &12.83 &15.09 \\
\hline 
\textbf{6} & 0.87 &1.24 &1.64 &2.20 &10.64 &12.59 &14.45 &16.81 \\
\hline 
\textbf{7} & 1.24 &1.69 &2.17 &2.83 &12.02 &14.07 &16.01 &18.48 \\
\hline 
\textbf{8} & 1.65 &2.18 &2.73 &3.49 &13.36 &15.51 &17.53 &20.09 \\
\hline 
\textbf{9} & 2.09 &2.70 &3.33 &4.17 &14.68 &16.92 &19.02 &21.67 \\
\hline 
\textbf{10} & 2.56 &3.25 &3.94 &4.87 &15.99 &18.31 &20.48 &23.21 \\
\hline 
\textbf{11} & 3.05 &3.82 &4.57 &5.58 &17.28 &19.68 &21.92 &24.72 \\
\hline 
\textbf{12} & 3.57 &4.40 &5.23 &6.30 &18.55 &21.03 &23.34 &26.22 \\
\hline 
\textbf{13} & 4.11 &5.01 &5.89 &7.04 &19.81 &22.36 &24.74 &27.69 \\
\hline 
\textbf{14} & 4.66 &5.63 &6.57 &7.79 &21.06 &23.68 &26.12 &29.14 \\
\hline 
\textbf{15} & 5.23 &6.26 &7.26 &8.55 &22.31 &25.00 &27.49 &30.58 \\
\hline 
\textbf{16} & 5.81 &6.91 &7.96 &9.31 &23.54 &26.30 &28.85 &32.00 \\
\hline 
\textbf{17} & 6.41 &7.56 &8.67 &10.09 &24.77 &27.59 &30.19 &33.41 \\
\hline 
\textbf{18} & 7.01 &8.23 &9.39 &10.86 &25.99 &28.87 &31.53 &34.81 \\
\hline 
\textbf{19} & 7.63 &8.91 &10.12 &11.65 &27.20 &30.14 &32.85 &36.19 \\
\hline 
\textbf{20} & 8.26 &9.59 &10.85 &12.44 &28.41 &31.41 &34.17 &37.57 \\
\hline 
\textbf{21} & 8.90 &10.28 &11.59 &13.24 &29.62 &32.67 &35.48 &38.93 \\
\hline 
\textbf{22} & 9.54 &10.98 &12.34 &14.04 &30.81 &33.92 &36.78 &40.29 \\
\hline 
\textbf{23} & 10.20 &11.69 &13.09 &14.85 &32.01 &35.17 &38.08 &41.64 \\
\hline 
\textbf{24} & 10.86 &12.40 &13.85 &15.66 &33.20 &36.42 &39.36 &42.98 \\
\hline 
\textbf{25} & 11.52 &13.12 &14.61 &16.47 &34.38 &37.65 &40.65 &44.31 \\
\hline 
\textbf{26} & 12.20 &13.84 &15.38 &17.29 &35.56 &38.89 &41.92 &45.64 \\
\hline 
\textbf{27} & 12.88 &14.57 &16.15 &18.11 &36.74 &40.11 &43.19 &46.96 \\
\hline 
\textbf{28} & 13.56 &15.31 &16.93 &18.94 &37.92 &41.34 &44.46 &48.28 \\
\hline 
\textbf{29} & 14.26 &16.05 &17.71 &19.77 &39.09 &42.56 &45.72 &49.59 \\
\hline 
\textbf{30} & 14.95 &16.79 &18.49 &20.60 &40.26 &43.77 &46.98 &50.89 \\
\hline 
\textbf{30} & 14.95 &16.79 &18.49 &20.60 &40.26 &43.77 &46.98 &50.89 \\
\hline 
\textbf{40} & 22.16 &24.43 &26.51 &29.05 &51.81 &55.76 &59.34 &63.69 \\
\hline 
\textbf{50} & 29.71 &32.36 &34.76 &37.69 &63.17 &67.50 &71.42 &76.15 \\
\hline 
\textbf{60} & 37.48 &40.48 &43.19 &46.46 &74.40 &79.08 &83.30 &88.38 \\
\hline 
\textbf{70} & 45.44 &48.76 &51.74 &55.33 &85.53 &90.53 &95.02 &100.43 \\
\hline 
\textbf{80} & 53.54 &57.15 &60.39 &64.28 &96.58 &101.88 &106.63 &112.33 \\
\hline 
\textbf{90} & 61.75 &65.65 &69.13 &73.29 &107.57 &113.15 &118.14 &124.12 \\
\hline 
\textbf{100} & 70.06 &74.22 &77.93 &82.36 &118.50 &124.34 &129.56 &135.81 \\
\hline 
\end{tabular}
\end{center}
\newpage
    \section*{Деякі квантилі розподілу $\mathrm{St}_n$}
        \addcontentsline{toc}{section}{Деякі квантилі розподілу \texorpdfstring{$\mathrm{St}_n$}{Stn}}
        % !TEX root = ../main.tex

Значення функції $t_{\alpha,n}$ визначаються з рівняння 
$ \int\limits_{t_{\alpha,n}}^{+\infty} f(x) dx = \alpha$
де $f(x)$ --- щільність розподілу $\mathrm{St}_n$ (Стьюдента з $n$ ступенями вільності).
Це значення є квантилем рівня $q=1-\alpha$.


Якщо для $\xi \sim \mathrm{St}_n$ необхідно знайти таке значення $t$, що
$\P\left\{ |\xi| < t\right\} = \beta$, то, внаслідок симетричності розподілу відносно нуля, можна шукати 
за таблицею значення $t = t_{(1-\beta)/2,n}$.
\begin{center}
    \begin{tikzpicture}[yscale = 6, xscale = 1.3, baseline={(current bounding box.center)}]
        \pgfmathsetmacro{\b}{12}; % n = 4
        \pgfmathsetmacro{\p}{1.5};
    
        \fill [lightgray, domain=\p:5, smooth, variable = \x] plot ({\x}, 
        {
            \b/((4+(\x)^2)^((4+1)/2))
        }) -- (5, 0) -- (\p, 0) -- (\p, {\b/((4+(\p)^2)^((4+1)/2))});
        \draw [->] (-5, 0) -- (5, 0);
        \draw [->] (0, -0.05) -- (0, 0.41);
        \draw [domain=-5:5, smooth, variable = \x, ultra thick] plot ({\x}, {\b/((4+(\x)^2)^((4+1)/2))});
        \node [below] at (5, 0) {$x$};
        \node [left] at (0, 0.41) {$f(x)$};
        \node [below left] at (0, 0) {$0$};
        \node [below] at (\p, 0) {$t_{\alpha,n}$};
        \draw [dashed] (\p, 0) -- (\p, {\b/((4+(\p)^2)^((4+1)/2))});
        \draw [->, thick] (\p+1, 0.2) -- (\p+0.3, 0.03);
        \node [right] at (\p+1, 0.2) {площа дорівнює $\alpha$};
    \end{tikzpicture}
\end{center}
\begin{center}
    \begin{tabular}{c c}
        \noindent\begin{tabular}{*{5}{|c}|}
    \hline
    \diagbox{$\pmb{n}$}{$\pmb{\alpha}$} & \textbf{0.050} & \textbf{0.025} & \textbf{0.010} & \textbf{0.005} \\
    \hline 
\textbf{1} & 6.314 &12.706 &31.821 &63.657 \\
\hline 
\textbf{2} & 2.920 &4.303 &6.965 &9.925 \\
\hline 
\textbf{3} & 2.353 &3.182 &4.541 &5.841 \\
\hline 
\textbf{4} & 2.132 &2.776 &3.747 &4.604 \\
\hline 
\textbf{5} & 2.015 &2.571 &3.365 &4.032 \\
\hline 
\textbf{6} & 1.943 &2.447 &3.143 &3.707 \\
\hline 
\textbf{7} & 1.895 &2.365 &2.998 &3.499 \\
\hline 
\textbf{8} & 1.860 &2.306 &2.896 &3.355 \\
\hline 
\textbf{9} & 1.833 &2.262 &2.821 &3.250 \\
\hline 
\textbf{10} & 1.812 &2.228 &2.764 &3.169 \\
\hline 
\textbf{11} & 1.796 &2.201 &2.718 &3.106 \\
\hline 
\textbf{12} & 1.782 &2.179 &2.681 &3.055 \\
\hline 
\textbf{13} & 1.771 &2.160 &2.650 &3.012 \\
\hline 
\textbf{14} & 1.761 &2.145 &2.624 &2.977 \\
\hline 
\textbf{15} & 1.753 &2.131 &2.602 &2.947 \\
\hline 
\textbf{16} & 1.746 &2.120 &2.583 &2.921 \\
\hline 
\textbf{17} & 1.740 &2.110 &2.567 &2.898 \\
\hline 
\textbf{18} & 1.734 &2.101 &2.552 &2.878 \\
\hline 
\textbf{19} & 1.729 &2.093 &2.539 &2.861 \\
\hline 
\textbf{20} & 1.725 &2.086 &2.528 &2.845 \\
\hline 
\end{tabular} 

 &
        \noindent\begin{tabular}{*{9}{|c}|}
    \hline
    \diagbox{$\pmb{n}$}{$\pmb{\alpha}$} & \textbf{0.050} & \textbf{0.025} & \textbf{0.010} & \textbf{0.005} \\
    \hline 
\textbf{21} & 1.721 &2.080 &2.518 &2.831 \\
\hline 
\textbf{22} & 1.717 &2.074 &2.508 &2.819 \\
\hline 
\textbf{23} & 1.714 &2.069 &2.500 &2.807 \\
\hline 
\textbf{24} & 1.711 &2.064 &2.492 &2.797 \\
\hline 
\textbf{25} & 1.708 &2.060 &2.485 &2.787 \\
\hline 
\textbf{26} & 1.706 &2.056 &2.479 &2.779 \\
\hline 
\textbf{27} & 1.703 &2.052 &2.473 &2.771 \\
\hline 
\textbf{28} & 1.701 &2.048 &2.467 &2.763 \\
\hline 
\textbf{29} & 1.699 &2.045 &2.462 &2.756 \\
\hline 
\textbf{30} & 1.697 &2.042 &2.457 &2.750 \\
\hline 
\textbf{40} & 1.684 &2.021 &2.423 &2.704 \\
\hline 
\textbf{50} & 1.676 &2.009 &2.403 &2.678 \\
\hline 
\textbf{60} & 1.671 &2.000 &2.390 &2.660 \\
\hline 
\textbf{70} & 1.667 &1.994 &2.381 &2.648 \\
\hline 
\textbf{80} & 1.664 &1.990 &2.374 &2.639 \\
\hline 
\textbf{90} & 1.662 &1.987 &2.368 &2.632 \\
\hline 
\textbf{100} & 1.660 &1.984 &2.364 &2.626 \\
\hline 
\textbf{120} & 1.658 &1.980 &2.358 &2.617 \\
\hline 
\textbf{150} & 1.655 &1.976 &2.351 &2.609 \\
\hline 
$\pmb{\infty}$& 1.645 &1.960 &2.326 &2.576 \\
\hline 
\end{tabular}
    \end{tabular}
\end{center}\newpage
    \section*{Квантилі рівня 0.95 розподілу $\mathrm{F}(n_1, n_2)$}
        \addcontentsline{toc}{section}{Квантилі рівня 0.95 розподілу \texorpdfstring{$\mathrm{F}(n_1, n_2)$}{Fn1n2}}
        % !TEX root = ../main.tex
Значення функції $F_{\alpha,n_1,n_2}$ визначаються з рівняння 
$ \int\limits_{F_{\alpha,n_1,n_2}}^{+\infty} f(x) dx = \alpha$
де $f(x)$ --- щільність розподілу $\mathrm{F}(n_1, n_2)$ (Фішера-Снедекора з $n_1$, $n_2$ ступенями вільності).
Це значення є квантилем рівня $q=1-\alpha$.
Нагадаємо, що даний розподіл --- це розподіл частки $\frac{\chi_{n_1}^2/n_1}{\chi_{n_2}^2/n_2}$ з незалежними чисельником і знаменником.
\begin{center}
    \begin{tikzpicture}[yscale = 6, xscale = 3, baseline={(current bounding box.center)}]
        \pgfmathsetmacro{\b}{64}; % n1 = 4, n2 = 2
        \pgfmathsetmacro{\p}{1};
        
        \fill [lightgray, domain=\p:4, smooth, variable = \x] plot ({\x}, 
        {
            \b*((\x)^(4/2 - 1))/(4*\x + 2)^((4+2)/2)
        }) -- (4, 0) -- (\p, 0) -- (\p, {\b*((\p)^(4/2 - 1))/(4*\p + 2)^((4+2)/2)});
        \draw [->] (-0.5, 0) -- (4, 0);
        \draw [->] (0, -0.05) -- (0, 0.7);
        \draw [domain=0:4, smooth, variable = \x, ultra thick, samples = 400] plot ({\x}, {\b*((\x)^(4/2 - 1))/(4*\x + 2)^((4+2)/2)});
        \node [below] at (4, 0) {$x$};
        \draw [ultra thick] (-0.5, 0) -- (0, 0);
        \node [left] at (0, 0.7) {$f(x)$};
        \node [below left] at (0, 0) {$0$};
        \node [below] at (\p, 0) {$F_{\alpha,n_1,n_2}$};
        \draw [dashed] (\p, 0) -- (\p, {
            \b*((\p)^(4/2 - 1))/(4*\p + 2)^((4+2)/2)
        });
        \draw [->, thick] (\p+0.7, 0.3) -- (\p+0.3, 0.1);
        \node [right] at (\p+0.7, 0.3) {площа дорівнює $\alpha$};
    \end{tikzpicture} 
\end{center}
Для наведеної таблиці $\alpha = 0.05$.
\begin{center}
    \noindent\begin{tabular}{*{11}{|c}|}
    \hline
    \diagbox{$\pmb{n_2}$}{$\pmb{n_1}$} & \textbf1 & \textbf2 & \textbf3 & \textbf4 & \textbf5 & \textbf6 & \textbf7 & \textbf8 & \textbf9 & \textbf10 \\ 
 \hline 
 \textbf{3} & 10.1 &9.55 &9.28 &9.12 &9.01 &8.94 &8.89 &8.85 &8.81 &8.79 \\
\hline 
\textbf{4} & 7.71 &6.94 &6.59 &6.39 &6.26 &6.16 &6.09 &6.04 &6.00 &5.96 \\
\hline 
\textbf{5} & 6.61 &5.79 &5.41 &5.19 &5.05 &4.95 &4.88 &4.82 &4.77 &4.74 \\
\hline 
\textbf{6} & 5.99 &5.14 &4.76 &4.53 &4.39 &4.28 &4.21 &4.15 &4.10 &4.06 \\
\hline 
\textbf{7} & 5.59 &4.74 &4.35 &4.12 &3.97 &3.87 &3.79 &3.73 &3.68 &3.64 \\
\hline 
\textbf{8} & 5.32 &4.46 &4.07 &3.84 &3.69 &3.58 &3.50 &3.44 &3.39 &3.35 \\
\hline 
\textbf{9} & 5.12 &4.26 &3.86 &3.63 &3.48 &3.37 &3.29 &3.23 &3.18 &3.14 \\
\hline 
\textbf{10} & 4.96 &4.10 &3.71 &3.48 &3.33 &3.22 &3.14 &3.07 &3.02 &2.98 \\
\hline 
\textbf{11} & 4.84 &3.98 &3.59 &3.36 &3.20 &3.09 &3.01 &2.95 &2.90 &2.85 \\
\hline 
\textbf{12} & 4.75 &3.89 &3.49 &3.26 &3.11 &3.00 &2.91 &2.85 &2.80 &2.75 \\
\hline 
\textbf{13} & 4.67 &3.81 &3.41 &3.18 &3.03 &2.92 &2.83 &2.77 &2.71 &2.67 \\
\hline 
\textbf{14} & 4.60 &3.74 &3.34 &3.11 &2.96 &2.85 &2.76 &2.70 &2.65 &2.60 \\
\hline 
\textbf{15} & 4.54 &3.68 &3.29 &3.06 &2.90 &2.79 &2.71 &2.64 &2.59 &2.54 \\
\hline 
\textbf{16} & 4.49 &3.63 &3.24 &3.01 &2.85 &2.74 &2.66 &2.59 &2.54 &2.49 \\
\hline 
\textbf{17} & 4.45 &3.59 &3.20 &2.96 &2.81 &2.70 &2.61 &2.55 &2.49 &2.45 \\
\hline 
\textbf{18} & 4.41 &3.55 &3.16 &2.93 &2.77 &2.66 &2.58 &2.51 &2.46 &2.41 \\
\hline 
\textbf{19} & 4.38 &3.52 &3.13 &2.90 &2.74 &2.63 &2.54 &2.48 &2.42 &2.38 \\
\hline 
\textbf{20} & 4.35 &3.49 &3.10 &2.87 &2.71 &2.60 &2.51 &2.45 &2.39 &2.35 \\
\hline 
\textbf{22} & 4.30 &3.44 &3.05 &2.82 &2.66 &2.55 &2.46 &2.40 &2.34 &2.30 \\
\hline 
\textbf{24} & 4.26 &3.40 &3.01 &2.78 &2.62 &2.51 &2.42 &2.36 &2.30 &2.25 \\
\hline 
\textbf{26} & 4.23 &3.37 &2.98 &2.74 &2.59 &2.47 &2.39 &2.32 &2.27 &2.22 \\
\hline 
\textbf{28} & 4.20 &3.34 &2.95 &2.71 &2.56 &2.45 &2.36 &2.29 &2.24 &2.19 \\
\hline 
\textbf{30} & 4.17 &3.32 &2.92 &2.69 &2.53 &2.42 &2.33 &2.27 &2.21 &2.16 \\
\hline 
\textbf{40} & 4.08 &3.23 &2.84 &2.61 &2.45 &2.34 &2.25 &2.18 &2.12 &2.08 \\
\hline 
\textbf{50} & 4.03 &3.18 &2.79 &2.56 &2.40 &2.29 &2.20 &2.13 &2.07 &2.03 \\
\hline 
\textbf{60} & 4.00 &3.15 &2.76 &2.53 &2.37 &2.25 &2.17 &2.10 &2.04 &1.99 \\
\hline 
\textbf{100} & 3.94 &3.09 &2.70 &2.46 &2.31 &2.19 &2.10 &2.03 &1.97 &1.93 \\
\hline 
\end{tabular}
\end{center}
\newpage
\begin{center}
    \noindent\begin{tabular}{*{11}{|c}|}
    \hline
    \diagbox{$\pmb{n_2}$}{$\pmb{n_1}$} & \textbf12 & \textbf14 & \textbf16 & \textbf18 & \textbf20 & \textbf30 & \textbf40 & \textbf50 & \textbf60 & \textbf100 \\ 
 \hline 
 \textbf{3} & 8.74 &8.71 &8.69 &8.67 &8.66 &8.62 &8.59 &8.58 &8.57 &8.55 \\
\hline 
\textbf{4} & 5.91 &5.87 &5.84 &5.82 &5.80 &5.75 &5.72 &5.70 &5.69 &5.66 \\
\hline 
\textbf{5} & 4.68 &4.64 &4.60 &4.58 &4.56 &4.50 &4.46 &4.44 &4.43 &4.41 \\
\hline 
\textbf{6} & 4.00 &3.96 &3.92 &3.90 &3.87 &3.81 &3.77 &3.75 &3.74 &3.71 \\
\hline 
\textbf{7} & 3.57 &3.53 &3.49 &3.47 &3.44 &3.38 &3.34 &3.32 &3.30 &3.27 \\
\hline 
\textbf{8} & 3.28 &3.24 &3.20 &3.17 &3.15 &3.08 &3.04 &3.02 &3.01 &2.97 \\
\hline 
\textbf{9} & 3.07 &3.03 &2.99 &2.96 &2.94 &2.86 &2.83 &2.80 &2.79 &2.76 \\
\hline 
\textbf{10} & 2.91 &2.86 &2.83 &2.80 &2.77 &2.70 &2.66 &2.64 &2.62 &2.59 \\
\hline 
\textbf{11} & 2.79 &2.74 &2.70 &2.67 &2.65 &2.57 &2.53 &2.51 &2.49 &2.46 \\
\hline 
\textbf{12} & 2.69 &2.64 &2.60 &2.57 &2.54 &2.47 &2.43 &2.40 &2.38 &2.35 \\
\hline 
\textbf{13} & 2.60 &2.55 &2.51 &2.48 &2.46 &2.38 &2.34 &2.31 &2.30 &2.26 \\
\hline 
\textbf{14} & 2.53 &2.48 &2.44 &2.41 &2.39 &2.31 &2.27 &2.24 &2.22 &2.19 \\
\hline 
\textbf{15} & 2.48 &2.42 &2.38 &2.35 &2.33 &2.25 &2.20 &2.18 &2.16 &2.12 \\
\hline 
\textbf{16} & 2.42 &2.37 &2.33 &2.30 &2.28 &2.19 &2.15 &2.12 &2.11 &2.07 \\
\hline 
\textbf{17} & 2.38 &2.33 &2.29 &2.26 &2.23 &2.15 &2.10 &2.08 &2.06 &2.02 \\
\hline 
\textbf{18} & 2.34 &2.29 &2.25 &2.22 &2.19 &2.11 &2.06 &2.04 &2.02 &1.98 \\
\hline 
\textbf{19} & 2.31 &2.26 &2.21 &2.18 &2.16 &2.07 &2.03 &2.00 &1.98 &1.94 \\
\hline 
\textbf{20} & 2.28 &2.22 &2.18 &2.15 &2.12 &2.04 &1.99 &1.97 &1.95 &1.91 \\
\hline 
\textbf{22} & 2.23 &2.17 &2.13 &2.10 &2.07 &1.98 &1.94 &1.91 &1.89 &1.85 \\
\hline 
\textbf{24} & 2.18 &2.13 &2.09 &2.05 &2.03 &1.94 &1.89 &1.86 &1.84 &1.80 \\
\hline 
\textbf{26} & 2.15 &2.09 &2.05 &2.02 &1.99 &1.90 &1.85 &1.82 &1.80 &1.76 \\
\hline 
\textbf{28} & 2.12 &2.06 &2.02 &1.99 &1.96 &1.87 &1.82 &1.79 &1.77 &1.73 \\
\hline 
\textbf{30} & 2.09 &2.04 &1.99 &1.96 &1.93 &1.84 &1.79 &1.76 &1.74 &1.70 \\
\hline 
\textbf{40} & 2.00 &1.95 &1.90 &1.87 &1.84 &1.74 &1.69 &1.66 &1.64 &1.59 \\
\hline 
\textbf{50} & 1.95 &1.89 &1.85 &1.81 &1.78 &1.69 &1.63 &1.60 &1.58 &1.52 \\
\hline 
\textbf{60} & 1.92 &1.86 &1.82 &1.78 &1.75 &1.65 &1.59 &1.56 &1.53 &1.48 \\
\hline 
\textbf{100} & 1.85 &1.79 &1.75 &1.71 &1.68 &1.57 &1.52 &1.48 &1.45 &1.39 \\
\hline 
\end{tabular}
\end{center}\newpage
\end{document}